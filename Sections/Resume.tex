\section{R\'esum\'e}
Wir haben in diesem Versuch die Materialeigenschaften zwei verschiedener Proben Anhand des Quanten-Hall-Effekts, also dem Verhalten zwei dimensionaler Elektronengase bei sehr niedrigen Temperaturen im externen Magnetfeld, untersucht. \\

Aus den daraus gewonnen Messdaten konnten wir erfolgreich die Elektronenkonzentrationen und -mobilitäten der Proben bestimmen und die Veränderungen dieser Größen unter einem dauerhaften Photoeffekt untersuchen. Wir stellten dabei eine Erhöhung dieser Werte fest. 
Während dies zu unseren physikalischen Intuitionen passt, gab es eine Reihe von Beobachtungen, die uns überraschten (zb. die Stärke der SdH-Oszillation).
