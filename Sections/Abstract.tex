\section*{\centerline{Einleitung}}
\begin{addmargin}{1cm}
Der klassische Hall-Effekt ist bereits seit dem 19. Jahrhundert bekannt und (im klassischen Sinn) gut verstanden. Lew Landau vermutete jedoch schon kurz nach Begründung der Quantenmechanik, dass bei tiefen Temperaturen und hohen Magnetfeldern an geeigneten Systemen quantenmechanische Effekte beobachtet werden können, was spätestens durch Klaus von Klitzing 1980 auch experimentell nachgewiesen wurde.\\
Dieser \emph{Quanten-Hall-Effekt} soll in diesem Praktikum untersucht werden. Dazu werden einerseits einfache Magnetotransportmessungen durchgeführt (\ref{sec:Magnetotransport} \& \ref{sec:Mag_trans}) und andererseits weiterführende Messungen zu einem \glqq dauerhaften Photoeffekt\grqq\ (\ref{sec:Photoeffekt} \& \ref{sec:Photo}).
\end{addmargin}
