\section{Auswertung}

Zunächst einige Worte zu den mit LabView aufgenommenen Messwerten: da die Abtastrate der Messgeräte höher gewählt war als die Rate, mit der das Magnetfeld erhöht wurde, kam es vor, dass für gleiche Werte des Magnetfelds mehrere Messwerte aufgenommen wurden.

Zwischen diesen Werten mit konstantem Magnetfeld kann es in den gemessenen Spannungen / Widerständen natürlich zu Fluktuationen gekommen sein. Um einem Wert des Magnetfeld eindeutige Messwerte zuordnen zu können und zu verhindern, dass Messwerte doppelt auftreten (was zum Beispiel beim Fitten für eine ungerechtfertige Gewichtung gesorgt hätte), wurde daher über alle diese Werte gemittelt.

\subsection{Der Magnetotransport}
Die Vermessung des Magnetotransport wurde wie bereits in Abschnitt ? ausgeführt an Probe P07 und über 4 verschiedene Kontaktpaare für den Hallwiderstand / -spannung bzw. der für den Strom angelegten Spannung / dem dort beobachteten Widerstand durchgeführt. Aus diesen Werten lässt sich dann mit den in Abschnitt ? erläuterten Gl. ? auf drei unterschiedlichen Wegen die Elektronenkonzentration und die Ladungsträgermobilität bestimmen. 

Während jede dieser Methoden auf alle 4 Messreihen angewendet wurden, werden im Folgenden nur Darstellungen für eine der Messreihen repräsentativ angeführt. Die Vorgehensweisen lassen sich auf die Übrigen Reihen ohne nennenswerten Anpassungen übertragen - auf Unterschiede wird eingegangen. Die so bestimmten Größen werden für alle Messreihen angegeben und (vergleiched) diskutiert.

\subsubsection{Methode 1: Bestimmung der Hallplateaus}
Nach Gl. ?? lässt sich aus der Lage der Hallplateaus sowie den entsprechenden Füllfaktoren die Elektronenkonzentration berechnet werden. Dafür muss nach Gl. ?? der Füllfaktor bestimmt werden - wir benötigen daher zusätzlich die zu den Plateaus korrespondierenden Hallwiderstände.  Betrachten wir daher die repräsentative Darstellung einer Messreihe in Abb. \ref{abb:P07_Meth_1}:

Man erkennt sehr gut die SdH-Oszillation sowie zwei klare Hallplateaus für Magnetfelder mit $B>\SI{2}{\tesla}$. Auch für schwächere Magnetfelder lassen sich Hallplateaus ausmachen - allerdings deutlich weniger sicher. Für Magnetfelder $B<\SI{1}{\tesla}$ lassen sich keine Plateaus mehr erkennen. Die SdH-Oszillationen sind hingegen über einen deutlich größeren Wertebereich klar und mit eindeutigen Minima versehen. Wir haben uns daher entschieden, die zu Plataeus korrespondierenden Magnetfeldstärken (und damit dazu gehörige Hallwiderstände) aus den Minima der Oszillation zu bestimmen.

\begin{figure}[h!]
	\centering
	%% Creator: Matplotlib, PGF backend
%%
%% To include the figure in your LaTeX document, write
%%   \input{<filename>.pgf}
%%
%% Make sure the required packages are loaded in your preamble
%%   \usepackage{pgf}
%%
%% Figures using additional raster images can only be included by \input if
%% they are in the same directory as the main LaTeX file. For loading figures
%% from other directories you can use the `import` package
%%   \usepackage{import}
%% and then include the figures with
%%   \import{<path to file>}{<filename>.pgf}
%%
%% Matplotlib used the following preamble
%%   \usepackage[utf8x]{inputenc}
%%   \usepackage[T1]{fontenc}
%%
\begingroup%
\makeatletter%
\begin{pgfpicture}%
\pgfpathrectangle{\pgfpointorigin}{\pgfqpoint{6.012886in}{3.716168in}}%
\pgfusepath{use as bounding box, clip}%
\begin{pgfscope}%
\pgfsetbuttcap%
\pgfsetmiterjoin%
\definecolor{currentfill}{rgb}{1.000000,1.000000,1.000000}%
\pgfsetfillcolor{currentfill}%
\pgfsetlinewidth{0.000000pt}%
\definecolor{currentstroke}{rgb}{1.000000,1.000000,1.000000}%
\pgfsetstrokecolor{currentstroke}%
\pgfsetdash{}{0pt}%
\pgfpathmoveto{\pgfqpoint{0.000000in}{0.000000in}}%
\pgfpathlineto{\pgfqpoint{6.012886in}{0.000000in}}%
\pgfpathlineto{\pgfqpoint{6.012886in}{3.716168in}}%
\pgfpathlineto{\pgfqpoint{0.000000in}{3.716168in}}%
\pgfpathclose%
\pgfusepath{fill}%
\end{pgfscope}%
\begin{pgfscope}%
\pgfsetbuttcap%
\pgfsetmiterjoin%
\definecolor{currentfill}{rgb}{1.000000,1.000000,1.000000}%
\pgfsetfillcolor{currentfill}%
\pgfsetlinewidth{0.000000pt}%
\definecolor{currentstroke}{rgb}{0.000000,0.000000,0.000000}%
\pgfsetstrokecolor{currentstroke}%
\pgfsetstrokeopacity{0.000000}%
\pgfsetdash{}{0pt}%
\pgfpathmoveto{\pgfqpoint{0.592318in}{0.451986in}}%
\pgfpathlineto{\pgfqpoint{5.481911in}{0.451986in}}%
\pgfpathlineto{\pgfqpoint{5.481911in}{3.591168in}}%
\pgfpathlineto{\pgfqpoint{0.592318in}{3.591168in}}%
\pgfpathclose%
\pgfusepath{fill}%
\end{pgfscope}%
\begin{pgfscope}%
\pgfpathrectangle{\pgfqpoint{0.592318in}{0.451986in}}{\pgfqpoint{4.889593in}{3.139182in}}%
\pgfusepath{clip}%
\pgfsetbuttcap%
\pgfsetroundjoin%
\pgfsetlinewidth{0.501875pt}%
\definecolor{currentstroke}{rgb}{0.690196,0.690196,0.690196}%
\pgfsetstrokecolor{currentstroke}%
\pgfsetdash{{1.850000pt}{0.800000pt}}{0.000000pt}%
\pgfpathmoveto{\pgfqpoint{0.814572in}{0.451986in}}%
\pgfpathlineto{\pgfqpoint{0.814572in}{3.591168in}}%
\pgfusepath{stroke}%
\end{pgfscope}%
\begin{pgfscope}%
\pgfsetbuttcap%
\pgfsetroundjoin%
\definecolor{currentfill}{rgb}{0.000000,0.000000,0.000000}%
\pgfsetfillcolor{currentfill}%
\pgfsetlinewidth{0.803000pt}%
\definecolor{currentstroke}{rgb}{0.000000,0.000000,0.000000}%
\pgfsetstrokecolor{currentstroke}%
\pgfsetdash{}{0pt}%
\pgfsys@defobject{currentmarker}{\pgfqpoint{0.000000in}{-0.048611in}}{\pgfqpoint{0.000000in}{0.000000in}}{%
\pgfpathmoveto{\pgfqpoint{0.000000in}{0.000000in}}%
\pgfpathlineto{\pgfqpoint{0.000000in}{-0.048611in}}%
\pgfusepath{stroke,fill}%
}%
\begin{pgfscope}%
\pgfsys@transformshift{0.814572in}{0.451986in}%
\pgfsys@useobject{currentmarker}{}%
\end{pgfscope}%
\end{pgfscope}%
\begin{pgfscope}%
\definecolor{textcolor}{rgb}{0.000000,0.000000,0.000000}%
\pgfsetstrokecolor{textcolor}%
\pgfsetfillcolor{textcolor}%
\pgftext[x=0.814572in,y=0.354764in,,top]{\color{textcolor}\rmfamily\fontsize{8.000000}{9.600000}\selectfont \(\displaystyle 0.0\)}%
\end{pgfscope}%
\begin{pgfscope}%
\pgfpathrectangle{\pgfqpoint{0.592318in}{0.451986in}}{\pgfqpoint{4.889593in}{3.139182in}}%
\pgfusepath{clip}%
\pgfsetbuttcap%
\pgfsetroundjoin%
\pgfsetlinewidth{0.501875pt}%
\definecolor{currentstroke}{rgb}{0.690196,0.690196,0.690196}%
\pgfsetstrokecolor{currentstroke}%
\pgfsetdash{{1.850000pt}{0.800000pt}}{0.000000pt}%
\pgfpathmoveto{\pgfqpoint{1.370347in}{0.451986in}}%
\pgfpathlineto{\pgfqpoint{1.370347in}{3.591168in}}%
\pgfusepath{stroke}%
\end{pgfscope}%
\begin{pgfscope}%
\pgfsetbuttcap%
\pgfsetroundjoin%
\definecolor{currentfill}{rgb}{0.000000,0.000000,0.000000}%
\pgfsetfillcolor{currentfill}%
\pgfsetlinewidth{0.803000pt}%
\definecolor{currentstroke}{rgb}{0.000000,0.000000,0.000000}%
\pgfsetstrokecolor{currentstroke}%
\pgfsetdash{}{0pt}%
\pgfsys@defobject{currentmarker}{\pgfqpoint{0.000000in}{-0.048611in}}{\pgfqpoint{0.000000in}{0.000000in}}{%
\pgfpathmoveto{\pgfqpoint{0.000000in}{0.000000in}}%
\pgfpathlineto{\pgfqpoint{0.000000in}{-0.048611in}}%
\pgfusepath{stroke,fill}%
}%
\begin{pgfscope}%
\pgfsys@transformshift{1.370347in}{0.451986in}%
\pgfsys@useobject{currentmarker}{}%
\end{pgfscope}%
\end{pgfscope}%
\begin{pgfscope}%
\definecolor{textcolor}{rgb}{0.000000,0.000000,0.000000}%
\pgfsetstrokecolor{textcolor}%
\pgfsetfillcolor{textcolor}%
\pgftext[x=1.370347in,y=0.354764in,,top]{\color{textcolor}\rmfamily\fontsize{8.000000}{9.600000}\selectfont \(\displaystyle 0.5\)}%
\end{pgfscope}%
\begin{pgfscope}%
\pgfpathrectangle{\pgfqpoint{0.592318in}{0.451986in}}{\pgfqpoint{4.889593in}{3.139182in}}%
\pgfusepath{clip}%
\pgfsetbuttcap%
\pgfsetroundjoin%
\pgfsetlinewidth{0.501875pt}%
\definecolor{currentstroke}{rgb}{0.690196,0.690196,0.690196}%
\pgfsetstrokecolor{currentstroke}%
\pgfsetdash{{1.850000pt}{0.800000pt}}{0.000000pt}%
\pgfpathmoveto{\pgfqpoint{1.926121in}{0.451986in}}%
\pgfpathlineto{\pgfqpoint{1.926121in}{3.591168in}}%
\pgfusepath{stroke}%
\end{pgfscope}%
\begin{pgfscope}%
\pgfsetbuttcap%
\pgfsetroundjoin%
\definecolor{currentfill}{rgb}{0.000000,0.000000,0.000000}%
\pgfsetfillcolor{currentfill}%
\pgfsetlinewidth{0.803000pt}%
\definecolor{currentstroke}{rgb}{0.000000,0.000000,0.000000}%
\pgfsetstrokecolor{currentstroke}%
\pgfsetdash{}{0pt}%
\pgfsys@defobject{currentmarker}{\pgfqpoint{0.000000in}{-0.048611in}}{\pgfqpoint{0.000000in}{0.000000in}}{%
\pgfpathmoveto{\pgfqpoint{0.000000in}{0.000000in}}%
\pgfpathlineto{\pgfqpoint{0.000000in}{-0.048611in}}%
\pgfusepath{stroke,fill}%
}%
\begin{pgfscope}%
\pgfsys@transformshift{1.926121in}{0.451986in}%
\pgfsys@useobject{currentmarker}{}%
\end{pgfscope}%
\end{pgfscope}%
\begin{pgfscope}%
\definecolor{textcolor}{rgb}{0.000000,0.000000,0.000000}%
\pgfsetstrokecolor{textcolor}%
\pgfsetfillcolor{textcolor}%
\pgftext[x=1.926121in,y=0.354764in,,top]{\color{textcolor}\rmfamily\fontsize{8.000000}{9.600000}\selectfont \(\displaystyle 1.0\)}%
\end{pgfscope}%
\begin{pgfscope}%
\pgfpathrectangle{\pgfqpoint{0.592318in}{0.451986in}}{\pgfqpoint{4.889593in}{3.139182in}}%
\pgfusepath{clip}%
\pgfsetbuttcap%
\pgfsetroundjoin%
\pgfsetlinewidth{0.501875pt}%
\definecolor{currentstroke}{rgb}{0.690196,0.690196,0.690196}%
\pgfsetstrokecolor{currentstroke}%
\pgfsetdash{{1.850000pt}{0.800000pt}}{0.000000pt}%
\pgfpathmoveto{\pgfqpoint{2.481896in}{0.451986in}}%
\pgfpathlineto{\pgfqpoint{2.481896in}{3.591168in}}%
\pgfusepath{stroke}%
\end{pgfscope}%
\begin{pgfscope}%
\pgfsetbuttcap%
\pgfsetroundjoin%
\definecolor{currentfill}{rgb}{0.000000,0.000000,0.000000}%
\pgfsetfillcolor{currentfill}%
\pgfsetlinewidth{0.803000pt}%
\definecolor{currentstroke}{rgb}{0.000000,0.000000,0.000000}%
\pgfsetstrokecolor{currentstroke}%
\pgfsetdash{}{0pt}%
\pgfsys@defobject{currentmarker}{\pgfqpoint{0.000000in}{-0.048611in}}{\pgfqpoint{0.000000in}{0.000000in}}{%
\pgfpathmoveto{\pgfqpoint{0.000000in}{0.000000in}}%
\pgfpathlineto{\pgfqpoint{0.000000in}{-0.048611in}}%
\pgfusepath{stroke,fill}%
}%
\begin{pgfscope}%
\pgfsys@transformshift{2.481896in}{0.451986in}%
\pgfsys@useobject{currentmarker}{}%
\end{pgfscope}%
\end{pgfscope}%
\begin{pgfscope}%
\definecolor{textcolor}{rgb}{0.000000,0.000000,0.000000}%
\pgfsetstrokecolor{textcolor}%
\pgfsetfillcolor{textcolor}%
\pgftext[x=2.481896in,y=0.354764in,,top]{\color{textcolor}\rmfamily\fontsize{8.000000}{9.600000}\selectfont \(\displaystyle 1.5\)}%
\end{pgfscope}%
\begin{pgfscope}%
\pgfpathrectangle{\pgfqpoint{0.592318in}{0.451986in}}{\pgfqpoint{4.889593in}{3.139182in}}%
\pgfusepath{clip}%
\pgfsetbuttcap%
\pgfsetroundjoin%
\pgfsetlinewidth{0.501875pt}%
\definecolor{currentstroke}{rgb}{0.690196,0.690196,0.690196}%
\pgfsetstrokecolor{currentstroke}%
\pgfsetdash{{1.850000pt}{0.800000pt}}{0.000000pt}%
\pgfpathmoveto{\pgfqpoint{3.037670in}{0.451986in}}%
\pgfpathlineto{\pgfqpoint{3.037670in}{3.591168in}}%
\pgfusepath{stroke}%
\end{pgfscope}%
\begin{pgfscope}%
\pgfsetbuttcap%
\pgfsetroundjoin%
\definecolor{currentfill}{rgb}{0.000000,0.000000,0.000000}%
\pgfsetfillcolor{currentfill}%
\pgfsetlinewidth{0.803000pt}%
\definecolor{currentstroke}{rgb}{0.000000,0.000000,0.000000}%
\pgfsetstrokecolor{currentstroke}%
\pgfsetdash{}{0pt}%
\pgfsys@defobject{currentmarker}{\pgfqpoint{0.000000in}{-0.048611in}}{\pgfqpoint{0.000000in}{0.000000in}}{%
\pgfpathmoveto{\pgfqpoint{0.000000in}{0.000000in}}%
\pgfpathlineto{\pgfqpoint{0.000000in}{-0.048611in}}%
\pgfusepath{stroke,fill}%
}%
\begin{pgfscope}%
\pgfsys@transformshift{3.037670in}{0.451986in}%
\pgfsys@useobject{currentmarker}{}%
\end{pgfscope}%
\end{pgfscope}%
\begin{pgfscope}%
\definecolor{textcolor}{rgb}{0.000000,0.000000,0.000000}%
\pgfsetstrokecolor{textcolor}%
\pgfsetfillcolor{textcolor}%
\pgftext[x=3.037670in,y=0.354764in,,top]{\color{textcolor}\rmfamily\fontsize{8.000000}{9.600000}\selectfont \(\displaystyle 2.0\)}%
\end{pgfscope}%
\begin{pgfscope}%
\pgfpathrectangle{\pgfqpoint{0.592318in}{0.451986in}}{\pgfqpoint{4.889593in}{3.139182in}}%
\pgfusepath{clip}%
\pgfsetbuttcap%
\pgfsetroundjoin%
\pgfsetlinewidth{0.501875pt}%
\definecolor{currentstroke}{rgb}{0.690196,0.690196,0.690196}%
\pgfsetstrokecolor{currentstroke}%
\pgfsetdash{{1.850000pt}{0.800000pt}}{0.000000pt}%
\pgfpathmoveto{\pgfqpoint{3.593445in}{0.451986in}}%
\pgfpathlineto{\pgfqpoint{3.593445in}{3.591168in}}%
\pgfusepath{stroke}%
\end{pgfscope}%
\begin{pgfscope}%
\pgfsetbuttcap%
\pgfsetroundjoin%
\definecolor{currentfill}{rgb}{0.000000,0.000000,0.000000}%
\pgfsetfillcolor{currentfill}%
\pgfsetlinewidth{0.803000pt}%
\definecolor{currentstroke}{rgb}{0.000000,0.000000,0.000000}%
\pgfsetstrokecolor{currentstroke}%
\pgfsetdash{}{0pt}%
\pgfsys@defobject{currentmarker}{\pgfqpoint{0.000000in}{-0.048611in}}{\pgfqpoint{0.000000in}{0.000000in}}{%
\pgfpathmoveto{\pgfqpoint{0.000000in}{0.000000in}}%
\pgfpathlineto{\pgfqpoint{0.000000in}{-0.048611in}}%
\pgfusepath{stroke,fill}%
}%
\begin{pgfscope}%
\pgfsys@transformshift{3.593445in}{0.451986in}%
\pgfsys@useobject{currentmarker}{}%
\end{pgfscope}%
\end{pgfscope}%
\begin{pgfscope}%
\definecolor{textcolor}{rgb}{0.000000,0.000000,0.000000}%
\pgfsetstrokecolor{textcolor}%
\pgfsetfillcolor{textcolor}%
\pgftext[x=3.593445in,y=0.354764in,,top]{\color{textcolor}\rmfamily\fontsize{8.000000}{9.600000}\selectfont \(\displaystyle 2.5\)}%
\end{pgfscope}%
\begin{pgfscope}%
\pgfpathrectangle{\pgfqpoint{0.592318in}{0.451986in}}{\pgfqpoint{4.889593in}{3.139182in}}%
\pgfusepath{clip}%
\pgfsetbuttcap%
\pgfsetroundjoin%
\pgfsetlinewidth{0.501875pt}%
\definecolor{currentstroke}{rgb}{0.690196,0.690196,0.690196}%
\pgfsetstrokecolor{currentstroke}%
\pgfsetdash{{1.850000pt}{0.800000pt}}{0.000000pt}%
\pgfpathmoveto{\pgfqpoint{4.149219in}{0.451986in}}%
\pgfpathlineto{\pgfqpoint{4.149219in}{3.591168in}}%
\pgfusepath{stroke}%
\end{pgfscope}%
\begin{pgfscope}%
\pgfsetbuttcap%
\pgfsetroundjoin%
\definecolor{currentfill}{rgb}{0.000000,0.000000,0.000000}%
\pgfsetfillcolor{currentfill}%
\pgfsetlinewidth{0.803000pt}%
\definecolor{currentstroke}{rgb}{0.000000,0.000000,0.000000}%
\pgfsetstrokecolor{currentstroke}%
\pgfsetdash{}{0pt}%
\pgfsys@defobject{currentmarker}{\pgfqpoint{0.000000in}{-0.048611in}}{\pgfqpoint{0.000000in}{0.000000in}}{%
\pgfpathmoveto{\pgfqpoint{0.000000in}{0.000000in}}%
\pgfpathlineto{\pgfqpoint{0.000000in}{-0.048611in}}%
\pgfusepath{stroke,fill}%
}%
\begin{pgfscope}%
\pgfsys@transformshift{4.149219in}{0.451986in}%
\pgfsys@useobject{currentmarker}{}%
\end{pgfscope}%
\end{pgfscope}%
\begin{pgfscope}%
\definecolor{textcolor}{rgb}{0.000000,0.000000,0.000000}%
\pgfsetstrokecolor{textcolor}%
\pgfsetfillcolor{textcolor}%
\pgftext[x=4.149219in,y=0.354764in,,top]{\color{textcolor}\rmfamily\fontsize{8.000000}{9.600000}\selectfont \(\displaystyle 3.0\)}%
\end{pgfscope}%
\begin{pgfscope}%
\pgfpathrectangle{\pgfqpoint{0.592318in}{0.451986in}}{\pgfqpoint{4.889593in}{3.139182in}}%
\pgfusepath{clip}%
\pgfsetbuttcap%
\pgfsetroundjoin%
\pgfsetlinewidth{0.501875pt}%
\definecolor{currentstroke}{rgb}{0.690196,0.690196,0.690196}%
\pgfsetstrokecolor{currentstroke}%
\pgfsetdash{{1.850000pt}{0.800000pt}}{0.000000pt}%
\pgfpathmoveto{\pgfqpoint{4.704994in}{0.451986in}}%
\pgfpathlineto{\pgfqpoint{4.704994in}{3.591168in}}%
\pgfusepath{stroke}%
\end{pgfscope}%
\begin{pgfscope}%
\pgfsetbuttcap%
\pgfsetroundjoin%
\definecolor{currentfill}{rgb}{0.000000,0.000000,0.000000}%
\pgfsetfillcolor{currentfill}%
\pgfsetlinewidth{0.803000pt}%
\definecolor{currentstroke}{rgb}{0.000000,0.000000,0.000000}%
\pgfsetstrokecolor{currentstroke}%
\pgfsetdash{}{0pt}%
\pgfsys@defobject{currentmarker}{\pgfqpoint{0.000000in}{-0.048611in}}{\pgfqpoint{0.000000in}{0.000000in}}{%
\pgfpathmoveto{\pgfqpoint{0.000000in}{0.000000in}}%
\pgfpathlineto{\pgfqpoint{0.000000in}{-0.048611in}}%
\pgfusepath{stroke,fill}%
}%
\begin{pgfscope}%
\pgfsys@transformshift{4.704994in}{0.451986in}%
\pgfsys@useobject{currentmarker}{}%
\end{pgfscope}%
\end{pgfscope}%
\begin{pgfscope}%
\definecolor{textcolor}{rgb}{0.000000,0.000000,0.000000}%
\pgfsetstrokecolor{textcolor}%
\pgfsetfillcolor{textcolor}%
\pgftext[x=4.704994in,y=0.354764in,,top]{\color{textcolor}\rmfamily\fontsize{8.000000}{9.600000}\selectfont \(\displaystyle 3.5\)}%
\end{pgfscope}%
\begin{pgfscope}%
\pgfpathrectangle{\pgfqpoint{0.592318in}{0.451986in}}{\pgfqpoint{4.889593in}{3.139182in}}%
\pgfusepath{clip}%
\pgfsetbuttcap%
\pgfsetroundjoin%
\pgfsetlinewidth{0.501875pt}%
\definecolor{currentstroke}{rgb}{0.690196,0.690196,0.690196}%
\pgfsetstrokecolor{currentstroke}%
\pgfsetdash{{1.850000pt}{0.800000pt}}{0.000000pt}%
\pgfpathmoveto{\pgfqpoint{5.260768in}{0.451986in}}%
\pgfpathlineto{\pgfqpoint{5.260768in}{3.591168in}}%
\pgfusepath{stroke}%
\end{pgfscope}%
\begin{pgfscope}%
\pgfsetbuttcap%
\pgfsetroundjoin%
\definecolor{currentfill}{rgb}{0.000000,0.000000,0.000000}%
\pgfsetfillcolor{currentfill}%
\pgfsetlinewidth{0.803000pt}%
\definecolor{currentstroke}{rgb}{0.000000,0.000000,0.000000}%
\pgfsetstrokecolor{currentstroke}%
\pgfsetdash{}{0pt}%
\pgfsys@defobject{currentmarker}{\pgfqpoint{0.000000in}{-0.048611in}}{\pgfqpoint{0.000000in}{0.000000in}}{%
\pgfpathmoveto{\pgfqpoint{0.000000in}{0.000000in}}%
\pgfpathlineto{\pgfqpoint{0.000000in}{-0.048611in}}%
\pgfusepath{stroke,fill}%
}%
\begin{pgfscope}%
\pgfsys@transformshift{5.260768in}{0.451986in}%
\pgfsys@useobject{currentmarker}{}%
\end{pgfscope}%
\end{pgfscope}%
\begin{pgfscope}%
\definecolor{textcolor}{rgb}{0.000000,0.000000,0.000000}%
\pgfsetstrokecolor{textcolor}%
\pgfsetfillcolor{textcolor}%
\pgftext[x=5.260768in,y=0.354764in,,top]{\color{textcolor}\rmfamily\fontsize{8.000000}{9.600000}\selectfont \(\displaystyle 4.0\)}%
\end{pgfscope}%
\begin{pgfscope}%
\definecolor{textcolor}{rgb}{0.000000,0.000000,0.000000}%
\pgfsetstrokecolor{textcolor}%
\pgfsetfillcolor{textcolor}%
\pgftext[x=3.037115in,y=0.201084in,,top]{\color{textcolor}\rmfamily\fontsize{8.000000}{9.600000}\selectfont Magnetfeld B (T)}%
\end{pgfscope}%
\begin{pgfscope}%
\pgfpathrectangle{\pgfqpoint{0.592318in}{0.451986in}}{\pgfqpoint{4.889593in}{3.139182in}}%
\pgfusepath{clip}%
\pgfsetbuttcap%
\pgfsetroundjoin%
\pgfsetlinewidth{0.501875pt}%
\definecolor{currentstroke}{rgb}{0.690196,0.690196,0.690196}%
\pgfsetstrokecolor{currentstroke}%
\pgfsetdash{{1.850000pt}{0.800000pt}}{0.000000pt}%
\pgfpathmoveto{\pgfqpoint{0.592318in}{0.451986in}}%
\pgfpathlineto{\pgfqpoint{5.481911in}{0.451986in}}%
\pgfusepath{stroke}%
\end{pgfscope}%
\begin{pgfscope}%
\pgfsetbuttcap%
\pgfsetroundjoin%
\definecolor{currentfill}{rgb}{0.000000,0.000000,0.000000}%
\pgfsetfillcolor{currentfill}%
\pgfsetlinewidth{0.803000pt}%
\definecolor{currentstroke}{rgb}{0.000000,0.000000,0.000000}%
\pgfsetstrokecolor{currentstroke}%
\pgfsetdash{}{0pt}%
\pgfsys@defobject{currentmarker}{\pgfqpoint{-0.048611in}{0.000000in}}{\pgfqpoint{0.000000in}{0.000000in}}{%
\pgfpathmoveto{\pgfqpoint{0.000000in}{0.000000in}}%
\pgfpathlineto{\pgfqpoint{-0.048611in}{0.000000in}}%
\pgfusepath{stroke,fill}%
}%
\begin{pgfscope}%
\pgfsys@transformshift{0.592318in}{0.451986in}%
\pgfsys@useobject{currentmarker}{}%
\end{pgfscope}%
\end{pgfscope}%
\begin{pgfscope}%
\definecolor{textcolor}{rgb}{0.000000,0.000000,0.000000}%
\pgfsetstrokecolor{textcolor}%
\pgfsetfillcolor{textcolor}%
\pgftext[x=0.436067in,y=0.413724in,left,base]{\color{textcolor}\rmfamily\fontsize{8.000000}{9.600000}\selectfont \(\displaystyle 0\)}%
\end{pgfscope}%
\begin{pgfscope}%
\pgfpathrectangle{\pgfqpoint{0.592318in}{0.451986in}}{\pgfqpoint{4.889593in}{3.139182in}}%
\pgfusepath{clip}%
\pgfsetbuttcap%
\pgfsetroundjoin%
\pgfsetlinewidth{0.501875pt}%
\definecolor{currentstroke}{rgb}{0.690196,0.690196,0.690196}%
\pgfsetstrokecolor{currentstroke}%
\pgfsetdash{{1.850000pt}{0.800000pt}}{0.000000pt}%
\pgfpathmoveto{\pgfqpoint{0.592318in}{0.934937in}}%
\pgfpathlineto{\pgfqpoint{5.481911in}{0.934937in}}%
\pgfusepath{stroke}%
\end{pgfscope}%
\begin{pgfscope}%
\pgfsetbuttcap%
\pgfsetroundjoin%
\definecolor{currentfill}{rgb}{0.000000,0.000000,0.000000}%
\pgfsetfillcolor{currentfill}%
\pgfsetlinewidth{0.803000pt}%
\definecolor{currentstroke}{rgb}{0.000000,0.000000,0.000000}%
\pgfsetstrokecolor{currentstroke}%
\pgfsetdash{}{0pt}%
\pgfsys@defobject{currentmarker}{\pgfqpoint{-0.048611in}{0.000000in}}{\pgfqpoint{0.000000in}{0.000000in}}{%
\pgfpathmoveto{\pgfqpoint{0.000000in}{0.000000in}}%
\pgfpathlineto{\pgfqpoint{-0.048611in}{0.000000in}}%
\pgfusepath{stroke,fill}%
}%
\begin{pgfscope}%
\pgfsys@transformshift{0.592318in}{0.934937in}%
\pgfsys@useobject{currentmarker}{}%
\end{pgfscope}%
\end{pgfscope}%
\begin{pgfscope}%
\definecolor{textcolor}{rgb}{0.000000,0.000000,0.000000}%
\pgfsetstrokecolor{textcolor}%
\pgfsetfillcolor{textcolor}%
\pgftext[x=0.258982in,y=0.896675in,left,base]{\color{textcolor}\rmfamily\fontsize{8.000000}{9.600000}\selectfont \(\displaystyle 1000\)}%
\end{pgfscope}%
\begin{pgfscope}%
\pgfpathrectangle{\pgfqpoint{0.592318in}{0.451986in}}{\pgfqpoint{4.889593in}{3.139182in}}%
\pgfusepath{clip}%
\pgfsetbuttcap%
\pgfsetroundjoin%
\pgfsetlinewidth{0.501875pt}%
\definecolor{currentstroke}{rgb}{0.690196,0.690196,0.690196}%
\pgfsetstrokecolor{currentstroke}%
\pgfsetdash{{1.850000pt}{0.800000pt}}{0.000000pt}%
\pgfpathmoveto{\pgfqpoint{0.592318in}{1.417888in}}%
\pgfpathlineto{\pgfqpoint{5.481911in}{1.417888in}}%
\pgfusepath{stroke}%
\end{pgfscope}%
\begin{pgfscope}%
\pgfsetbuttcap%
\pgfsetroundjoin%
\definecolor{currentfill}{rgb}{0.000000,0.000000,0.000000}%
\pgfsetfillcolor{currentfill}%
\pgfsetlinewidth{0.803000pt}%
\definecolor{currentstroke}{rgb}{0.000000,0.000000,0.000000}%
\pgfsetstrokecolor{currentstroke}%
\pgfsetdash{}{0pt}%
\pgfsys@defobject{currentmarker}{\pgfqpoint{-0.048611in}{0.000000in}}{\pgfqpoint{0.000000in}{0.000000in}}{%
\pgfpathmoveto{\pgfqpoint{0.000000in}{0.000000in}}%
\pgfpathlineto{\pgfqpoint{-0.048611in}{0.000000in}}%
\pgfusepath{stroke,fill}%
}%
\begin{pgfscope}%
\pgfsys@transformshift{0.592318in}{1.417888in}%
\pgfsys@useobject{currentmarker}{}%
\end{pgfscope}%
\end{pgfscope}%
\begin{pgfscope}%
\definecolor{textcolor}{rgb}{0.000000,0.000000,0.000000}%
\pgfsetstrokecolor{textcolor}%
\pgfsetfillcolor{textcolor}%
\pgftext[x=0.258982in,y=1.379626in,left,base]{\color{textcolor}\rmfamily\fontsize{8.000000}{9.600000}\selectfont \(\displaystyle 2000\)}%
\end{pgfscope}%
\begin{pgfscope}%
\pgfpathrectangle{\pgfqpoint{0.592318in}{0.451986in}}{\pgfqpoint{4.889593in}{3.139182in}}%
\pgfusepath{clip}%
\pgfsetbuttcap%
\pgfsetroundjoin%
\pgfsetlinewidth{0.501875pt}%
\definecolor{currentstroke}{rgb}{0.690196,0.690196,0.690196}%
\pgfsetstrokecolor{currentstroke}%
\pgfsetdash{{1.850000pt}{0.800000pt}}{0.000000pt}%
\pgfpathmoveto{\pgfqpoint{0.592318in}{1.900839in}}%
\pgfpathlineto{\pgfqpoint{5.481911in}{1.900839in}}%
\pgfusepath{stroke}%
\end{pgfscope}%
\begin{pgfscope}%
\pgfsetbuttcap%
\pgfsetroundjoin%
\definecolor{currentfill}{rgb}{0.000000,0.000000,0.000000}%
\pgfsetfillcolor{currentfill}%
\pgfsetlinewidth{0.803000pt}%
\definecolor{currentstroke}{rgb}{0.000000,0.000000,0.000000}%
\pgfsetstrokecolor{currentstroke}%
\pgfsetdash{}{0pt}%
\pgfsys@defobject{currentmarker}{\pgfqpoint{-0.048611in}{0.000000in}}{\pgfqpoint{0.000000in}{0.000000in}}{%
\pgfpathmoveto{\pgfqpoint{0.000000in}{0.000000in}}%
\pgfpathlineto{\pgfqpoint{-0.048611in}{0.000000in}}%
\pgfusepath{stroke,fill}%
}%
\begin{pgfscope}%
\pgfsys@transformshift{0.592318in}{1.900839in}%
\pgfsys@useobject{currentmarker}{}%
\end{pgfscope}%
\end{pgfscope}%
\begin{pgfscope}%
\definecolor{textcolor}{rgb}{0.000000,0.000000,0.000000}%
\pgfsetstrokecolor{textcolor}%
\pgfsetfillcolor{textcolor}%
\pgftext[x=0.258982in,y=1.862577in,left,base]{\color{textcolor}\rmfamily\fontsize{8.000000}{9.600000}\selectfont \(\displaystyle 3000\)}%
\end{pgfscope}%
\begin{pgfscope}%
\pgfpathrectangle{\pgfqpoint{0.592318in}{0.451986in}}{\pgfqpoint{4.889593in}{3.139182in}}%
\pgfusepath{clip}%
\pgfsetbuttcap%
\pgfsetroundjoin%
\pgfsetlinewidth{0.501875pt}%
\definecolor{currentstroke}{rgb}{0.690196,0.690196,0.690196}%
\pgfsetstrokecolor{currentstroke}%
\pgfsetdash{{1.850000pt}{0.800000pt}}{0.000000pt}%
\pgfpathmoveto{\pgfqpoint{0.592318in}{2.383790in}}%
\pgfpathlineto{\pgfqpoint{5.481911in}{2.383790in}}%
\pgfusepath{stroke}%
\end{pgfscope}%
\begin{pgfscope}%
\pgfsetbuttcap%
\pgfsetroundjoin%
\definecolor{currentfill}{rgb}{0.000000,0.000000,0.000000}%
\pgfsetfillcolor{currentfill}%
\pgfsetlinewidth{0.803000pt}%
\definecolor{currentstroke}{rgb}{0.000000,0.000000,0.000000}%
\pgfsetstrokecolor{currentstroke}%
\pgfsetdash{}{0pt}%
\pgfsys@defobject{currentmarker}{\pgfqpoint{-0.048611in}{0.000000in}}{\pgfqpoint{0.000000in}{0.000000in}}{%
\pgfpathmoveto{\pgfqpoint{0.000000in}{0.000000in}}%
\pgfpathlineto{\pgfqpoint{-0.048611in}{0.000000in}}%
\pgfusepath{stroke,fill}%
}%
\begin{pgfscope}%
\pgfsys@transformshift{0.592318in}{2.383790in}%
\pgfsys@useobject{currentmarker}{}%
\end{pgfscope}%
\end{pgfscope}%
\begin{pgfscope}%
\definecolor{textcolor}{rgb}{0.000000,0.000000,0.000000}%
\pgfsetstrokecolor{textcolor}%
\pgfsetfillcolor{textcolor}%
\pgftext[x=0.258982in,y=2.345528in,left,base]{\color{textcolor}\rmfamily\fontsize{8.000000}{9.600000}\selectfont \(\displaystyle 4000\)}%
\end{pgfscope}%
\begin{pgfscope}%
\pgfpathrectangle{\pgfqpoint{0.592318in}{0.451986in}}{\pgfqpoint{4.889593in}{3.139182in}}%
\pgfusepath{clip}%
\pgfsetbuttcap%
\pgfsetroundjoin%
\pgfsetlinewidth{0.501875pt}%
\definecolor{currentstroke}{rgb}{0.690196,0.690196,0.690196}%
\pgfsetstrokecolor{currentstroke}%
\pgfsetdash{{1.850000pt}{0.800000pt}}{0.000000pt}%
\pgfpathmoveto{\pgfqpoint{0.592318in}{2.866741in}}%
\pgfpathlineto{\pgfqpoint{5.481911in}{2.866741in}}%
\pgfusepath{stroke}%
\end{pgfscope}%
\begin{pgfscope}%
\pgfsetbuttcap%
\pgfsetroundjoin%
\definecolor{currentfill}{rgb}{0.000000,0.000000,0.000000}%
\pgfsetfillcolor{currentfill}%
\pgfsetlinewidth{0.803000pt}%
\definecolor{currentstroke}{rgb}{0.000000,0.000000,0.000000}%
\pgfsetstrokecolor{currentstroke}%
\pgfsetdash{}{0pt}%
\pgfsys@defobject{currentmarker}{\pgfqpoint{-0.048611in}{0.000000in}}{\pgfqpoint{0.000000in}{0.000000in}}{%
\pgfpathmoveto{\pgfqpoint{0.000000in}{0.000000in}}%
\pgfpathlineto{\pgfqpoint{-0.048611in}{0.000000in}}%
\pgfusepath{stroke,fill}%
}%
\begin{pgfscope}%
\pgfsys@transformshift{0.592318in}{2.866741in}%
\pgfsys@useobject{currentmarker}{}%
\end{pgfscope}%
\end{pgfscope}%
\begin{pgfscope}%
\definecolor{textcolor}{rgb}{0.000000,0.000000,0.000000}%
\pgfsetstrokecolor{textcolor}%
\pgfsetfillcolor{textcolor}%
\pgftext[x=0.258982in,y=2.828479in,left,base]{\color{textcolor}\rmfamily\fontsize{8.000000}{9.600000}\selectfont \(\displaystyle 5000\)}%
\end{pgfscope}%
\begin{pgfscope}%
\pgfpathrectangle{\pgfqpoint{0.592318in}{0.451986in}}{\pgfqpoint{4.889593in}{3.139182in}}%
\pgfusepath{clip}%
\pgfsetbuttcap%
\pgfsetroundjoin%
\pgfsetlinewidth{0.501875pt}%
\definecolor{currentstroke}{rgb}{0.690196,0.690196,0.690196}%
\pgfsetstrokecolor{currentstroke}%
\pgfsetdash{{1.850000pt}{0.800000pt}}{0.000000pt}%
\pgfpathmoveto{\pgfqpoint{0.592318in}{3.349692in}}%
\pgfpathlineto{\pgfqpoint{5.481911in}{3.349692in}}%
\pgfusepath{stroke}%
\end{pgfscope}%
\begin{pgfscope}%
\pgfsetbuttcap%
\pgfsetroundjoin%
\definecolor{currentfill}{rgb}{0.000000,0.000000,0.000000}%
\pgfsetfillcolor{currentfill}%
\pgfsetlinewidth{0.803000pt}%
\definecolor{currentstroke}{rgb}{0.000000,0.000000,0.000000}%
\pgfsetstrokecolor{currentstroke}%
\pgfsetdash{}{0pt}%
\pgfsys@defobject{currentmarker}{\pgfqpoint{-0.048611in}{0.000000in}}{\pgfqpoint{0.000000in}{0.000000in}}{%
\pgfpathmoveto{\pgfqpoint{0.000000in}{0.000000in}}%
\pgfpathlineto{\pgfqpoint{-0.048611in}{0.000000in}}%
\pgfusepath{stroke,fill}%
}%
\begin{pgfscope}%
\pgfsys@transformshift{0.592318in}{3.349692in}%
\pgfsys@useobject{currentmarker}{}%
\end{pgfscope}%
\end{pgfscope}%
\begin{pgfscope}%
\definecolor{textcolor}{rgb}{0.000000,0.000000,0.000000}%
\pgfsetstrokecolor{textcolor}%
\pgfsetfillcolor{textcolor}%
\pgftext[x=0.258982in,y=3.311430in,left,base]{\color{textcolor}\rmfamily\fontsize{8.000000}{9.600000}\selectfont \(\displaystyle 6000\)}%
\end{pgfscope}%
\begin{pgfscope}%
\definecolor{textcolor}{rgb}{0.000000,0.000000,0.000000}%
\pgfsetstrokecolor{textcolor}%
\pgfsetfillcolor{textcolor}%
\pgftext[x=0.203426in,y=2.021577in,,bottom,rotate=90.000000]{\color{textcolor}\rmfamily\fontsize{8.000000}{9.600000}\selectfont Hallwiderstand \(\displaystyle R_{xy} (\Omega)\)}%
\end{pgfscope}%
\begin{pgfscope}%
\pgfpathrectangle{\pgfqpoint{0.592318in}{0.451986in}}{\pgfqpoint{4.889593in}{3.139182in}}%
\pgfusepath{clip}%
\pgfsetrectcap%
\pgfsetroundjoin%
\pgfsetlinewidth{1.003750pt}%
\definecolor{currentstroke}{rgb}{0.760784,0.211765,0.086275}%
\pgfsetstrokecolor{currentstroke}%
\pgfsetdash{}{0pt}%
\pgfpathmoveto{\pgfqpoint{0.814572in}{0.453528in}}%
\pgfpathlineto{\pgfqpoint{0.819019in}{0.453527in}}%
\pgfpathlineto{\pgfqpoint{0.821242in}{0.454970in}}%
\pgfpathlineto{\pgfqpoint{0.822353in}{0.454970in}}%
\pgfpathlineto{\pgfqpoint{0.824576in}{0.456955in}}%
\pgfpathlineto{\pgfqpoint{0.830134in}{0.460959in}}%
\pgfpathlineto{\pgfqpoint{0.832357in}{0.460959in}}%
\pgfpathlineto{\pgfqpoint{0.834580in}{0.462956in}}%
\pgfpathlineto{\pgfqpoint{0.835692in}{0.462956in}}%
\pgfpathlineto{\pgfqpoint{0.837915in}{0.464955in}}%
\pgfpathlineto{\pgfqpoint{0.839026in}{0.464955in}}%
\pgfpathlineto{\pgfqpoint{0.841250in}{0.466965in}}%
\pgfpathlineto{\pgfqpoint{0.842361in}{0.466965in}}%
\pgfpathlineto{\pgfqpoint{0.844584in}{0.468967in}}%
\pgfpathlineto{\pgfqpoint{0.845696in}{0.468967in}}%
\pgfpathlineto{\pgfqpoint{0.847919in}{0.470942in}}%
\pgfpathlineto{\pgfqpoint{0.849030in}{0.470942in}}%
\pgfpathlineto{\pgfqpoint{0.851254in}{0.472965in}}%
\pgfpathlineto{\pgfqpoint{0.853477in}{0.473970in}}%
\pgfpathlineto{\pgfqpoint{0.857923in}{0.476971in}}%
\pgfpathlineto{\pgfqpoint{0.859034in}{0.476971in}}%
\pgfpathlineto{\pgfqpoint{0.861257in}{0.478967in}}%
\pgfpathlineto{\pgfqpoint{0.862369in}{0.478967in}}%
\pgfpathlineto{\pgfqpoint{0.864592in}{0.480968in}}%
\pgfpathlineto{\pgfqpoint{0.866815in}{0.480968in}}%
\pgfpathlineto{\pgfqpoint{0.869038in}{0.482960in}}%
\pgfpathlineto{\pgfqpoint{0.870150in}{0.482960in}}%
\pgfpathlineto{\pgfqpoint{0.872373in}{0.484963in}}%
\pgfpathlineto{\pgfqpoint{0.873485in}{0.484963in}}%
\pgfpathlineto{\pgfqpoint{0.875708in}{0.486949in}}%
\pgfpathlineto{\pgfqpoint{0.881265in}{0.490944in}}%
\pgfpathlineto{\pgfqpoint{0.883488in}{0.490944in}}%
\pgfpathlineto{\pgfqpoint{0.885712in}{0.492943in}}%
\pgfpathlineto{\pgfqpoint{0.886823in}{0.492943in}}%
\pgfpathlineto{\pgfqpoint{0.889046in}{0.494911in}}%
\pgfpathlineto{\pgfqpoint{0.890158in}{0.494911in}}%
\pgfpathlineto{\pgfqpoint{0.892381in}{0.496908in}}%
\pgfpathlineto{\pgfqpoint{0.893492in}{0.496908in}}%
\pgfpathlineto{\pgfqpoint{0.895715in}{0.498884in}}%
\pgfpathlineto{\pgfqpoint{0.896827in}{0.498884in}}%
\pgfpathlineto{\pgfqpoint{0.899050in}{0.500863in}}%
\pgfpathlineto{\pgfqpoint{0.901273in}{0.501864in}}%
\pgfpathlineto{\pgfqpoint{0.905719in}{0.504854in}}%
\pgfpathlineto{\pgfqpoint{0.906831in}{0.504854in}}%
\pgfpathlineto{\pgfqpoint{0.909054in}{0.506857in}}%
\pgfpathlineto{\pgfqpoint{0.910166in}{0.506857in}}%
\pgfpathlineto{\pgfqpoint{0.912389in}{0.508838in}}%
\pgfpathlineto{\pgfqpoint{0.914612in}{0.508838in}}%
\pgfpathlineto{\pgfqpoint{0.916835in}{0.510831in}}%
\pgfpathlineto{\pgfqpoint{0.917946in}{0.510831in}}%
\pgfpathlineto{\pgfqpoint{0.921281in}{0.514805in}}%
\pgfpathlineto{\pgfqpoint{0.923504in}{0.514805in}}%
\pgfpathlineto{\pgfqpoint{0.925727in}{0.516778in}}%
\pgfpathlineto{\pgfqpoint{0.927950in}{0.517783in}}%
\pgfpathlineto{\pgfqpoint{0.930174in}{0.518787in}}%
\pgfpathlineto{\pgfqpoint{0.931285in}{0.518787in}}%
\pgfpathlineto{\pgfqpoint{0.933508in}{0.520774in}}%
\pgfpathlineto{\pgfqpoint{0.934620in}{0.520774in}}%
\pgfpathlineto{\pgfqpoint{0.936843in}{0.522779in}}%
\pgfpathlineto{\pgfqpoint{0.942401in}{0.526725in}}%
\pgfpathlineto{\pgfqpoint{0.944624in}{0.526725in}}%
\pgfpathlineto{\pgfqpoint{0.946847in}{0.528732in}}%
\pgfpathlineto{\pgfqpoint{0.947958in}{0.528732in}}%
\pgfpathlineto{\pgfqpoint{0.950181in}{0.530736in}}%
\pgfpathlineto{\pgfqpoint{0.951293in}{0.530736in}}%
\pgfpathlineto{\pgfqpoint{0.953516in}{0.532745in}}%
\pgfpathlineto{\pgfqpoint{0.954628in}{0.532745in}}%
\pgfpathlineto{\pgfqpoint{0.956851in}{0.534710in}}%
\pgfpathlineto{\pgfqpoint{0.957962in}{0.534710in}}%
\pgfpathlineto{\pgfqpoint{0.960185in}{0.536737in}}%
\pgfpathlineto{\pgfqpoint{0.962408in}{0.537740in}}%
\pgfpathlineto{\pgfqpoint{0.966855in}{0.540724in}}%
\pgfpathlineto{\pgfqpoint{0.967966in}{0.540724in}}%
\pgfpathlineto{\pgfqpoint{0.970189in}{0.542740in}}%
\pgfpathlineto{\pgfqpoint{0.971301in}{0.542740in}}%
\pgfpathlineto{\pgfqpoint{0.973524in}{0.544727in}}%
\pgfpathlineto{\pgfqpoint{0.974635in}{0.544727in}}%
\pgfpathlineto{\pgfqpoint{0.976859in}{0.546631in}}%
\pgfpathlineto{\pgfqpoint{0.979082in}{0.546631in}}%
\pgfpathlineto{\pgfqpoint{0.981305in}{0.548775in}}%
\pgfpathlineto{\pgfqpoint{0.986863in}{0.552670in}}%
\pgfpathlineto{\pgfqpoint{0.989086in}{0.553658in}}%
\pgfpathlineto{\pgfqpoint{0.991309in}{0.554647in}}%
\pgfpathlineto{\pgfqpoint{0.992420in}{0.554647in}}%
\pgfpathlineto{\pgfqpoint{0.994643in}{0.556674in}}%
\pgfpathlineto{\pgfqpoint{0.995755in}{0.556674in}}%
\pgfpathlineto{\pgfqpoint{0.997978in}{0.558662in}}%
\pgfpathlineto{\pgfqpoint{0.999090in}{0.558662in}}%
\pgfpathlineto{\pgfqpoint{1.001313in}{0.560629in}}%
\pgfpathlineto{\pgfqpoint{1.002424in}{0.560629in}}%
\pgfpathlineto{\pgfqpoint{1.004647in}{0.562638in}}%
\pgfpathlineto{\pgfqpoint{1.005759in}{0.562638in}}%
\pgfpathlineto{\pgfqpoint{1.007982in}{0.564638in}}%
\pgfpathlineto{\pgfqpoint{1.010205in}{0.565625in}}%
\pgfpathlineto{\pgfqpoint{1.014651in}{0.568617in}}%
\pgfpathlineto{\pgfqpoint{1.015763in}{0.568617in}}%
\pgfpathlineto{\pgfqpoint{1.017986in}{0.570603in}}%
\pgfpathlineto{\pgfqpoint{1.019097in}{0.570603in}}%
\pgfpathlineto{\pgfqpoint{1.021321in}{0.572636in}}%
\pgfpathlineto{\pgfqpoint{1.022432in}{0.572636in}}%
\pgfpathlineto{\pgfqpoint{1.024655in}{0.574604in}}%
\pgfpathlineto{\pgfqpoint{1.026878in}{0.575605in}}%
\pgfpathlineto{\pgfqpoint{1.031324in}{0.578575in}}%
\pgfpathlineto{\pgfqpoint{1.033548in}{0.579569in}}%
\pgfpathlineto{\pgfqpoint{1.037994in}{0.582574in}}%
\pgfpathlineto{\pgfqpoint{1.039105in}{0.582574in}}%
\pgfpathlineto{\pgfqpoint{1.041328in}{0.584578in}}%
\pgfpathlineto{\pgfqpoint{1.043552in}{0.584578in}}%
\pgfpathlineto{\pgfqpoint{1.045775in}{0.586572in}}%
\pgfpathlineto{\pgfqpoint{1.046886in}{0.586572in}}%
\pgfpathlineto{\pgfqpoint{1.049109in}{0.588550in}}%
\pgfpathlineto{\pgfqpoint{1.050221in}{0.588550in}}%
\pgfpathlineto{\pgfqpoint{1.052444in}{0.590569in}}%
\pgfpathlineto{\pgfqpoint{1.053555in}{0.590569in}}%
\pgfpathlineto{\pgfqpoint{1.055779in}{0.592572in}}%
\pgfpathlineto{\pgfqpoint{1.058002in}{0.593557in}}%
\pgfpathlineto{\pgfqpoint{1.062448in}{0.596531in}}%
\pgfpathlineto{\pgfqpoint{1.063559in}{0.596531in}}%
\pgfpathlineto{\pgfqpoint{1.065782in}{0.598552in}}%
\pgfpathlineto{\pgfqpoint{1.066894in}{0.598552in}}%
\pgfpathlineto{\pgfqpoint{1.069117in}{0.600552in}}%
\pgfpathlineto{\pgfqpoint{1.071340in}{0.601557in}}%
\pgfpathlineto{\pgfqpoint{1.073563in}{0.602561in}}%
\pgfpathlineto{\pgfqpoint{1.074675in}{0.602561in}}%
\pgfpathlineto{\pgfqpoint{1.078010in}{0.606534in}}%
\pgfpathlineto{\pgfqpoint{1.080233in}{0.606534in}}%
\pgfpathlineto{\pgfqpoint{1.082456in}{0.608541in}}%
\pgfpathlineto{\pgfqpoint{1.083567in}{0.608541in}}%
\pgfpathlineto{\pgfqpoint{1.085790in}{0.610542in}}%
\pgfpathlineto{\pgfqpoint{1.086902in}{0.610542in}}%
\pgfpathlineto{\pgfqpoint{1.089125in}{0.612488in}}%
\pgfpathlineto{\pgfqpoint{1.091348in}{0.612488in}}%
\pgfpathlineto{\pgfqpoint{1.093571in}{0.614494in}}%
\pgfpathlineto{\pgfqpoint{1.094683in}{0.614494in}}%
\pgfpathlineto{\pgfqpoint{1.096906in}{0.616482in}}%
\pgfpathlineto{\pgfqpoint{1.098017in}{0.616482in}}%
\pgfpathlineto{\pgfqpoint{1.100240in}{0.618497in}}%
\pgfpathlineto{\pgfqpoint{1.105798in}{0.622495in}}%
\pgfpathlineto{\pgfqpoint{1.108021in}{0.622495in}}%
\pgfpathlineto{\pgfqpoint{1.110244in}{0.624503in}}%
\pgfpathlineto{\pgfqpoint{1.111356in}{0.624503in}}%
\pgfpathlineto{\pgfqpoint{1.113579in}{0.626520in}}%
\pgfpathlineto{\pgfqpoint{1.114691in}{0.626520in}}%
\pgfpathlineto{\pgfqpoint{1.116914in}{0.628530in}}%
\pgfpathlineto{\pgfqpoint{1.118025in}{0.628530in}}%
\pgfpathlineto{\pgfqpoint{1.120248in}{0.630534in}}%
\pgfpathlineto{\pgfqpoint{1.122471in}{0.631529in}}%
\pgfpathlineto{\pgfqpoint{1.126918in}{0.634530in}}%
\pgfpathlineto{\pgfqpoint{1.128029in}{0.634530in}}%
\pgfpathlineto{\pgfqpoint{1.130252in}{0.636548in}}%
\pgfpathlineto{\pgfqpoint{1.131364in}{0.636548in}}%
\pgfpathlineto{\pgfqpoint{1.133587in}{0.638560in}}%
\pgfpathlineto{\pgfqpoint{1.134699in}{0.638560in}}%
\pgfpathlineto{\pgfqpoint{1.136922in}{0.640553in}}%
\pgfpathlineto{\pgfqpoint{1.139145in}{0.641559in}}%
\pgfpathlineto{\pgfqpoint{1.143591in}{0.644533in}}%
\pgfpathlineto{\pgfqpoint{1.145814in}{0.645535in}}%
\pgfpathlineto{\pgfqpoint{1.148037in}{0.646536in}}%
\pgfpathlineto{\pgfqpoint{1.149149in}{0.646536in}}%
\pgfpathlineto{\pgfqpoint{1.152483in}{0.650506in}}%
\pgfpathlineto{\pgfqpoint{1.155818in}{0.650506in}}%
\pgfpathlineto{\pgfqpoint{1.158041in}{0.652522in}}%
\pgfpathlineto{\pgfqpoint{1.159153in}{0.652522in}}%
\pgfpathlineto{\pgfqpoint{1.161376in}{0.654485in}}%
\pgfpathlineto{\pgfqpoint{1.162487in}{0.654485in}}%
\pgfpathlineto{\pgfqpoint{1.164710in}{0.656498in}}%
\pgfpathlineto{\pgfqpoint{1.165822in}{0.656498in}}%
\pgfpathlineto{\pgfqpoint{1.168045in}{0.658464in}}%
\pgfpathlineto{\pgfqpoint{1.169157in}{0.658464in}}%
\pgfpathlineto{\pgfqpoint{1.171380in}{0.660466in}}%
\pgfpathlineto{\pgfqpoint{1.172491in}{0.660466in}}%
\pgfpathlineto{\pgfqpoint{1.174714in}{0.662479in}}%
\pgfpathlineto{\pgfqpoint{1.175826in}{0.662479in}}%
\pgfpathlineto{\pgfqpoint{1.178049in}{0.664473in}}%
\pgfpathlineto{\pgfqpoint{1.179160in}{0.664473in}}%
\pgfpathlineto{\pgfqpoint{1.181384in}{0.666470in}}%
\pgfpathlineto{\pgfqpoint{1.182495in}{0.666470in}}%
\pgfpathlineto{\pgfqpoint{1.184718in}{0.668447in}}%
\pgfpathlineto{\pgfqpoint{1.186941in}{0.669449in}}%
\pgfpathlineto{\pgfqpoint{1.191388in}{0.672446in}}%
\pgfpathlineto{\pgfqpoint{1.192499in}{0.672446in}}%
\pgfpathlineto{\pgfqpoint{1.194722in}{0.674456in}}%
\pgfpathlineto{\pgfqpoint{1.195834in}{0.674456in}}%
\pgfpathlineto{\pgfqpoint{1.198057in}{0.676513in}}%
\pgfpathlineto{\pgfqpoint{1.200280in}{0.676513in}}%
\pgfpathlineto{\pgfqpoint{1.202503in}{0.678503in}}%
\pgfpathlineto{\pgfqpoint{1.203615in}{0.678503in}}%
\pgfpathlineto{\pgfqpoint{1.205838in}{0.680443in}}%
\pgfpathlineto{\pgfqpoint{1.206949in}{0.680443in}}%
\pgfpathlineto{\pgfqpoint{1.210284in}{0.684468in}}%
\pgfpathlineto{\pgfqpoint{1.213618in}{0.684468in}}%
\pgfpathlineto{\pgfqpoint{1.215842in}{0.686470in}}%
\pgfpathlineto{\pgfqpoint{1.216953in}{0.686470in}}%
\pgfpathlineto{\pgfqpoint{1.219176in}{0.688494in}}%
\pgfpathlineto{\pgfqpoint{1.220288in}{0.688494in}}%
\pgfpathlineto{\pgfqpoint{1.222511in}{0.690488in}}%
\pgfpathlineto{\pgfqpoint{1.223622in}{0.690488in}}%
\pgfpathlineto{\pgfqpoint{1.225846in}{0.692479in}}%
\pgfpathlineto{\pgfqpoint{1.226957in}{0.692479in}}%
\pgfpathlineto{\pgfqpoint{1.229180in}{0.694478in}}%
\pgfpathlineto{\pgfqpoint{1.230292in}{0.694478in}}%
\pgfpathlineto{\pgfqpoint{1.232515in}{0.696457in}}%
\pgfpathlineto{\pgfqpoint{1.233626in}{0.696457in}}%
\pgfpathlineto{\pgfqpoint{1.235849in}{0.698436in}}%
\pgfpathlineto{\pgfqpoint{1.238073in}{0.699442in}}%
\pgfpathlineto{\pgfqpoint{1.240296in}{0.700448in}}%
\pgfpathlineto{\pgfqpoint{1.241407in}{0.700448in}}%
\pgfpathlineto{\pgfqpoint{1.244742in}{0.704453in}}%
\pgfpathlineto{\pgfqpoint{1.246965in}{0.704453in}}%
\pgfpathlineto{\pgfqpoint{1.249188in}{0.706469in}}%
\pgfpathlineto{\pgfqpoint{1.250300in}{0.706469in}}%
\pgfpathlineto{\pgfqpoint{1.252523in}{0.708461in}}%
\pgfpathlineto{\pgfqpoint{1.254746in}{0.708461in}}%
\pgfpathlineto{\pgfqpoint{1.256969in}{0.710475in}}%
\pgfpathlineto{\pgfqpoint{1.258080in}{0.710475in}}%
\pgfpathlineto{\pgfqpoint{1.260304in}{0.712464in}}%
\pgfpathlineto{\pgfqpoint{1.261415in}{0.712464in}}%
\pgfpathlineto{\pgfqpoint{1.263638in}{0.714475in}}%
\pgfpathlineto{\pgfqpoint{1.264750in}{0.714475in}}%
\pgfpathlineto{\pgfqpoint{1.266973in}{0.716464in}}%
\pgfpathlineto{\pgfqpoint{1.268084in}{0.716464in}}%
\pgfpathlineto{\pgfqpoint{1.270307in}{0.718471in}}%
\pgfpathlineto{\pgfqpoint{1.271419in}{0.718471in}}%
\pgfpathlineto{\pgfqpoint{1.273642in}{0.720471in}}%
\pgfpathlineto{\pgfqpoint{1.274754in}{0.720471in}}%
\pgfpathlineto{\pgfqpoint{1.276977in}{0.722459in}}%
\pgfpathlineto{\pgfqpoint{1.278088in}{0.722459in}}%
\pgfpathlineto{\pgfqpoint{1.280311in}{0.724452in}}%
\pgfpathlineto{\pgfqpoint{1.281423in}{0.724452in}}%
\pgfpathlineto{\pgfqpoint{1.283646in}{0.726428in}}%
\pgfpathlineto{\pgfqpoint{1.285869in}{0.726428in}}%
\pgfpathlineto{\pgfqpoint{1.288092in}{0.728463in}}%
\pgfpathlineto{\pgfqpoint{1.293650in}{0.732463in}}%
\pgfpathlineto{\pgfqpoint{1.294762in}{0.732463in}}%
\pgfpathlineto{\pgfqpoint{1.296985in}{0.734515in}}%
\pgfpathlineto{\pgfqpoint{1.298096in}{0.734515in}}%
\pgfpathlineto{\pgfqpoint{1.300319in}{0.736466in}}%
\pgfpathlineto{\pgfqpoint{1.302542in}{0.737468in}}%
\pgfpathlineto{\pgfqpoint{1.306989in}{0.740434in}}%
\pgfpathlineto{\pgfqpoint{1.309212in}{0.741435in}}%
\pgfpathlineto{\pgfqpoint{1.313658in}{0.744435in}}%
\pgfpathlineto{\pgfqpoint{1.315881in}{0.744435in}}%
\pgfpathlineto{\pgfqpoint{1.318104in}{0.746409in}}%
\pgfpathlineto{\pgfqpoint{1.319216in}{0.746409in}}%
\pgfpathlineto{\pgfqpoint{1.321439in}{0.748420in}}%
\pgfpathlineto{\pgfqpoint{1.322550in}{0.748420in}}%
\pgfpathlineto{\pgfqpoint{1.324773in}{0.750377in}}%
\pgfpathlineto{\pgfqpoint{1.325885in}{0.750377in}}%
\pgfpathlineto{\pgfqpoint{1.328108in}{0.752328in}}%
\pgfpathlineto{\pgfqpoint{1.331443in}{0.754395in}}%
\pgfpathlineto{\pgfqpoint{1.333666in}{0.755384in}}%
\pgfpathlineto{\pgfqpoint{1.338112in}{0.758419in}}%
\pgfpathlineto{\pgfqpoint{1.339224in}{0.758419in}}%
\pgfpathlineto{\pgfqpoint{1.341447in}{0.760405in}}%
\pgfpathlineto{\pgfqpoint{1.342558in}{0.760405in}}%
\pgfpathlineto{\pgfqpoint{1.344781in}{0.762439in}}%
\pgfpathlineto{\pgfqpoint{1.345893in}{0.762439in}}%
\pgfpathlineto{\pgfqpoint{1.348116in}{0.764461in}}%
\pgfpathlineto{\pgfqpoint{1.350339in}{0.765450in}}%
\pgfpathlineto{\pgfqpoint{1.354785in}{0.768452in}}%
\pgfpathlineto{\pgfqpoint{1.357008in}{0.769434in}}%
\pgfpathlineto{\pgfqpoint{1.361455in}{0.772437in}}%
\pgfpathlineto{\pgfqpoint{1.363678in}{0.773421in}}%
\pgfpathlineto{\pgfqpoint{1.368124in}{0.776403in}}%
\pgfpathlineto{\pgfqpoint{1.370347in}{0.776403in}}%
\pgfpathlineto{\pgfqpoint{1.372570in}{0.778393in}}%
\pgfpathlineto{\pgfqpoint{1.373682in}{0.778393in}}%
\pgfpathlineto{\pgfqpoint{1.375905in}{0.780364in}}%
\pgfpathlineto{\pgfqpoint{1.379239in}{0.782403in}}%
\pgfpathlineto{\pgfqpoint{1.380351in}{0.782403in}}%
\pgfpathlineto{\pgfqpoint{1.382574in}{0.784386in}}%
\pgfpathlineto{\pgfqpoint{1.383685in}{0.784386in}}%
\pgfpathlineto{\pgfqpoint{1.385909in}{0.786369in}}%
\pgfpathlineto{\pgfqpoint{1.387020in}{0.786369in}}%
\pgfpathlineto{\pgfqpoint{1.389243in}{0.788356in}}%
\pgfpathlineto{\pgfqpoint{1.390355in}{0.788356in}}%
\pgfpathlineto{\pgfqpoint{1.392578in}{0.790386in}}%
\pgfpathlineto{\pgfqpoint{1.394801in}{0.791390in}}%
\pgfpathlineto{\pgfqpoint{1.399247in}{0.794377in}}%
\pgfpathlineto{\pgfqpoint{1.401470in}{0.795385in}}%
\pgfpathlineto{\pgfqpoint{1.405916in}{0.798393in}}%
\pgfpathlineto{\pgfqpoint{1.407028in}{0.798393in}}%
\pgfpathlineto{\pgfqpoint{1.409251in}{0.800369in}}%
\pgfpathlineto{\pgfqpoint{1.411474in}{0.800369in}}%
\pgfpathlineto{\pgfqpoint{1.413697in}{0.802341in}}%
\pgfpathlineto{\pgfqpoint{1.414809in}{0.802341in}}%
\pgfpathlineto{\pgfqpoint{1.417032in}{0.804311in}}%
\pgfpathlineto{\pgfqpoint{1.418144in}{0.804311in}}%
\pgfpathlineto{\pgfqpoint{1.420367in}{0.806307in}}%
\pgfpathlineto{\pgfqpoint{1.425924in}{0.810371in}}%
\pgfpathlineto{\pgfqpoint{1.428147in}{0.810371in}}%
\pgfpathlineto{\pgfqpoint{1.430371in}{0.812368in}}%
\pgfpathlineto{\pgfqpoint{1.431482in}{0.812368in}}%
\pgfpathlineto{\pgfqpoint{1.433705in}{0.814377in}}%
\pgfpathlineto{\pgfqpoint{1.434817in}{0.814377in}}%
\pgfpathlineto{\pgfqpoint{1.437040in}{0.816361in}}%
\pgfpathlineto{\pgfqpoint{1.438151in}{0.816361in}}%
\pgfpathlineto{\pgfqpoint{1.440374in}{0.818344in}}%
\pgfpathlineto{\pgfqpoint{1.442598in}{0.819332in}}%
\pgfpathlineto{\pgfqpoint{1.447044in}{0.822318in}}%
\pgfpathlineto{\pgfqpoint{1.448155in}{0.822318in}}%
\pgfpathlineto{\pgfqpoint{1.450378in}{0.824268in}}%
\pgfpathlineto{\pgfqpoint{1.451490in}{0.824268in}}%
\pgfpathlineto{\pgfqpoint{1.453713in}{0.826269in}}%
\pgfpathlineto{\pgfqpoint{1.454825in}{0.826269in}}%
\pgfpathlineto{\pgfqpoint{1.457048in}{0.828260in}}%
\pgfpathlineto{\pgfqpoint{1.459271in}{0.829270in}}%
\pgfpathlineto{\pgfqpoint{1.463717in}{0.832281in}}%
\pgfpathlineto{\pgfqpoint{1.464829in}{0.832281in}}%
\pgfpathlineto{\pgfqpoint{1.467052in}{0.834315in}}%
\pgfpathlineto{\pgfqpoint{1.468163in}{0.834315in}}%
\pgfpathlineto{\pgfqpoint{1.470386in}{0.836326in}}%
\pgfpathlineto{\pgfqpoint{1.472609in}{0.836326in}}%
\pgfpathlineto{\pgfqpoint{1.474833in}{0.838292in}}%
\pgfpathlineto{\pgfqpoint{1.480390in}{0.842238in}}%
\pgfpathlineto{\pgfqpoint{1.482613in}{0.843209in}}%
\pgfpathlineto{\pgfqpoint{1.484836in}{0.844180in}}%
\pgfpathlineto{\pgfqpoint{1.485948in}{0.844180in}}%
\pgfpathlineto{\pgfqpoint{1.488171in}{0.846129in}}%
\pgfpathlineto{\pgfqpoint{1.489283in}{0.846129in}}%
\pgfpathlineto{\pgfqpoint{1.491506in}{0.848107in}}%
\pgfpathlineto{\pgfqpoint{1.492617in}{0.848107in}}%
\pgfpathlineto{\pgfqpoint{1.494840in}{0.850076in}}%
\pgfpathlineto{\pgfqpoint{1.495952in}{0.850076in}}%
\pgfpathlineto{\pgfqpoint{1.498175in}{0.852112in}}%
\pgfpathlineto{\pgfqpoint{1.499287in}{0.852112in}}%
\pgfpathlineto{\pgfqpoint{1.501510in}{0.854127in}}%
\pgfpathlineto{\pgfqpoint{1.502621in}{0.854127in}}%
\pgfpathlineto{\pgfqpoint{1.504844in}{0.856187in}}%
\pgfpathlineto{\pgfqpoint{1.505956in}{0.856187in}}%
\pgfpathlineto{\pgfqpoint{1.508179in}{0.858216in}}%
\pgfpathlineto{\pgfqpoint{1.509291in}{0.858216in}}%
\pgfpathlineto{\pgfqpoint{1.511514in}{0.860288in}}%
\pgfpathlineto{\pgfqpoint{1.512625in}{0.860288in}}%
\pgfpathlineto{\pgfqpoint{1.514848in}{0.862281in}}%
\pgfpathlineto{\pgfqpoint{1.517071in}{0.862281in}}%
\pgfpathlineto{\pgfqpoint{1.519294in}{0.864283in}}%
\pgfpathlineto{\pgfqpoint{1.524852in}{0.868184in}}%
\pgfpathlineto{\pgfqpoint{1.527075in}{0.869150in}}%
\pgfpathlineto{\pgfqpoint{1.531522in}{0.872042in}}%
\pgfpathlineto{\pgfqpoint{1.533745in}{0.872042in}}%
\pgfpathlineto{\pgfqpoint{1.535968in}{0.874018in}}%
\pgfpathlineto{\pgfqpoint{1.537079in}{0.874018in}}%
\pgfpathlineto{\pgfqpoint{1.539302in}{0.875998in}}%
\pgfpathlineto{\pgfqpoint{1.540414in}{0.875998in}}%
\pgfpathlineto{\pgfqpoint{1.542637in}{0.878005in}}%
\pgfpathlineto{\pgfqpoint{1.543749in}{0.878005in}}%
\pgfpathlineto{\pgfqpoint{1.545972in}{0.880025in}}%
\pgfpathlineto{\pgfqpoint{1.551529in}{0.884113in}}%
\pgfpathlineto{\pgfqpoint{1.554864in}{0.885143in}}%
\pgfpathlineto{\pgfqpoint{1.559310in}{0.888229in}}%
\pgfpathlineto{\pgfqpoint{1.560422in}{0.888229in}}%
\pgfpathlineto{\pgfqpoint{1.562645in}{0.890253in}}%
\pgfpathlineto{\pgfqpoint{1.563756in}{0.890253in}}%
\pgfpathlineto{\pgfqpoint{1.565980in}{0.892234in}}%
\pgfpathlineto{\pgfqpoint{1.567091in}{0.892234in}}%
\pgfpathlineto{\pgfqpoint{1.569314in}{0.894177in}}%
\pgfpathlineto{\pgfqpoint{1.571537in}{0.895148in}}%
\pgfpathlineto{\pgfqpoint{1.573760in}{0.896120in}}%
\pgfpathlineto{\pgfqpoint{1.574872in}{0.896120in}}%
\pgfpathlineto{\pgfqpoint{1.577095in}{0.898052in}}%
\pgfpathlineto{\pgfqpoint{1.582653in}{0.901893in}}%
\pgfpathlineto{\pgfqpoint{1.583764in}{0.901893in}}%
\pgfpathlineto{\pgfqpoint{1.585987in}{0.903795in}}%
\pgfpathlineto{\pgfqpoint{1.588210in}{0.903795in}}%
\pgfpathlineto{\pgfqpoint{1.590434in}{0.905756in}}%
\pgfpathlineto{\pgfqpoint{1.591545in}{0.905756in}}%
\pgfpathlineto{\pgfqpoint{1.593768in}{0.907747in}}%
\pgfpathlineto{\pgfqpoint{1.594880in}{0.907747in}}%
\pgfpathlineto{\pgfqpoint{1.597103in}{0.909760in}}%
\pgfpathlineto{\pgfqpoint{1.602661in}{0.913903in}}%
\pgfpathlineto{\pgfqpoint{1.604884in}{0.913903in}}%
\pgfpathlineto{\pgfqpoint{1.607107in}{0.915993in}}%
\pgfpathlineto{\pgfqpoint{1.608218in}{0.915993in}}%
\pgfpathlineto{\pgfqpoint{1.610441in}{0.918067in}}%
\pgfpathlineto{\pgfqpoint{1.611553in}{0.918067in}}%
\pgfpathlineto{\pgfqpoint{1.613776in}{0.920140in}}%
\pgfpathlineto{\pgfqpoint{1.614888in}{0.920140in}}%
\pgfpathlineto{\pgfqpoint{1.617111in}{0.922183in}}%
\pgfpathlineto{\pgfqpoint{1.618222in}{0.922183in}}%
\pgfpathlineto{\pgfqpoint{1.620445in}{0.924207in}}%
\pgfpathlineto{\pgfqpoint{1.622669in}{0.925198in}}%
\pgfpathlineto{\pgfqpoint{1.627115in}{0.928147in}}%
\pgfpathlineto{\pgfqpoint{1.628226in}{0.928147in}}%
\pgfpathlineto{\pgfqpoint{1.630449in}{0.930077in}}%
\pgfpathlineto{\pgfqpoint{1.631561in}{0.930077in}}%
\pgfpathlineto{\pgfqpoint{1.633784in}{0.932007in}}%
\pgfpathlineto{\pgfqpoint{1.636007in}{0.932007in}}%
\pgfpathlineto{\pgfqpoint{1.638230in}{0.933896in}}%
\pgfpathlineto{\pgfqpoint{1.639342in}{0.933896in}}%
\pgfpathlineto{\pgfqpoint{1.641565in}{0.935753in}}%
\pgfpathlineto{\pgfqpoint{1.647123in}{0.939573in}}%
\pgfpathlineto{\pgfqpoint{1.649346in}{0.940564in}}%
\pgfpathlineto{\pgfqpoint{1.651569in}{0.941554in}}%
\pgfpathlineto{\pgfqpoint{1.652680in}{0.941554in}}%
\pgfpathlineto{\pgfqpoint{1.654903in}{0.943500in}}%
\pgfpathlineto{\pgfqpoint{1.656015in}{0.943500in}}%
\pgfpathlineto{\pgfqpoint{1.658238in}{0.945523in}}%
\pgfpathlineto{\pgfqpoint{1.663796in}{0.949633in}}%
\pgfpathlineto{\pgfqpoint{1.667130in}{0.950691in}}%
\pgfpathlineto{\pgfqpoint{1.671577in}{0.953845in}}%
\pgfpathlineto{\pgfqpoint{1.672688in}{0.953845in}}%
\pgfpathlineto{\pgfqpoint{1.674911in}{0.955945in}}%
\pgfpathlineto{\pgfqpoint{1.676023in}{0.955945in}}%
\pgfpathlineto{\pgfqpoint{1.678246in}{0.958008in}}%
\pgfpathlineto{\pgfqpoint{1.680469in}{0.959063in}}%
\pgfpathlineto{\pgfqpoint{1.684915in}{0.962142in}}%
\pgfpathlineto{\pgfqpoint{1.687138in}{0.963141in}}%
\pgfpathlineto{\pgfqpoint{1.691585in}{0.966097in}}%
\pgfpathlineto{\pgfqpoint{1.692696in}{0.966097in}}%
\pgfpathlineto{\pgfqpoint{1.694919in}{0.968005in}}%
\pgfpathlineto{\pgfqpoint{1.697142in}{0.968005in}}%
\pgfpathlineto{\pgfqpoint{1.699365in}{0.969893in}}%
\pgfpathlineto{\pgfqpoint{1.700477in}{0.969893in}}%
\pgfpathlineto{\pgfqpoint{1.702700in}{0.971719in}}%
\pgfpathlineto{\pgfqpoint{1.706035in}{0.973554in}}%
\pgfpathlineto{\pgfqpoint{1.707146in}{0.973554in}}%
\pgfpathlineto{\pgfqpoint{1.709369in}{0.975375in}}%
\pgfpathlineto{\pgfqpoint{1.710481in}{0.975375in}}%
\pgfpathlineto{\pgfqpoint{1.712704in}{0.977234in}}%
\pgfpathlineto{\pgfqpoint{1.713816in}{0.977234in}}%
\pgfpathlineto{\pgfqpoint{1.716039in}{0.979132in}}%
\pgfpathlineto{\pgfqpoint{1.717150in}{0.979132in}}%
\pgfpathlineto{\pgfqpoint{1.719373in}{0.981011in}}%
\pgfpathlineto{\pgfqpoint{1.720485in}{0.981011in}}%
\pgfpathlineto{\pgfqpoint{1.722708in}{0.982947in}}%
\pgfpathlineto{\pgfqpoint{1.723819in}{0.982947in}}%
\pgfpathlineto{\pgfqpoint{1.726043in}{0.984952in}}%
\pgfpathlineto{\pgfqpoint{1.728266in}{0.985961in}}%
\pgfpathlineto{\pgfqpoint{1.732712in}{0.989052in}}%
\pgfpathlineto{\pgfqpoint{1.733823in}{0.989052in}}%
\pgfpathlineto{\pgfqpoint{1.736047in}{0.991138in}}%
\pgfpathlineto{\pgfqpoint{1.737158in}{0.991138in}}%
\pgfpathlineto{\pgfqpoint{1.739381in}{0.993287in}}%
\pgfpathlineto{\pgfqpoint{1.741604in}{0.993287in}}%
\pgfpathlineto{\pgfqpoint{1.743827in}{0.995456in}}%
\pgfpathlineto{\pgfqpoint{1.744939in}{0.995456in}}%
\pgfpathlineto{\pgfqpoint{1.747162in}{0.997615in}}%
\pgfpathlineto{\pgfqpoint{1.752720in}{1.001913in}}%
\pgfpathlineto{\pgfqpoint{1.754943in}{1.001913in}}%
\pgfpathlineto{\pgfqpoint{1.757166in}{1.004052in}}%
\pgfpathlineto{\pgfqpoint{1.758277in}{1.004052in}}%
\pgfpathlineto{\pgfqpoint{1.760501in}{1.006168in}}%
\pgfpathlineto{\pgfqpoint{1.761612in}{1.006168in}}%
\pgfpathlineto{\pgfqpoint{1.763835in}{1.008220in}}%
\pgfpathlineto{\pgfqpoint{1.764947in}{1.008220in}}%
\pgfpathlineto{\pgfqpoint{1.767170in}{1.010210in}}%
\pgfpathlineto{\pgfqpoint{1.768281in}{1.010210in}}%
\pgfpathlineto{\pgfqpoint{1.770505in}{1.012151in}}%
\pgfpathlineto{\pgfqpoint{1.771616in}{1.012151in}}%
\pgfpathlineto{\pgfqpoint{1.773839in}{1.014049in}}%
\pgfpathlineto{\pgfqpoint{1.774951in}{1.014049in}}%
\pgfpathlineto{\pgfqpoint{1.777174in}{1.015885in}}%
\pgfpathlineto{\pgfqpoint{1.778285in}{1.015885in}}%
\pgfpathlineto{\pgfqpoint{1.780508in}{1.017705in}}%
\pgfpathlineto{\pgfqpoint{1.781620in}{1.017705in}}%
\pgfpathlineto{\pgfqpoint{1.783843in}{1.019492in}}%
\pgfpathlineto{\pgfqpoint{1.786066in}{1.020378in}}%
\pgfpathlineto{\pgfqpoint{1.790512in}{1.023027in}}%
\pgfpathlineto{\pgfqpoint{1.792736in}{1.023921in}}%
\pgfpathlineto{\pgfqpoint{1.797182in}{1.026577in}}%
\pgfpathlineto{\pgfqpoint{1.798293in}{1.026577in}}%
\pgfpathlineto{\pgfqpoint{1.800516in}{1.028446in}}%
\pgfpathlineto{\pgfqpoint{1.802739in}{1.028446in}}%
\pgfpathlineto{\pgfqpoint{1.804963in}{1.030344in}}%
\pgfpathlineto{\pgfqpoint{1.810520in}{1.034275in}}%
\pgfpathlineto{\pgfqpoint{1.812743in}{1.035314in}}%
\pgfpathlineto{\pgfqpoint{1.814966in}{1.036352in}}%
\pgfpathlineto{\pgfqpoint{1.816078in}{1.036352in}}%
\pgfpathlineto{\pgfqpoint{1.818301in}{1.038414in}}%
\pgfpathlineto{\pgfqpoint{1.819413in}{1.038414in}}%
\pgfpathlineto{\pgfqpoint{1.821636in}{1.040549in}}%
\pgfpathlineto{\pgfqpoint{1.822747in}{1.040549in}}%
\pgfpathlineto{\pgfqpoint{1.824970in}{1.042712in}}%
\pgfpathlineto{\pgfqpoint{1.826082in}{1.042712in}}%
\pgfpathlineto{\pgfqpoint{1.828305in}{1.044886in}}%
\pgfpathlineto{\pgfqpoint{1.829417in}{1.044886in}}%
\pgfpathlineto{\pgfqpoint{1.834974in}{1.049339in}}%
\pgfpathlineto{\pgfqpoint{1.836086in}{1.049339in}}%
\pgfpathlineto{\pgfqpoint{1.838309in}{1.051570in}}%
\pgfpathlineto{\pgfqpoint{1.839421in}{1.051570in}}%
\pgfpathlineto{\pgfqpoint{1.841644in}{1.053772in}}%
\pgfpathlineto{\pgfqpoint{1.842755in}{1.053772in}}%
\pgfpathlineto{\pgfqpoint{1.844978in}{1.055950in}}%
\pgfpathlineto{\pgfqpoint{1.847201in}{1.057020in}}%
\pgfpathlineto{\pgfqpoint{1.851648in}{1.060229in}}%
\pgfpathlineto{\pgfqpoint{1.853871in}{1.061282in}}%
\pgfpathlineto{\pgfqpoint{1.858317in}{1.064329in}}%
\pgfpathlineto{\pgfqpoint{1.860540in}{1.064329in}}%
\pgfpathlineto{\pgfqpoint{1.861652in}{1.066290in}}%
\pgfpathlineto{\pgfqpoint{1.863875in}{1.066290in}}%
\pgfpathlineto{\pgfqpoint{1.866098in}{1.068159in}}%
\pgfpathlineto{\pgfqpoint{1.867209in}{1.068159in}}%
\pgfpathlineto{\pgfqpoint{1.869432in}{1.069994in}}%
\pgfpathlineto{\pgfqpoint{1.870544in}{1.069994in}}%
\pgfpathlineto{\pgfqpoint{1.872767in}{1.071781in}}%
\pgfpathlineto{\pgfqpoint{1.873879in}{1.071781in}}%
\pgfpathlineto{\pgfqpoint{1.876102in}{1.073481in}}%
\pgfpathlineto{\pgfqpoint{1.877213in}{1.073481in}}%
\pgfpathlineto{\pgfqpoint{1.879436in}{1.075196in}}%
\pgfpathlineto{\pgfqpoint{1.880548in}{1.075196in}}%
\pgfpathlineto{\pgfqpoint{1.882771in}{1.076886in}}%
\pgfpathlineto{\pgfqpoint{1.883883in}{1.076886in}}%
\pgfpathlineto{\pgfqpoint{1.886106in}{1.078543in}}%
\pgfpathlineto{\pgfqpoint{1.887217in}{1.078543in}}%
\pgfpathlineto{\pgfqpoint{1.889440in}{1.080218in}}%
\pgfpathlineto{\pgfqpoint{1.890552in}{1.080218in}}%
\pgfpathlineto{\pgfqpoint{1.892775in}{1.081928in}}%
\pgfpathlineto{\pgfqpoint{1.894998in}{1.082812in}}%
\pgfpathlineto{\pgfqpoint{1.899444in}{1.085458in}}%
\pgfpathlineto{\pgfqpoint{1.900556in}{1.085458in}}%
\pgfpathlineto{\pgfqpoint{1.902779in}{1.087323in}}%
\pgfpathlineto{\pgfqpoint{1.903890in}{1.087323in}}%
\pgfpathlineto{\pgfqpoint{1.906114in}{1.089201in}}%
\pgfpathlineto{\pgfqpoint{1.908337in}{1.089201in}}%
\pgfpathlineto{\pgfqpoint{1.910560in}{1.091167in}}%
\pgfpathlineto{\pgfqpoint{1.911671in}{1.091167in}}%
\pgfpathlineto{\pgfqpoint{1.913894in}{1.093176in}}%
\pgfpathlineto{\pgfqpoint{1.915006in}{1.093176in}}%
\pgfpathlineto{\pgfqpoint{1.917229in}{1.095262in}}%
\pgfpathlineto{\pgfqpoint{1.922787in}{1.099517in}}%
\pgfpathlineto{\pgfqpoint{1.925010in}{1.099517in}}%
\pgfpathlineto{\pgfqpoint{1.927233in}{1.101729in}}%
\pgfpathlineto{\pgfqpoint{1.928344in}{1.101729in}}%
\pgfpathlineto{\pgfqpoint{1.930568in}{1.103975in}}%
\pgfpathlineto{\pgfqpoint{1.931679in}{1.103975in}}%
\pgfpathlineto{\pgfqpoint{1.933902in}{1.106264in}}%
\pgfpathlineto{\pgfqpoint{1.935014in}{1.106264in}}%
\pgfpathlineto{\pgfqpoint{1.937237in}{1.108524in}}%
\pgfpathlineto{\pgfqpoint{1.938348in}{1.108524in}}%
\pgfpathlineto{\pgfqpoint{1.943906in}{1.113146in}}%
\pgfpathlineto{\pgfqpoint{1.945018in}{1.113146in}}%
\pgfpathlineto{\pgfqpoint{1.947241in}{1.115459in}}%
\pgfpathlineto{\pgfqpoint{1.948352in}{1.115459in}}%
\pgfpathlineto{\pgfqpoint{1.950575in}{1.117720in}}%
\pgfpathlineto{\pgfqpoint{1.951687in}{1.117720in}}%
\pgfpathlineto{\pgfqpoint{1.953910in}{1.119985in}}%
\pgfpathlineto{\pgfqpoint{1.956133in}{1.119985in}}%
\pgfpathlineto{\pgfqpoint{1.958356in}{1.122221in}}%
\pgfpathlineto{\pgfqpoint{1.963914in}{1.126572in}}%
\pgfpathlineto{\pgfqpoint{1.966137in}{1.127623in}}%
\pgfpathlineto{\pgfqpoint{1.970583in}{1.130696in}}%
\pgfpathlineto{\pgfqpoint{1.972806in}{1.130696in}}%
\pgfpathlineto{\pgfqpoint{1.975030in}{1.132657in}}%
\pgfpathlineto{\pgfqpoint{1.976141in}{1.132657in}}%
\pgfpathlineto{\pgfqpoint{1.978364in}{1.134550in}}%
\pgfpathlineto{\pgfqpoint{1.979476in}{1.134550in}}%
\pgfpathlineto{\pgfqpoint{1.981699in}{1.136381in}}%
\pgfpathlineto{\pgfqpoint{1.982810in}{1.136381in}}%
\pgfpathlineto{\pgfqpoint{1.985033in}{1.138168in}}%
\pgfpathlineto{\pgfqpoint{1.986145in}{1.138168in}}%
\pgfpathlineto{\pgfqpoint{1.988368in}{1.139829in}}%
\pgfpathlineto{\pgfqpoint{1.989480in}{1.139829in}}%
\pgfpathlineto{\pgfqpoint{1.991703in}{1.141510in}}%
\pgfpathlineto{\pgfqpoint{1.992814in}{1.141510in}}%
\pgfpathlineto{\pgfqpoint{1.995037in}{1.143089in}}%
\pgfpathlineto{\pgfqpoint{1.996149in}{1.143089in}}%
\pgfpathlineto{\pgfqpoint{1.998372in}{1.144668in}}%
\pgfpathlineto{\pgfqpoint{1.999484in}{1.144668in}}%
\pgfpathlineto{\pgfqpoint{2.001707in}{1.146219in}}%
\pgfpathlineto{\pgfqpoint{2.003930in}{1.146986in}}%
\pgfpathlineto{\pgfqpoint{2.008376in}{1.149324in}}%
\pgfpathlineto{\pgfqpoint{2.010599in}{1.150109in}}%
\pgfpathlineto{\pgfqpoint{2.015045in}{1.152536in}}%
\pgfpathlineto{\pgfqpoint{2.017268in}{1.152536in}}%
\pgfpathlineto{\pgfqpoint{2.019492in}{1.154207in}}%
\pgfpathlineto{\pgfqpoint{2.020603in}{1.154207in}}%
\pgfpathlineto{\pgfqpoint{2.022826in}{1.155955in}}%
\pgfpathlineto{\pgfqpoint{2.028384in}{1.159601in}}%
\pgfpathlineto{\pgfqpoint{2.029495in}{1.159601in}}%
\pgfpathlineto{\pgfqpoint{2.031719in}{1.161533in}}%
\pgfpathlineto{\pgfqpoint{2.033942in}{1.161533in}}%
\pgfpathlineto{\pgfqpoint{2.036165in}{1.163542in}}%
\pgfpathlineto{\pgfqpoint{2.037276in}{1.163542in}}%
\pgfpathlineto{\pgfqpoint{2.039499in}{1.165580in}}%
\pgfpathlineto{\pgfqpoint{2.040611in}{1.165580in}}%
\pgfpathlineto{\pgfqpoint{2.042834in}{1.167666in}}%
\pgfpathlineto{\pgfqpoint{2.043946in}{1.167666in}}%
\pgfpathlineto{\pgfqpoint{2.046169in}{1.169830in}}%
\pgfpathlineto{\pgfqpoint{2.047280in}{1.169830in}}%
\pgfpathlineto{\pgfqpoint{2.049503in}{1.172032in}}%
\pgfpathlineto{\pgfqpoint{2.050615in}{1.172032in}}%
\pgfpathlineto{\pgfqpoint{2.052838in}{1.174297in}}%
\pgfpathlineto{\pgfqpoint{2.053950in}{1.174297in}}%
\pgfpathlineto{\pgfqpoint{2.056173in}{1.176577in}}%
\pgfpathlineto{\pgfqpoint{2.057284in}{1.176577in}}%
\pgfpathlineto{\pgfqpoint{2.059507in}{1.178929in}}%
\pgfpathlineto{\pgfqpoint{2.060619in}{1.178929in}}%
\pgfpathlineto{\pgfqpoint{2.062842in}{1.181237in}}%
\pgfpathlineto{\pgfqpoint{2.063953in}{1.181237in}}%
\pgfpathlineto{\pgfqpoint{2.069511in}{1.186009in}}%
\pgfpathlineto{\pgfqpoint{2.070623in}{1.186009in}}%
\pgfpathlineto{\pgfqpoint{2.072846in}{1.188428in}}%
\pgfpathlineto{\pgfqpoint{2.073957in}{1.188428in}}%
\pgfpathlineto{\pgfqpoint{2.076180in}{1.190795in}}%
\pgfpathlineto{\pgfqpoint{2.078404in}{1.190795in}}%
\pgfpathlineto{\pgfqpoint{2.080627in}{1.193219in}}%
\pgfpathlineto{\pgfqpoint{2.081738in}{1.193219in}}%
\pgfpathlineto{\pgfqpoint{2.083961in}{1.195605in}}%
\pgfpathlineto{\pgfqpoint{2.085073in}{1.195605in}}%
\pgfpathlineto{\pgfqpoint{2.087296in}{1.197976in}}%
\pgfpathlineto{\pgfqpoint{2.088408in}{1.197976in}}%
\pgfpathlineto{\pgfqpoint{2.090631in}{1.200333in}}%
\pgfpathlineto{\pgfqpoint{2.091742in}{1.200333in}}%
\pgfpathlineto{\pgfqpoint{2.093965in}{1.202642in}}%
\pgfpathlineto{\pgfqpoint{2.095077in}{1.202642in}}%
\pgfpathlineto{\pgfqpoint{2.097300in}{1.204926in}}%
\pgfpathlineto{\pgfqpoint{2.098411in}{1.204926in}}%
\pgfpathlineto{\pgfqpoint{2.100635in}{1.207210in}}%
\pgfpathlineto{\pgfqpoint{2.101746in}{1.207210in}}%
\pgfpathlineto{\pgfqpoint{2.103969in}{1.209418in}}%
\pgfpathlineto{\pgfqpoint{2.105081in}{1.209418in}}%
\pgfpathlineto{\pgfqpoint{2.107304in}{1.211542in}}%
\pgfpathlineto{\pgfqpoint{2.108415in}{1.211542in}}%
\pgfpathlineto{\pgfqpoint{2.110639in}{1.213624in}}%
\pgfpathlineto{\pgfqpoint{2.112862in}{1.213624in}}%
\pgfpathlineto{\pgfqpoint{2.115085in}{1.215672in}}%
\pgfpathlineto{\pgfqpoint{2.116196in}{1.215672in}}%
\pgfpathlineto{\pgfqpoint{2.119531in}{1.219458in}}%
\pgfpathlineto{\pgfqpoint{2.121754in}{1.219458in}}%
\pgfpathlineto{\pgfqpoint{2.123977in}{1.221269in}}%
\pgfpathlineto{\pgfqpoint{2.125089in}{1.221269in}}%
\pgfpathlineto{\pgfqpoint{2.127312in}{1.222998in}}%
\pgfpathlineto{\pgfqpoint{2.129535in}{1.222998in}}%
\pgfpathlineto{\pgfqpoint{2.131758in}{1.224645in}}%
\pgfpathlineto{\pgfqpoint{2.135093in}{1.226190in}}%
\pgfpathlineto{\pgfqpoint{2.136204in}{1.226190in}}%
\pgfpathlineto{\pgfqpoint{2.138427in}{1.227726in}}%
\pgfpathlineto{\pgfqpoint{2.139539in}{1.227726in}}%
\pgfpathlineto{\pgfqpoint{2.141762in}{1.229170in}}%
\pgfpathlineto{\pgfqpoint{2.142873in}{1.229170in}}%
\pgfpathlineto{\pgfqpoint{2.145097in}{1.230580in}}%
\pgfpathlineto{\pgfqpoint{2.146208in}{1.230580in}}%
\pgfpathlineto{\pgfqpoint{2.148431in}{1.231976in}}%
\pgfpathlineto{\pgfqpoint{2.149543in}{1.231976in}}%
\pgfpathlineto{\pgfqpoint{2.151766in}{1.233338in}}%
\pgfpathlineto{\pgfqpoint{2.152877in}{1.233338in}}%
\pgfpathlineto{\pgfqpoint{2.155100in}{1.234719in}}%
\pgfpathlineto{\pgfqpoint{2.156212in}{1.234719in}}%
\pgfpathlineto{\pgfqpoint{2.158435in}{1.236115in}}%
\pgfpathlineto{\pgfqpoint{2.159547in}{1.236115in}}%
\pgfpathlineto{\pgfqpoint{2.161770in}{1.237545in}}%
\pgfpathlineto{\pgfqpoint{2.162881in}{1.237545in}}%
\pgfpathlineto{\pgfqpoint{2.165104in}{1.239013in}}%
\pgfpathlineto{\pgfqpoint{2.167328in}{1.239773in}}%
\pgfpathlineto{\pgfqpoint{2.171774in}{1.242157in}}%
\pgfpathlineto{\pgfqpoint{2.172885in}{1.242157in}}%
\pgfpathlineto{\pgfqpoint{2.175108in}{1.243818in}}%
\pgfpathlineto{\pgfqpoint{2.177331in}{1.244685in}}%
\pgfpathlineto{\pgfqpoint{2.181778in}{1.247363in}}%
\pgfpathlineto{\pgfqpoint{2.184001in}{1.248295in}}%
\pgfpathlineto{\pgfqpoint{2.188447in}{1.251188in}}%
\pgfpathlineto{\pgfqpoint{2.190670in}{1.252207in}}%
\pgfpathlineto{\pgfqpoint{2.192893in}{1.253226in}}%
\pgfpathlineto{\pgfqpoint{2.194005in}{1.253226in}}%
\pgfpathlineto{\pgfqpoint{2.196228in}{1.255308in}}%
\pgfpathlineto{\pgfqpoint{2.197339in}{1.255308in}}%
\pgfpathlineto{\pgfqpoint{2.199562in}{1.257442in}}%
\pgfpathlineto{\pgfqpoint{2.200674in}{1.257442in}}%
\pgfpathlineto{\pgfqpoint{2.202897in}{1.259640in}}%
\pgfpathlineto{\pgfqpoint{2.204009in}{1.259640in}}%
\pgfpathlineto{\pgfqpoint{2.206232in}{1.261866in}}%
\pgfpathlineto{\pgfqpoint{2.207343in}{1.261866in}}%
\pgfpathlineto{\pgfqpoint{2.209566in}{1.264189in}}%
\pgfpathlineto{\pgfqpoint{2.210678in}{1.264189in}}%
\pgfpathlineto{\pgfqpoint{2.212901in}{1.266507in}}%
\pgfpathlineto{\pgfqpoint{2.214013in}{1.266507in}}%
\pgfpathlineto{\pgfqpoint{2.219570in}{1.271298in}}%
\pgfpathlineto{\pgfqpoint{2.220682in}{1.271298in}}%
\pgfpathlineto{\pgfqpoint{2.222905in}{1.273703in}}%
\pgfpathlineto{\pgfqpoint{2.224017in}{1.273703in}}%
\pgfpathlineto{\pgfqpoint{2.229574in}{1.278600in}}%
\pgfpathlineto{\pgfqpoint{2.230686in}{1.278600in}}%
\pgfpathlineto{\pgfqpoint{2.232909in}{1.281039in}}%
\pgfpathlineto{\pgfqpoint{2.234020in}{1.281039in}}%
\pgfpathlineto{\pgfqpoint{2.239578in}{1.286018in}}%
\pgfpathlineto{\pgfqpoint{2.241801in}{1.286018in}}%
\pgfpathlineto{\pgfqpoint{2.244024in}{1.288506in}}%
\pgfpathlineto{\pgfqpoint{2.245136in}{1.288506in}}%
\pgfpathlineto{\pgfqpoint{2.247359in}{1.291022in}}%
\pgfpathlineto{\pgfqpoint{2.248471in}{1.291022in}}%
\pgfpathlineto{\pgfqpoint{2.250694in}{1.293490in}}%
\pgfpathlineto{\pgfqpoint{2.251805in}{1.293490in}}%
\pgfpathlineto{\pgfqpoint{2.255140in}{1.298425in}}%
\pgfpathlineto{\pgfqpoint{2.258475in}{1.298425in}}%
\pgfpathlineto{\pgfqpoint{2.260698in}{1.300908in}}%
\pgfpathlineto{\pgfqpoint{2.261809in}{1.300908in}}%
\pgfpathlineto{\pgfqpoint{2.264032in}{1.303327in}}%
\pgfpathlineto{\pgfqpoint{2.265144in}{1.303327in}}%
\pgfpathlineto{\pgfqpoint{2.267367in}{1.305761in}}%
\pgfpathlineto{\pgfqpoint{2.268478in}{1.305761in}}%
\pgfpathlineto{\pgfqpoint{2.270702in}{1.308109in}}%
\pgfpathlineto{\pgfqpoint{2.271813in}{1.308109in}}%
\pgfpathlineto{\pgfqpoint{2.275148in}{1.310490in}}%
\pgfpathlineto{\pgfqpoint{2.276259in}{1.310490in}}%
\pgfpathlineto{\pgfqpoint{2.279594in}{1.315092in}}%
\pgfpathlineto{\pgfqpoint{2.281817in}{1.315092in}}%
\pgfpathlineto{\pgfqpoint{2.284040in}{1.317265in}}%
\pgfpathlineto{\pgfqpoint{2.285152in}{1.317265in}}%
\pgfpathlineto{\pgfqpoint{2.287375in}{1.319414in}}%
\pgfpathlineto{\pgfqpoint{2.289598in}{1.319414in}}%
\pgfpathlineto{\pgfqpoint{2.291821in}{1.321515in}}%
\pgfpathlineto{\pgfqpoint{2.297379in}{1.325519in}}%
\pgfpathlineto{\pgfqpoint{2.299602in}{1.326463in}}%
\pgfpathlineto{\pgfqpoint{2.304048in}{1.329228in}}%
\pgfpathlineto{\pgfqpoint{2.306271in}{1.329228in}}%
\pgfpathlineto{\pgfqpoint{2.308494in}{1.330957in}}%
\pgfpathlineto{\pgfqpoint{2.309606in}{1.330957in}}%
\pgfpathlineto{\pgfqpoint{2.311829in}{1.332589in}}%
\pgfpathlineto{\pgfqpoint{2.317387in}{1.335632in}}%
\pgfpathlineto{\pgfqpoint{2.319610in}{1.335632in}}%
\pgfpathlineto{\pgfqpoint{2.321833in}{1.337032in}}%
\pgfpathlineto{\pgfqpoint{2.322944in}{1.337032in}}%
\pgfpathlineto{\pgfqpoint{2.325167in}{1.338346in}}%
\pgfpathlineto{\pgfqpoint{2.326279in}{1.338346in}}%
\pgfpathlineto{\pgfqpoint{2.328502in}{1.339611in}}%
\pgfpathlineto{\pgfqpoint{2.329614in}{1.339611in}}%
\pgfpathlineto{\pgfqpoint{2.331837in}{1.340819in}}%
\pgfpathlineto{\pgfqpoint{2.332948in}{1.340819in}}%
\pgfpathlineto{\pgfqpoint{2.335171in}{1.341997in}}%
\pgfpathlineto{\pgfqpoint{2.337395in}{1.342562in}}%
\pgfpathlineto{\pgfqpoint{2.345175in}{1.345383in}}%
\pgfpathlineto{\pgfqpoint{2.346287in}{1.345383in}}%
\pgfpathlineto{\pgfqpoint{2.348510in}{1.346532in}}%
\pgfpathlineto{\pgfqpoint{2.349622in}{1.346532in}}%
\pgfpathlineto{\pgfqpoint{2.351845in}{1.347735in}}%
\pgfpathlineto{\pgfqpoint{2.354068in}{1.347735in}}%
\pgfpathlineto{\pgfqpoint{2.356291in}{1.348961in}}%
\pgfpathlineto{\pgfqpoint{2.357402in}{1.348961in}}%
\pgfpathlineto{\pgfqpoint{2.359625in}{1.350251in}}%
\pgfpathlineto{\pgfqpoint{2.368518in}{1.354530in}}%
\pgfpathlineto{\pgfqpoint{2.370741in}{1.354530in}}%
\pgfpathlineto{\pgfqpoint{2.372964in}{1.356148in}}%
\pgfpathlineto{\pgfqpoint{2.374076in}{1.356148in}}%
\pgfpathlineto{\pgfqpoint{2.376299in}{1.357785in}}%
\pgfpathlineto{\pgfqpoint{2.377410in}{1.357785in}}%
\pgfpathlineto{\pgfqpoint{2.379633in}{1.359562in}}%
\pgfpathlineto{\pgfqpoint{2.380745in}{1.359562in}}%
\pgfpathlineto{\pgfqpoint{2.382968in}{1.361378in}}%
\pgfpathlineto{\pgfqpoint{2.385191in}{1.361378in}}%
\pgfpathlineto{\pgfqpoint{2.387414in}{1.363300in}}%
\pgfpathlineto{\pgfqpoint{2.392972in}{1.367357in}}%
\pgfpathlineto{\pgfqpoint{2.394084in}{1.367357in}}%
\pgfpathlineto{\pgfqpoint{2.396307in}{1.369496in}}%
\pgfpathlineto{\pgfqpoint{2.397418in}{1.369496in}}%
\pgfpathlineto{\pgfqpoint{2.399641in}{1.371670in}}%
\pgfpathlineto{\pgfqpoint{2.400753in}{1.371670in}}%
\pgfpathlineto{\pgfqpoint{2.404087in}{1.373906in}}%
\pgfpathlineto{\pgfqpoint{2.405199in}{1.373906in}}%
\pgfpathlineto{\pgfqpoint{2.407422in}{1.376166in}}%
\pgfpathlineto{\pgfqpoint{2.408534in}{1.376166in}}%
\pgfpathlineto{\pgfqpoint{2.410757in}{1.378446in}}%
\pgfpathlineto{\pgfqpoint{2.411868in}{1.378446in}}%
\pgfpathlineto{\pgfqpoint{2.414091in}{1.380836in}}%
\pgfpathlineto{\pgfqpoint{2.415203in}{1.380836in}}%
\pgfpathlineto{\pgfqpoint{2.417426in}{1.383251in}}%
\pgfpathlineto{\pgfqpoint{2.418538in}{1.383251in}}%
\pgfpathlineto{\pgfqpoint{2.420761in}{1.385632in}}%
\pgfpathlineto{\pgfqpoint{2.421872in}{1.385632in}}%
\pgfpathlineto{\pgfqpoint{2.424095in}{1.388056in}}%
\pgfpathlineto{\pgfqpoint{2.425207in}{1.388056in}}%
\pgfpathlineto{\pgfqpoint{2.427430in}{1.390524in}}%
\pgfpathlineto{\pgfqpoint{2.428542in}{1.390524in}}%
\pgfpathlineto{\pgfqpoint{2.430765in}{1.393026in}}%
\pgfpathlineto{\pgfqpoint{2.431876in}{1.393026in}}%
\pgfpathlineto{\pgfqpoint{2.437434in}{1.398053in}}%
\pgfpathlineto{\pgfqpoint{2.438545in}{1.398053in}}%
\pgfpathlineto{\pgfqpoint{2.440769in}{1.400574in}}%
\pgfpathlineto{\pgfqpoint{2.441880in}{1.400574in}}%
\pgfpathlineto{\pgfqpoint{2.444103in}{1.403139in}}%
\pgfpathlineto{\pgfqpoint{2.445215in}{1.403139in}}%
\pgfpathlineto{\pgfqpoint{2.450773in}{1.408215in}}%
\pgfpathlineto{\pgfqpoint{2.451884in}{1.408215in}}%
\pgfpathlineto{\pgfqpoint{2.457442in}{1.413344in}}%
\pgfpathlineto{\pgfqpoint{2.458553in}{1.413344in}}%
\pgfpathlineto{\pgfqpoint{2.460776in}{1.415918in}}%
\pgfpathlineto{\pgfqpoint{2.463000in}{1.415918in}}%
\pgfpathlineto{\pgfqpoint{2.465223in}{1.418535in}}%
\pgfpathlineto{\pgfqpoint{2.466334in}{1.418535in}}%
\pgfpathlineto{\pgfqpoint{2.468557in}{1.421100in}}%
\pgfpathlineto{\pgfqpoint{2.469669in}{1.421100in}}%
\pgfpathlineto{\pgfqpoint{2.471892in}{1.423664in}}%
\pgfpathlineto{\pgfqpoint{2.473003in}{1.423664in}}%
\pgfpathlineto{\pgfqpoint{2.475227in}{1.426234in}}%
\pgfpathlineto{\pgfqpoint{2.476338in}{1.426234in}}%
\pgfpathlineto{\pgfqpoint{2.478561in}{1.428779in}}%
\pgfpathlineto{\pgfqpoint{2.479673in}{1.428779in}}%
\pgfpathlineto{\pgfqpoint{2.481896in}{1.431348in}}%
\pgfpathlineto{\pgfqpoint{2.483007in}{1.431348in}}%
\pgfpathlineto{\pgfqpoint{2.485231in}{1.433898in}}%
\pgfpathlineto{\pgfqpoint{2.486342in}{1.433898in}}%
\pgfpathlineto{\pgfqpoint{2.488565in}{1.436419in}}%
\pgfpathlineto{\pgfqpoint{2.489677in}{1.436419in}}%
\pgfpathlineto{\pgfqpoint{2.495234in}{1.441471in}}%
\pgfpathlineto{\pgfqpoint{2.496346in}{1.441471in}}%
\pgfpathlineto{\pgfqpoint{2.501904in}{1.446397in}}%
\pgfpathlineto{\pgfqpoint{2.503015in}{1.446397in}}%
\pgfpathlineto{\pgfqpoint{2.505238in}{1.448855in}}%
\pgfpathlineto{\pgfqpoint{2.507462in}{1.448855in}}%
\pgfpathlineto{\pgfqpoint{2.509685in}{1.451313in}}%
\pgfpathlineto{\pgfqpoint{2.515242in}{1.456003in}}%
\pgfpathlineto{\pgfqpoint{2.516354in}{1.456003in}}%
\pgfpathlineto{\pgfqpoint{2.518577in}{1.458321in}}%
\pgfpathlineto{\pgfqpoint{2.520800in}{1.458321in}}%
\pgfpathlineto{\pgfqpoint{2.523023in}{1.460576in}}%
\pgfpathlineto{\pgfqpoint{2.524135in}{1.460576in}}%
\pgfpathlineto{\pgfqpoint{2.526358in}{1.462812in}}%
\pgfpathlineto{\pgfqpoint{2.527469in}{1.462812in}}%
\pgfpathlineto{\pgfqpoint{2.529692in}{1.464981in}}%
\pgfpathlineto{\pgfqpoint{2.533027in}{1.467086in}}%
\pgfpathlineto{\pgfqpoint{2.534139in}{1.467086in}}%
\pgfpathlineto{\pgfqpoint{2.536362in}{1.469095in}}%
\pgfpathlineto{\pgfqpoint{2.537473in}{1.469095in}}%
\pgfpathlineto{\pgfqpoint{2.539696in}{1.471100in}}%
\pgfpathlineto{\pgfqpoint{2.540808in}{1.471100in}}%
\pgfpathlineto{\pgfqpoint{2.543031in}{1.473007in}}%
\pgfpathlineto{\pgfqpoint{2.544143in}{1.473007in}}%
\pgfpathlineto{\pgfqpoint{2.546366in}{1.474818in}}%
\pgfpathlineto{\pgfqpoint{2.547477in}{1.474818in}}%
\pgfpathlineto{\pgfqpoint{2.549700in}{1.476567in}}%
\pgfpathlineto{\pgfqpoint{2.550812in}{1.476567in}}%
\pgfpathlineto{\pgfqpoint{2.553035in}{1.478257in}}%
\pgfpathlineto{\pgfqpoint{2.555258in}{1.479049in}}%
\pgfpathlineto{\pgfqpoint{2.559704in}{1.481314in}}%
\pgfpathlineto{\pgfqpoint{2.560816in}{1.481314in}}%
\pgfpathlineto{\pgfqpoint{2.563039in}{1.482695in}}%
\pgfpathlineto{\pgfqpoint{2.565262in}{1.483343in}}%
\pgfpathlineto{\pgfqpoint{2.567485in}{1.483990in}}%
\pgfpathlineto{\pgfqpoint{2.568597in}{1.483990in}}%
\pgfpathlineto{\pgfqpoint{2.570820in}{1.485207in}}%
\pgfpathlineto{\pgfqpoint{2.571931in}{1.485207in}}%
\pgfpathlineto{\pgfqpoint{2.574154in}{1.486332in}}%
\pgfpathlineto{\pgfqpoint{2.578601in}{1.487375in}}%
\pgfpathlineto{\pgfqpoint{2.581935in}{1.489278in}}%
\pgfpathlineto{\pgfqpoint{2.588605in}{1.490142in}}%
\pgfpathlineto{\pgfqpoint{2.590828in}{1.490959in}}%
\pgfpathlineto{\pgfqpoint{2.595274in}{1.491765in}}%
\pgfpathlineto{\pgfqpoint{2.597497in}{1.492548in}}%
\pgfpathlineto{\pgfqpoint{2.601943in}{1.493340in}}%
\pgfpathlineto{\pgfqpoint{2.604166in}{1.494141in}}%
\pgfpathlineto{\pgfqpoint{2.608612in}{1.494996in}}%
\pgfpathlineto{\pgfqpoint{2.610836in}{1.495885in}}%
\pgfpathlineto{\pgfqpoint{2.616393in}{1.496841in}}%
\pgfpathlineto{\pgfqpoint{2.618616in}{1.497850in}}%
\pgfpathlineto{\pgfqpoint{2.623063in}{1.498956in}}%
\pgfpathlineto{\pgfqpoint{2.626397in}{1.501419in}}%
\pgfpathlineto{\pgfqpoint{2.629732in}{1.501419in}}%
\pgfpathlineto{\pgfqpoint{2.631955in}{1.502781in}}%
\pgfpathlineto{\pgfqpoint{2.633067in}{1.502781in}}%
\pgfpathlineto{\pgfqpoint{2.635290in}{1.504225in}}%
\pgfpathlineto{\pgfqpoint{2.636401in}{1.504225in}}%
\pgfpathlineto{\pgfqpoint{2.638624in}{1.505771in}}%
\pgfpathlineto{\pgfqpoint{2.639736in}{1.505771in}}%
\pgfpathlineto{\pgfqpoint{2.641959in}{1.507393in}}%
\pgfpathlineto{\pgfqpoint{2.643070in}{1.507393in}}%
\pgfpathlineto{\pgfqpoint{2.645294in}{1.509147in}}%
\pgfpathlineto{\pgfqpoint{2.646405in}{1.509147in}}%
\pgfpathlineto{\pgfqpoint{2.648628in}{1.510933in}}%
\pgfpathlineto{\pgfqpoint{2.649740in}{1.510933in}}%
\pgfpathlineto{\pgfqpoint{2.651963in}{1.512831in}}%
\pgfpathlineto{\pgfqpoint{2.653074in}{1.512831in}}%
\pgfpathlineto{\pgfqpoint{2.655298in}{1.514739in}}%
\pgfpathlineto{\pgfqpoint{2.656409in}{1.514739in}}%
\pgfpathlineto{\pgfqpoint{2.658632in}{1.516758in}}%
\pgfpathlineto{\pgfqpoint{2.659744in}{1.516758in}}%
\pgfpathlineto{\pgfqpoint{2.661967in}{1.518839in}}%
\pgfpathlineto{\pgfqpoint{2.664190in}{1.518839in}}%
\pgfpathlineto{\pgfqpoint{2.666413in}{1.521008in}}%
\pgfpathlineto{\pgfqpoint{2.671971in}{1.525403in}}%
\pgfpathlineto{\pgfqpoint{2.673082in}{1.525403in}}%
\pgfpathlineto{\pgfqpoint{2.675305in}{1.527697in}}%
\pgfpathlineto{\pgfqpoint{2.677528in}{1.527697in}}%
\pgfpathlineto{\pgfqpoint{2.679752in}{1.530005in}}%
\pgfpathlineto{\pgfqpoint{2.680863in}{1.530005in}}%
\pgfpathlineto{\pgfqpoint{2.683086in}{1.532367in}}%
\pgfpathlineto{\pgfqpoint{2.684198in}{1.532367in}}%
\pgfpathlineto{\pgfqpoint{2.686421in}{1.534748in}}%
\pgfpathlineto{\pgfqpoint{2.691979in}{1.539597in}}%
\pgfpathlineto{\pgfqpoint{2.694202in}{1.539597in}}%
\pgfpathlineto{\pgfqpoint{2.696425in}{1.542098in}}%
\pgfpathlineto{\pgfqpoint{2.697536in}{1.542098in}}%
\pgfpathlineto{\pgfqpoint{2.699759in}{1.544508in}}%
\pgfpathlineto{\pgfqpoint{2.700871in}{1.544508in}}%
\pgfpathlineto{\pgfqpoint{2.703094in}{1.547058in}}%
\pgfpathlineto{\pgfqpoint{2.704206in}{1.547058in}}%
\pgfpathlineto{\pgfqpoint{2.706429in}{1.549570in}}%
\pgfpathlineto{\pgfqpoint{2.707540in}{1.549570in}}%
\pgfpathlineto{\pgfqpoint{2.713098in}{1.554665in}}%
\pgfpathlineto{\pgfqpoint{2.714210in}{1.554665in}}%
\pgfpathlineto{\pgfqpoint{2.719767in}{1.559765in}}%
\pgfpathlineto{\pgfqpoint{2.720879in}{1.559765in}}%
\pgfpathlineto{\pgfqpoint{2.723102in}{1.562348in}}%
\pgfpathlineto{\pgfqpoint{2.724214in}{1.562348in}}%
\pgfpathlineto{\pgfqpoint{2.729771in}{1.567545in}}%
\pgfpathlineto{\pgfqpoint{2.730883in}{1.567545in}}%
\pgfpathlineto{\pgfqpoint{2.736441in}{1.572814in}}%
\pgfpathlineto{\pgfqpoint{2.737552in}{1.572814in}}%
\pgfpathlineto{\pgfqpoint{2.739775in}{1.575388in}}%
\pgfpathlineto{\pgfqpoint{2.741998in}{1.575388in}}%
\pgfpathlineto{\pgfqpoint{2.744221in}{1.578049in}}%
\pgfpathlineto{\pgfqpoint{2.745333in}{1.578049in}}%
\pgfpathlineto{\pgfqpoint{2.747556in}{1.580647in}}%
\pgfpathlineto{\pgfqpoint{2.748668in}{1.580647in}}%
\pgfpathlineto{\pgfqpoint{2.752002in}{1.585970in}}%
\pgfpathlineto{\pgfqpoint{2.755337in}{1.585970in}}%
\pgfpathlineto{\pgfqpoint{2.757560in}{1.588635in}}%
\pgfpathlineto{\pgfqpoint{2.758672in}{1.588635in}}%
\pgfpathlineto{\pgfqpoint{2.760895in}{1.591287in}}%
\pgfpathlineto{\pgfqpoint{2.762006in}{1.591287in}}%
\pgfpathlineto{\pgfqpoint{2.764229in}{1.593933in}}%
\pgfpathlineto{\pgfqpoint{2.765341in}{1.593933in}}%
\pgfpathlineto{\pgfqpoint{2.767564in}{1.596575in}}%
\pgfpathlineto{\pgfqpoint{2.768676in}{1.596575in}}%
\pgfpathlineto{\pgfqpoint{2.770899in}{1.599236in}}%
\pgfpathlineto{\pgfqpoint{2.773122in}{1.599236in}}%
\pgfpathlineto{\pgfqpoint{2.776456in}{1.604418in}}%
\pgfpathlineto{\pgfqpoint{2.778679in}{1.604418in}}%
\pgfpathlineto{\pgfqpoint{2.780903in}{1.607094in}}%
\pgfpathlineto{\pgfqpoint{2.782014in}{1.607094in}}%
\pgfpathlineto{\pgfqpoint{2.784237in}{1.609736in}}%
\pgfpathlineto{\pgfqpoint{2.785349in}{1.609736in}}%
\pgfpathlineto{\pgfqpoint{2.787572in}{1.612373in}}%
\pgfpathlineto{\pgfqpoint{2.789795in}{1.612373in}}%
\pgfpathlineto{\pgfqpoint{2.792018in}{1.614995in}}%
\pgfpathlineto{\pgfqpoint{2.793130in}{1.614995in}}%
\pgfpathlineto{\pgfqpoint{2.795353in}{1.617656in}}%
\pgfpathlineto{\pgfqpoint{2.796464in}{1.617656in}}%
\pgfpathlineto{\pgfqpoint{2.798687in}{1.620303in}}%
\pgfpathlineto{\pgfqpoint{2.802022in}{1.622968in}}%
\pgfpathlineto{\pgfqpoint{2.803134in}{1.622968in}}%
\pgfpathlineto{\pgfqpoint{2.805357in}{1.625557in}}%
\pgfpathlineto{\pgfqpoint{2.806468in}{1.625557in}}%
\pgfpathlineto{\pgfqpoint{2.808691in}{1.628194in}}%
\pgfpathlineto{\pgfqpoint{2.809803in}{1.628194in}}%
\pgfpathlineto{\pgfqpoint{2.812026in}{1.630729in}}%
\pgfpathlineto{\pgfqpoint{2.813137in}{1.630729in}}%
\pgfpathlineto{\pgfqpoint{2.815361in}{1.633352in}}%
\pgfpathlineto{\pgfqpoint{2.818695in}{1.635897in}}%
\pgfpathlineto{\pgfqpoint{2.819807in}{1.635897in}}%
\pgfpathlineto{\pgfqpoint{2.825365in}{1.641040in}}%
\pgfpathlineto{\pgfqpoint{2.826476in}{1.641040in}}%
\pgfpathlineto{\pgfqpoint{2.828699in}{1.643571in}}%
\pgfpathlineto{\pgfqpoint{2.829811in}{1.643571in}}%
\pgfpathlineto{\pgfqpoint{2.832034in}{1.646020in}}%
\pgfpathlineto{\pgfqpoint{2.833145in}{1.646020in}}%
\pgfpathlineto{\pgfqpoint{2.835368in}{1.648541in}}%
\pgfpathlineto{\pgfqpoint{2.837592in}{1.648541in}}%
\pgfpathlineto{\pgfqpoint{2.839815in}{1.650946in}}%
\pgfpathlineto{\pgfqpoint{2.845372in}{1.655814in}}%
\pgfpathlineto{\pgfqpoint{2.846484in}{1.655814in}}%
\pgfpathlineto{\pgfqpoint{2.848707in}{1.658185in}}%
\pgfpathlineto{\pgfqpoint{2.850930in}{1.658185in}}%
\pgfpathlineto{\pgfqpoint{2.853153in}{1.660518in}}%
\pgfpathlineto{\pgfqpoint{2.854265in}{1.660518in}}%
\pgfpathlineto{\pgfqpoint{2.856488in}{1.662831in}}%
\pgfpathlineto{\pgfqpoint{2.857599in}{1.662831in}}%
\pgfpathlineto{\pgfqpoint{2.859823in}{1.665082in}}%
\pgfpathlineto{\pgfqpoint{2.865380in}{1.669472in}}%
\pgfpathlineto{\pgfqpoint{2.867603in}{1.669472in}}%
\pgfpathlineto{\pgfqpoint{2.869826in}{1.671558in}}%
\pgfpathlineto{\pgfqpoint{2.870938in}{1.671558in}}%
\pgfpathlineto{\pgfqpoint{2.873161in}{1.673625in}}%
\pgfpathlineto{\pgfqpoint{2.874273in}{1.673625in}}%
\pgfpathlineto{\pgfqpoint{2.876496in}{1.675639in}}%
\pgfpathlineto{\pgfqpoint{2.877607in}{1.675639in}}%
\pgfpathlineto{\pgfqpoint{2.879830in}{1.677585in}}%
\pgfpathlineto{\pgfqpoint{2.882054in}{1.678520in}}%
\pgfpathlineto{\pgfqpoint{2.886500in}{1.681294in}}%
\pgfpathlineto{\pgfqpoint{2.888723in}{1.682149in}}%
\pgfpathlineto{\pgfqpoint{2.893169in}{1.684704in}}%
\pgfpathlineto{\pgfqpoint{2.894281in}{1.684704in}}%
\pgfpathlineto{\pgfqpoint{2.896504in}{1.686240in}}%
\pgfpathlineto{\pgfqpoint{2.898727in}{1.686240in}}%
\pgfpathlineto{\pgfqpoint{2.900950in}{1.687727in}}%
\pgfpathlineto{\pgfqpoint{2.902061in}{1.687727in}}%
\pgfpathlineto{\pgfqpoint{2.904284in}{1.689104in}}%
\pgfpathlineto{\pgfqpoint{2.909842in}{1.691630in}}%
\pgfpathlineto{\pgfqpoint{2.915400in}{1.692731in}}%
\pgfpathlineto{\pgfqpoint{2.917623in}{1.693759in}}%
\pgfpathlineto{\pgfqpoint{2.922069in}{1.694663in}}%
\pgfpathlineto{\pgfqpoint{2.924292in}{1.695503in}}%
\pgfpathlineto{\pgfqpoint{2.929850in}{1.696575in}}%
\pgfpathlineto{\pgfqpoint{2.937631in}{1.698024in}}%
\pgfpathlineto{\pgfqpoint{2.946523in}{1.698937in}}%
\pgfpathlineto{\pgfqpoint{2.953193in}{1.700091in}}%
\pgfpathlineto{\pgfqpoint{2.963197in}{1.700864in}}%
\pgfpathlineto{\pgfqpoint{2.968754in}{1.701738in}}%
\pgfpathlineto{\pgfqpoint{2.976535in}{1.702849in}}%
\pgfpathlineto{\pgfqpoint{2.989874in}{1.705954in}}%
\pgfpathlineto{\pgfqpoint{2.993208in}{1.706968in}}%
\pgfpathlineto{\pgfqpoint{3.002101in}{1.710561in}}%
\pgfpathlineto{\pgfqpoint{3.004324in}{1.711242in}}%
\pgfpathlineto{\pgfqpoint{3.006547in}{1.711923in}}%
\pgfpathlineto{\pgfqpoint{3.007659in}{1.711923in}}%
\pgfpathlineto{\pgfqpoint{3.009882in}{1.713411in}}%
\pgfpathlineto{\pgfqpoint{3.010993in}{1.713411in}}%
\pgfpathlineto{\pgfqpoint{3.013216in}{1.714951in}}%
\pgfpathlineto{\pgfqpoint{3.018774in}{1.718308in}}%
\pgfpathlineto{\pgfqpoint{3.020997in}{1.718308in}}%
\pgfpathlineto{\pgfqpoint{3.023220in}{1.720100in}}%
\pgfpathlineto{\pgfqpoint{3.024332in}{1.720100in}}%
\pgfpathlineto{\pgfqpoint{3.026555in}{1.721949in}}%
\pgfpathlineto{\pgfqpoint{3.027666in}{1.721949in}}%
\pgfpathlineto{\pgfqpoint{3.029890in}{1.723896in}}%
\pgfpathlineto{\pgfqpoint{3.031001in}{1.723896in}}%
\pgfpathlineto{\pgfqpoint{3.033224in}{1.725842in}}%
\pgfpathlineto{\pgfqpoint{3.034336in}{1.725842in}}%
\pgfpathlineto{\pgfqpoint{3.036559in}{1.727890in}}%
\pgfpathlineto{\pgfqpoint{3.038782in}{1.728940in}}%
\pgfpathlineto{\pgfqpoint{3.043228in}{1.732149in}}%
\pgfpathlineto{\pgfqpoint{3.044340in}{1.732149in}}%
\pgfpathlineto{\pgfqpoint{3.046563in}{1.734313in}}%
\pgfpathlineto{\pgfqpoint{3.047674in}{1.734313in}}%
\pgfpathlineto{\pgfqpoint{3.049897in}{1.736563in}}%
\pgfpathlineto{\pgfqpoint{3.052121in}{1.736563in}}%
\pgfpathlineto{\pgfqpoint{3.054344in}{1.738809in}}%
\pgfpathlineto{\pgfqpoint{3.057678in}{1.741103in}}%
\pgfpathlineto{\pgfqpoint{3.058790in}{1.741103in}}%
\pgfpathlineto{\pgfqpoint{3.062124in}{1.745817in}}%
\pgfpathlineto{\pgfqpoint{3.064348in}{1.745817in}}%
\pgfpathlineto{\pgfqpoint{3.066571in}{1.748183in}}%
\pgfpathlineto{\pgfqpoint{3.068794in}{1.748183in}}%
\pgfpathlineto{\pgfqpoint{3.071017in}{1.750588in}}%
\pgfpathlineto{\pgfqpoint{3.072128in}{1.750588in}}%
\pgfpathlineto{\pgfqpoint{3.074351in}{1.753032in}}%
\pgfpathlineto{\pgfqpoint{3.075463in}{1.753032in}}%
\pgfpathlineto{\pgfqpoint{3.077686in}{1.755418in}}%
\pgfpathlineto{\pgfqpoint{3.078798in}{1.755418in}}%
\pgfpathlineto{\pgfqpoint{3.081021in}{1.757915in}}%
\pgfpathlineto{\pgfqpoint{3.084355in}{1.760354in}}%
\pgfpathlineto{\pgfqpoint{3.085467in}{1.760354in}}%
\pgfpathlineto{\pgfqpoint{3.091025in}{1.765357in}}%
\pgfpathlineto{\pgfqpoint{3.092136in}{1.765357in}}%
\pgfpathlineto{\pgfqpoint{3.094359in}{1.767897in}}%
\pgfpathlineto{\pgfqpoint{3.095471in}{1.767897in}}%
\pgfpathlineto{\pgfqpoint{3.097694in}{1.770409in}}%
\pgfpathlineto{\pgfqpoint{3.098806in}{1.770409in}}%
\pgfpathlineto{\pgfqpoint{3.101029in}{1.772997in}}%
\pgfpathlineto{\pgfqpoint{3.102140in}{1.772997in}}%
\pgfpathlineto{\pgfqpoint{3.107698in}{1.778092in}}%
\pgfpathlineto{\pgfqpoint{3.108809in}{1.778092in}}%
\pgfpathlineto{\pgfqpoint{3.111033in}{1.780695in}}%
\pgfpathlineto{\pgfqpoint{3.112144in}{1.780695in}}%
\pgfpathlineto{\pgfqpoint{3.114367in}{1.783255in}}%
\pgfpathlineto{\pgfqpoint{3.115479in}{1.783255in}}%
\pgfpathlineto{\pgfqpoint{3.117702in}{1.785849in}}%
\pgfpathlineto{\pgfqpoint{3.118813in}{1.785849in}}%
\pgfpathlineto{\pgfqpoint{3.121037in}{1.788418in}}%
\pgfpathlineto{\pgfqpoint{3.123260in}{1.788418in}}%
\pgfpathlineto{\pgfqpoint{3.125483in}{1.791045in}}%
\pgfpathlineto{\pgfqpoint{3.126594in}{1.791045in}}%
\pgfpathlineto{\pgfqpoint{3.128817in}{1.793610in}}%
\pgfpathlineto{\pgfqpoint{3.129929in}{1.793610in}}%
\pgfpathlineto{\pgfqpoint{3.132152in}{1.796256in}}%
\pgfpathlineto{\pgfqpoint{3.133264in}{1.796256in}}%
\pgfpathlineto{\pgfqpoint{3.135487in}{1.798850in}}%
\pgfpathlineto{\pgfqpoint{3.138821in}{1.801501in}}%
\pgfpathlineto{\pgfqpoint{3.139933in}{1.801501in}}%
\pgfpathlineto{\pgfqpoint{3.142156in}{1.804109in}}%
\pgfpathlineto{\pgfqpoint{3.143268in}{1.804109in}}%
\pgfpathlineto{\pgfqpoint{3.145491in}{1.806760in}}%
\pgfpathlineto{\pgfqpoint{3.146602in}{1.806760in}}%
\pgfpathlineto{\pgfqpoint{3.148825in}{1.809373in}}%
\pgfpathlineto{\pgfqpoint{3.149937in}{1.809373in}}%
\pgfpathlineto{\pgfqpoint{3.152160in}{1.812020in}}%
\pgfpathlineto{\pgfqpoint{3.153271in}{1.812020in}}%
\pgfpathlineto{\pgfqpoint{3.155495in}{1.814637in}}%
\pgfpathlineto{\pgfqpoint{3.156606in}{1.814637in}}%
\pgfpathlineto{\pgfqpoint{3.159941in}{1.817245in}}%
\pgfpathlineto{\pgfqpoint{3.161052in}{1.817245in}}%
\pgfpathlineto{\pgfqpoint{3.164387in}{1.822582in}}%
\pgfpathlineto{\pgfqpoint{3.166610in}{1.822582in}}%
\pgfpathlineto{\pgfqpoint{3.169945in}{1.825277in}}%
\pgfpathlineto{\pgfqpoint{3.171056in}{1.825277in}}%
\pgfpathlineto{\pgfqpoint{3.173279in}{1.827904in}}%
\pgfpathlineto{\pgfqpoint{3.174391in}{1.827904in}}%
\pgfpathlineto{\pgfqpoint{3.176614in}{1.830604in}}%
\pgfpathlineto{\pgfqpoint{3.177726in}{1.830604in}}%
\pgfpathlineto{\pgfqpoint{3.179949in}{1.833255in}}%
\pgfpathlineto{\pgfqpoint{3.181060in}{1.833255in}}%
\pgfpathlineto{\pgfqpoint{3.183283in}{1.835979in}}%
\pgfpathlineto{\pgfqpoint{3.184395in}{1.835979in}}%
\pgfpathlineto{\pgfqpoint{3.186618in}{1.838669in}}%
\pgfpathlineto{\pgfqpoint{3.187729in}{1.838669in}}%
\pgfpathlineto{\pgfqpoint{3.189953in}{1.841378in}}%
\pgfpathlineto{\pgfqpoint{3.191064in}{1.841378in}}%
\pgfpathlineto{\pgfqpoint{3.193287in}{1.844059in}}%
\pgfpathlineto{\pgfqpoint{3.194399in}{1.844059in}}%
\pgfpathlineto{\pgfqpoint{3.196622in}{1.846763in}}%
\pgfpathlineto{\pgfqpoint{3.197733in}{1.846763in}}%
\pgfpathlineto{\pgfqpoint{3.199957in}{1.849501in}}%
\pgfpathlineto{\pgfqpoint{3.201068in}{1.849501in}}%
\pgfpathlineto{\pgfqpoint{3.203291in}{1.852172in}}%
\pgfpathlineto{\pgfqpoint{3.205514in}{1.852172in}}%
\pgfpathlineto{\pgfqpoint{3.207737in}{1.854920in}}%
\pgfpathlineto{\pgfqpoint{3.213295in}{1.860363in}}%
\pgfpathlineto{\pgfqpoint{3.214407in}{1.860363in}}%
\pgfpathlineto{\pgfqpoint{3.216630in}{1.863077in}}%
\pgfpathlineto{\pgfqpoint{3.217741in}{1.863077in}}%
\pgfpathlineto{\pgfqpoint{3.219964in}{1.865801in}}%
\pgfpathlineto{\pgfqpoint{3.221076in}{1.865801in}}%
\pgfpathlineto{\pgfqpoint{3.226634in}{1.871283in}}%
\pgfpathlineto{\pgfqpoint{3.228857in}{1.871283in}}%
\pgfpathlineto{\pgfqpoint{3.232191in}{1.876774in}}%
\pgfpathlineto{\pgfqpoint{3.234415in}{1.876774in}}%
\pgfpathlineto{\pgfqpoint{3.237749in}{1.879527in}}%
\pgfpathlineto{\pgfqpoint{3.238861in}{1.879527in}}%
\pgfpathlineto{\pgfqpoint{3.241084in}{1.882246in}}%
\pgfpathlineto{\pgfqpoint{3.242195in}{1.882246in}}%
\pgfpathlineto{\pgfqpoint{3.244418in}{1.884965in}}%
\pgfpathlineto{\pgfqpoint{3.245530in}{1.884965in}}%
\pgfpathlineto{\pgfqpoint{3.247753in}{1.887727in}}%
\pgfpathlineto{\pgfqpoint{3.248865in}{1.887727in}}%
\pgfpathlineto{\pgfqpoint{3.251088in}{1.890436in}}%
\pgfpathlineto{\pgfqpoint{3.252199in}{1.890436in}}%
\pgfpathlineto{\pgfqpoint{3.254422in}{1.893199in}}%
\pgfpathlineto{\pgfqpoint{3.255534in}{1.893199in}}%
\pgfpathlineto{\pgfqpoint{3.261092in}{1.898666in}}%
\pgfpathlineto{\pgfqpoint{3.262203in}{1.898666in}}%
\pgfpathlineto{\pgfqpoint{3.264426in}{1.901361in}}%
\pgfpathlineto{\pgfqpoint{3.265538in}{1.901361in}}%
\pgfpathlineto{\pgfqpoint{3.267761in}{1.904051in}}%
\pgfpathlineto{\pgfqpoint{3.269984in}{1.904051in}}%
\pgfpathlineto{\pgfqpoint{3.272207in}{1.906770in}}%
\pgfpathlineto{\pgfqpoint{3.273319in}{1.906770in}}%
\pgfpathlineto{\pgfqpoint{3.275542in}{1.909503in}}%
\pgfpathlineto{\pgfqpoint{3.281100in}{1.914888in}}%
\pgfpathlineto{\pgfqpoint{3.282211in}{1.914888in}}%
\pgfpathlineto{\pgfqpoint{3.284434in}{1.917583in}}%
\pgfpathlineto{\pgfqpoint{3.286657in}{1.917583in}}%
\pgfpathlineto{\pgfqpoint{3.288880in}{1.920244in}}%
\pgfpathlineto{\pgfqpoint{3.289992in}{1.920244in}}%
\pgfpathlineto{\pgfqpoint{3.292215in}{1.922910in}}%
\pgfpathlineto{\pgfqpoint{3.293327in}{1.922910in}}%
\pgfpathlineto{\pgfqpoint{3.295550in}{1.925566in}}%
\pgfpathlineto{\pgfqpoint{3.298884in}{1.928189in}}%
\pgfpathlineto{\pgfqpoint{3.299996in}{1.928189in}}%
\pgfpathlineto{\pgfqpoint{3.302219in}{1.930792in}}%
\pgfpathlineto{\pgfqpoint{3.303331in}{1.930792in}}%
\pgfpathlineto{\pgfqpoint{3.305554in}{1.933385in}}%
\pgfpathlineto{\pgfqpoint{3.306665in}{1.933385in}}%
\pgfpathlineto{\pgfqpoint{3.308888in}{1.935950in}}%
\pgfpathlineto{\pgfqpoint{3.310000in}{1.935950in}}%
\pgfpathlineto{\pgfqpoint{3.312223in}{1.938500in}}%
\pgfpathlineto{\pgfqpoint{3.313335in}{1.938500in}}%
\pgfpathlineto{\pgfqpoint{3.315558in}{1.941055in}}%
\pgfpathlineto{\pgfqpoint{3.316669in}{1.941055in}}%
\pgfpathlineto{\pgfqpoint{3.322227in}{1.946130in}}%
\pgfpathlineto{\pgfqpoint{3.323338in}{1.946130in}}%
\pgfpathlineto{\pgfqpoint{3.325562in}{1.948598in}}%
\pgfpathlineto{\pgfqpoint{3.326673in}{1.948598in}}%
\pgfpathlineto{\pgfqpoint{3.328896in}{1.951061in}}%
\pgfpathlineto{\pgfqpoint{3.330008in}{1.951061in}}%
\pgfpathlineto{\pgfqpoint{3.333342in}{1.953529in}}%
\pgfpathlineto{\pgfqpoint{3.334454in}{1.953529in}}%
\pgfpathlineto{\pgfqpoint{3.336677in}{1.955920in}}%
\pgfpathlineto{\pgfqpoint{3.337789in}{1.955920in}}%
\pgfpathlineto{\pgfqpoint{3.340012in}{1.958325in}}%
\pgfpathlineto{\pgfqpoint{3.345569in}{1.963063in}}%
\pgfpathlineto{\pgfqpoint{3.347793in}{1.963063in}}%
\pgfpathlineto{\pgfqpoint{3.350016in}{1.965361in}}%
\pgfpathlineto{\pgfqpoint{3.351127in}{1.965361in}}%
\pgfpathlineto{\pgfqpoint{3.353350in}{1.967655in}}%
\pgfpathlineto{\pgfqpoint{3.354462in}{1.967655in}}%
\pgfpathlineto{\pgfqpoint{3.356685in}{1.969882in}}%
\pgfpathlineto{\pgfqpoint{3.357796in}{1.969882in}}%
\pgfpathlineto{\pgfqpoint{3.360020in}{1.972113in}}%
\pgfpathlineto{\pgfqpoint{3.361131in}{1.972113in}}%
\pgfpathlineto{\pgfqpoint{3.363354in}{1.974286in}}%
\pgfpathlineto{\pgfqpoint{3.365577in}{1.974286in}}%
\pgfpathlineto{\pgfqpoint{3.368912in}{1.978551in}}%
\pgfpathlineto{\pgfqpoint{3.372247in}{1.979580in}}%
\pgfpathlineto{\pgfqpoint{3.376693in}{1.982632in}}%
\pgfpathlineto{\pgfqpoint{3.377804in}{1.982632in}}%
\pgfpathlineto{\pgfqpoint{3.380027in}{1.984588in}}%
\pgfpathlineto{\pgfqpoint{3.382251in}{1.985556in}}%
\pgfpathlineto{\pgfqpoint{3.386697in}{1.988374in}}%
\pgfpathlineto{\pgfqpoint{3.388920in}{1.989287in}}%
\pgfpathlineto{\pgfqpoint{3.393366in}{1.991924in}}%
\pgfpathlineto{\pgfqpoint{3.395589in}{1.992783in}}%
\pgfpathlineto{\pgfqpoint{3.397812in}{1.993643in}}%
\pgfpathlineto{\pgfqpoint{3.398924in}{1.993643in}}%
\pgfpathlineto{\pgfqpoint{3.401147in}{1.995266in}}%
\pgfpathlineto{\pgfqpoint{3.402258in}{1.995266in}}%
\pgfpathlineto{\pgfqpoint{3.404482in}{1.996806in}}%
\pgfpathlineto{\pgfqpoint{3.405593in}{1.996806in}}%
\pgfpathlineto{\pgfqpoint{3.407816in}{1.998347in}}%
\pgfpathlineto{\pgfqpoint{3.408928in}{1.998347in}}%
\pgfpathlineto{\pgfqpoint{3.411151in}{1.999743in}}%
\pgfpathlineto{\pgfqpoint{3.412262in}{1.999743in}}%
\pgfpathlineto{\pgfqpoint{3.414485in}{2.001095in}}%
\pgfpathlineto{\pgfqpoint{3.415597in}{2.001095in}}%
\pgfpathlineto{\pgfqpoint{3.417820in}{2.002346in}}%
\pgfpathlineto{\pgfqpoint{3.418932in}{2.002346in}}%
\pgfpathlineto{\pgfqpoint{3.421155in}{2.003500in}}%
\pgfpathlineto{\pgfqpoint{3.425601in}{2.004587in}}%
\pgfpathlineto{\pgfqpoint{3.427824in}{2.005596in}}%
\pgfpathlineto{\pgfqpoint{3.432270in}{2.006523in}}%
\pgfpathlineto{\pgfqpoint{3.441163in}{2.008697in}}%
\pgfpathlineto{\pgfqpoint{3.450055in}{2.009759in}}%
\pgfpathlineto{\pgfqpoint{3.454501in}{2.010503in}}%
\pgfpathlineto{\pgfqpoint{3.507856in}{2.011541in}}%
\pgfpathlineto{\pgfqpoint{3.547871in}{2.016743in}}%
\pgfpathlineto{\pgfqpoint{3.550094in}{2.017680in}}%
\pgfpathlineto{\pgfqpoint{3.555652in}{2.018674in}}%
\pgfpathlineto{\pgfqpoint{3.557875in}{2.019771in}}%
\pgfpathlineto{\pgfqpoint{3.558987in}{2.019771in}}%
\pgfpathlineto{\pgfqpoint{3.561210in}{2.020949in}}%
\pgfpathlineto{\pgfqpoint{3.562321in}{2.020949in}}%
\pgfpathlineto{\pgfqpoint{3.564545in}{2.022185in}}%
\pgfpathlineto{\pgfqpoint{3.567879in}{2.023523in}}%
\pgfpathlineto{\pgfqpoint{3.568991in}{2.023523in}}%
\pgfpathlineto{\pgfqpoint{3.571214in}{2.024904in}}%
\pgfpathlineto{\pgfqpoint{3.572325in}{2.024904in}}%
\pgfpathlineto{\pgfqpoint{3.574549in}{2.026165in}}%
\pgfpathlineto{\pgfqpoint{3.575660in}{2.026165in}}%
\pgfpathlineto{\pgfqpoint{3.577883in}{2.028039in}}%
\pgfpathlineto{\pgfqpoint{3.578995in}{2.028039in}}%
\pgfpathlineto{\pgfqpoint{3.581218in}{2.029594in}}%
\pgfpathlineto{\pgfqpoint{3.583441in}{2.029594in}}%
\pgfpathlineto{\pgfqpoint{3.585664in}{2.031246in}}%
\pgfpathlineto{\pgfqpoint{3.591222in}{2.034766in}}%
\pgfpathlineto{\pgfqpoint{3.592333in}{2.034766in}}%
\pgfpathlineto{\pgfqpoint{3.594556in}{2.036635in}}%
\pgfpathlineto{\pgfqpoint{3.595668in}{2.036635in}}%
\pgfpathlineto{\pgfqpoint{3.597891in}{2.038533in}}%
\pgfpathlineto{\pgfqpoint{3.600114in}{2.038533in}}%
\pgfpathlineto{\pgfqpoint{3.602337in}{2.040480in}}%
\pgfpathlineto{\pgfqpoint{3.605672in}{2.042465in}}%
\pgfpathlineto{\pgfqpoint{3.606783in}{2.042465in}}%
\pgfpathlineto{\pgfqpoint{3.609007in}{2.044522in}}%
\pgfpathlineto{\pgfqpoint{3.614564in}{2.048656in}}%
\pgfpathlineto{\pgfqpoint{3.616787in}{2.048656in}}%
\pgfpathlineto{\pgfqpoint{3.619010in}{2.050800in}}%
\pgfpathlineto{\pgfqpoint{3.620122in}{2.050800in}}%
\pgfpathlineto{\pgfqpoint{3.622345in}{2.052959in}}%
\pgfpathlineto{\pgfqpoint{3.623457in}{2.052959in}}%
\pgfpathlineto{\pgfqpoint{3.625680in}{2.055181in}}%
\pgfpathlineto{\pgfqpoint{3.626791in}{2.055181in}}%
\pgfpathlineto{\pgfqpoint{3.629014in}{2.057388in}}%
\pgfpathlineto{\pgfqpoint{3.630126in}{2.057388in}}%
\pgfpathlineto{\pgfqpoint{3.632349in}{2.059648in}}%
\pgfpathlineto{\pgfqpoint{3.633461in}{2.059648in}}%
\pgfpathlineto{\pgfqpoint{3.635684in}{2.061942in}}%
\pgfpathlineto{\pgfqpoint{3.636795in}{2.061942in}}%
\pgfpathlineto{\pgfqpoint{3.639018in}{2.064197in}}%
\pgfpathlineto{\pgfqpoint{3.640130in}{2.064197in}}%
\pgfpathlineto{\pgfqpoint{3.642353in}{2.066554in}}%
\pgfpathlineto{\pgfqpoint{3.643465in}{2.066554in}}%
\pgfpathlineto{\pgfqpoint{3.645688in}{2.068853in}}%
\pgfpathlineto{\pgfqpoint{3.646799in}{2.068853in}}%
\pgfpathlineto{\pgfqpoint{3.649022in}{2.071224in}}%
\pgfpathlineto{\pgfqpoint{3.650134in}{2.071224in}}%
\pgfpathlineto{\pgfqpoint{3.655692in}{2.075943in}}%
\pgfpathlineto{\pgfqpoint{3.656803in}{2.075943in}}%
\pgfpathlineto{\pgfqpoint{3.659026in}{2.078367in}}%
\pgfpathlineto{\pgfqpoint{3.661249in}{2.078367in}}%
\pgfpathlineto{\pgfqpoint{3.663472in}{2.080792in}}%
\pgfpathlineto{\pgfqpoint{3.664584in}{2.080792in}}%
\pgfpathlineto{\pgfqpoint{3.666807in}{2.083201in}}%
\pgfpathlineto{\pgfqpoint{3.672365in}{2.088041in}}%
\pgfpathlineto{\pgfqpoint{3.673476in}{2.088041in}}%
\pgfpathlineto{\pgfqpoint{3.676811in}{2.090494in}}%
\pgfpathlineto{\pgfqpoint{3.677923in}{2.090494in}}%
\pgfpathlineto{\pgfqpoint{3.680146in}{2.092962in}}%
\pgfpathlineto{\pgfqpoint{3.681257in}{2.092962in}}%
\pgfpathlineto{\pgfqpoint{3.683480in}{2.095449in}}%
\pgfpathlineto{\pgfqpoint{3.684592in}{2.095449in}}%
\pgfpathlineto{\pgfqpoint{3.686815in}{2.097946in}}%
\pgfpathlineto{\pgfqpoint{3.687927in}{2.097946in}}%
\pgfpathlineto{\pgfqpoint{3.690150in}{2.100399in}}%
\pgfpathlineto{\pgfqpoint{3.691261in}{2.100399in}}%
\pgfpathlineto{\pgfqpoint{3.696819in}{2.105379in}}%
\pgfpathlineto{\pgfqpoint{3.697930in}{2.105379in}}%
\pgfpathlineto{\pgfqpoint{3.700154in}{2.107871in}}%
\pgfpathlineto{\pgfqpoint{3.701265in}{2.107871in}}%
\pgfpathlineto{\pgfqpoint{3.703488in}{2.110382in}}%
\pgfpathlineto{\pgfqpoint{3.705711in}{2.110382in}}%
\pgfpathlineto{\pgfqpoint{3.707934in}{2.112917in}}%
\pgfpathlineto{\pgfqpoint{3.709046in}{2.112917in}}%
\pgfpathlineto{\pgfqpoint{3.711269in}{2.115438in}}%
\pgfpathlineto{\pgfqpoint{3.716827in}{2.120519in}}%
\pgfpathlineto{\pgfqpoint{3.717938in}{2.120519in}}%
\pgfpathlineto{\pgfqpoint{3.720161in}{2.123006in}}%
\pgfpathlineto{\pgfqpoint{3.722385in}{2.123006in}}%
\pgfpathlineto{\pgfqpoint{3.724608in}{2.125576in}}%
\pgfpathlineto{\pgfqpoint{3.725719in}{2.125576in}}%
\pgfpathlineto{\pgfqpoint{3.727942in}{2.128097in}}%
\pgfpathlineto{\pgfqpoint{3.733500in}{2.133177in}}%
\pgfpathlineto{\pgfqpoint{3.735723in}{2.133177in}}%
\pgfpathlineto{\pgfqpoint{3.737946in}{2.135718in}}%
\pgfpathlineto{\pgfqpoint{3.739058in}{2.135718in}}%
\pgfpathlineto{\pgfqpoint{3.744616in}{2.140793in}}%
\pgfpathlineto{\pgfqpoint{3.745727in}{2.140793in}}%
\pgfpathlineto{\pgfqpoint{3.747950in}{2.143372in}}%
\pgfpathlineto{\pgfqpoint{3.749062in}{2.143372in}}%
\pgfpathlineto{\pgfqpoint{3.751285in}{2.145937in}}%
\pgfpathlineto{\pgfqpoint{3.753508in}{2.145937in}}%
\pgfpathlineto{\pgfqpoint{3.755731in}{2.148492in}}%
\pgfpathlineto{\pgfqpoint{3.761289in}{2.153592in}}%
\pgfpathlineto{\pgfqpoint{3.762400in}{2.153592in}}%
\pgfpathlineto{\pgfqpoint{3.767958in}{2.158716in}}%
\pgfpathlineto{\pgfqpoint{3.769070in}{2.158716in}}%
\pgfpathlineto{\pgfqpoint{3.771293in}{2.161304in}}%
\pgfpathlineto{\pgfqpoint{3.773516in}{2.161304in}}%
\pgfpathlineto{\pgfqpoint{3.775739in}{2.163878in}}%
\pgfpathlineto{\pgfqpoint{3.776850in}{2.163878in}}%
\pgfpathlineto{\pgfqpoint{3.779074in}{2.166467in}}%
\pgfpathlineto{\pgfqpoint{3.780185in}{2.166467in}}%
\pgfpathlineto{\pgfqpoint{3.782408in}{2.169065in}}%
\pgfpathlineto{\pgfqpoint{3.783520in}{2.169065in}}%
\pgfpathlineto{\pgfqpoint{3.785743in}{2.171659in}}%
\pgfpathlineto{\pgfqpoint{3.786854in}{2.171659in}}%
\pgfpathlineto{\pgfqpoint{3.792412in}{2.176870in}}%
\pgfpathlineto{\pgfqpoint{3.793524in}{2.176870in}}%
\pgfpathlineto{\pgfqpoint{3.795747in}{2.179425in}}%
\pgfpathlineto{\pgfqpoint{3.796858in}{2.179425in}}%
\pgfpathlineto{\pgfqpoint{3.799081in}{2.181994in}}%
\pgfpathlineto{\pgfqpoint{3.800193in}{2.181994in}}%
\pgfpathlineto{\pgfqpoint{3.802416in}{2.184616in}}%
\pgfpathlineto{\pgfqpoint{3.803528in}{2.184616in}}%
\pgfpathlineto{\pgfqpoint{3.809085in}{2.189842in}}%
\pgfpathlineto{\pgfqpoint{3.810197in}{2.189842in}}%
\pgfpathlineto{\pgfqpoint{3.812420in}{2.192474in}}%
\pgfpathlineto{\pgfqpoint{3.813532in}{2.192474in}}%
\pgfpathlineto{\pgfqpoint{3.815755in}{2.195125in}}%
\pgfpathlineto{\pgfqpoint{3.817978in}{2.195125in}}%
\pgfpathlineto{\pgfqpoint{3.820201in}{2.197762in}}%
\pgfpathlineto{\pgfqpoint{3.821312in}{2.197762in}}%
\pgfpathlineto{\pgfqpoint{3.823535in}{2.200365in}}%
\pgfpathlineto{\pgfqpoint{3.829093in}{2.205697in}}%
\pgfpathlineto{\pgfqpoint{3.830205in}{2.205697in}}%
\pgfpathlineto{\pgfqpoint{3.833539in}{2.208300in}}%
\pgfpathlineto{\pgfqpoint{3.834651in}{2.208300in}}%
\pgfpathlineto{\pgfqpoint{3.836874in}{2.210976in}}%
\pgfpathlineto{\pgfqpoint{3.837986in}{2.210976in}}%
\pgfpathlineto{\pgfqpoint{3.840209in}{2.213671in}}%
\pgfpathlineto{\pgfqpoint{3.841320in}{2.213671in}}%
\pgfpathlineto{\pgfqpoint{3.843543in}{2.216351in}}%
\pgfpathlineto{\pgfqpoint{3.844655in}{2.216351in}}%
\pgfpathlineto{\pgfqpoint{3.846878in}{2.219070in}}%
\pgfpathlineto{\pgfqpoint{3.850213in}{2.221736in}}%
\pgfpathlineto{\pgfqpoint{3.851324in}{2.221736in}}%
\pgfpathlineto{\pgfqpoint{3.856882in}{2.227150in}}%
\pgfpathlineto{\pgfqpoint{3.857994in}{2.227150in}}%
\pgfpathlineto{\pgfqpoint{3.860217in}{2.229830in}}%
\pgfpathlineto{\pgfqpoint{3.861328in}{2.229830in}}%
\pgfpathlineto{\pgfqpoint{3.863551in}{2.232578in}}%
\pgfpathlineto{\pgfqpoint{3.864663in}{2.232578in}}%
\pgfpathlineto{\pgfqpoint{3.866886in}{2.235297in}}%
\pgfpathlineto{\pgfqpoint{3.867997in}{2.235297in}}%
\pgfpathlineto{\pgfqpoint{3.873555in}{2.240779in}}%
\pgfpathlineto{\pgfqpoint{3.874667in}{2.240779in}}%
\pgfpathlineto{\pgfqpoint{3.880224in}{2.246260in}}%
\pgfpathlineto{\pgfqpoint{3.881336in}{2.246260in}}%
\pgfpathlineto{\pgfqpoint{3.883559in}{2.249037in}}%
\pgfpathlineto{\pgfqpoint{3.885782in}{2.249037in}}%
\pgfpathlineto{\pgfqpoint{3.888005in}{2.251795in}}%
\pgfpathlineto{\pgfqpoint{3.889117in}{2.251795in}}%
\pgfpathlineto{\pgfqpoint{3.891340in}{2.254596in}}%
\pgfpathlineto{\pgfqpoint{3.892452in}{2.254596in}}%
\pgfpathlineto{\pgfqpoint{3.894675in}{2.257349in}}%
\pgfpathlineto{\pgfqpoint{3.895786in}{2.257349in}}%
\pgfpathlineto{\pgfqpoint{3.898009in}{2.260145in}}%
\pgfpathlineto{\pgfqpoint{3.899121in}{2.260145in}}%
\pgfpathlineto{\pgfqpoint{3.901344in}{2.262946in}}%
\pgfpathlineto{\pgfqpoint{3.902455in}{2.262946in}}%
\pgfpathlineto{\pgfqpoint{3.904679in}{2.265738in}}%
\pgfpathlineto{\pgfqpoint{3.905790in}{2.265738in}}%
\pgfpathlineto{\pgfqpoint{3.908013in}{2.268582in}}%
\pgfpathlineto{\pgfqpoint{3.909125in}{2.268582in}}%
\pgfpathlineto{\pgfqpoint{3.911348in}{2.271383in}}%
\pgfpathlineto{\pgfqpoint{3.912459in}{2.271383in}}%
\pgfpathlineto{\pgfqpoint{3.914683in}{2.274199in}}%
\pgfpathlineto{\pgfqpoint{3.915794in}{2.274199in}}%
\pgfpathlineto{\pgfqpoint{3.921352in}{2.279883in}}%
\pgfpathlineto{\pgfqpoint{3.922463in}{2.279883in}}%
\pgfpathlineto{\pgfqpoint{3.924686in}{2.282766in}}%
\pgfpathlineto{\pgfqpoint{3.925798in}{2.282766in}}%
\pgfpathlineto{\pgfqpoint{3.928021in}{2.285630in}}%
\pgfpathlineto{\pgfqpoint{3.930244in}{2.285630in}}%
\pgfpathlineto{\pgfqpoint{3.932467in}{2.288485in}}%
\pgfpathlineto{\pgfqpoint{3.933579in}{2.288485in}}%
\pgfpathlineto{\pgfqpoint{3.935802in}{2.291339in}}%
\pgfpathlineto{\pgfqpoint{3.941360in}{2.297023in}}%
\pgfpathlineto{\pgfqpoint{3.942471in}{2.297023in}}%
\pgfpathlineto{\pgfqpoint{3.945806in}{2.299984in}}%
\pgfpathlineto{\pgfqpoint{3.946917in}{2.299984in}}%
\pgfpathlineto{\pgfqpoint{3.949141in}{2.302833in}}%
\pgfpathlineto{\pgfqpoint{3.950252in}{2.302833in}}%
\pgfpathlineto{\pgfqpoint{3.952475in}{2.305707in}}%
\pgfpathlineto{\pgfqpoint{3.953587in}{2.305707in}}%
\pgfpathlineto{\pgfqpoint{3.955810in}{2.308595in}}%
\pgfpathlineto{\pgfqpoint{3.956921in}{2.308595in}}%
\pgfpathlineto{\pgfqpoint{3.959144in}{2.311483in}}%
\pgfpathlineto{\pgfqpoint{3.960256in}{2.311483in}}%
\pgfpathlineto{\pgfqpoint{3.965814in}{2.317264in}}%
\pgfpathlineto{\pgfqpoint{3.966925in}{2.317264in}}%
\pgfpathlineto{\pgfqpoint{3.969148in}{2.320210in}}%
\pgfpathlineto{\pgfqpoint{3.970260in}{2.320210in}}%
\pgfpathlineto{\pgfqpoint{3.972483in}{2.323078in}}%
\pgfpathlineto{\pgfqpoint{3.973595in}{2.323078in}}%
\pgfpathlineto{\pgfqpoint{3.975818in}{2.326005in}}%
\pgfpathlineto{\pgfqpoint{3.978041in}{2.326005in}}%
\pgfpathlineto{\pgfqpoint{3.980264in}{2.328932in}}%
\pgfpathlineto{\pgfqpoint{3.981375in}{2.328932in}}%
\pgfpathlineto{\pgfqpoint{3.983599in}{2.331858in}}%
\pgfpathlineto{\pgfqpoint{3.989156in}{2.337683in}}%
\pgfpathlineto{\pgfqpoint{3.990268in}{2.337683in}}%
\pgfpathlineto{\pgfqpoint{3.992491in}{2.340552in}}%
\pgfpathlineto{\pgfqpoint{3.994714in}{2.340552in}}%
\pgfpathlineto{\pgfqpoint{3.996937in}{2.356692in}}%
\pgfpathlineto{\pgfqpoint{3.998049in}{2.356692in}}%
\pgfpathlineto{\pgfqpoint{4.000272in}{2.358455in}}%
\pgfpathlineto{\pgfqpoint{4.001383in}{2.358455in}}%
\pgfpathlineto{\pgfqpoint{4.004718in}{2.361695in}}%
\pgfpathlineto{\pgfqpoint{4.010276in}{2.368379in}}%
\pgfpathlineto{\pgfqpoint{4.011387in}{2.368379in}}%
\pgfpathlineto{\pgfqpoint{4.013610in}{2.370548in}}%
\pgfpathlineto{\pgfqpoint{4.014722in}{2.370548in}}%
\pgfpathlineto{\pgfqpoint{4.016945in}{2.374160in}}%
\pgfpathlineto{\pgfqpoint{4.018057in}{2.374160in}}%
\pgfpathlineto{\pgfqpoint{4.020280in}{2.376894in}}%
\pgfpathlineto{\pgfqpoint{4.021391in}{2.376894in}}%
\pgfpathlineto{\pgfqpoint{4.023614in}{2.378116in}}%
\pgfpathlineto{\pgfqpoint{4.028061in}{2.378850in}}%
\pgfpathlineto{\pgfqpoint{4.033618in}{2.381902in}}%
\pgfpathlineto{\pgfqpoint{4.039176in}{2.382899in}}%
\pgfpathlineto{\pgfqpoint{4.041399in}{2.384317in}}%
\pgfpathlineto{\pgfqpoint{4.045845in}{2.384114in}}%
\pgfpathlineto{\pgfqpoint{4.048068in}{2.387007in}}%
\pgfpathlineto{\pgfqpoint{4.049180in}{2.387007in}}%
\pgfpathlineto{\pgfqpoint{4.051403in}{2.389846in}}%
\pgfpathlineto{\pgfqpoint{4.052515in}{2.389846in}}%
\pgfpathlineto{\pgfqpoint{4.054738in}{2.392710in}}%
\pgfpathlineto{\pgfqpoint{4.055849in}{2.392710in}}%
\pgfpathlineto{\pgfqpoint{4.058072in}{2.395502in}}%
\pgfpathlineto{\pgfqpoint{4.059184in}{2.395502in}}%
\pgfpathlineto{\pgfqpoint{4.064742in}{2.401157in}}%
\pgfpathlineto{\pgfqpoint{4.065853in}{2.401157in}}%
\pgfpathlineto{\pgfqpoint{4.068076in}{2.403924in}}%
\pgfpathlineto{\pgfqpoint{4.069188in}{2.403924in}}%
\pgfpathlineto{\pgfqpoint{4.071411in}{2.406721in}}%
\pgfpathlineto{\pgfqpoint{4.072522in}{2.406721in}}%
\pgfpathlineto{\pgfqpoint{4.074746in}{2.409498in}}%
\pgfpathlineto{\pgfqpoint{4.075857in}{2.409498in}}%
\pgfpathlineto{\pgfqpoint{4.081415in}{2.414965in}}%
\pgfpathlineto{\pgfqpoint{4.082526in}{2.414965in}}%
\pgfpathlineto{\pgfqpoint{4.088084in}{2.420388in}}%
\pgfpathlineto{\pgfqpoint{4.089196in}{2.420388in}}%
\pgfpathlineto{\pgfqpoint{4.091419in}{2.423107in}}%
\pgfpathlineto{\pgfqpoint{4.093642in}{2.423107in}}%
\pgfpathlineto{\pgfqpoint{4.095865in}{2.425763in}}%
\pgfpathlineto{\pgfqpoint{4.096977in}{2.425763in}}%
\pgfpathlineto{\pgfqpoint{4.099200in}{2.428391in}}%
\pgfpathlineto{\pgfqpoint{4.104757in}{2.433636in}}%
\pgfpathlineto{\pgfqpoint{4.106980in}{2.433636in}}%
\pgfpathlineto{\pgfqpoint{4.109204in}{2.436224in}}%
\pgfpathlineto{\pgfqpoint{4.110315in}{2.436224in}}%
\pgfpathlineto{\pgfqpoint{4.112538in}{2.438827in}}%
\pgfpathlineto{\pgfqpoint{4.113650in}{2.438827in}}%
\pgfpathlineto{\pgfqpoint{4.115873in}{2.441363in}}%
\pgfpathlineto{\pgfqpoint{4.116984in}{2.441363in}}%
\pgfpathlineto{\pgfqpoint{4.119208in}{2.443889in}}%
\pgfpathlineto{\pgfqpoint{4.120319in}{2.443889in}}%
\pgfpathlineto{\pgfqpoint{4.122542in}{2.446390in}}%
\pgfpathlineto{\pgfqpoint{4.123654in}{2.446390in}}%
\pgfpathlineto{\pgfqpoint{4.125877in}{2.448824in}}%
\pgfpathlineto{\pgfqpoint{4.126988in}{2.448824in}}%
\pgfpathlineto{\pgfqpoint{4.132546in}{2.453731in}}%
\pgfpathlineto{\pgfqpoint{4.133658in}{2.453731in}}%
\pgfpathlineto{\pgfqpoint{4.135881in}{2.456122in}}%
\pgfpathlineto{\pgfqpoint{4.136992in}{2.456122in}}%
\pgfpathlineto{\pgfqpoint{4.139215in}{2.458508in}}%
\pgfpathlineto{\pgfqpoint{4.141438in}{2.458508in}}%
\pgfpathlineto{\pgfqpoint{4.143662in}{2.460826in}}%
\pgfpathlineto{\pgfqpoint{4.144773in}{2.460826in}}%
\pgfpathlineto{\pgfqpoint{4.146996in}{2.463129in}}%
\pgfpathlineto{\pgfqpoint{4.148108in}{2.463129in}}%
\pgfpathlineto{\pgfqpoint{4.150331in}{2.465423in}}%
\pgfpathlineto{\pgfqpoint{4.151442in}{2.465423in}}%
\pgfpathlineto{\pgfqpoint{4.154777in}{2.469910in}}%
\pgfpathlineto{\pgfqpoint{4.158112in}{2.469910in}}%
\pgfpathlineto{\pgfqpoint{4.160335in}{2.472127in}}%
\pgfpathlineto{\pgfqpoint{4.161446in}{2.472127in}}%
\pgfpathlineto{\pgfqpoint{4.163669in}{2.474252in}}%
\pgfpathlineto{\pgfqpoint{4.164781in}{2.474252in}}%
\pgfpathlineto{\pgfqpoint{4.167004in}{2.476415in}}%
\pgfpathlineto{\pgfqpoint{4.168116in}{2.476415in}}%
\pgfpathlineto{\pgfqpoint{4.170339in}{2.478511in}}%
\pgfpathlineto{\pgfqpoint{4.171450in}{2.478511in}}%
\pgfpathlineto{\pgfqpoint{4.173673in}{2.480540in}}%
\pgfpathlineto{\pgfqpoint{4.175897in}{2.481576in}}%
\pgfpathlineto{\pgfqpoint{4.180343in}{2.484635in}}%
\pgfpathlineto{\pgfqpoint{4.181454in}{2.484635in}}%
\pgfpathlineto{\pgfqpoint{4.183677in}{2.486562in}}%
\pgfpathlineto{\pgfqpoint{4.184789in}{2.486562in}}%
\pgfpathlineto{\pgfqpoint{4.187012in}{2.488533in}}%
\pgfpathlineto{\pgfqpoint{4.188124in}{2.488533in}}%
\pgfpathlineto{\pgfqpoint{4.190347in}{2.490440in}}%
\pgfpathlineto{\pgfqpoint{4.192570in}{2.490440in}}%
\pgfpathlineto{\pgfqpoint{4.194793in}{2.492290in}}%
\pgfpathlineto{\pgfqpoint{4.195904in}{2.492290in}}%
\pgfpathlineto{\pgfqpoint{4.198127in}{2.494173in}}%
\pgfpathlineto{\pgfqpoint{4.199239in}{2.494173in}}%
\pgfpathlineto{\pgfqpoint{4.201462in}{2.495956in}}%
\pgfpathlineto{\pgfqpoint{4.207020in}{2.499423in}}%
\pgfpathlineto{\pgfqpoint{4.209243in}{2.499423in}}%
\pgfpathlineto{\pgfqpoint{4.211466in}{2.501142in}}%
\pgfpathlineto{\pgfqpoint{4.212578in}{2.501142in}}%
\pgfpathlineto{\pgfqpoint{4.214801in}{2.502823in}}%
\pgfpathlineto{\pgfqpoint{4.215912in}{2.502823in}}%
\pgfpathlineto{\pgfqpoint{4.218135in}{2.504470in}}%
\pgfpathlineto{\pgfqpoint{4.223693in}{2.507624in}}%
\pgfpathlineto{\pgfqpoint{4.225916in}{2.507624in}}%
\pgfpathlineto{\pgfqpoint{4.228139in}{2.509164in}}%
\pgfpathlineto{\pgfqpoint{4.229251in}{2.509164in}}%
\pgfpathlineto{\pgfqpoint{4.231474in}{2.510642in}}%
\pgfpathlineto{\pgfqpoint{4.232586in}{2.510642in}}%
\pgfpathlineto{\pgfqpoint{4.234809in}{2.512086in}}%
\pgfpathlineto{\pgfqpoint{4.235920in}{2.512086in}}%
\pgfpathlineto{\pgfqpoint{4.238143in}{2.513506in}}%
\pgfpathlineto{\pgfqpoint{4.239255in}{2.513506in}}%
\pgfpathlineto{\pgfqpoint{4.241478in}{2.514849in}}%
\pgfpathlineto{\pgfqpoint{4.243701in}{2.515515in}}%
\pgfpathlineto{\pgfqpoint{4.248147in}{2.517447in}}%
\pgfpathlineto{\pgfqpoint{4.249259in}{2.517447in}}%
\pgfpathlineto{\pgfqpoint{4.251482in}{2.518654in}}%
\pgfpathlineto{\pgfqpoint{4.252593in}{2.518654in}}%
\pgfpathlineto{\pgfqpoint{4.254816in}{2.519842in}}%
\pgfpathlineto{\pgfqpoint{4.260374in}{2.520934in}}%
\pgfpathlineto{\pgfqpoint{4.262597in}{2.521991in}}%
\pgfpathlineto{\pgfqpoint{4.269267in}{2.523957in}}%
\pgfpathlineto{\pgfqpoint{4.273713in}{2.524846in}}%
\pgfpathlineto{\pgfqpoint{4.275936in}{2.525705in}}%
\pgfpathlineto{\pgfqpoint{4.280382in}{2.526439in}}%
\pgfpathlineto{\pgfqpoint{4.282605in}{2.527149in}}%
\pgfpathlineto{\pgfqpoint{4.287051in}{2.527768in}}%
\pgfpathlineto{\pgfqpoint{4.291498in}{2.528835in}}%
\pgfpathlineto{\pgfqpoint{4.301502in}{2.529806in}}%
\pgfpathlineto{\pgfqpoint{4.313729in}{2.530593in}}%
\pgfpathlineto{\pgfqpoint{4.395983in}{2.529482in}}%
\pgfpathlineto{\pgfqpoint{4.434887in}{2.529385in}}%
\pgfpathlineto{\pgfqpoint{4.458230in}{2.530226in}}%
\pgfpathlineto{\pgfqpoint{4.466011in}{2.531158in}}%
\pgfpathlineto{\pgfqpoint{4.474903in}{2.532003in}}%
\pgfpathlineto{\pgfqpoint{4.479349in}{2.533032in}}%
\pgfpathlineto{\pgfqpoint{4.484907in}{2.533679in}}%
\pgfpathlineto{\pgfqpoint{4.489353in}{2.535046in}}%
\pgfpathlineto{\pgfqpoint{4.494911in}{2.535789in}}%
\pgfpathlineto{\pgfqpoint{4.497134in}{2.536639in}}%
\pgfpathlineto{\pgfqpoint{4.501580in}{2.537557in}}%
\pgfpathlineto{\pgfqpoint{4.503803in}{2.538446in}}%
\pgfpathlineto{\pgfqpoint{4.508250in}{2.539513in}}%
\pgfpathlineto{\pgfqpoint{4.517142in}{2.542865in}}%
\pgfpathlineto{\pgfqpoint{4.518254in}{2.542865in}}%
\pgfpathlineto{\pgfqpoint{4.520477in}{2.544144in}}%
\pgfpathlineto{\pgfqpoint{4.522700in}{2.544144in}}%
\pgfpathlineto{\pgfqpoint{4.524923in}{2.545410in}}%
\pgfpathlineto{\pgfqpoint{4.526034in}{2.545410in}}%
\pgfpathlineto{\pgfqpoint{4.528258in}{2.546767in}}%
\pgfpathlineto{\pgfqpoint{4.529369in}{2.546767in}}%
\pgfpathlineto{\pgfqpoint{4.531592in}{2.548163in}}%
\pgfpathlineto{\pgfqpoint{4.538261in}{2.551104in}}%
\pgfpathlineto{\pgfqpoint{4.539373in}{2.551104in}}%
\pgfpathlineto{\pgfqpoint{4.541596in}{2.552625in}}%
\pgfpathlineto{\pgfqpoint{4.542708in}{2.552625in}}%
\pgfpathlineto{\pgfqpoint{4.544931in}{2.554214in}}%
\pgfpathlineto{\pgfqpoint{4.546042in}{2.554214in}}%
\pgfpathlineto{\pgfqpoint{4.548265in}{2.555856in}}%
\pgfpathlineto{\pgfqpoint{4.549377in}{2.555856in}}%
\pgfpathlineto{\pgfqpoint{4.551600in}{2.557532in}}%
\pgfpathlineto{\pgfqpoint{4.552712in}{2.557532in}}%
\pgfpathlineto{\pgfqpoint{4.554935in}{2.559217in}}%
\pgfpathlineto{\pgfqpoint{4.557158in}{2.560106in}}%
\pgfpathlineto{\pgfqpoint{4.561604in}{2.562791in}}%
\pgfpathlineto{\pgfqpoint{4.562716in}{2.562791in}}%
\pgfpathlineto{\pgfqpoint{4.564939in}{2.564675in}}%
\pgfpathlineto{\pgfqpoint{4.566050in}{2.564675in}}%
\pgfpathlineto{\pgfqpoint{4.568273in}{2.566491in}}%
\pgfpathlineto{\pgfqpoint{4.569385in}{2.566491in}}%
\pgfpathlineto{\pgfqpoint{4.571608in}{2.568432in}}%
\pgfpathlineto{\pgfqpoint{4.573831in}{2.569396in}}%
\pgfpathlineto{\pgfqpoint{4.578277in}{2.572291in}}%
\pgfpathlineto{\pgfqpoint{4.580500in}{2.573329in}}%
\pgfpathlineto{\pgfqpoint{4.584947in}{2.576367in}}%
\pgfpathlineto{\pgfqpoint{4.587170in}{2.576367in}}%
\pgfpathlineto{\pgfqpoint{4.589393in}{2.578434in}}%
\pgfpathlineto{\pgfqpoint{4.590504in}{2.578434in}}%
\pgfpathlineto{\pgfqpoint{4.592727in}{2.580515in}}%
\pgfpathlineto{\pgfqpoint{4.593839in}{2.580515in}}%
\pgfpathlineto{\pgfqpoint{4.596062in}{2.582582in}}%
\pgfpathlineto{\pgfqpoint{4.601620in}{2.586881in}}%
\pgfpathlineto{\pgfqpoint{4.604954in}{2.587992in}}%
\pgfpathlineto{\pgfqpoint{4.609401in}{2.591271in}}%
\pgfpathlineto{\pgfqpoint{4.610512in}{2.591271in}}%
\pgfpathlineto{\pgfqpoint{4.612735in}{2.593492in}}%
\pgfpathlineto{\pgfqpoint{4.613847in}{2.593492in}}%
\pgfpathlineto{\pgfqpoint{4.616070in}{2.595709in}}%
\pgfpathlineto{\pgfqpoint{4.617181in}{2.595709in}}%
\pgfpathlineto{\pgfqpoint{4.619405in}{2.598027in}}%
\pgfpathlineto{\pgfqpoint{4.620516in}{2.598027in}}%
\pgfpathlineto{\pgfqpoint{4.626074in}{2.602567in}}%
\pgfpathlineto{\pgfqpoint{4.627185in}{2.602567in}}%
\pgfpathlineto{\pgfqpoint{4.632743in}{2.607203in}}%
\pgfpathlineto{\pgfqpoint{4.633855in}{2.607203in}}%
\pgfpathlineto{\pgfqpoint{4.636078in}{2.609550in}}%
\pgfpathlineto{\pgfqpoint{4.637189in}{2.609550in}}%
\pgfpathlineto{\pgfqpoint{4.640524in}{2.611902in}}%
\pgfpathlineto{\pgfqpoint{4.641636in}{2.611902in}}%
\pgfpathlineto{\pgfqpoint{4.643859in}{2.614245in}}%
\pgfpathlineto{\pgfqpoint{4.644970in}{2.614245in}}%
\pgfpathlineto{\pgfqpoint{4.647193in}{2.616645in}}%
\pgfpathlineto{\pgfqpoint{4.648305in}{2.616645in}}%
\pgfpathlineto{\pgfqpoint{4.650528in}{2.619040in}}%
\pgfpathlineto{\pgfqpoint{4.651639in}{2.619040in}}%
\pgfpathlineto{\pgfqpoint{4.653863in}{2.621470in}}%
\pgfpathlineto{\pgfqpoint{4.654974in}{2.621470in}}%
\pgfpathlineto{\pgfqpoint{4.657197in}{2.623880in}}%
\pgfpathlineto{\pgfqpoint{4.658309in}{2.623880in}}%
\pgfpathlineto{\pgfqpoint{4.660532in}{2.626309in}}%
\pgfpathlineto{\pgfqpoint{4.661643in}{2.626309in}}%
\pgfpathlineto{\pgfqpoint{4.663867in}{2.628738in}}%
\pgfpathlineto{\pgfqpoint{4.664978in}{2.628738in}}%
\pgfpathlineto{\pgfqpoint{4.667201in}{2.631167in}}%
\pgfpathlineto{\pgfqpoint{4.668313in}{2.631167in}}%
\pgfpathlineto{\pgfqpoint{4.670536in}{2.633645in}}%
\pgfpathlineto{\pgfqpoint{4.671647in}{2.633645in}}%
\pgfpathlineto{\pgfqpoint{4.677205in}{2.638528in}}%
\pgfpathlineto{\pgfqpoint{4.678317in}{2.638528in}}%
\pgfpathlineto{\pgfqpoint{4.680540in}{2.641005in}}%
\pgfpathlineto{\pgfqpoint{4.681651in}{2.641005in}}%
\pgfpathlineto{\pgfqpoint{4.683874in}{2.643473in}}%
\pgfpathlineto{\pgfqpoint{4.686097in}{2.643473in}}%
\pgfpathlineto{\pgfqpoint{4.688321in}{2.645946in}}%
\pgfpathlineto{\pgfqpoint{4.689432in}{2.645946in}}%
\pgfpathlineto{\pgfqpoint{4.692767in}{2.650910in}}%
\pgfpathlineto{\pgfqpoint{4.694990in}{2.650910in}}%
\pgfpathlineto{\pgfqpoint{4.697213in}{2.653417in}}%
\pgfpathlineto{\pgfqpoint{4.698325in}{2.653417in}}%
\pgfpathlineto{\pgfqpoint{4.701659in}{2.655923in}}%
\pgfpathlineto{\pgfqpoint{4.702771in}{2.655923in}}%
\pgfpathlineto{\pgfqpoint{4.704994in}{2.658420in}}%
\pgfpathlineto{\pgfqpoint{4.706105in}{2.658420in}}%
\pgfpathlineto{\pgfqpoint{4.708328in}{2.660951in}}%
\pgfpathlineto{\pgfqpoint{4.709440in}{2.660951in}}%
\pgfpathlineto{\pgfqpoint{4.711663in}{2.663462in}}%
\pgfpathlineto{\pgfqpoint{4.712775in}{2.663462in}}%
\pgfpathlineto{\pgfqpoint{4.714998in}{2.665964in}}%
\pgfpathlineto{\pgfqpoint{4.716109in}{2.665964in}}%
\pgfpathlineto{\pgfqpoint{4.718332in}{2.668490in}}%
\pgfpathlineto{\pgfqpoint{4.719444in}{2.668490in}}%
\pgfpathlineto{\pgfqpoint{4.721667in}{2.671016in}}%
\pgfpathlineto{\pgfqpoint{4.722779in}{2.671016in}}%
\pgfpathlineto{\pgfqpoint{4.725002in}{2.673488in}}%
\pgfpathlineto{\pgfqpoint{4.726113in}{2.673488in}}%
\pgfpathlineto{\pgfqpoint{4.728336in}{2.676024in}}%
\pgfpathlineto{\pgfqpoint{4.729448in}{2.676024in}}%
\pgfpathlineto{\pgfqpoint{4.731671in}{2.678530in}}%
\pgfpathlineto{\pgfqpoint{4.732783in}{2.678530in}}%
\pgfpathlineto{\pgfqpoint{4.735006in}{2.681003in}}%
\pgfpathlineto{\pgfqpoint{4.737229in}{2.681003in}}%
\pgfpathlineto{\pgfqpoint{4.739452in}{2.683514in}}%
\pgfpathlineto{\pgfqpoint{4.745010in}{2.688610in}}%
\pgfpathlineto{\pgfqpoint{4.746121in}{2.688610in}}%
\pgfpathlineto{\pgfqpoint{4.748344in}{2.691087in}}%
\pgfpathlineto{\pgfqpoint{4.750567in}{2.691087in}}%
\pgfpathlineto{\pgfqpoint{4.752790in}{2.693589in}}%
\pgfpathlineto{\pgfqpoint{4.753902in}{2.693589in}}%
\pgfpathlineto{\pgfqpoint{4.756125in}{2.696100in}}%
\pgfpathlineto{\pgfqpoint{4.757237in}{2.696100in}}%
\pgfpathlineto{\pgfqpoint{4.759460in}{2.698660in}}%
\pgfpathlineto{\pgfqpoint{4.760571in}{2.698660in}}%
\pgfpathlineto{\pgfqpoint{4.762794in}{2.701142in}}%
\pgfpathlineto{\pgfqpoint{4.766129in}{2.703678in}}%
\pgfpathlineto{\pgfqpoint{4.767241in}{2.703678in}}%
\pgfpathlineto{\pgfqpoint{4.769464in}{2.706247in}}%
\pgfpathlineto{\pgfqpoint{4.770575in}{2.706247in}}%
\pgfpathlineto{\pgfqpoint{4.772798in}{2.708763in}}%
\pgfpathlineto{\pgfqpoint{4.773910in}{2.708763in}}%
\pgfpathlineto{\pgfqpoint{4.776133in}{2.711318in}}%
\pgfpathlineto{\pgfqpoint{4.777245in}{2.711318in}}%
\pgfpathlineto{\pgfqpoint{4.779468in}{2.713820in}}%
\pgfpathlineto{\pgfqpoint{4.780579in}{2.713820in}}%
\pgfpathlineto{\pgfqpoint{4.786137in}{2.718924in}}%
\pgfpathlineto{\pgfqpoint{4.787248in}{2.718924in}}%
\pgfpathlineto{\pgfqpoint{4.789472in}{2.721484in}}%
\pgfpathlineto{\pgfqpoint{4.790583in}{2.721484in}}%
\pgfpathlineto{\pgfqpoint{4.792806in}{2.723942in}}%
\pgfpathlineto{\pgfqpoint{4.793918in}{2.723942in}}%
\pgfpathlineto{\pgfqpoint{4.797252in}{2.726483in}}%
\pgfpathlineto{\pgfqpoint{4.798364in}{2.726483in}}%
\pgfpathlineto{\pgfqpoint{4.800587in}{2.728994in}}%
\pgfpathlineto{\pgfqpoint{4.806145in}{2.734012in}}%
\pgfpathlineto{\pgfqpoint{4.807256in}{2.734012in}}%
\pgfpathlineto{\pgfqpoint{4.810591in}{2.736576in}}%
\pgfpathlineto{\pgfqpoint{4.812814in}{2.744265in}}%
\pgfpathlineto{\pgfqpoint{4.815037in}{2.744265in}}%
\pgfpathlineto{\pgfqpoint{4.817260in}{2.747341in}}%
\pgfpathlineto{\pgfqpoint{4.818372in}{2.747341in}}%
\pgfpathlineto{\pgfqpoint{4.820595in}{2.749027in}}%
\pgfpathlineto{\pgfqpoint{4.828376in}{2.749481in}}%
\pgfpathlineto{\pgfqpoint{4.833934in}{2.754668in}}%
\pgfpathlineto{\pgfqpoint{4.835045in}{2.754668in}}%
\pgfpathlineto{\pgfqpoint{4.837268in}{2.772213in}}%
\pgfpathlineto{\pgfqpoint{4.838380in}{2.772213in}}%
\pgfpathlineto{\pgfqpoint{4.840603in}{2.775410in}}%
\pgfpathlineto{\pgfqpoint{4.841714in}{2.775410in}}%
\pgfpathlineto{\pgfqpoint{4.843937in}{2.778086in}}%
\pgfpathlineto{\pgfqpoint{4.846161in}{2.778086in}}%
\pgfpathlineto{\pgfqpoint{4.848384in}{2.782669in}}%
\pgfpathlineto{\pgfqpoint{4.849495in}{2.782669in}}%
\pgfpathlineto{\pgfqpoint{4.851718in}{2.785543in}}%
\pgfpathlineto{\pgfqpoint{4.856164in}{2.786079in}}%
\pgfpathlineto{\pgfqpoint{4.860611in}{2.791309in}}%
\pgfpathlineto{\pgfqpoint{4.866168in}{2.790976in}}%
\pgfpathlineto{\pgfqpoint{4.868392in}{2.792724in}}%
\pgfpathlineto{\pgfqpoint{4.869503in}{2.792724in}}%
\pgfpathlineto{\pgfqpoint{4.871726in}{2.796235in}}%
\pgfpathlineto{\pgfqpoint{4.872838in}{2.796235in}}%
\pgfpathlineto{\pgfqpoint{4.875061in}{2.799447in}}%
\pgfpathlineto{\pgfqpoint{4.876172in}{2.799447in}}%
\pgfpathlineto{\pgfqpoint{4.878395in}{2.801519in}}%
\pgfpathlineto{\pgfqpoint{4.879507in}{2.801519in}}%
\pgfpathlineto{\pgfqpoint{4.881730in}{2.802716in}}%
\pgfpathlineto{\pgfqpoint{4.882842in}{2.802716in}}%
\pgfpathlineto{\pgfqpoint{4.885065in}{2.795970in}}%
\pgfpathlineto{\pgfqpoint{4.889511in}{2.796168in}}%
\pgfpathlineto{\pgfqpoint{4.891734in}{2.808662in}}%
\pgfpathlineto{\pgfqpoint{4.892846in}{2.808662in}}%
\pgfpathlineto{\pgfqpoint{4.896180in}{2.801441in}}%
\pgfpathlineto{\pgfqpoint{4.898403in}{2.813641in}}%
\pgfpathlineto{\pgfqpoint{4.900626in}{2.812631in}}%
\pgfpathlineto{\pgfqpoint{4.906184in}{2.807652in}}%
\pgfpathlineto{\pgfqpoint{4.909519in}{2.823367in}}%
\pgfpathlineto{\pgfqpoint{4.910630in}{2.823367in}}%
\pgfpathlineto{\pgfqpoint{4.911742in}{2.825178in}}%
\pgfpathlineto{\pgfqpoint{4.913965in}{2.826441in}}%
\pgfpathlineto{\pgfqpoint{4.916188in}{2.828965in}}%
\pgfpathlineto{\pgfqpoint{4.919523in}{2.832886in}}%
\pgfpathlineto{\pgfqpoint{4.922857in}{2.835301in}}%
\pgfpathlineto{\pgfqpoint{4.927304in}{2.836320in}}%
\pgfpathlineto{\pgfqpoint{4.929527in}{2.839899in}}%
\pgfpathlineto{\pgfqpoint{4.930638in}{2.839899in}}%
\pgfpathlineto{\pgfqpoint{4.932861in}{2.842676in}}%
\pgfpathlineto{\pgfqpoint{4.937308in}{2.842956in}}%
\pgfpathlineto{\pgfqpoint{4.939531in}{2.846631in}}%
\pgfpathlineto{\pgfqpoint{4.940642in}{2.846631in}}%
\pgfpathlineto{\pgfqpoint{4.942865in}{2.850471in}}%
\pgfpathlineto{\pgfqpoint{4.947312in}{2.851234in}}%
\pgfpathlineto{\pgfqpoint{4.950646in}{2.853267in}}%
\pgfpathlineto{\pgfqpoint{4.952869in}{2.843729in}}%
\pgfpathlineto{\pgfqpoint{4.953981in}{2.843729in}}%
\pgfpathlineto{\pgfqpoint{4.956204in}{2.846404in}}%
\pgfpathlineto{\pgfqpoint{4.957315in}{2.846404in}}%
\pgfpathlineto{\pgfqpoint{4.959539in}{2.857594in}}%
\pgfpathlineto{\pgfqpoint{4.960650in}{2.857594in}}%
\pgfpathlineto{\pgfqpoint{4.962873in}{2.861327in}}%
\pgfpathlineto{\pgfqpoint{4.965096in}{2.861950in}}%
\pgfpathlineto{\pgfqpoint{4.970654in}{2.864607in}}%
\pgfpathlineto{\pgfqpoint{4.973989in}{2.861994in}}%
\pgfpathlineto{\pgfqpoint{4.977323in}{2.862361in}}%
\pgfpathlineto{\pgfqpoint{4.980658in}{2.866746in}}%
\pgfpathlineto{\pgfqpoint{4.981770in}{2.866746in}}%
\pgfpathlineto{\pgfqpoint{4.983993in}{2.869117in}}%
\pgfpathlineto{\pgfqpoint{4.985104in}{2.869117in}}%
\pgfpathlineto{\pgfqpoint{4.987327in}{2.870701in}}%
\pgfpathlineto{\pgfqpoint{4.988439in}{2.870701in}}%
\pgfpathlineto{\pgfqpoint{4.990662in}{2.873990in}}%
\pgfpathlineto{\pgfqpoint{4.991773in}{2.873990in}}%
\pgfpathlineto{\pgfqpoint{4.993997in}{2.900712in}}%
\pgfpathlineto{\pgfqpoint{4.995108in}{2.900712in}}%
\pgfpathlineto{\pgfqpoint{4.997331in}{2.903653in}}%
\pgfpathlineto{\pgfqpoint{4.999554in}{2.904537in}}%
\pgfpathlineto{\pgfqpoint{5.001777in}{2.905421in}}%
\pgfpathlineto{\pgfqpoint{5.004001in}{2.908855in}}%
\pgfpathlineto{\pgfqpoint{5.005112in}{2.908855in}}%
\pgfpathlineto{\pgfqpoint{5.007335in}{2.912071in}}%
\pgfpathlineto{\pgfqpoint{5.008447in}{2.912071in}}%
\pgfpathlineto{\pgfqpoint{5.010670in}{2.914998in}}%
\pgfpathlineto{\pgfqpoint{5.012893in}{2.914998in}}%
\pgfpathlineto{\pgfqpoint{5.015116in}{2.917620in}}%
\pgfpathlineto{\pgfqpoint{5.016228in}{2.917620in}}%
\pgfpathlineto{\pgfqpoint{5.018451in}{2.911313in}}%
\pgfpathlineto{\pgfqpoint{5.019562in}{2.911313in}}%
\pgfpathlineto{\pgfqpoint{5.021785in}{2.921150in}}%
\pgfpathlineto{\pgfqpoint{5.025120in}{2.916562in}}%
\pgfpathlineto{\pgfqpoint{5.027343in}{2.921006in}}%
\pgfpathlineto{\pgfqpoint{5.029566in}{2.921006in}}%
\pgfpathlineto{\pgfqpoint{5.031789in}{2.919837in}}%
\pgfpathlineto{\pgfqpoint{5.032901in}{2.919837in}}%
\pgfpathlineto{\pgfqpoint{5.035124in}{2.924777in}}%
\pgfpathlineto{\pgfqpoint{5.036235in}{2.924777in}}%
\pgfpathlineto{\pgfqpoint{5.038459in}{2.909067in}}%
\pgfpathlineto{\pgfqpoint{5.039570in}{2.909067in}}%
\pgfpathlineto{\pgfqpoint{5.041793in}{2.914095in}}%
\pgfpathlineto{\pgfqpoint{5.046239in}{2.914495in}}%
\pgfpathlineto{\pgfqpoint{5.048462in}{2.917185in}}%
\pgfpathlineto{\pgfqpoint{5.049574in}{2.917185in}}%
\pgfpathlineto{\pgfqpoint{5.051797in}{2.919818in}}%
\pgfpathlineto{\pgfqpoint{5.052909in}{2.919818in}}%
\pgfpathlineto{\pgfqpoint{5.055132in}{2.924855in}}%
\pgfpathlineto{\pgfqpoint{5.056243in}{2.924855in}}%
\pgfpathlineto{\pgfqpoint{5.058466in}{2.926584in}}%
\pgfpathlineto{\pgfqpoint{5.060689in}{2.926584in}}%
\pgfpathlineto{\pgfqpoint{5.062913in}{2.929419in}}%
\pgfpathlineto{\pgfqpoint{5.066247in}{2.932210in}}%
\pgfpathlineto{\pgfqpoint{5.069582in}{2.933171in}}%
\pgfpathlineto{\pgfqpoint{5.071805in}{2.935760in}}%
\pgfpathlineto{\pgfqpoint{5.072917in}{2.935760in}}%
\pgfpathlineto{\pgfqpoint{5.075140in}{2.940024in}}%
\pgfpathlineto{\pgfqpoint{5.080697in}{2.941130in}}%
\pgfpathlineto{\pgfqpoint{5.082920in}{2.943757in}}%
\pgfpathlineto{\pgfqpoint{5.084032in}{2.943757in}}%
\pgfpathlineto{\pgfqpoint{5.086255in}{2.946346in}}%
\pgfpathlineto{\pgfqpoint{5.091813in}{2.951745in}}%
\pgfpathlineto{\pgfqpoint{5.094036in}{2.951745in}}%
\pgfpathlineto{\pgfqpoint{5.096259in}{2.954392in}}%
\pgfpathlineto{\pgfqpoint{5.097371in}{2.954392in}}%
\pgfpathlineto{\pgfqpoint{5.099594in}{2.957029in}}%
\pgfpathlineto{\pgfqpoint{5.100705in}{2.957029in}}%
\pgfpathlineto{\pgfqpoint{5.102928in}{2.959729in}}%
\pgfpathlineto{\pgfqpoint{5.104040in}{2.959729in}}%
\pgfpathlineto{\pgfqpoint{5.106263in}{2.962390in}}%
\pgfpathlineto{\pgfqpoint{5.107375in}{2.962390in}}%
\pgfpathlineto{\pgfqpoint{5.112932in}{2.967799in}}%
\pgfpathlineto{\pgfqpoint{5.114044in}{2.967799in}}%
\pgfpathlineto{\pgfqpoint{5.116267in}{2.970450in}}%
\pgfpathlineto{\pgfqpoint{5.117378in}{2.970450in}}%
\pgfpathlineto{\pgfqpoint{5.119602in}{2.988860in}}%
\pgfpathlineto{\pgfqpoint{5.120713in}{2.988860in}}%
\pgfpathlineto{\pgfqpoint{5.122936in}{2.993883in}}%
\pgfpathlineto{\pgfqpoint{5.125159in}{2.993883in}}%
\pgfpathlineto{\pgfqpoint{5.127382in}{2.997689in}}%
\pgfpathlineto{\pgfqpoint{5.132940in}{3.002219in}}%
\pgfpathlineto{\pgfqpoint{5.134052in}{3.002219in}}%
\pgfpathlineto{\pgfqpoint{5.138498in}{3.007130in}}%
\pgfpathlineto{\pgfqpoint{5.140721in}{3.007130in}}%
\pgfpathlineto{\pgfqpoint{5.144056in}{3.010603in}}%
\pgfpathlineto{\pgfqpoint{5.145167in}{3.010603in}}%
\pgfpathlineto{\pgfqpoint{5.147390in}{3.013669in}}%
\pgfpathlineto{\pgfqpoint{5.148502in}{3.013669in}}%
\pgfpathlineto{\pgfqpoint{5.150725in}{3.015210in}}%
\pgfpathlineto{\pgfqpoint{5.151837in}{3.015210in}}%
\pgfpathlineto{\pgfqpoint{5.154060in}{3.017456in}}%
\pgfpathlineto{\pgfqpoint{5.155171in}{3.017456in}}%
\pgfpathlineto{\pgfqpoint{5.157394in}{3.020460in}}%
\pgfpathlineto{\pgfqpoint{5.159617in}{3.021423in}}%
\pgfpathlineto{\pgfqpoint{5.171844in}{3.030708in}}%
\pgfpathlineto{\pgfqpoint{5.174067in}{3.034977in}}%
\pgfpathlineto{\pgfqpoint{5.175179in}{3.034977in}}%
\pgfpathlineto{\pgfqpoint{5.178514in}{3.039135in}}%
\pgfpathlineto{\pgfqpoint{5.179625in}{3.039135in}}%
\pgfpathlineto{\pgfqpoint{5.181848in}{3.041700in}}%
\pgfpathlineto{\pgfqpoint{5.182960in}{3.041700in}}%
\pgfpathlineto{\pgfqpoint{5.186295in}{3.046602in}}%
\pgfpathlineto{\pgfqpoint{5.189629in}{3.046602in}}%
\pgfpathlineto{\pgfqpoint{5.191852in}{3.049244in}}%
\pgfpathlineto{\pgfqpoint{5.192964in}{3.049244in}}%
\pgfpathlineto{\pgfqpoint{5.195187in}{3.052103in}}%
\pgfpathlineto{\pgfqpoint{5.196298in}{3.052103in}}%
\pgfpathlineto{\pgfqpoint{5.198522in}{3.055184in}}%
\pgfpathlineto{\pgfqpoint{5.199633in}{3.055184in}}%
\pgfpathlineto{\pgfqpoint{5.201856in}{3.057536in}}%
\pgfpathlineto{\pgfqpoint{5.202968in}{3.057536in}}%
\pgfpathlineto{\pgfqpoint{5.205191in}{3.060936in}}%
\pgfpathlineto{\pgfqpoint{5.206302in}{3.060936in}}%
\pgfpathlineto{\pgfqpoint{5.208526in}{3.064278in}}%
\pgfpathlineto{\pgfqpoint{5.209637in}{3.064278in}}%
\pgfpathlineto{\pgfqpoint{5.211860in}{3.066403in}}%
\pgfpathlineto{\pgfqpoint{5.212972in}{3.066403in}}%
\pgfpathlineto{\pgfqpoint{5.215195in}{3.070155in}}%
\pgfpathlineto{\pgfqpoint{5.216306in}{3.070155in}}%
\pgfpathlineto{\pgfqpoint{5.218529in}{3.073121in}}%
\pgfpathlineto{\pgfqpoint{5.219641in}{3.073121in}}%
\pgfpathlineto{\pgfqpoint{5.221864in}{3.076004in}}%
\pgfpathlineto{\pgfqpoint{5.222976in}{3.076004in}}%
\pgfpathlineto{\pgfqpoint{5.225199in}{3.078660in}}%
\pgfpathlineto{\pgfqpoint{5.226310in}{3.078660in}}%
\pgfpathlineto{\pgfqpoint{5.228533in}{3.081036in}}%
\pgfpathlineto{\pgfqpoint{5.229645in}{3.081036in}}%
\pgfpathlineto{\pgfqpoint{5.231868in}{3.082746in}}%
\pgfpathlineto{\pgfqpoint{5.232980in}{3.082746in}}%
\pgfpathlineto{\pgfqpoint{5.235203in}{3.084726in}}%
\pgfpathlineto{\pgfqpoint{5.237426in}{3.084726in}}%
\pgfpathlineto{\pgfqpoint{5.239649in}{3.083403in}}%
\pgfpathlineto{\pgfqpoint{5.240760in}{3.083403in}}%
\pgfpathlineto{\pgfqpoint{5.242984in}{3.088073in}}%
\pgfpathlineto{\pgfqpoint{5.246318in}{3.088826in}}%
\pgfpathlineto{\pgfqpoint{5.249653in}{3.091033in}}%
\pgfpathlineto{\pgfqpoint{5.252987in}{3.091154in}}%
\pgfpathlineto{\pgfqpoint{5.255211in}{3.084919in}}%
\pgfpathlineto{\pgfqpoint{5.257434in}{3.084919in}}%
\pgfpathlineto{\pgfqpoint{5.259657in}{3.086827in}}%
\pgfpathlineto{\pgfqpoint{5.259657in}{3.086827in}}%
\pgfusepath{stroke}%
\end{pgfscope}%
\begin{pgfscope}%
\pgfsetrectcap%
\pgfsetmiterjoin%
\pgfsetlinewidth{0.803000pt}%
\definecolor{currentstroke}{rgb}{0.000000,0.000000,0.000000}%
\pgfsetstrokecolor{currentstroke}%
\pgfsetdash{}{0pt}%
\pgfpathmoveto{\pgfqpoint{0.592318in}{0.451986in}}%
\pgfpathlineto{\pgfqpoint{0.592318in}{3.591168in}}%
\pgfusepath{stroke}%
\end{pgfscope}%
\begin{pgfscope}%
\pgfsetrectcap%
\pgfsetmiterjoin%
\pgfsetlinewidth{0.803000pt}%
\definecolor{currentstroke}{rgb}{0.000000,0.000000,0.000000}%
\pgfsetstrokecolor{currentstroke}%
\pgfsetdash{}{0pt}%
\pgfpathmoveto{\pgfqpoint{5.481911in}{0.451986in}}%
\pgfpathlineto{\pgfqpoint{5.481911in}{3.591168in}}%
\pgfusepath{stroke}%
\end{pgfscope}%
\begin{pgfscope}%
\pgfsetrectcap%
\pgfsetmiterjoin%
\pgfsetlinewidth{0.803000pt}%
\definecolor{currentstroke}{rgb}{0.000000,0.000000,0.000000}%
\pgfsetstrokecolor{currentstroke}%
\pgfsetdash{}{0pt}%
\pgfpathmoveto{\pgfqpoint{0.592318in}{0.451986in}}%
\pgfpathlineto{\pgfqpoint{5.481911in}{0.451986in}}%
\pgfusepath{stroke}%
\end{pgfscope}%
\begin{pgfscope}%
\pgfsetrectcap%
\pgfsetmiterjoin%
\pgfsetlinewidth{0.000000pt}%
\definecolor{currentstroke}{rgb}{0.000000,0.000000,0.000000}%
\pgfsetstrokecolor{currentstroke}%
\pgfsetstrokeopacity{0.000000}%
\pgfsetdash{}{0pt}%
\pgfpathmoveto{\pgfqpoint{0.592318in}{3.591168in}}%
\pgfpathlineto{\pgfqpoint{5.481911in}{3.591168in}}%
\pgfusepath{}%
\end{pgfscope}%
\begin{pgfscope}%
\definecolor{textcolor}{rgb}{0.000000,0.000000,0.000000}%
\pgfsetstrokecolor{textcolor}%
\pgfsetfillcolor{textcolor}%
\pgftext[x=2.002818in,y=1.194514in,,base]{\color{textcolor}\rmfamily\fontsize{8.000000}{9.600000}\selectfont 18}%
\end{pgfscope}%
\begin{pgfscope}%
\definecolor{textcolor}{rgb}{0.000000,0.000000,0.000000}%
\pgfsetstrokecolor{textcolor}%
\pgfsetfillcolor{textcolor}%
\pgftext[x=2.152877in,y=1.281633in,,base]{\color{textcolor}\rmfamily\fontsize{8.000000}{9.600000}\selectfont 16}%
\end{pgfscope}%
\begin{pgfscope}%
\definecolor{textcolor}{rgb}{0.000000,0.000000,0.000000}%
\pgfsetstrokecolor{textcolor}%
\pgfsetfillcolor{textcolor}%
\pgftext[x=2.347398in,y=1.394827in,,base]{\color{textcolor}\rmfamily\fontsize{8.000000}{9.600000}\selectfont 14}%
\end{pgfscope}%
\begin{pgfscope}%
\definecolor{textcolor}{rgb}{0.000000,0.000000,0.000000}%
\pgfsetstrokecolor{textcolor}%
\pgfsetfillcolor{textcolor}%
\pgftext[x=2.601943in,y=1.541635in,,base]{\color{textcolor}\rmfamily\fontsize{8.000000}{9.600000}\selectfont 12}%
\end{pgfscope}%
\begin{pgfscope}%
\definecolor{textcolor}{rgb}{0.000000,0.000000,0.000000}%
\pgfsetstrokecolor{textcolor}%
\pgfsetfillcolor{textcolor}%
\pgftext[x=2.959862in,y=1.748768in,,base]{\color{textcolor}\rmfamily\fontsize{8.000000}{9.600000}\selectfont 10}%
\end{pgfscope}%
\begin{pgfscope}%
\definecolor{textcolor}{rgb}{0.000000,0.000000,0.000000}%
\pgfsetstrokecolor{textcolor}%
\pgfsetfillcolor{textcolor}%
\pgftext[x=3.497852in,y=2.059691in,,base]{\color{textcolor}\rmfamily\fontsize{8.000000}{9.600000}\selectfont 8}%
\end{pgfscope}%
\begin{pgfscope}%
\definecolor{textcolor}{rgb}{0.000000,0.000000,0.000000}%
\pgfsetstrokecolor{textcolor}%
\pgfsetfillcolor{textcolor}%
\pgftext[x=4.395983in,y=2.577777in,,base]{\color{textcolor}\rmfamily\fontsize{8.000000}{9.600000}\selectfont \(\displaystyle \nu = 6\)}%
\end{pgfscope}%
\begin{pgfscope}%
\pgfsetbuttcap%
\pgfsetmiterjoin%
\definecolor{currentfill}{rgb}{1.000000,1.000000,1.000000}%
\pgfsetfillcolor{currentfill}%
\pgfsetfillopacity{0.800000}%
\pgfsetlinewidth{1.003750pt}%
\definecolor{currentstroke}{rgb}{0.800000,0.800000,0.800000}%
\pgfsetstrokecolor{currentstroke}%
\pgfsetstrokeopacity{0.800000}%
\pgfsetdash{}{0pt}%
\pgfpathmoveto{\pgfqpoint{0.670096in}{3.183920in}}%
\pgfpathlineto{\pgfqpoint{2.145690in}{3.183920in}}%
\pgfpathquadraticcurveto{\pgfqpoint{2.167912in}{3.183920in}}{\pgfqpoint{2.167912in}{3.206142in}}%
\pgfpathlineto{\pgfqpoint{2.167912in}{3.513390in}}%
\pgfpathquadraticcurveto{\pgfqpoint{2.167912in}{3.535612in}}{\pgfqpoint{2.145690in}{3.535612in}}%
\pgfpathlineto{\pgfqpoint{0.670096in}{3.535612in}}%
\pgfpathquadraticcurveto{\pgfqpoint{0.647874in}{3.535612in}}{\pgfqpoint{0.647874in}{3.513390in}}%
\pgfpathlineto{\pgfqpoint{0.647874in}{3.206142in}}%
\pgfpathquadraticcurveto{\pgfqpoint{0.647874in}{3.183920in}}{\pgfqpoint{0.670096in}{3.183920in}}%
\pgfpathclose%
\pgfusepath{stroke,fill}%
\end{pgfscope}%
\begin{pgfscope}%
\pgfsetrectcap%
\pgfsetroundjoin%
\pgfsetlinewidth{1.003750pt}%
\definecolor{currentstroke}{rgb}{0.760784,0.211765,0.086275}%
\pgfsetstrokecolor{currentstroke}%
\pgfsetdash{}{0pt}%
\pgfpathmoveto{\pgfqpoint{0.692318in}{3.452279in}}%
\pgfpathlineto{\pgfqpoint{0.914540in}{3.452279in}}%
\pgfusepath{stroke}%
\end{pgfscope}%
\begin{pgfscope}%
\definecolor{textcolor}{rgb}{0.000000,0.000000,0.000000}%
\pgfsetstrokecolor{textcolor}%
\pgfsetfillcolor{textcolor}%
\pgftext[x=1.003429in,y=3.413390in,left,base]{\color{textcolor}\rmfamily\fontsize{8.000000}{9.600000}\selectfont Hallwiderstand \(\displaystyle R_{xy}\)}%
\end{pgfscope}%
\begin{pgfscope}%
\pgfsetrectcap%
\pgfsetroundjoin%
\pgfsetlinewidth{1.003750pt}%
\definecolor{currentstroke}{rgb}{0.152941,0.235294,0.458824}%
\pgfsetstrokecolor{currentstroke}%
\pgfsetdash{}{0pt}%
\pgfpathmoveto{\pgfqpoint{0.692318in}{3.288853in}}%
\pgfpathlineto{\pgfqpoint{0.914540in}{3.288853in}}%
\pgfusepath{stroke}%
\end{pgfscope}%
\begin{pgfscope}%
\definecolor{textcolor}{rgb}{0.000000,0.000000,0.000000}%
\pgfsetstrokecolor{textcolor}%
\pgfsetfillcolor{textcolor}%
\pgftext[x=1.003429in,y=3.249964in,left,base]{\color{textcolor}\rmfamily\fontsize{8.000000}{9.600000}\selectfont Längswiderstand \(\displaystyle R_{xx}\)}%
\end{pgfscope}%
\begin{pgfscope}%
\pgfsetroundcap%
\pgfsetroundjoin%
\pgfsetlinewidth{1.003750pt}%
\definecolor{currentstroke}{rgb}{0.207843,0.231373,0.282353}%
\pgfsetstrokecolor{currentstroke}%
\pgfsetdash{}{0pt}%
\pgfpathmoveto{\pgfqpoint{2.002818in}{1.130690in}}%
\pgfpathquadraticcurveto{\pgfqpoint{2.002818in}{1.059064in}}{\pgfqpoint{2.002818in}{0.987438in}}%
\pgfusepath{stroke}%
\end{pgfscope}%
\begin{pgfscope}%
\pgfsetroundcap%
\pgfsetroundjoin%
\pgfsetlinewidth{1.003750pt}%
\definecolor{currentstroke}{rgb}{0.207843,0.231373,0.282353}%
\pgfsetstrokecolor{currentstroke}%
\pgfsetdash{}{0pt}%
\pgfpathmoveto{\pgfqpoint{1.975040in}{1.075135in}}%
\pgfpathlineto{\pgfqpoint{2.002818in}{1.130690in}}%
\pgfpathlineto{\pgfqpoint{2.030596in}{1.075135in}}%
\pgfusepath{stroke}%
\end{pgfscope}%
\begin{pgfscope}%
\pgfsetroundcap%
\pgfsetroundjoin%
\pgfsetlinewidth{1.003750pt}%
\definecolor{currentstroke}{rgb}{0.207843,0.231373,0.282353}%
\pgfsetstrokecolor{currentstroke}%
\pgfsetdash{}{0pt}%
\pgfpathmoveto{\pgfqpoint{2.030596in}{1.042993in}}%
\pgfpathlineto{\pgfqpoint{2.002818in}{0.987438in}}%
\pgfpathlineto{\pgfqpoint{1.975040in}{1.042993in}}%
\pgfusepath{stroke}%
\end{pgfscope}%
\begin{pgfscope}%
\pgfsetroundcap%
\pgfsetroundjoin%
\pgfsetlinewidth{1.003750pt}%
\definecolor{currentstroke}{rgb}{0.207843,0.231373,0.282353}%
\pgfsetstrokecolor{currentstroke}%
\pgfsetdash{}{0pt}%
\pgfpathmoveto{\pgfqpoint{2.152877in}{1.217810in}}%
\pgfpathquadraticcurveto{\pgfqpoint{2.152877in}{1.088196in}}{\pgfqpoint{2.152877in}{0.958582in}}%
\pgfusepath{stroke}%
\end{pgfscope}%
\begin{pgfscope}%
\pgfsetroundcap%
\pgfsetroundjoin%
\pgfsetlinewidth{1.003750pt}%
\definecolor{currentstroke}{rgb}{0.207843,0.231373,0.282353}%
\pgfsetstrokecolor{currentstroke}%
\pgfsetdash{}{0pt}%
\pgfpathmoveto{\pgfqpoint{2.125100in}{1.162254in}}%
\pgfpathlineto{\pgfqpoint{2.152877in}{1.217810in}}%
\pgfpathlineto{\pgfqpoint{2.180655in}{1.162254in}}%
\pgfusepath{stroke}%
\end{pgfscope}%
\begin{pgfscope}%
\pgfsetroundcap%
\pgfsetroundjoin%
\pgfsetlinewidth{1.003750pt}%
\definecolor{currentstroke}{rgb}{0.207843,0.231373,0.282353}%
\pgfsetstrokecolor{currentstroke}%
\pgfsetdash{}{0pt}%
\pgfpathmoveto{\pgfqpoint{2.180655in}{1.014137in}}%
\pgfpathlineto{\pgfqpoint{2.152877in}{0.958582in}}%
\pgfpathlineto{\pgfqpoint{2.125100in}{1.014137in}}%
\pgfusepath{stroke}%
\end{pgfscope}%
\begin{pgfscope}%
\pgfsetroundcap%
\pgfsetroundjoin%
\pgfsetlinewidth{1.003750pt}%
\definecolor{currentstroke}{rgb}{0.207843,0.231373,0.282353}%
\pgfsetstrokecolor{currentstroke}%
\pgfsetdash{}{0pt}%
\pgfpathmoveto{\pgfqpoint{2.347398in}{1.331004in}}%
\pgfpathquadraticcurveto{\pgfqpoint{2.347398in}{1.124253in}}{\pgfqpoint{2.347398in}{0.917503in}}%
\pgfusepath{stroke}%
\end{pgfscope}%
\begin{pgfscope}%
\pgfsetroundcap%
\pgfsetroundjoin%
\pgfsetlinewidth{1.003750pt}%
\definecolor{currentstroke}{rgb}{0.207843,0.231373,0.282353}%
\pgfsetstrokecolor{currentstroke}%
\pgfsetdash{}{0pt}%
\pgfpathmoveto{\pgfqpoint{2.319621in}{1.275448in}}%
\pgfpathlineto{\pgfqpoint{2.347398in}{1.331004in}}%
\pgfpathlineto{\pgfqpoint{2.375176in}{1.275448in}}%
\pgfusepath{stroke}%
\end{pgfscope}%
\begin{pgfscope}%
\pgfsetroundcap%
\pgfsetroundjoin%
\pgfsetlinewidth{1.003750pt}%
\definecolor{currentstroke}{rgb}{0.207843,0.231373,0.282353}%
\pgfsetstrokecolor{currentstroke}%
\pgfsetdash{}{0pt}%
\pgfpathmoveto{\pgfqpoint{2.375176in}{0.973058in}}%
\pgfpathlineto{\pgfqpoint{2.347398in}{0.917503in}}%
\pgfpathlineto{\pgfqpoint{2.319621in}{0.973058in}}%
\pgfusepath{stroke}%
\end{pgfscope}%
\begin{pgfscope}%
\pgfsetroundcap%
\pgfsetroundjoin%
\pgfsetlinewidth{1.003750pt}%
\definecolor{currentstroke}{rgb}{0.207843,0.231373,0.282353}%
\pgfsetstrokecolor{currentstroke}%
\pgfsetdash{}{0pt}%
\pgfpathmoveto{\pgfqpoint{2.601943in}{1.477811in}}%
\pgfpathquadraticcurveto{\pgfqpoint{2.601943in}{1.168891in}}{\pgfqpoint{2.601943in}{0.859970in}}%
\pgfusepath{stroke}%
\end{pgfscope}%
\begin{pgfscope}%
\pgfsetroundcap%
\pgfsetroundjoin%
\pgfsetlinewidth{1.003750pt}%
\definecolor{currentstroke}{rgb}{0.207843,0.231373,0.282353}%
\pgfsetstrokecolor{currentstroke}%
\pgfsetdash{}{0pt}%
\pgfpathmoveto{\pgfqpoint{2.574165in}{1.422256in}}%
\pgfpathlineto{\pgfqpoint{2.601943in}{1.477811in}}%
\pgfpathlineto{\pgfqpoint{2.629721in}{1.422256in}}%
\pgfusepath{stroke}%
\end{pgfscope}%
\begin{pgfscope}%
\pgfsetroundcap%
\pgfsetroundjoin%
\pgfsetlinewidth{1.003750pt}%
\definecolor{currentstroke}{rgb}{0.207843,0.231373,0.282353}%
\pgfsetstrokecolor{currentstroke}%
\pgfsetdash{}{0pt}%
\pgfpathmoveto{\pgfqpoint{2.629721in}{0.915526in}}%
\pgfpathlineto{\pgfqpoint{2.601943in}{0.859970in}}%
\pgfpathlineto{\pgfqpoint{2.574165in}{0.915526in}}%
\pgfusepath{stroke}%
\end{pgfscope}%
\begin{pgfscope}%
\pgfsetroundcap%
\pgfsetroundjoin%
\pgfsetlinewidth{1.003750pt}%
\definecolor{currentstroke}{rgb}{0.207843,0.231373,0.282353}%
\pgfsetstrokecolor{currentstroke}%
\pgfsetdash{}{0pt}%
\pgfpathmoveto{\pgfqpoint{2.959862in}{1.684944in}}%
\pgfpathquadraticcurveto{\pgfqpoint{2.959862in}{1.233631in}}{\pgfqpoint{2.959862in}{0.782318in}}%
\pgfusepath{stroke}%
\end{pgfscope}%
\begin{pgfscope}%
\pgfsetroundcap%
\pgfsetroundjoin%
\pgfsetlinewidth{1.003750pt}%
\definecolor{currentstroke}{rgb}{0.207843,0.231373,0.282353}%
\pgfsetstrokecolor{currentstroke}%
\pgfsetdash{}{0pt}%
\pgfpathmoveto{\pgfqpoint{2.932084in}{1.629389in}}%
\pgfpathlineto{\pgfqpoint{2.959862in}{1.684944in}}%
\pgfpathlineto{\pgfqpoint{2.987640in}{1.629389in}}%
\pgfusepath{stroke}%
\end{pgfscope}%
\begin{pgfscope}%
\pgfsetroundcap%
\pgfsetroundjoin%
\pgfsetlinewidth{1.003750pt}%
\definecolor{currentstroke}{rgb}{0.207843,0.231373,0.282353}%
\pgfsetstrokecolor{currentstroke}%
\pgfsetdash{}{0pt}%
\pgfpathmoveto{\pgfqpoint{2.987640in}{0.837873in}}%
\pgfpathlineto{\pgfqpoint{2.959862in}{0.782318in}}%
\pgfpathlineto{\pgfqpoint{2.932084in}{0.837873in}}%
\pgfusepath{stroke}%
\end{pgfscope}%
\begin{pgfscope}%
\pgfsetroundcap%
\pgfsetroundjoin%
\pgfsetlinewidth{1.003750pt}%
\definecolor{currentstroke}{rgb}{0.207843,0.231373,0.282353}%
\pgfsetstrokecolor{currentstroke}%
\pgfsetdash{}{0pt}%
\pgfpathmoveto{\pgfqpoint{3.497852in}{1.995868in}}%
\pgfpathquadraticcurveto{\pgfqpoint{3.497852in}{1.341248in}}{\pgfqpoint{3.497852in}{0.686629in}}%
\pgfusepath{stroke}%
\end{pgfscope}%
\begin{pgfscope}%
\pgfsetroundcap%
\pgfsetroundjoin%
\pgfsetlinewidth{1.003750pt}%
\definecolor{currentstroke}{rgb}{0.207843,0.231373,0.282353}%
\pgfsetstrokecolor{currentstroke}%
\pgfsetdash{}{0pt}%
\pgfpathmoveto{\pgfqpoint{3.470074in}{1.940313in}}%
\pgfpathlineto{\pgfqpoint{3.497852in}{1.995868in}}%
\pgfpathlineto{\pgfqpoint{3.525629in}{1.940313in}}%
\pgfusepath{stroke}%
\end{pgfscope}%
\begin{pgfscope}%
\pgfsetroundcap%
\pgfsetroundjoin%
\pgfsetlinewidth{1.003750pt}%
\definecolor{currentstroke}{rgb}{0.207843,0.231373,0.282353}%
\pgfsetstrokecolor{currentstroke}%
\pgfsetdash{}{0pt}%
\pgfpathmoveto{\pgfqpoint{3.525629in}{0.742184in}}%
\pgfpathlineto{\pgfqpoint{3.497852in}{0.686629in}}%
\pgfpathlineto{\pgfqpoint{3.470074in}{0.742184in}}%
\pgfusepath{stroke}%
\end{pgfscope}%
\begin{pgfscope}%
\pgfsetroundcap%
\pgfsetroundjoin%
\pgfsetlinewidth{1.003750pt}%
\definecolor{currentstroke}{rgb}{0.207843,0.231373,0.282353}%
\pgfsetstrokecolor{currentstroke}%
\pgfsetdash{}{0pt}%
\pgfpathmoveto{\pgfqpoint{4.395983in}{2.513954in}}%
\pgfpathquadraticcurveto{\pgfqpoint{4.395983in}{1.556975in}}{\pgfqpoint{4.395983in}{0.599997in}}%
\pgfusepath{stroke}%
\end{pgfscope}%
\begin{pgfscope}%
\pgfsetroundcap%
\pgfsetroundjoin%
\pgfsetlinewidth{1.003750pt}%
\definecolor{currentstroke}{rgb}{0.207843,0.231373,0.282353}%
\pgfsetstrokecolor{currentstroke}%
\pgfsetdash{}{0pt}%
\pgfpathmoveto{\pgfqpoint{4.368205in}{2.458398in}}%
\pgfpathlineto{\pgfqpoint{4.395983in}{2.513954in}}%
\pgfpathlineto{\pgfqpoint{4.423761in}{2.458398in}}%
\pgfusepath{stroke}%
\end{pgfscope}%
\begin{pgfscope}%
\pgfsetroundcap%
\pgfsetroundjoin%
\pgfsetlinewidth{1.003750pt}%
\definecolor{currentstroke}{rgb}{0.207843,0.231373,0.282353}%
\pgfsetstrokecolor{currentstroke}%
\pgfsetdash{}{0pt}%
\pgfpathmoveto{\pgfqpoint{4.423761in}{0.655552in}}%
\pgfpathlineto{\pgfqpoint{4.395983in}{0.599997in}}%
\pgfpathlineto{\pgfqpoint{4.368205in}{0.655552in}}%
\pgfusepath{stroke}%
\end{pgfscope}%
\begin{pgfscope}%
\pgfpathrectangle{\pgfqpoint{0.592318in}{0.451986in}}{\pgfqpoint{4.889593in}{3.139182in}}%
\pgfusepath{clip}%
\pgfsetbuttcap%
\pgfsetroundjoin%
\pgfsetlinewidth{0.501875pt}%
\definecolor{currentstroke}{rgb}{0.690196,0.690196,0.690196}%
\pgfsetstrokecolor{currentstroke}%
\pgfsetdash{{1.850000pt}{0.800000pt}}{0.000000pt}%
\pgfpathmoveto{\pgfqpoint{0.592318in}{0.451986in}}%
\pgfpathlineto{\pgfqpoint{5.481911in}{0.451986in}}%
\pgfusepath{stroke}%
\end{pgfscope}%
\begin{pgfscope}%
\pgfsetbuttcap%
\pgfsetroundjoin%
\definecolor{currentfill}{rgb}{0.000000,0.000000,0.000000}%
\pgfsetfillcolor{currentfill}%
\pgfsetlinewidth{0.803000pt}%
\definecolor{currentstroke}{rgb}{0.000000,0.000000,0.000000}%
\pgfsetstrokecolor{currentstroke}%
\pgfsetdash{}{0pt}%
\pgfsys@defobject{currentmarker}{\pgfqpoint{0.000000in}{0.000000in}}{\pgfqpoint{0.048611in}{0.000000in}}{%
\pgfpathmoveto{\pgfqpoint{0.000000in}{0.000000in}}%
\pgfpathlineto{\pgfqpoint{0.048611in}{0.000000in}}%
\pgfusepath{stroke,fill}%
}%
\begin{pgfscope}%
\pgfsys@transformshift{5.481911in}{0.451986in}%
\pgfsys@useobject{currentmarker}{}%
\end{pgfscope}%
\end{pgfscope}%
\begin{pgfscope}%
\definecolor{textcolor}{rgb}{0.000000,0.000000,0.000000}%
\pgfsetstrokecolor{textcolor}%
\pgfsetfillcolor{textcolor}%
\pgftext[x=5.579133in,y=0.413724in,left,base]{\color{textcolor}\rmfamily\fontsize{8.000000}{9.600000}\selectfont \(\displaystyle 0\)}%
\end{pgfscope}%
\begin{pgfscope}%
\pgfpathrectangle{\pgfqpoint{0.592318in}{0.451986in}}{\pgfqpoint{4.889593in}{3.139182in}}%
\pgfusepath{clip}%
\pgfsetbuttcap%
\pgfsetroundjoin%
\pgfsetlinewidth{0.501875pt}%
\definecolor{currentstroke}{rgb}{0.690196,0.690196,0.690196}%
\pgfsetstrokecolor{currentstroke}%
\pgfsetdash{{1.850000pt}{0.800000pt}}{0.000000pt}%
\pgfpathmoveto{\pgfqpoint{0.592318in}{0.934937in}}%
\pgfpathlineto{\pgfqpoint{5.481911in}{0.934937in}}%
\pgfusepath{stroke}%
\end{pgfscope}%
\begin{pgfscope}%
\pgfsetbuttcap%
\pgfsetroundjoin%
\definecolor{currentfill}{rgb}{0.000000,0.000000,0.000000}%
\pgfsetfillcolor{currentfill}%
\pgfsetlinewidth{0.803000pt}%
\definecolor{currentstroke}{rgb}{0.000000,0.000000,0.000000}%
\pgfsetstrokecolor{currentstroke}%
\pgfsetdash{}{0pt}%
\pgfsys@defobject{currentmarker}{\pgfqpoint{0.000000in}{0.000000in}}{\pgfqpoint{0.048611in}{0.000000in}}{%
\pgfpathmoveto{\pgfqpoint{0.000000in}{0.000000in}}%
\pgfpathlineto{\pgfqpoint{0.048611in}{0.000000in}}%
\pgfusepath{stroke,fill}%
}%
\begin{pgfscope}%
\pgfsys@transformshift{5.481911in}{0.934937in}%
\pgfsys@useobject{currentmarker}{}%
\end{pgfscope}%
\end{pgfscope}%
\begin{pgfscope}%
\definecolor{textcolor}{rgb}{0.000000,0.000000,0.000000}%
\pgfsetstrokecolor{textcolor}%
\pgfsetfillcolor{textcolor}%
\pgftext[x=5.579133in,y=0.896675in,left,base]{\color{textcolor}\rmfamily\fontsize{8.000000}{9.600000}\selectfont \(\displaystyle 100\)}%
\end{pgfscope}%
\begin{pgfscope}%
\pgfpathrectangle{\pgfqpoint{0.592318in}{0.451986in}}{\pgfqpoint{4.889593in}{3.139182in}}%
\pgfusepath{clip}%
\pgfsetbuttcap%
\pgfsetroundjoin%
\pgfsetlinewidth{0.501875pt}%
\definecolor{currentstroke}{rgb}{0.690196,0.690196,0.690196}%
\pgfsetstrokecolor{currentstroke}%
\pgfsetdash{{1.850000pt}{0.800000pt}}{0.000000pt}%
\pgfpathmoveto{\pgfqpoint{0.592318in}{1.417888in}}%
\pgfpathlineto{\pgfqpoint{5.481911in}{1.417888in}}%
\pgfusepath{stroke}%
\end{pgfscope}%
\begin{pgfscope}%
\pgfsetbuttcap%
\pgfsetroundjoin%
\definecolor{currentfill}{rgb}{0.000000,0.000000,0.000000}%
\pgfsetfillcolor{currentfill}%
\pgfsetlinewidth{0.803000pt}%
\definecolor{currentstroke}{rgb}{0.000000,0.000000,0.000000}%
\pgfsetstrokecolor{currentstroke}%
\pgfsetdash{}{0pt}%
\pgfsys@defobject{currentmarker}{\pgfqpoint{0.000000in}{0.000000in}}{\pgfqpoint{0.048611in}{0.000000in}}{%
\pgfpathmoveto{\pgfqpoint{0.000000in}{0.000000in}}%
\pgfpathlineto{\pgfqpoint{0.048611in}{0.000000in}}%
\pgfusepath{stroke,fill}%
}%
\begin{pgfscope}%
\pgfsys@transformshift{5.481911in}{1.417888in}%
\pgfsys@useobject{currentmarker}{}%
\end{pgfscope}%
\end{pgfscope}%
\begin{pgfscope}%
\definecolor{textcolor}{rgb}{0.000000,0.000000,0.000000}%
\pgfsetstrokecolor{textcolor}%
\pgfsetfillcolor{textcolor}%
\pgftext[x=5.579133in,y=1.379626in,left,base]{\color{textcolor}\rmfamily\fontsize{8.000000}{9.600000}\selectfont \(\displaystyle 200\)}%
\end{pgfscope}%
\begin{pgfscope}%
\pgfpathrectangle{\pgfqpoint{0.592318in}{0.451986in}}{\pgfqpoint{4.889593in}{3.139182in}}%
\pgfusepath{clip}%
\pgfsetbuttcap%
\pgfsetroundjoin%
\pgfsetlinewidth{0.501875pt}%
\definecolor{currentstroke}{rgb}{0.690196,0.690196,0.690196}%
\pgfsetstrokecolor{currentstroke}%
\pgfsetdash{{1.850000pt}{0.800000pt}}{0.000000pt}%
\pgfpathmoveto{\pgfqpoint{0.592318in}{1.900839in}}%
\pgfpathlineto{\pgfqpoint{5.481911in}{1.900839in}}%
\pgfusepath{stroke}%
\end{pgfscope}%
\begin{pgfscope}%
\pgfsetbuttcap%
\pgfsetroundjoin%
\definecolor{currentfill}{rgb}{0.000000,0.000000,0.000000}%
\pgfsetfillcolor{currentfill}%
\pgfsetlinewidth{0.803000pt}%
\definecolor{currentstroke}{rgb}{0.000000,0.000000,0.000000}%
\pgfsetstrokecolor{currentstroke}%
\pgfsetdash{}{0pt}%
\pgfsys@defobject{currentmarker}{\pgfqpoint{0.000000in}{0.000000in}}{\pgfqpoint{0.048611in}{0.000000in}}{%
\pgfpathmoveto{\pgfqpoint{0.000000in}{0.000000in}}%
\pgfpathlineto{\pgfqpoint{0.048611in}{0.000000in}}%
\pgfusepath{stroke,fill}%
}%
\begin{pgfscope}%
\pgfsys@transformshift{5.481911in}{1.900839in}%
\pgfsys@useobject{currentmarker}{}%
\end{pgfscope}%
\end{pgfscope}%
\begin{pgfscope}%
\definecolor{textcolor}{rgb}{0.000000,0.000000,0.000000}%
\pgfsetstrokecolor{textcolor}%
\pgfsetfillcolor{textcolor}%
\pgftext[x=5.579133in,y=1.862577in,left,base]{\color{textcolor}\rmfamily\fontsize{8.000000}{9.600000}\selectfont \(\displaystyle 300\)}%
\end{pgfscope}%
\begin{pgfscope}%
\pgfpathrectangle{\pgfqpoint{0.592318in}{0.451986in}}{\pgfqpoint{4.889593in}{3.139182in}}%
\pgfusepath{clip}%
\pgfsetbuttcap%
\pgfsetroundjoin%
\pgfsetlinewidth{0.501875pt}%
\definecolor{currentstroke}{rgb}{0.690196,0.690196,0.690196}%
\pgfsetstrokecolor{currentstroke}%
\pgfsetdash{{1.850000pt}{0.800000pt}}{0.000000pt}%
\pgfpathmoveto{\pgfqpoint{0.592318in}{2.383790in}}%
\pgfpathlineto{\pgfqpoint{5.481911in}{2.383790in}}%
\pgfusepath{stroke}%
\end{pgfscope}%
\begin{pgfscope}%
\pgfsetbuttcap%
\pgfsetroundjoin%
\definecolor{currentfill}{rgb}{0.000000,0.000000,0.000000}%
\pgfsetfillcolor{currentfill}%
\pgfsetlinewidth{0.803000pt}%
\definecolor{currentstroke}{rgb}{0.000000,0.000000,0.000000}%
\pgfsetstrokecolor{currentstroke}%
\pgfsetdash{}{0pt}%
\pgfsys@defobject{currentmarker}{\pgfqpoint{0.000000in}{0.000000in}}{\pgfqpoint{0.048611in}{0.000000in}}{%
\pgfpathmoveto{\pgfqpoint{0.000000in}{0.000000in}}%
\pgfpathlineto{\pgfqpoint{0.048611in}{0.000000in}}%
\pgfusepath{stroke,fill}%
}%
\begin{pgfscope}%
\pgfsys@transformshift{5.481911in}{2.383790in}%
\pgfsys@useobject{currentmarker}{}%
\end{pgfscope}%
\end{pgfscope}%
\begin{pgfscope}%
\definecolor{textcolor}{rgb}{0.000000,0.000000,0.000000}%
\pgfsetstrokecolor{textcolor}%
\pgfsetfillcolor{textcolor}%
\pgftext[x=5.579133in,y=2.345528in,left,base]{\color{textcolor}\rmfamily\fontsize{8.000000}{9.600000}\selectfont \(\displaystyle 400\)}%
\end{pgfscope}%
\begin{pgfscope}%
\pgfpathrectangle{\pgfqpoint{0.592318in}{0.451986in}}{\pgfqpoint{4.889593in}{3.139182in}}%
\pgfusepath{clip}%
\pgfsetbuttcap%
\pgfsetroundjoin%
\pgfsetlinewidth{0.501875pt}%
\definecolor{currentstroke}{rgb}{0.690196,0.690196,0.690196}%
\pgfsetstrokecolor{currentstroke}%
\pgfsetdash{{1.850000pt}{0.800000pt}}{0.000000pt}%
\pgfpathmoveto{\pgfqpoint{0.592318in}{2.866741in}}%
\pgfpathlineto{\pgfqpoint{5.481911in}{2.866741in}}%
\pgfusepath{stroke}%
\end{pgfscope}%
\begin{pgfscope}%
\pgfsetbuttcap%
\pgfsetroundjoin%
\definecolor{currentfill}{rgb}{0.000000,0.000000,0.000000}%
\pgfsetfillcolor{currentfill}%
\pgfsetlinewidth{0.803000pt}%
\definecolor{currentstroke}{rgb}{0.000000,0.000000,0.000000}%
\pgfsetstrokecolor{currentstroke}%
\pgfsetdash{}{0pt}%
\pgfsys@defobject{currentmarker}{\pgfqpoint{0.000000in}{0.000000in}}{\pgfqpoint{0.048611in}{0.000000in}}{%
\pgfpathmoveto{\pgfqpoint{0.000000in}{0.000000in}}%
\pgfpathlineto{\pgfqpoint{0.048611in}{0.000000in}}%
\pgfusepath{stroke,fill}%
}%
\begin{pgfscope}%
\pgfsys@transformshift{5.481911in}{2.866741in}%
\pgfsys@useobject{currentmarker}{}%
\end{pgfscope}%
\end{pgfscope}%
\begin{pgfscope}%
\definecolor{textcolor}{rgb}{0.000000,0.000000,0.000000}%
\pgfsetstrokecolor{textcolor}%
\pgfsetfillcolor{textcolor}%
\pgftext[x=5.579133in,y=2.828479in,left,base]{\color{textcolor}\rmfamily\fontsize{8.000000}{9.600000}\selectfont \(\displaystyle 500\)}%
\end{pgfscope}%
\begin{pgfscope}%
\pgfpathrectangle{\pgfqpoint{0.592318in}{0.451986in}}{\pgfqpoint{4.889593in}{3.139182in}}%
\pgfusepath{clip}%
\pgfsetbuttcap%
\pgfsetroundjoin%
\pgfsetlinewidth{0.501875pt}%
\definecolor{currentstroke}{rgb}{0.690196,0.690196,0.690196}%
\pgfsetstrokecolor{currentstroke}%
\pgfsetdash{{1.850000pt}{0.800000pt}}{0.000000pt}%
\pgfpathmoveto{\pgfqpoint{0.592318in}{3.349692in}}%
\pgfpathlineto{\pgfqpoint{5.481911in}{3.349692in}}%
\pgfusepath{stroke}%
\end{pgfscope}%
\begin{pgfscope}%
\pgfsetbuttcap%
\pgfsetroundjoin%
\definecolor{currentfill}{rgb}{0.000000,0.000000,0.000000}%
\pgfsetfillcolor{currentfill}%
\pgfsetlinewidth{0.803000pt}%
\definecolor{currentstroke}{rgb}{0.000000,0.000000,0.000000}%
\pgfsetstrokecolor{currentstroke}%
\pgfsetdash{}{0pt}%
\pgfsys@defobject{currentmarker}{\pgfqpoint{0.000000in}{0.000000in}}{\pgfqpoint{0.048611in}{0.000000in}}{%
\pgfpathmoveto{\pgfqpoint{0.000000in}{0.000000in}}%
\pgfpathlineto{\pgfqpoint{0.048611in}{0.000000in}}%
\pgfusepath{stroke,fill}%
}%
\begin{pgfscope}%
\pgfsys@transformshift{5.481911in}{3.349692in}%
\pgfsys@useobject{currentmarker}{}%
\end{pgfscope}%
\end{pgfscope}%
\begin{pgfscope}%
\definecolor{textcolor}{rgb}{0.000000,0.000000,0.000000}%
\pgfsetstrokecolor{textcolor}%
\pgfsetfillcolor{textcolor}%
\pgftext[x=5.579133in,y=3.311430in,left,base]{\color{textcolor}\rmfamily\fontsize{8.000000}{9.600000}\selectfont \(\displaystyle 600\)}%
\end{pgfscope}%
\begin{pgfscope}%
\definecolor{textcolor}{rgb}{0.000000,0.000000,0.000000}%
\pgfsetstrokecolor{textcolor}%
\pgfsetfillcolor{textcolor}%
\pgftext[x=5.811775in,y=2.021577in,,top,rotate=90.000000]{\color{textcolor}\rmfamily\fontsize{8.000000}{9.600000}\selectfont Längswiderstand \(\displaystyle R_{xx} (\Omega)\)}%
\end{pgfscope}%
\begin{pgfscope}%
\pgfpathrectangle{\pgfqpoint{0.592318in}{0.451986in}}{\pgfqpoint{4.889593in}{3.139182in}}%
\pgfusepath{clip}%
\pgfsetrectcap%
\pgfsetroundjoin%
\pgfsetlinewidth{1.003750pt}%
\definecolor{currentstroke}{rgb}{0.152941,0.235294,0.458824}%
\pgfsetstrokecolor{currentstroke}%
\pgfsetdash{}{0pt}%
\pgfpathmoveto{\pgfqpoint{0.814572in}{0.967502in}}%
\pgfpathlineto{\pgfqpoint{0.821242in}{0.967539in}}%
\pgfpathlineto{\pgfqpoint{1.001313in}{0.961427in}}%
\pgfpathlineto{\pgfqpoint{1.016874in}{0.960683in}}%
\pgfpathlineto{\pgfqpoint{1.044663in}{0.959737in}}%
\pgfpathlineto{\pgfqpoint{1.050221in}{0.959350in}}%
\pgfpathlineto{\pgfqpoint{1.075786in}{0.958244in}}%
\pgfpathlineto{\pgfqpoint{1.098017in}{0.957501in}}%
\pgfpathlineto{\pgfqpoint{1.120248in}{0.956443in}}%
\pgfpathlineto{\pgfqpoint{1.128029in}{0.955965in}}%
\pgfpathlineto{\pgfqpoint{1.150260in}{0.954931in}}%
\pgfpathlineto{\pgfqpoint{1.186941in}{0.952840in}}%
\pgfpathlineto{\pgfqpoint{1.208061in}{0.951826in}}%
\pgfpathlineto{\pgfqpoint{1.224734in}{0.950677in}}%
\pgfpathlineto{\pgfqpoint{1.245853in}{0.949566in}}%
\pgfpathlineto{\pgfqpoint{1.251411in}{0.949112in}}%
\pgfpathlineto{\pgfqpoint{1.270307in}{0.948083in}}%
\pgfpathlineto{\pgfqpoint{1.326996in}{0.944495in}}%
\pgfpathlineto{\pgfqpoint{1.344781in}{0.943712in}}%
\pgfpathlineto{\pgfqpoint{1.348116in}{0.943234in}}%
\pgfpathlineto{\pgfqpoint{1.355897in}{0.942867in}}%
\pgfpathlineto{\pgfqpoint{1.383685in}{0.941037in}}%
\pgfpathlineto{\pgfqpoint{1.403693in}{0.939955in}}%
\pgfpathlineto{\pgfqpoint{1.408140in}{0.939385in}}%
\pgfpathlineto{\pgfqpoint{1.437040in}{0.938492in}}%
\pgfpathlineto{\pgfqpoint{1.439263in}{0.937724in}}%
\pgfpathlineto{\pgfqpoint{1.473721in}{0.936854in}}%
\pgfpathlineto{\pgfqpoint{1.475944in}{0.935642in}}%
\pgfpathlineto{\pgfqpoint{1.481502in}{0.934821in}}%
\pgfpathlineto{\pgfqpoint{1.484836in}{0.934305in}}%
\pgfpathlineto{\pgfqpoint{1.491506in}{0.934724in}}%
\pgfpathlineto{\pgfqpoint{1.493729in}{0.935628in}}%
\pgfpathlineto{\pgfqpoint{1.498175in}{0.936584in}}%
\pgfpathlineto{\pgfqpoint{1.500398in}{0.937318in}}%
\pgfpathlineto{\pgfqpoint{1.511514in}{0.937246in}}%
\pgfpathlineto{\pgfqpoint{1.513737in}{0.935956in}}%
\pgfpathlineto{\pgfqpoint{1.514848in}{0.935956in}}%
\pgfpathlineto{\pgfqpoint{1.517071in}{0.934619in}}%
\pgfpathlineto{\pgfqpoint{1.519294in}{0.933966in}}%
\pgfpathlineto{\pgfqpoint{1.524852in}{0.931910in}}%
\pgfpathlineto{\pgfqpoint{1.535968in}{0.932521in}}%
\pgfpathlineto{\pgfqpoint{1.548195in}{0.937531in}}%
\pgfpathlineto{\pgfqpoint{1.560422in}{0.936806in}}%
\pgfpathlineto{\pgfqpoint{1.568203in}{0.931907in}}%
\pgfpathlineto{\pgfqpoint{1.569314in}{0.931907in}}%
\pgfpathlineto{\pgfqpoint{1.571537in}{0.930032in}}%
\pgfpathlineto{\pgfqpoint{1.572649in}{0.930032in}}%
\pgfpathlineto{\pgfqpoint{1.574872in}{0.928483in}}%
\pgfpathlineto{\pgfqpoint{1.587099in}{0.929548in}}%
\pgfpathlineto{\pgfqpoint{1.592657in}{0.932716in}}%
\pgfpathlineto{\pgfqpoint{1.593768in}{0.932716in}}%
\pgfpathlineto{\pgfqpoint{1.595991in}{0.934976in}}%
\pgfpathlineto{\pgfqpoint{1.597103in}{0.934976in}}%
\pgfpathlineto{\pgfqpoint{1.599326in}{0.937183in}}%
\pgfpathlineto{\pgfqpoint{1.600438in}{0.937183in}}%
\pgfpathlineto{\pgfqpoint{1.602661in}{0.938965in}}%
\pgfpathlineto{\pgfqpoint{1.607107in}{0.940015in}}%
\pgfpathlineto{\pgfqpoint{1.611553in}{0.939371in}}%
\pgfpathlineto{\pgfqpoint{1.613776in}{0.939371in}}%
\pgfpathlineto{\pgfqpoint{1.615999in}{0.937598in}}%
\pgfpathlineto{\pgfqpoint{1.617111in}{0.937598in}}%
\pgfpathlineto{\pgfqpoint{1.619334in}{0.935140in}}%
\pgfpathlineto{\pgfqpoint{1.620445in}{0.935140in}}%
\pgfpathlineto{\pgfqpoint{1.622669in}{0.932314in}}%
\pgfpathlineto{\pgfqpoint{1.623780in}{0.932314in}}%
\pgfpathlineto{\pgfqpoint{1.628226in}{0.926632in}}%
\pgfpathlineto{\pgfqpoint{1.630449in}{0.926632in}}%
\pgfpathlineto{\pgfqpoint{1.632672in}{0.924337in}}%
\pgfpathlineto{\pgfqpoint{1.633784in}{0.924337in}}%
\pgfpathlineto{\pgfqpoint{1.636007in}{0.923130in}}%
\pgfpathlineto{\pgfqpoint{1.644899in}{0.923937in}}%
\pgfpathlineto{\pgfqpoint{1.647123in}{0.925752in}}%
\pgfpathlineto{\pgfqpoint{1.648234in}{0.925752in}}%
\pgfpathlineto{\pgfqpoint{1.650457in}{0.928537in}}%
\pgfpathlineto{\pgfqpoint{1.651569in}{0.928537in}}%
\pgfpathlineto{\pgfqpoint{1.653792in}{0.931800in}}%
\pgfpathlineto{\pgfqpoint{1.654903in}{0.931800in}}%
\pgfpathlineto{\pgfqpoint{1.657127in}{0.935396in}}%
\pgfpathlineto{\pgfqpoint{1.658238in}{0.935396in}}%
\pgfpathlineto{\pgfqpoint{1.660461in}{0.938608in}}%
\pgfpathlineto{\pgfqpoint{1.661573in}{0.938608in}}%
\pgfpathlineto{\pgfqpoint{1.663796in}{0.941423in}}%
\pgfpathlineto{\pgfqpoint{1.664907in}{0.941423in}}%
\pgfpathlineto{\pgfqpoint{1.667130in}{0.943297in}}%
\pgfpathlineto{\pgfqpoint{1.678246in}{0.942631in}}%
\pgfpathlineto{\pgfqpoint{1.680469in}{0.940158in}}%
\pgfpathlineto{\pgfqpoint{1.681581in}{0.940158in}}%
\pgfpathlineto{\pgfqpoint{1.683804in}{0.936883in}}%
\pgfpathlineto{\pgfqpoint{1.686027in}{0.936883in}}%
\pgfpathlineto{\pgfqpoint{1.688250in}{0.932886in}}%
\pgfpathlineto{\pgfqpoint{1.693808in}{0.924695in}}%
\pgfpathlineto{\pgfqpoint{1.694919in}{0.924695in}}%
\pgfpathlineto{\pgfqpoint{1.698254in}{0.921125in}}%
\pgfpathlineto{\pgfqpoint{1.699365in}{0.921125in}}%
\pgfpathlineto{\pgfqpoint{1.701588in}{0.918406in}}%
\pgfpathlineto{\pgfqpoint{1.702700in}{0.918406in}}%
\pgfpathlineto{\pgfqpoint{1.704923in}{0.916700in}}%
\pgfpathlineto{\pgfqpoint{1.711592in}{0.917244in}}%
\pgfpathlineto{\pgfqpoint{1.714927in}{0.919490in}}%
\pgfpathlineto{\pgfqpoint{1.716039in}{0.919490in}}%
\pgfpathlineto{\pgfqpoint{1.718262in}{0.922811in}}%
\pgfpathlineto{\pgfqpoint{1.719373in}{0.922811in}}%
\pgfpathlineto{\pgfqpoint{1.724931in}{0.931794in}}%
\pgfpathlineto{\pgfqpoint{1.726043in}{0.931794in}}%
\pgfpathlineto{\pgfqpoint{1.728266in}{0.936695in}}%
\pgfpathlineto{\pgfqpoint{1.729377in}{0.936695in}}%
\pgfpathlineto{\pgfqpoint{1.732712in}{0.941669in}}%
\pgfpathlineto{\pgfqpoint{1.736047in}{0.945948in}}%
\pgfpathlineto{\pgfqpoint{1.740493in}{0.951782in}}%
\pgfpathlineto{\pgfqpoint{1.752720in}{0.950821in}}%
\pgfpathlineto{\pgfqpoint{1.756054in}{0.947967in}}%
\pgfpathlineto{\pgfqpoint{1.757166in}{0.947967in}}%
\pgfpathlineto{\pgfqpoint{1.759389in}{0.943862in}}%
\pgfpathlineto{\pgfqpoint{1.760501in}{0.943862in}}%
\pgfpathlineto{\pgfqpoint{1.762724in}{0.938979in}}%
\pgfpathlineto{\pgfqpoint{1.763835in}{0.938979in}}%
\pgfpathlineto{\pgfqpoint{1.766058in}{0.933472in}}%
\pgfpathlineto{\pgfqpoint{1.767170in}{0.933472in}}%
\pgfpathlineto{\pgfqpoint{1.769393in}{0.927721in}}%
\pgfpathlineto{\pgfqpoint{1.770505in}{0.927721in}}%
\pgfpathlineto{\pgfqpoint{1.772728in}{0.922185in}}%
\pgfpathlineto{\pgfqpoint{1.773839in}{0.922185in}}%
\pgfpathlineto{\pgfqpoint{1.778285in}{0.912762in}}%
\pgfpathlineto{\pgfqpoint{1.780508in}{0.912762in}}%
\pgfpathlineto{\pgfqpoint{1.784955in}{0.907647in}}%
\pgfpathlineto{\pgfqpoint{1.791624in}{0.907250in}}%
\pgfpathlineto{\pgfqpoint{1.796070in}{0.911492in}}%
\pgfpathlineto{\pgfqpoint{1.797182in}{0.911492in}}%
\pgfpathlineto{\pgfqpoint{1.802739in}{0.921311in}}%
\pgfpathlineto{\pgfqpoint{1.804963in}{0.921311in}}%
\pgfpathlineto{\pgfqpoint{1.807186in}{0.927800in}}%
\pgfpathlineto{\pgfqpoint{1.808297in}{0.927800in}}%
\pgfpathlineto{\pgfqpoint{1.810520in}{0.934557in}}%
\pgfpathlineto{\pgfqpoint{1.811632in}{0.934557in}}%
\pgfpathlineto{\pgfqpoint{1.813855in}{0.941491in}}%
\pgfpathlineto{\pgfqpoint{1.817190in}{0.948228in}}%
\pgfpathlineto{\pgfqpoint{1.820524in}{0.954593in}}%
\pgfpathlineto{\pgfqpoint{1.821636in}{0.954593in}}%
\pgfpathlineto{\pgfqpoint{1.826082in}{0.964127in}}%
\pgfpathlineto{\pgfqpoint{1.828305in}{0.964127in}}%
\pgfpathlineto{\pgfqpoint{1.830528in}{0.966880in}}%
\pgfpathlineto{\pgfqpoint{1.831640in}{0.966880in}}%
\pgfpathlineto{\pgfqpoint{1.833863in}{0.968237in}}%
\pgfpathlineto{\pgfqpoint{1.839421in}{0.967507in}}%
\pgfpathlineto{\pgfqpoint{1.841644in}{0.966657in}}%
\pgfpathlineto{\pgfqpoint{1.843867in}{0.963373in}}%
\pgfpathlineto{\pgfqpoint{1.844978in}{0.963373in}}%
\pgfpathlineto{\pgfqpoint{1.847201in}{0.959075in}}%
\pgfpathlineto{\pgfqpoint{1.849425in}{0.959075in}}%
\pgfpathlineto{\pgfqpoint{1.851648in}{0.953318in}}%
\pgfpathlineto{\pgfqpoint{1.852759in}{0.953318in}}%
\pgfpathlineto{\pgfqpoint{1.854982in}{0.946716in}}%
\pgfpathlineto{\pgfqpoint{1.860540in}{0.931903in}}%
\pgfpathlineto{\pgfqpoint{1.861652in}{0.931903in}}%
\pgfpathlineto{\pgfqpoint{1.864986in}{0.924055in}}%
\pgfpathlineto{\pgfqpoint{1.866098in}{0.924055in}}%
\pgfpathlineto{\pgfqpoint{1.868321in}{0.916449in}}%
\pgfpathlineto{\pgfqpoint{1.869432in}{0.916449in}}%
\pgfpathlineto{\pgfqpoint{1.871655in}{0.909747in}}%
\pgfpathlineto{\pgfqpoint{1.872767in}{0.909747in}}%
\pgfpathlineto{\pgfqpoint{1.874990in}{0.903841in}}%
\pgfpathlineto{\pgfqpoint{1.878325in}{0.899249in}}%
\pgfpathlineto{\pgfqpoint{1.879436in}{0.899249in}}%
\pgfpathlineto{\pgfqpoint{1.881659in}{0.895925in}}%
\pgfpathlineto{\pgfqpoint{1.882771in}{0.895925in}}%
\pgfpathlineto{\pgfqpoint{1.884994in}{0.894543in}}%
\pgfpathlineto{\pgfqpoint{1.889440in}{0.894983in}}%
\pgfpathlineto{\pgfqpoint{1.891663in}{0.897233in}}%
\pgfpathlineto{\pgfqpoint{1.892775in}{0.897233in}}%
\pgfpathlineto{\pgfqpoint{1.894998in}{0.901074in}}%
\pgfpathlineto{\pgfqpoint{1.897221in}{0.901074in}}%
\pgfpathlineto{\pgfqpoint{1.899444in}{0.906661in}}%
\pgfpathlineto{\pgfqpoint{1.905002in}{0.921891in}}%
\pgfpathlineto{\pgfqpoint{1.906114in}{0.921891in}}%
\pgfpathlineto{\pgfqpoint{1.911671in}{0.940428in}}%
\pgfpathlineto{\pgfqpoint{1.913894in}{0.940428in}}%
\pgfpathlineto{\pgfqpoint{1.916117in}{0.949962in}}%
\pgfpathlineto{\pgfqpoint{1.917229in}{0.949962in}}%
\pgfpathlineto{\pgfqpoint{1.919452in}{0.959254in}}%
\pgfpathlineto{\pgfqpoint{1.922787in}{0.967676in}}%
\pgfpathlineto{\pgfqpoint{1.923898in}{0.967676in}}%
\pgfpathlineto{\pgfqpoint{1.926121in}{0.975568in}}%
\pgfpathlineto{\pgfqpoint{1.927233in}{0.975568in}}%
\pgfpathlineto{\pgfqpoint{1.931679in}{0.987907in}}%
\pgfpathlineto{\pgfqpoint{1.933902in}{0.987907in}}%
\pgfpathlineto{\pgfqpoint{1.936125in}{0.991901in}}%
\pgfpathlineto{\pgfqpoint{1.937237in}{0.991901in}}%
\pgfpathlineto{\pgfqpoint{1.939460in}{0.994514in}}%
\pgfpathlineto{\pgfqpoint{1.948352in}{0.993449in}}%
\pgfpathlineto{\pgfqpoint{1.953910in}{0.988361in}}%
\pgfpathlineto{\pgfqpoint{1.956133in}{0.982947in}}%
\pgfpathlineto{\pgfqpoint{1.957245in}{0.982947in}}%
\pgfpathlineto{\pgfqpoint{1.959468in}{0.976297in}}%
\pgfpathlineto{\pgfqpoint{1.961691in}{0.976297in}}%
\pgfpathlineto{\pgfqpoint{1.963914in}{0.968493in}}%
\pgfpathlineto{\pgfqpoint{1.965026in}{0.968493in}}%
\pgfpathlineto{\pgfqpoint{1.967249in}{0.959824in}}%
\pgfpathlineto{\pgfqpoint{1.968360in}{0.959824in}}%
\pgfpathlineto{\pgfqpoint{1.970583in}{0.950227in}}%
\pgfpathlineto{\pgfqpoint{1.971695in}{0.950227in}}%
\pgfpathlineto{\pgfqpoint{1.973918in}{0.940196in}}%
\pgfpathlineto{\pgfqpoint{1.975030in}{0.940196in}}%
\pgfpathlineto{\pgfqpoint{1.977253in}{0.930026in}}%
\pgfpathlineto{\pgfqpoint{1.978364in}{0.930026in}}%
\pgfpathlineto{\pgfqpoint{1.980587in}{0.919955in}}%
\pgfpathlineto{\pgfqpoint{1.981699in}{0.919955in}}%
\pgfpathlineto{\pgfqpoint{1.983922in}{0.910159in}}%
\pgfpathlineto{\pgfqpoint{1.985033in}{0.910159in}}%
\pgfpathlineto{\pgfqpoint{1.987257in}{0.901018in}}%
\pgfpathlineto{\pgfqpoint{1.988368in}{0.901018in}}%
\pgfpathlineto{\pgfqpoint{1.990591in}{0.892740in}}%
\pgfpathlineto{\pgfqpoint{1.991703in}{0.892740in}}%
\pgfpathlineto{\pgfqpoint{1.996149in}{0.880489in}}%
\pgfpathlineto{\pgfqpoint{1.998372in}{0.880489in}}%
\pgfpathlineto{\pgfqpoint{2.000595in}{0.877032in}}%
\pgfpathlineto{\pgfqpoint{2.001707in}{0.877032in}}%
\pgfpathlineto{\pgfqpoint{2.003930in}{0.875319in}}%
\pgfpathlineto{\pgfqpoint{2.009488in}{0.875457in}}%
\pgfpathlineto{\pgfqpoint{2.012822in}{0.882246in}}%
\pgfpathlineto{\pgfqpoint{2.015045in}{0.882246in}}%
\pgfpathlineto{\pgfqpoint{2.017268in}{0.888477in}}%
\pgfpathlineto{\pgfqpoint{2.018380in}{0.888477in}}%
\pgfpathlineto{\pgfqpoint{2.020603in}{0.896440in}}%
\pgfpathlineto{\pgfqpoint{2.022826in}{0.896440in}}%
\pgfpathlineto{\pgfqpoint{2.025049in}{0.905996in}}%
\pgfpathlineto{\pgfqpoint{2.026161in}{0.905996in}}%
\pgfpathlineto{\pgfqpoint{2.028384in}{0.916851in}}%
\pgfpathlineto{\pgfqpoint{2.033942in}{0.940819in}}%
\pgfpathlineto{\pgfqpoint{2.036165in}{0.940819in}}%
\pgfpathlineto{\pgfqpoint{2.038388in}{0.953270in}}%
\pgfpathlineto{\pgfqpoint{2.039499in}{0.953270in}}%
\pgfpathlineto{\pgfqpoint{2.041722in}{0.965740in}}%
\pgfpathlineto{\pgfqpoint{2.042834in}{0.965740in}}%
\pgfpathlineto{\pgfqpoint{2.045057in}{0.977968in}}%
\pgfpathlineto{\pgfqpoint{2.046169in}{0.977968in}}%
\pgfpathlineto{\pgfqpoint{2.048392in}{0.989651in}}%
\pgfpathlineto{\pgfqpoint{2.049503in}{0.989651in}}%
\pgfpathlineto{\pgfqpoint{2.051726in}{1.000305in}}%
\pgfpathlineto{\pgfqpoint{2.052838in}{1.000305in}}%
\pgfpathlineto{\pgfqpoint{2.056173in}{1.010200in}}%
\pgfpathlineto{\pgfqpoint{2.059507in}{1.018710in}}%
\pgfpathlineto{\pgfqpoint{2.062842in}{1.025925in}}%
\pgfpathlineto{\pgfqpoint{2.066177in}{1.031822in}}%
\pgfpathlineto{\pgfqpoint{2.067288in}{1.031822in}}%
\pgfpathlineto{\pgfqpoint{2.069511in}{1.035927in}}%
\pgfpathlineto{\pgfqpoint{2.072846in}{1.038680in}}%
\pgfpathlineto{\pgfqpoint{2.076180in}{1.039713in}}%
\pgfpathlineto{\pgfqpoint{2.083961in}{1.036782in}}%
\pgfpathlineto{\pgfqpoint{2.086184in}{1.032976in}}%
\pgfpathlineto{\pgfqpoint{2.087296in}{1.032976in}}%
\pgfpathlineto{\pgfqpoint{2.089519in}{1.027625in}}%
\pgfpathlineto{\pgfqpoint{2.090631in}{1.027625in}}%
\pgfpathlineto{\pgfqpoint{2.092854in}{1.021023in}}%
\pgfpathlineto{\pgfqpoint{2.093965in}{1.021023in}}%
\pgfpathlineto{\pgfqpoint{2.096188in}{1.012924in}}%
\pgfpathlineto{\pgfqpoint{2.097300in}{1.012924in}}%
\pgfpathlineto{\pgfqpoint{2.102858in}{0.993519in}}%
\pgfpathlineto{\pgfqpoint{2.103969in}{0.993519in}}%
\pgfpathlineto{\pgfqpoint{2.106192in}{0.982170in}}%
\pgfpathlineto{\pgfqpoint{2.107304in}{0.982170in}}%
\pgfpathlineto{\pgfqpoint{2.109527in}{0.970366in}}%
\pgfpathlineto{\pgfqpoint{2.110639in}{0.970366in}}%
\pgfpathlineto{\pgfqpoint{2.112862in}{0.957597in}}%
\pgfpathlineto{\pgfqpoint{2.115085in}{0.957597in}}%
\pgfpathlineto{\pgfqpoint{2.117308in}{0.944519in}}%
\pgfpathlineto{\pgfqpoint{2.121754in}{0.925160in}}%
\pgfpathlineto{\pgfqpoint{2.125089in}{0.912410in}}%
\pgfpathlineto{\pgfqpoint{2.127312in}{0.906074in}}%
\pgfpathlineto{\pgfqpoint{2.130646in}{0.894206in}}%
\pgfpathlineto{\pgfqpoint{2.131758in}{0.894206in}}%
\pgfpathlineto{\pgfqpoint{2.133981in}{0.882951in}}%
\pgfpathlineto{\pgfqpoint{2.135093in}{0.882951in}}%
\pgfpathlineto{\pgfqpoint{2.137316in}{0.872890in}}%
\pgfpathlineto{\pgfqpoint{2.138427in}{0.872890in}}%
\pgfpathlineto{\pgfqpoint{2.140650in}{0.863884in}}%
\pgfpathlineto{\pgfqpoint{2.141762in}{0.863884in}}%
\pgfpathlineto{\pgfqpoint{2.143985in}{0.856718in}}%
\pgfpathlineto{\pgfqpoint{2.145097in}{0.856718in}}%
\pgfpathlineto{\pgfqpoint{2.147320in}{0.851422in}}%
\pgfpathlineto{\pgfqpoint{2.148431in}{0.851422in}}%
\pgfpathlineto{\pgfqpoint{2.151766in}{0.847799in}}%
\pgfpathlineto{\pgfqpoint{2.153989in}{0.846463in}}%
\pgfpathlineto{\pgfqpoint{2.158435in}{0.847230in}}%
\pgfpathlineto{\pgfqpoint{2.160658in}{0.850286in}}%
\pgfpathlineto{\pgfqpoint{2.161770in}{0.850286in}}%
\pgfpathlineto{\pgfqpoint{2.165104in}{0.855494in}}%
\pgfpathlineto{\pgfqpoint{2.166216in}{0.855494in}}%
\pgfpathlineto{\pgfqpoint{2.169551in}{0.871892in}}%
\pgfpathlineto{\pgfqpoint{2.171774in}{0.871892in}}%
\pgfpathlineto{\pgfqpoint{2.173997in}{0.882945in}}%
\pgfpathlineto{\pgfqpoint{2.175108in}{0.882945in}}%
\pgfpathlineto{\pgfqpoint{2.177331in}{0.895343in}}%
\pgfpathlineto{\pgfqpoint{2.179555in}{0.895343in}}%
\pgfpathlineto{\pgfqpoint{2.181778in}{0.909222in}}%
\pgfpathlineto{\pgfqpoint{2.182889in}{0.909222in}}%
\pgfpathlineto{\pgfqpoint{2.185112in}{0.924219in}}%
\pgfpathlineto{\pgfqpoint{2.188447in}{0.939752in}}%
\pgfpathlineto{\pgfqpoint{2.189558in}{0.939752in}}%
\pgfpathlineto{\pgfqpoint{2.191782in}{0.955603in}}%
\pgfpathlineto{\pgfqpoint{2.192893in}{0.955603in}}%
\pgfpathlineto{\pgfqpoint{2.195116in}{0.971844in}}%
\pgfpathlineto{\pgfqpoint{2.196228in}{0.971844in}}%
\pgfpathlineto{\pgfqpoint{2.198451in}{0.987941in}}%
\pgfpathlineto{\pgfqpoint{2.199562in}{0.987941in}}%
\pgfpathlineto{\pgfqpoint{2.201786in}{1.003598in}}%
\pgfpathlineto{\pgfqpoint{2.202897in}{1.003598in}}%
\pgfpathlineto{\pgfqpoint{2.205120in}{1.018782in}}%
\pgfpathlineto{\pgfqpoint{2.206232in}{1.018782in}}%
\pgfpathlineto{\pgfqpoint{2.208455in}{1.033358in}}%
\pgfpathlineto{\pgfqpoint{2.209566in}{1.033358in}}%
\pgfpathlineto{\pgfqpoint{2.211789in}{1.047127in}}%
\pgfpathlineto{\pgfqpoint{2.214013in}{1.047127in}}%
\pgfpathlineto{\pgfqpoint{2.216236in}{1.059567in}}%
\pgfpathlineto{\pgfqpoint{2.218459in}{1.070931in}}%
\pgfpathlineto{\pgfqpoint{2.219570in}{1.070931in}}%
\pgfpathlineto{\pgfqpoint{2.224017in}{1.089771in}}%
\pgfpathlineto{\pgfqpoint{2.227351in}{1.089771in}}%
\pgfpathlineto{\pgfqpoint{2.229574in}{1.097286in}}%
\pgfpathlineto{\pgfqpoint{2.232909in}{1.103304in}}%
\pgfpathlineto{\pgfqpoint{2.234020in}{1.103304in}}%
\pgfpathlineto{\pgfqpoint{2.236244in}{1.107892in}}%
\pgfpathlineto{\pgfqpoint{2.239578in}{1.111089in}}%
\pgfpathlineto{\pgfqpoint{2.242913in}{1.112682in}}%
\pgfpathlineto{\pgfqpoint{2.247359in}{1.112803in}}%
\pgfpathlineto{\pgfqpoint{2.249582in}{1.111301in}}%
\pgfpathlineto{\pgfqpoint{2.250694in}{1.111301in}}%
\pgfpathlineto{\pgfqpoint{2.252917in}{1.108466in}}%
\pgfpathlineto{\pgfqpoint{2.256251in}{1.104294in}}%
\pgfpathlineto{\pgfqpoint{2.257363in}{1.104294in}}%
\pgfpathlineto{\pgfqpoint{2.259586in}{1.098672in}}%
\pgfpathlineto{\pgfqpoint{2.260698in}{1.098672in}}%
\pgfpathlineto{\pgfqpoint{2.266255in}{1.083565in}}%
\pgfpathlineto{\pgfqpoint{2.267367in}{1.083565in}}%
\pgfpathlineto{\pgfqpoint{2.269590in}{1.074365in}}%
\pgfpathlineto{\pgfqpoint{2.270702in}{1.074365in}}%
\pgfpathlineto{\pgfqpoint{2.272925in}{1.063808in}}%
\pgfpathlineto{\pgfqpoint{2.274036in}{1.063808in}}%
\pgfpathlineto{\pgfqpoint{2.276259in}{1.052222in}}%
\pgfpathlineto{\pgfqpoint{2.277371in}{1.052222in}}%
\pgfpathlineto{\pgfqpoint{2.282929in}{1.026413in}}%
\pgfpathlineto{\pgfqpoint{2.284040in}{1.026413in}}%
\pgfpathlineto{\pgfqpoint{2.289598in}{0.997880in}}%
\pgfpathlineto{\pgfqpoint{2.291821in}{0.997880in}}%
\pgfpathlineto{\pgfqpoint{2.294044in}{0.982590in}}%
\pgfpathlineto{\pgfqpoint{2.295156in}{0.982590in}}%
\pgfpathlineto{\pgfqpoint{2.297379in}{0.966710in}}%
\pgfpathlineto{\pgfqpoint{2.298490in}{0.966710in}}%
\pgfpathlineto{\pgfqpoint{2.300713in}{0.951005in}}%
\pgfpathlineto{\pgfqpoint{2.301825in}{0.951005in}}%
\pgfpathlineto{\pgfqpoint{2.304048in}{0.935019in}}%
\pgfpathlineto{\pgfqpoint{2.305160in}{0.935019in}}%
\pgfpathlineto{\pgfqpoint{2.307383in}{0.918977in}}%
\pgfpathlineto{\pgfqpoint{2.308494in}{0.918977in}}%
\pgfpathlineto{\pgfqpoint{2.312940in}{0.888461in}}%
\pgfpathlineto{\pgfqpoint{2.315164in}{0.888461in}}%
\pgfpathlineto{\pgfqpoint{2.317387in}{0.873910in}}%
\pgfpathlineto{\pgfqpoint{2.318498in}{0.873910in}}%
\pgfpathlineto{\pgfqpoint{2.320721in}{0.860312in}}%
\pgfpathlineto{\pgfqpoint{2.321833in}{0.860312in}}%
\pgfpathlineto{\pgfqpoint{2.325167in}{0.847594in}}%
\pgfpathlineto{\pgfqpoint{2.328502in}{0.836392in}}%
\pgfpathlineto{\pgfqpoint{2.332948in}{0.818502in}}%
\pgfpathlineto{\pgfqpoint{2.335171in}{0.818502in}}%
\pgfpathlineto{\pgfqpoint{2.337395in}{0.812097in}}%
\pgfpathlineto{\pgfqpoint{2.339618in}{0.812097in}}%
\pgfpathlineto{\pgfqpoint{2.341841in}{0.807688in}}%
\pgfpathlineto{\pgfqpoint{2.342952in}{0.807688in}}%
\pgfpathlineto{\pgfqpoint{2.345175in}{0.805484in}}%
\pgfpathlineto{\pgfqpoint{2.348510in}{0.805384in}}%
\pgfpathlineto{\pgfqpoint{2.356291in}{0.812330in}}%
\pgfpathlineto{\pgfqpoint{2.358514in}{0.819143in}}%
\pgfpathlineto{\pgfqpoint{2.359625in}{0.819143in}}%
\pgfpathlineto{\pgfqpoint{2.361849in}{0.827936in}}%
\pgfpathlineto{\pgfqpoint{2.362960in}{0.827936in}}%
\pgfpathlineto{\pgfqpoint{2.365183in}{0.839239in}}%
\pgfpathlineto{\pgfqpoint{2.370741in}{0.866904in}}%
\pgfpathlineto{\pgfqpoint{2.372964in}{0.866904in}}%
\pgfpathlineto{\pgfqpoint{2.375187in}{0.882908in}}%
\pgfpathlineto{\pgfqpoint{2.376299in}{0.882908in}}%
\pgfpathlineto{\pgfqpoint{2.378522in}{0.900308in}}%
\pgfpathlineto{\pgfqpoint{2.379633in}{0.900308in}}%
\pgfpathlineto{\pgfqpoint{2.381856in}{0.918639in}}%
\pgfpathlineto{\pgfqpoint{2.382968in}{0.918639in}}%
\pgfpathlineto{\pgfqpoint{2.385191in}{0.937985in}}%
\pgfpathlineto{\pgfqpoint{2.387414in}{0.937985in}}%
\pgfpathlineto{\pgfqpoint{2.389637in}{0.957732in}}%
\pgfpathlineto{\pgfqpoint{2.395195in}{0.997880in}}%
\pgfpathlineto{\pgfqpoint{2.396307in}{0.997880in}}%
\pgfpathlineto{\pgfqpoint{2.401864in}{1.037854in}}%
\pgfpathlineto{\pgfqpoint{2.404087in}{1.037854in}}%
\pgfpathlineto{\pgfqpoint{2.406311in}{1.056776in}}%
\pgfpathlineto{\pgfqpoint{2.407422in}{1.056776in}}%
\pgfpathlineto{\pgfqpoint{2.409645in}{1.075669in}}%
\pgfpathlineto{\pgfqpoint{2.410757in}{1.075669in}}%
\pgfpathlineto{\pgfqpoint{2.412980in}{1.093688in}}%
\pgfpathlineto{\pgfqpoint{2.414091in}{1.093688in}}%
\pgfpathlineto{\pgfqpoint{2.416314in}{1.111002in}}%
\pgfpathlineto{\pgfqpoint{2.417426in}{1.111002in}}%
\pgfpathlineto{\pgfqpoint{2.419649in}{1.127282in}}%
\pgfpathlineto{\pgfqpoint{2.420761in}{1.127282in}}%
\pgfpathlineto{\pgfqpoint{2.422984in}{1.142674in}}%
\pgfpathlineto{\pgfqpoint{2.424095in}{1.142674in}}%
\pgfpathlineto{\pgfqpoint{2.426318in}{1.156926in}}%
\pgfpathlineto{\pgfqpoint{2.427430in}{1.156926in}}%
\pgfpathlineto{\pgfqpoint{2.429653in}{1.170236in}}%
\pgfpathlineto{\pgfqpoint{2.430765in}{1.170236in}}%
\pgfpathlineto{\pgfqpoint{2.432988in}{1.182280in}}%
\pgfpathlineto{\pgfqpoint{2.434099in}{1.182280in}}%
\pgfpathlineto{\pgfqpoint{2.438545in}{1.202956in}}%
\pgfpathlineto{\pgfqpoint{2.440769in}{1.202956in}}%
\pgfpathlineto{\pgfqpoint{2.442992in}{1.211287in}}%
\pgfpathlineto{\pgfqpoint{2.444103in}{1.211287in}}%
\pgfpathlineto{\pgfqpoint{2.446326in}{1.218618in}}%
\pgfpathlineto{\pgfqpoint{2.447438in}{1.218618in}}%
\pgfpathlineto{\pgfqpoint{2.449661in}{1.224548in}}%
\pgfpathlineto{\pgfqpoint{2.451884in}{1.224548in}}%
\pgfpathlineto{\pgfqpoint{2.454107in}{1.229257in}}%
\pgfpathlineto{\pgfqpoint{2.455219in}{1.229257in}}%
\pgfpathlineto{\pgfqpoint{2.457442in}{1.232599in}}%
\pgfpathlineto{\pgfqpoint{2.460776in}{1.234845in}}%
\pgfpathlineto{\pgfqpoint{2.465223in}{1.235816in}}%
\pgfpathlineto{\pgfqpoint{2.468557in}{1.235825in}}%
\pgfpathlineto{\pgfqpoint{2.470780in}{1.234029in}}%
\pgfpathlineto{\pgfqpoint{2.471892in}{1.234029in}}%
\pgfpathlineto{\pgfqpoint{2.474115in}{1.231421in}}%
\pgfpathlineto{\pgfqpoint{2.479673in}{1.222385in}}%
\pgfpathlineto{\pgfqpoint{2.481896in}{1.222385in}}%
\pgfpathlineto{\pgfqpoint{2.487454in}{1.209046in}}%
\pgfpathlineto{\pgfqpoint{2.488565in}{1.209046in}}%
\pgfpathlineto{\pgfqpoint{2.490788in}{1.200541in}}%
\pgfpathlineto{\pgfqpoint{2.491900in}{1.200541in}}%
\pgfpathlineto{\pgfqpoint{2.494123in}{1.191147in}}%
\pgfpathlineto{\pgfqpoint{2.496346in}{1.191147in}}%
\pgfpathlineto{\pgfqpoint{2.498569in}{1.180667in}}%
\pgfpathlineto{\pgfqpoint{2.499681in}{1.180667in}}%
\pgfpathlineto{\pgfqpoint{2.503015in}{1.157119in}}%
\pgfpathlineto{\pgfqpoint{2.505238in}{1.157119in}}%
\pgfpathlineto{\pgfqpoint{2.507462in}{1.143920in}}%
\pgfpathlineto{\pgfqpoint{2.508573in}{1.143920in}}%
\pgfpathlineto{\pgfqpoint{2.510796in}{1.129943in}}%
\pgfpathlineto{\pgfqpoint{2.513019in}{1.129943in}}%
\pgfpathlineto{\pgfqpoint{2.515242in}{1.115112in}}%
\pgfpathlineto{\pgfqpoint{2.520800in}{1.083580in}}%
\pgfpathlineto{\pgfqpoint{2.523023in}{1.083580in}}%
\pgfpathlineto{\pgfqpoint{2.525246in}{1.066860in}}%
\pgfpathlineto{\pgfqpoint{2.526358in}{1.066860in}}%
\pgfpathlineto{\pgfqpoint{2.528581in}{1.049768in}}%
\pgfpathlineto{\pgfqpoint{2.529692in}{1.049768in}}%
\pgfpathlineto{\pgfqpoint{2.531916in}{1.032377in}}%
\pgfpathlineto{\pgfqpoint{2.533027in}{1.032377in}}%
\pgfpathlineto{\pgfqpoint{2.535250in}{1.014402in}}%
\pgfpathlineto{\pgfqpoint{2.538585in}{0.996084in}}%
\pgfpathlineto{\pgfqpoint{2.539696in}{0.996084in}}%
\pgfpathlineto{\pgfqpoint{2.541920in}{0.977529in}}%
\pgfpathlineto{\pgfqpoint{2.543031in}{0.977529in}}%
\pgfpathlineto{\pgfqpoint{2.547477in}{0.940317in}}%
\pgfpathlineto{\pgfqpoint{2.549700in}{0.940317in}}%
\pgfpathlineto{\pgfqpoint{2.551923in}{0.921578in}}%
\pgfpathlineto{\pgfqpoint{2.553035in}{0.921578in}}%
\pgfpathlineto{\pgfqpoint{2.555258in}{0.903630in}}%
\pgfpathlineto{\pgfqpoint{2.557481in}{0.903630in}}%
\pgfpathlineto{\pgfqpoint{2.559704in}{0.885675in}}%
\pgfpathlineto{\pgfqpoint{2.563039in}{0.868059in}}%
\pgfpathlineto{\pgfqpoint{2.564151in}{0.868059in}}%
\pgfpathlineto{\pgfqpoint{2.566374in}{0.851273in}}%
\pgfpathlineto{\pgfqpoint{2.568597in}{0.835207in}}%
\pgfpathlineto{\pgfqpoint{2.569708in}{0.835207in}}%
\pgfpathlineto{\pgfqpoint{2.573043in}{0.819689in}}%
\pgfpathlineto{\pgfqpoint{2.574154in}{0.819689in}}%
\pgfpathlineto{\pgfqpoint{2.576378in}{0.805571in}}%
\pgfpathlineto{\pgfqpoint{2.577489in}{0.805571in}}%
\pgfpathlineto{\pgfqpoint{2.579712in}{0.792568in}}%
\pgfpathlineto{\pgfqpoint{2.583047in}{0.781007in}}%
\pgfpathlineto{\pgfqpoint{2.586381in}{0.771057in}}%
\pgfpathlineto{\pgfqpoint{2.587493in}{0.771057in}}%
\pgfpathlineto{\pgfqpoint{2.589716in}{0.762269in}}%
\pgfpathlineto{\pgfqpoint{2.590828in}{0.762269in}}%
\pgfpathlineto{\pgfqpoint{2.593051in}{0.755793in}}%
\pgfpathlineto{\pgfqpoint{2.594162in}{0.755793in}}%
\pgfpathlineto{\pgfqpoint{2.596385in}{0.750960in}}%
\pgfpathlineto{\pgfqpoint{2.597497in}{0.750960in}}%
\pgfpathlineto{\pgfqpoint{2.599720in}{0.748384in}}%
\pgfpathlineto{\pgfqpoint{2.607501in}{0.749404in}}%
\pgfpathlineto{\pgfqpoint{2.614170in}{0.759771in}}%
\pgfpathlineto{\pgfqpoint{2.616393in}{0.768442in}}%
\pgfpathlineto{\pgfqpoint{2.617505in}{0.768442in}}%
\pgfpathlineto{\pgfqpoint{2.619728in}{0.779133in}}%
\pgfpathlineto{\pgfqpoint{2.621951in}{0.779133in}}%
\pgfpathlineto{\pgfqpoint{2.624174in}{0.791941in}}%
\pgfpathlineto{\pgfqpoint{2.629732in}{0.823352in}}%
\pgfpathlineto{\pgfqpoint{2.630843in}{0.823352in}}%
\pgfpathlineto{\pgfqpoint{2.633067in}{0.841439in}}%
\pgfpathlineto{\pgfqpoint{2.635290in}{0.841439in}}%
\pgfpathlineto{\pgfqpoint{2.637513in}{0.861370in}}%
\pgfpathlineto{\pgfqpoint{2.638624in}{0.861370in}}%
\pgfpathlineto{\pgfqpoint{2.640847in}{0.882186in}}%
\pgfpathlineto{\pgfqpoint{2.641959in}{0.882186in}}%
\pgfpathlineto{\pgfqpoint{2.644182in}{0.903841in}}%
\pgfpathlineto{\pgfqpoint{2.645294in}{0.903841in}}%
\pgfpathlineto{\pgfqpoint{2.647517in}{0.926591in}}%
\pgfpathlineto{\pgfqpoint{2.650851in}{0.950140in}}%
\pgfpathlineto{\pgfqpoint{2.651963in}{0.950140in}}%
\pgfpathlineto{\pgfqpoint{2.654186in}{0.973911in}}%
\pgfpathlineto{\pgfqpoint{2.655298in}{0.973911in}}%
\pgfpathlineto{\pgfqpoint{2.657521in}{0.997817in}}%
\pgfpathlineto{\pgfqpoint{2.658632in}{0.997817in}}%
\pgfpathlineto{\pgfqpoint{2.660855in}{1.022313in}}%
\pgfpathlineto{\pgfqpoint{2.661967in}{1.022313in}}%
\pgfpathlineto{\pgfqpoint{2.664190in}{1.046248in}}%
\pgfpathlineto{\pgfqpoint{2.666413in}{1.046248in}}%
\pgfpathlineto{\pgfqpoint{2.668636in}{1.070521in}}%
\pgfpathlineto{\pgfqpoint{2.673082in}{1.117304in}}%
\pgfpathlineto{\pgfqpoint{2.675305in}{1.117304in}}%
\pgfpathlineto{\pgfqpoint{2.677528in}{1.139988in}}%
\pgfpathlineto{\pgfqpoint{2.679752in}{1.139988in}}%
\pgfpathlineto{\pgfqpoint{2.681975in}{1.162929in}}%
\pgfpathlineto{\pgfqpoint{2.683086in}{1.162929in}}%
\pgfpathlineto{\pgfqpoint{2.685309in}{1.184483in}}%
\pgfpathlineto{\pgfqpoint{2.686421in}{1.184483in}}%
\pgfpathlineto{\pgfqpoint{2.688644in}{1.205559in}}%
\pgfpathlineto{\pgfqpoint{2.691979in}{1.225794in}}%
\pgfpathlineto{\pgfqpoint{2.693090in}{1.225794in}}%
\pgfpathlineto{\pgfqpoint{2.695313in}{1.245528in}}%
\pgfpathlineto{\pgfqpoint{2.696425in}{1.245528in}}%
\pgfpathlineto{\pgfqpoint{2.698648in}{1.264252in}}%
\pgfpathlineto{\pgfqpoint{2.699759in}{1.264252in}}%
\pgfpathlineto{\pgfqpoint{2.701983in}{1.282044in}}%
\pgfpathlineto{\pgfqpoint{2.703094in}{1.282044in}}%
\pgfpathlineto{\pgfqpoint{2.705317in}{1.298995in}}%
\pgfpathlineto{\pgfqpoint{2.706429in}{1.298995in}}%
\pgfpathlineto{\pgfqpoint{2.708652in}{1.315039in}}%
\pgfpathlineto{\pgfqpoint{2.709763in}{1.315039in}}%
\pgfpathlineto{\pgfqpoint{2.711987in}{1.329996in}}%
\pgfpathlineto{\pgfqpoint{2.714210in}{1.329996in}}%
\pgfpathlineto{\pgfqpoint{2.716433in}{1.344117in}}%
\pgfpathlineto{\pgfqpoint{2.718656in}{1.357336in}}%
\pgfpathlineto{\pgfqpoint{2.719767in}{1.357336in}}%
\pgfpathlineto{\pgfqpoint{2.721990in}{1.369665in}}%
\pgfpathlineto{\pgfqpoint{2.723102in}{1.369665in}}%
\pgfpathlineto{\pgfqpoint{2.725325in}{1.381184in}}%
\pgfpathlineto{\pgfqpoint{2.727548in}{1.381184in}}%
\pgfpathlineto{\pgfqpoint{2.729771in}{1.391529in}}%
\pgfpathlineto{\pgfqpoint{2.730883in}{1.391529in}}%
\pgfpathlineto{\pgfqpoint{2.733106in}{1.400961in}}%
\pgfpathlineto{\pgfqpoint{2.734217in}{1.400961in}}%
\pgfpathlineto{\pgfqpoint{2.736441in}{1.409408in}}%
\pgfpathlineto{\pgfqpoint{2.739775in}{1.416748in}}%
\pgfpathlineto{\pgfqpoint{2.743110in}{1.423225in}}%
\pgfpathlineto{\pgfqpoint{2.744221in}{1.423225in}}%
\pgfpathlineto{\pgfqpoint{2.746445in}{1.428745in}}%
\pgfpathlineto{\pgfqpoint{2.747556in}{1.428745in}}%
\pgfpathlineto{\pgfqpoint{2.749779in}{1.433454in}}%
\pgfpathlineto{\pgfqpoint{2.756448in}{1.439650in}}%
\pgfpathlineto{\pgfqpoint{2.758672in}{1.440560in}}%
\pgfpathlineto{\pgfqpoint{2.762006in}{1.441867in}}%
\pgfpathlineto{\pgfqpoint{2.765341in}{1.442176in}}%
\pgfpathlineto{\pgfqpoint{2.770899in}{1.441108in}}%
\pgfpathlineto{\pgfqpoint{2.773122in}{1.439206in}}%
\pgfpathlineto{\pgfqpoint{2.774233in}{1.439206in}}%
\pgfpathlineto{\pgfqpoint{2.785349in}{1.425548in}}%
\pgfpathlineto{\pgfqpoint{2.789795in}{1.416671in}}%
\pgfpathlineto{\pgfqpoint{2.792018in}{1.416671in}}%
\pgfpathlineto{\pgfqpoint{2.794241in}{1.409596in}}%
\pgfpathlineto{\pgfqpoint{2.795353in}{1.409596in}}%
\pgfpathlineto{\pgfqpoint{2.797576in}{1.401811in}}%
\pgfpathlineto{\pgfqpoint{2.798687in}{1.401811in}}%
\pgfpathlineto{\pgfqpoint{2.800910in}{1.392949in}}%
\pgfpathlineto{\pgfqpoint{2.802022in}{1.392949in}}%
\pgfpathlineto{\pgfqpoint{2.804245in}{1.383502in}}%
\pgfpathlineto{\pgfqpoint{2.805357in}{1.383502in}}%
\pgfpathlineto{\pgfqpoint{2.810914in}{1.362329in}}%
\pgfpathlineto{\pgfqpoint{2.812026in}{1.362329in}}%
\pgfpathlineto{\pgfqpoint{2.814249in}{1.350459in}}%
\pgfpathlineto{\pgfqpoint{2.815361in}{1.350459in}}%
\pgfpathlineto{\pgfqpoint{2.817584in}{1.338071in}}%
\pgfpathlineto{\pgfqpoint{2.818695in}{1.338071in}}%
\pgfpathlineto{\pgfqpoint{2.820918in}{1.324997in}}%
\pgfpathlineto{\pgfqpoint{2.823141in}{1.324997in}}%
\pgfpathlineto{\pgfqpoint{2.825365in}{1.311045in}}%
\pgfpathlineto{\pgfqpoint{2.830922in}{1.281609in}}%
\pgfpathlineto{\pgfqpoint{2.832034in}{1.281609in}}%
\pgfpathlineto{\pgfqpoint{2.834257in}{1.266077in}}%
\pgfpathlineto{\pgfqpoint{2.835368in}{1.266077in}}%
\pgfpathlineto{\pgfqpoint{2.837592in}{1.249865in}}%
\pgfpathlineto{\pgfqpoint{2.839815in}{1.249865in}}%
\pgfpathlineto{\pgfqpoint{2.842038in}{1.232913in}}%
\pgfpathlineto{\pgfqpoint{2.843149in}{1.232913in}}%
\pgfpathlineto{\pgfqpoint{2.845372in}{1.215575in}}%
\pgfpathlineto{\pgfqpoint{2.850930in}{1.179822in}}%
\pgfpathlineto{\pgfqpoint{2.853153in}{1.179822in}}%
\pgfpathlineto{\pgfqpoint{2.855376in}{1.161262in}}%
\pgfpathlineto{\pgfqpoint{2.856488in}{1.161262in}}%
\pgfpathlineto{\pgfqpoint{2.858711in}{1.142205in}}%
\pgfpathlineto{\pgfqpoint{2.859823in}{1.142205in}}%
\pgfpathlineto{\pgfqpoint{2.862046in}{1.122762in}}%
\pgfpathlineto{\pgfqpoint{2.863157in}{1.122762in}}%
\pgfpathlineto{\pgfqpoint{2.865380in}{1.103149in}}%
\pgfpathlineto{\pgfqpoint{2.868715in}{1.083401in}}%
\pgfpathlineto{\pgfqpoint{2.869826in}{1.083401in}}%
\pgfpathlineto{\pgfqpoint{2.872050in}{1.063248in}}%
\pgfpathlineto{\pgfqpoint{2.873161in}{1.063248in}}%
\pgfpathlineto{\pgfqpoint{2.878719in}{1.022588in}}%
\pgfpathlineto{\pgfqpoint{2.879830in}{1.022588in}}%
\pgfpathlineto{\pgfqpoint{2.882054in}{1.002149in}}%
\pgfpathlineto{\pgfqpoint{2.883165in}{1.002149in}}%
\pgfpathlineto{\pgfqpoint{2.885388in}{0.981416in}}%
\pgfpathlineto{\pgfqpoint{2.887611in}{0.981416in}}%
\pgfpathlineto{\pgfqpoint{2.889834in}{0.961055in}}%
\pgfpathlineto{\pgfqpoint{2.893169in}{0.940612in}}%
\pgfpathlineto{\pgfqpoint{2.897615in}{0.900691in}}%
\pgfpathlineto{\pgfqpoint{2.900950in}{0.900691in}}%
\pgfpathlineto{\pgfqpoint{2.903173in}{0.880789in}}%
\pgfpathlineto{\pgfqpoint{2.904284in}{0.880789in}}%
\pgfpathlineto{\pgfqpoint{2.906508in}{0.861367in}}%
\pgfpathlineto{\pgfqpoint{2.907619in}{0.861367in}}%
\pgfpathlineto{\pgfqpoint{2.909842in}{0.841906in}}%
\pgfpathlineto{\pgfqpoint{2.910954in}{0.841906in}}%
\pgfpathlineto{\pgfqpoint{2.913177in}{0.823427in}}%
\pgfpathlineto{\pgfqpoint{2.914288in}{0.823427in}}%
\pgfpathlineto{\pgfqpoint{2.916512in}{0.805616in}}%
\pgfpathlineto{\pgfqpoint{2.917623in}{0.805616in}}%
\pgfpathlineto{\pgfqpoint{2.919846in}{0.788416in}}%
\pgfpathlineto{\pgfqpoint{2.920958in}{0.788416in}}%
\pgfpathlineto{\pgfqpoint{2.923181in}{0.771834in}}%
\pgfpathlineto{\pgfqpoint{2.924292in}{0.771834in}}%
\pgfpathlineto{\pgfqpoint{2.926515in}{0.756136in}}%
\pgfpathlineto{\pgfqpoint{2.927627in}{0.756136in}}%
\pgfpathlineto{\pgfqpoint{2.929850in}{0.741481in}}%
\pgfpathlineto{\pgfqpoint{2.932073in}{0.741481in}}%
\pgfpathlineto{\pgfqpoint{2.934296in}{0.728045in}}%
\pgfpathlineto{\pgfqpoint{2.938743in}{0.704134in}}%
\pgfpathlineto{\pgfqpoint{2.940966in}{0.704134in}}%
\pgfpathlineto{\pgfqpoint{2.943189in}{0.694442in}}%
\pgfpathlineto{\pgfqpoint{2.944300in}{0.694442in}}%
\pgfpathlineto{\pgfqpoint{2.946523in}{0.686140in}}%
\pgfpathlineto{\pgfqpoint{2.948746in}{0.686140in}}%
\pgfpathlineto{\pgfqpoint{2.950970in}{0.679221in}}%
\pgfpathlineto{\pgfqpoint{2.954304in}{0.674426in}}%
\pgfpathlineto{\pgfqpoint{2.957639in}{0.671416in}}%
\pgfpathlineto{\pgfqpoint{2.962085in}{0.670199in}}%
\pgfpathlineto{\pgfqpoint{2.964308in}{0.671126in}}%
\pgfpathlineto{\pgfqpoint{2.965420in}{0.671126in}}%
\pgfpathlineto{\pgfqpoint{2.967643in}{0.674205in}}%
\pgfpathlineto{\pgfqpoint{2.968754in}{0.674205in}}%
\pgfpathlineto{\pgfqpoint{2.970977in}{0.679175in}}%
\pgfpathlineto{\pgfqpoint{2.972089in}{0.679175in}}%
\pgfpathlineto{\pgfqpoint{2.974312in}{0.686448in}}%
\pgfpathlineto{\pgfqpoint{2.975424in}{0.686448in}}%
\pgfpathlineto{\pgfqpoint{2.977647in}{0.695633in}}%
\pgfpathlineto{\pgfqpoint{2.978758in}{0.695633in}}%
\pgfpathlineto{\pgfqpoint{2.984316in}{0.720971in}}%
\pgfpathlineto{\pgfqpoint{2.985428in}{0.720971in}}%
\pgfpathlineto{\pgfqpoint{2.987651in}{0.736362in}}%
\pgfpathlineto{\pgfqpoint{2.988762in}{0.736362in}}%
\pgfpathlineto{\pgfqpoint{2.990985in}{0.753486in}}%
\pgfpathlineto{\pgfqpoint{2.992097in}{0.753486in}}%
\pgfpathlineto{\pgfqpoint{2.994320in}{0.772610in}}%
\pgfpathlineto{\pgfqpoint{2.996543in}{0.772610in}}%
\pgfpathlineto{\pgfqpoint{2.998766in}{0.793220in}}%
\pgfpathlineto{\pgfqpoint{2.999878in}{0.793220in}}%
\pgfpathlineto{\pgfqpoint{3.002101in}{0.815470in}}%
\pgfpathlineto{\pgfqpoint{3.003212in}{0.815470in}}%
\pgfpathlineto{\pgfqpoint{3.005435in}{0.838295in}}%
\pgfpathlineto{\pgfqpoint{3.008770in}{0.862742in}}%
\pgfpathlineto{\pgfqpoint{3.009882in}{0.862742in}}%
\pgfpathlineto{\pgfqpoint{3.012105in}{0.887791in}}%
\pgfpathlineto{\pgfqpoint{3.013216in}{0.887791in}}%
\pgfpathlineto{\pgfqpoint{3.015439in}{0.913927in}}%
\pgfpathlineto{\pgfqpoint{3.016551in}{0.913927in}}%
\pgfpathlineto{\pgfqpoint{3.018774in}{0.940491in}}%
\pgfpathlineto{\pgfqpoint{3.019886in}{0.940491in}}%
\pgfpathlineto{\pgfqpoint{3.022109in}{0.967541in}}%
\pgfpathlineto{\pgfqpoint{3.023220in}{0.967541in}}%
\pgfpathlineto{\pgfqpoint{3.025443in}{0.995398in}}%
\pgfpathlineto{\pgfqpoint{3.026555in}{0.995398in}}%
\pgfpathlineto{\pgfqpoint{3.031001in}{1.051000in}}%
\pgfpathlineto{\pgfqpoint{3.033224in}{1.051000in}}%
\pgfpathlineto{\pgfqpoint{3.035447in}{1.078910in}}%
\pgfpathlineto{\pgfqpoint{3.036559in}{1.078910in}}%
\pgfpathlineto{\pgfqpoint{3.038782in}{1.106834in}}%
\pgfpathlineto{\pgfqpoint{3.041005in}{1.106834in}}%
\pgfpathlineto{\pgfqpoint{3.043228in}{1.134763in}}%
\pgfpathlineto{\pgfqpoint{3.045451in}{1.162441in}}%
\pgfpathlineto{\pgfqpoint{3.046563in}{1.162441in}}%
\pgfpathlineto{\pgfqpoint{3.048786in}{1.189269in}}%
\pgfpathlineto{\pgfqpoint{3.049897in}{1.189269in}}%
\pgfpathlineto{\pgfqpoint{3.052121in}{1.216309in}}%
\pgfpathlineto{\pgfqpoint{3.053232in}{1.216309in}}%
\pgfpathlineto{\pgfqpoint{3.055455in}{1.242461in}}%
\pgfpathlineto{\pgfqpoint{3.057678in}{1.242461in}}%
\pgfpathlineto{\pgfqpoint{3.059901in}{1.268999in}}%
\pgfpathlineto{\pgfqpoint{3.065459in}{1.320028in}}%
\pgfpathlineto{\pgfqpoint{3.066571in}{1.320028in}}%
\pgfpathlineto{\pgfqpoint{3.069905in}{1.344253in}}%
\pgfpathlineto{\pgfqpoint{3.073240in}{1.368188in}}%
\pgfpathlineto{\pgfqpoint{3.074351in}{1.368188in}}%
\pgfpathlineto{\pgfqpoint{3.076575in}{1.391577in}}%
\pgfpathlineto{\pgfqpoint{3.077686in}{1.391577in}}%
\pgfpathlineto{\pgfqpoint{3.079909in}{1.414590in}}%
\pgfpathlineto{\pgfqpoint{3.081021in}{1.414590in}}%
\pgfpathlineto{\pgfqpoint{3.083244in}{1.436554in}}%
\pgfpathlineto{\pgfqpoint{3.084355in}{1.436554in}}%
\pgfpathlineto{\pgfqpoint{3.086579in}{1.458292in}}%
\pgfpathlineto{\pgfqpoint{3.087690in}{1.458292in}}%
\pgfpathlineto{\pgfqpoint{3.091025in}{1.479271in}}%
\pgfpathlineto{\pgfqpoint{3.095471in}{1.519318in}}%
\pgfpathlineto{\pgfqpoint{3.097694in}{1.519318in}}%
\pgfpathlineto{\pgfqpoint{3.099917in}{1.538003in}}%
\pgfpathlineto{\pgfqpoint{3.101029in}{1.538003in}}%
\pgfpathlineto{\pgfqpoint{3.103252in}{1.556350in}}%
\pgfpathlineto{\pgfqpoint{3.104363in}{1.556350in}}%
\pgfpathlineto{\pgfqpoint{3.107698in}{1.574147in}}%
\pgfpathlineto{\pgfqpoint{3.111033in}{1.591340in}}%
\pgfpathlineto{\pgfqpoint{3.114367in}{1.607635in}}%
\pgfpathlineto{\pgfqpoint{3.117702in}{1.623418in}}%
\pgfpathlineto{\pgfqpoint{3.122148in}{1.645404in}}%
\pgfpathlineto{\pgfqpoint{3.125483in}{1.666197in}}%
\pgfpathlineto{\pgfqpoint{3.128817in}{1.666197in}}%
\pgfpathlineto{\pgfqpoint{3.131040in}{1.679396in}}%
\pgfpathlineto{\pgfqpoint{3.132152in}{1.679396in}}%
\pgfpathlineto{\pgfqpoint{3.134375in}{1.691586in}}%
\pgfpathlineto{\pgfqpoint{3.135487in}{1.691586in}}%
\pgfpathlineto{\pgfqpoint{3.137710in}{1.703559in}}%
\pgfpathlineto{\pgfqpoint{3.138821in}{1.703559in}}%
\pgfpathlineto{\pgfqpoint{3.141044in}{1.714333in}}%
\pgfpathlineto{\pgfqpoint{3.142156in}{1.714333in}}%
\pgfpathlineto{\pgfqpoint{3.144379in}{1.724446in}}%
\pgfpathlineto{\pgfqpoint{3.145491in}{1.724446in}}%
\pgfpathlineto{\pgfqpoint{3.147714in}{1.734294in}}%
\pgfpathlineto{\pgfqpoint{3.148825in}{1.734294in}}%
\pgfpathlineto{\pgfqpoint{3.151048in}{1.743267in}}%
\pgfpathlineto{\pgfqpoint{3.152160in}{1.743267in}}%
\pgfpathlineto{\pgfqpoint{3.154383in}{1.751810in}}%
\pgfpathlineto{\pgfqpoint{3.155495in}{1.751810in}}%
\pgfpathlineto{\pgfqpoint{3.157718in}{1.759349in}}%
\pgfpathlineto{\pgfqpoint{3.158829in}{1.759349in}}%
\pgfpathlineto{\pgfqpoint{3.161052in}{1.766313in}}%
\pgfpathlineto{\pgfqpoint{3.162164in}{1.766313in}}%
\pgfpathlineto{\pgfqpoint{3.165498in}{1.772649in}}%
\pgfpathlineto{\pgfqpoint{3.166610in}{1.772649in}}%
\pgfpathlineto{\pgfqpoint{3.168833in}{1.778595in}}%
\pgfpathlineto{\pgfqpoint{3.171056in}{1.783849in}}%
\pgfpathlineto{\pgfqpoint{3.172168in}{1.783849in}}%
\pgfpathlineto{\pgfqpoint{3.174391in}{1.788191in}}%
\pgfpathlineto{\pgfqpoint{3.175502in}{1.788191in}}%
\pgfpathlineto{\pgfqpoint{3.177726in}{1.791972in}}%
\pgfpathlineto{\pgfqpoint{3.179949in}{1.791972in}}%
\pgfpathlineto{\pgfqpoint{3.182172in}{1.795179in}}%
\pgfpathlineto{\pgfqpoint{3.183283in}{1.795179in}}%
\pgfpathlineto{\pgfqpoint{3.185506in}{1.797855in}}%
\pgfpathlineto{\pgfqpoint{3.186618in}{1.797855in}}%
\pgfpathlineto{\pgfqpoint{3.188841in}{1.799893in}}%
\pgfpathlineto{\pgfqpoint{3.189953in}{1.799893in}}%
\pgfpathlineto{\pgfqpoint{3.192176in}{1.801332in}}%
\pgfpathlineto{\pgfqpoint{3.198845in}{1.802274in}}%
\pgfpathlineto{\pgfqpoint{3.203291in}{1.801849in}}%
\pgfpathlineto{\pgfqpoint{3.205514in}{1.800723in}}%
\pgfpathlineto{\pgfqpoint{3.206626in}{1.800723in}}%
\pgfpathlineto{\pgfqpoint{3.208849in}{1.799163in}}%
\pgfpathlineto{\pgfqpoint{3.211072in}{1.799163in}}%
\pgfpathlineto{\pgfqpoint{3.213295in}{1.796942in}}%
\pgfpathlineto{\pgfqpoint{3.218853in}{1.790968in}}%
\pgfpathlineto{\pgfqpoint{3.219964in}{1.790968in}}%
\pgfpathlineto{\pgfqpoint{3.225522in}{1.782342in}}%
\pgfpathlineto{\pgfqpoint{3.227745in}{1.782342in}}%
\pgfpathlineto{\pgfqpoint{3.229968in}{1.777228in}}%
\pgfpathlineto{\pgfqpoint{3.233303in}{1.771606in}}%
\pgfpathlineto{\pgfqpoint{3.234415in}{1.771606in}}%
\pgfpathlineto{\pgfqpoint{3.236638in}{1.765371in}}%
\pgfpathlineto{\pgfqpoint{3.242195in}{1.751332in}}%
\pgfpathlineto{\pgfqpoint{3.244418in}{1.751332in}}%
\pgfpathlineto{\pgfqpoint{3.246642in}{1.743407in}}%
\pgfpathlineto{\pgfqpoint{3.247753in}{1.743407in}}%
\pgfpathlineto{\pgfqpoint{3.249976in}{1.734926in}}%
\pgfpathlineto{\pgfqpoint{3.251088in}{1.734926in}}%
\pgfpathlineto{\pgfqpoint{3.253311in}{1.726146in}}%
\pgfpathlineto{\pgfqpoint{3.254422in}{1.726146in}}%
\pgfpathlineto{\pgfqpoint{3.256646in}{1.716801in}}%
\pgfpathlineto{\pgfqpoint{3.257757in}{1.716801in}}%
\pgfpathlineto{\pgfqpoint{3.259980in}{1.706471in}}%
\pgfpathlineto{\pgfqpoint{3.261092in}{1.706471in}}%
\pgfpathlineto{\pgfqpoint{3.266649in}{1.685313in}}%
\pgfpathlineto{\pgfqpoint{3.267761in}{1.685313in}}%
\pgfpathlineto{\pgfqpoint{3.273319in}{1.661687in}}%
\pgfpathlineto{\pgfqpoint{3.275542in}{1.661687in}}%
\pgfpathlineto{\pgfqpoint{3.277765in}{1.648908in}}%
\pgfpathlineto{\pgfqpoint{3.283323in}{1.622766in}}%
\pgfpathlineto{\pgfqpoint{3.284434in}{1.622766in}}%
\pgfpathlineto{\pgfqpoint{3.289992in}{1.594160in}}%
\pgfpathlineto{\pgfqpoint{3.292215in}{1.594160in}}%
\pgfpathlineto{\pgfqpoint{3.297773in}{1.564498in}}%
\pgfpathlineto{\pgfqpoint{3.298884in}{1.564498in}}%
\pgfpathlineto{\pgfqpoint{3.301107in}{1.549005in}}%
\pgfpathlineto{\pgfqpoint{3.302219in}{1.549005in}}%
\pgfpathlineto{\pgfqpoint{3.304442in}{1.533140in}}%
\pgfpathlineto{\pgfqpoint{3.305554in}{1.533140in}}%
\pgfpathlineto{\pgfqpoint{3.311111in}{1.499826in}}%
\pgfpathlineto{\pgfqpoint{3.312223in}{1.499826in}}%
\pgfpathlineto{\pgfqpoint{3.317781in}{1.464981in}}%
\pgfpathlineto{\pgfqpoint{3.318892in}{1.464981in}}%
\pgfpathlineto{\pgfqpoint{3.321115in}{1.447121in}}%
\pgfpathlineto{\pgfqpoint{3.322227in}{1.447121in}}%
\pgfpathlineto{\pgfqpoint{3.327785in}{1.410707in}}%
\pgfpathlineto{\pgfqpoint{3.330008in}{1.410707in}}%
\pgfpathlineto{\pgfqpoint{3.332231in}{1.391872in}}%
\pgfpathlineto{\pgfqpoint{3.337789in}{1.353192in}}%
\pgfpathlineto{\pgfqpoint{3.340012in}{1.353192in}}%
\pgfpathlineto{\pgfqpoint{3.342235in}{1.333816in}}%
\pgfpathlineto{\pgfqpoint{3.343346in}{1.333816in}}%
\pgfpathlineto{\pgfqpoint{3.345569in}{1.314083in}}%
\pgfpathlineto{\pgfqpoint{3.346681in}{1.314083in}}%
\pgfpathlineto{\pgfqpoint{3.348904in}{1.294069in}}%
\pgfpathlineto{\pgfqpoint{3.350016in}{1.294069in}}%
\pgfpathlineto{\pgfqpoint{3.352239in}{1.273742in}}%
\pgfpathlineto{\pgfqpoint{3.353350in}{1.273742in}}%
\pgfpathlineto{\pgfqpoint{3.355573in}{1.253395in}}%
\pgfpathlineto{\pgfqpoint{3.358908in}{1.232532in}}%
\pgfpathlineto{\pgfqpoint{3.360020in}{1.232532in}}%
\pgfpathlineto{\pgfqpoint{3.362243in}{1.211731in}}%
\pgfpathlineto{\pgfqpoint{3.363354in}{1.211731in}}%
\pgfpathlineto{\pgfqpoint{3.365577in}{1.191147in}}%
\pgfpathlineto{\pgfqpoint{3.366689in}{1.191147in}}%
\pgfpathlineto{\pgfqpoint{3.368912in}{1.170081in}}%
\pgfpathlineto{\pgfqpoint{3.371135in}{1.170081in}}%
\pgfpathlineto{\pgfqpoint{3.373358in}{1.149015in}}%
\pgfpathlineto{\pgfqpoint{3.377804in}{1.117258in}}%
\pgfpathlineto{\pgfqpoint{3.382251in}{1.085343in}}%
\pgfpathlineto{\pgfqpoint{3.383362in}{1.085343in}}%
\pgfpathlineto{\pgfqpoint{3.385585in}{1.064054in}}%
\pgfpathlineto{\pgfqpoint{3.387808in}{1.064054in}}%
\pgfpathlineto{\pgfqpoint{3.390031in}{1.043133in}}%
\pgfpathlineto{\pgfqpoint{3.391143in}{1.043133in}}%
\pgfpathlineto{\pgfqpoint{3.393366in}{1.022033in}}%
\pgfpathlineto{\pgfqpoint{3.394478in}{1.022033in}}%
\pgfpathlineto{\pgfqpoint{3.396701in}{1.001174in}}%
\pgfpathlineto{\pgfqpoint{3.397812in}{1.001174in}}%
\pgfpathlineto{\pgfqpoint{3.400035in}{0.980368in}}%
\pgfpathlineto{\pgfqpoint{3.401147in}{0.980368in}}%
\pgfpathlineto{\pgfqpoint{3.405593in}{0.938666in}}%
\pgfpathlineto{\pgfqpoint{3.407816in}{0.938666in}}%
\pgfpathlineto{\pgfqpoint{3.410039in}{0.918383in}}%
\pgfpathlineto{\pgfqpoint{3.411151in}{0.918383in}}%
\pgfpathlineto{\pgfqpoint{3.413374in}{0.898036in}}%
\pgfpathlineto{\pgfqpoint{3.414485in}{0.898036in}}%
\pgfpathlineto{\pgfqpoint{3.416709in}{0.878117in}}%
\pgfpathlineto{\pgfqpoint{3.417820in}{0.878117in}}%
\pgfpathlineto{\pgfqpoint{3.420043in}{0.858707in}}%
\pgfpathlineto{\pgfqpoint{3.421155in}{0.858707in}}%
\pgfpathlineto{\pgfqpoint{3.425601in}{0.820024in}}%
\pgfpathlineto{\pgfqpoint{3.427824in}{0.820024in}}%
\pgfpathlineto{\pgfqpoint{3.430047in}{0.801386in}}%
\pgfpathlineto{\pgfqpoint{3.432270in}{0.801386in}}%
\pgfpathlineto{\pgfqpoint{3.434493in}{0.783242in}}%
\pgfpathlineto{\pgfqpoint{3.435605in}{0.783242in}}%
\pgfpathlineto{\pgfqpoint{3.437828in}{0.765397in}}%
\pgfpathlineto{\pgfqpoint{3.441163in}{0.748074in}}%
\pgfpathlineto{\pgfqpoint{3.444497in}{0.731287in}}%
\pgfpathlineto{\pgfqpoint{3.445609in}{0.731287in}}%
\pgfpathlineto{\pgfqpoint{3.447832in}{0.714937in}}%
\pgfpathlineto{\pgfqpoint{3.448943in}{0.714937in}}%
\pgfpathlineto{\pgfqpoint{3.451167in}{0.699370in}}%
\pgfpathlineto{\pgfqpoint{3.452278in}{0.699370in}}%
\pgfpathlineto{\pgfqpoint{3.454501in}{0.684209in}}%
\pgfpathlineto{\pgfqpoint{3.457836in}{0.669812in}}%
\pgfpathlineto{\pgfqpoint{3.458947in}{0.669812in}}%
\pgfpathlineto{\pgfqpoint{3.461171in}{0.656004in}}%
\pgfpathlineto{\pgfqpoint{3.462282in}{0.656004in}}%
\pgfpathlineto{\pgfqpoint{3.466728in}{0.631304in}}%
\pgfpathlineto{\pgfqpoint{3.468951in}{0.631304in}}%
\pgfpathlineto{\pgfqpoint{3.471174in}{0.620461in}}%
\pgfpathlineto{\pgfqpoint{3.472286in}{0.620461in}}%
\pgfpathlineto{\pgfqpoint{3.474509in}{0.610382in}}%
\pgfpathlineto{\pgfqpoint{3.476732in}{0.610382in}}%
\pgfpathlineto{\pgfqpoint{3.478955in}{0.601636in}}%
\pgfpathlineto{\pgfqpoint{3.481178in}{0.593920in}}%
\pgfpathlineto{\pgfqpoint{3.482290in}{0.593920in}}%
\pgfpathlineto{\pgfqpoint{3.486736in}{0.582240in}}%
\pgfpathlineto{\pgfqpoint{3.488959in}{0.582240in}}%
\pgfpathlineto{\pgfqpoint{3.491182in}{0.578194in}}%
\pgfpathlineto{\pgfqpoint{3.492294in}{0.578194in}}%
\pgfpathlineto{\pgfqpoint{3.495629in}{0.575718in}}%
\pgfpathlineto{\pgfqpoint{3.496740in}{0.575718in}}%
\pgfpathlineto{\pgfqpoint{3.498963in}{0.574510in}}%
\pgfpathlineto{\pgfqpoint{3.502298in}{0.574930in}}%
\pgfpathlineto{\pgfqpoint{3.505632in}{0.576714in}}%
\pgfpathlineto{\pgfqpoint{3.506744in}{0.576714in}}%
\pgfpathlineto{\pgfqpoint{3.508967in}{0.580279in}}%
\pgfpathlineto{\pgfqpoint{3.510079in}{0.580279in}}%
\pgfpathlineto{\pgfqpoint{3.512302in}{0.585254in}}%
\pgfpathlineto{\pgfqpoint{3.513413in}{0.585254in}}%
\pgfpathlineto{\pgfqpoint{3.515636in}{0.592025in}}%
\pgfpathlineto{\pgfqpoint{3.516748in}{0.592025in}}%
\pgfpathlineto{\pgfqpoint{3.518971in}{0.600546in}}%
\pgfpathlineto{\pgfqpoint{3.520083in}{0.600546in}}%
\pgfpathlineto{\pgfqpoint{3.522306in}{0.610642in}}%
\pgfpathlineto{\pgfqpoint{3.523417in}{0.610642in}}%
\pgfpathlineto{\pgfqpoint{3.528975in}{0.635771in}}%
\pgfpathlineto{\pgfqpoint{3.530087in}{0.635771in}}%
\pgfpathlineto{\pgfqpoint{3.532310in}{0.651071in}}%
\pgfpathlineto{\pgfqpoint{3.533421in}{0.651071in}}%
\pgfpathlineto{\pgfqpoint{3.535644in}{0.668017in}}%
\pgfpathlineto{\pgfqpoint{3.536756in}{0.668017in}}%
\pgfpathlineto{\pgfqpoint{3.538979in}{0.686390in}}%
\pgfpathlineto{\pgfqpoint{3.541202in}{0.686390in}}%
\pgfpathlineto{\pgfqpoint{3.543425in}{0.706757in}}%
\pgfpathlineto{\pgfqpoint{3.548983in}{0.750932in}}%
\pgfpathlineto{\pgfqpoint{3.550094in}{0.750932in}}%
\pgfpathlineto{\pgfqpoint{3.555652in}{0.799991in}}%
\pgfpathlineto{\pgfqpoint{3.557875in}{0.799991in}}%
\pgfpathlineto{\pgfqpoint{3.560098in}{0.826688in}}%
\pgfpathlineto{\pgfqpoint{3.561210in}{0.826688in}}%
\pgfpathlineto{\pgfqpoint{3.563433in}{0.853511in}}%
\pgfpathlineto{\pgfqpoint{3.564545in}{0.853511in}}%
\pgfpathlineto{\pgfqpoint{3.566768in}{0.881535in}}%
\pgfpathlineto{\pgfqpoint{3.567879in}{0.881535in}}%
\pgfpathlineto{\pgfqpoint{3.570102in}{0.909982in}}%
\pgfpathlineto{\pgfqpoint{3.571214in}{0.909982in}}%
\pgfpathlineto{\pgfqpoint{3.576772in}{0.971661in}}%
\pgfpathlineto{\pgfqpoint{3.577883in}{0.971661in}}%
\pgfpathlineto{\pgfqpoint{3.580106in}{0.999561in}}%
\pgfpathlineto{\pgfqpoint{3.581218in}{0.999561in}}%
\pgfpathlineto{\pgfqpoint{3.583441in}{1.029760in}}%
\pgfpathlineto{\pgfqpoint{3.584552in}{1.029760in}}%
\pgfpathlineto{\pgfqpoint{3.586776in}{1.060219in}}%
\pgfpathlineto{\pgfqpoint{3.587887in}{1.060219in}}%
\pgfpathlineto{\pgfqpoint{3.593445in}{1.122023in}}%
\pgfpathlineto{\pgfqpoint{3.594556in}{1.122023in}}%
\pgfpathlineto{\pgfqpoint{3.600114in}{1.184309in}}%
\pgfpathlineto{\pgfqpoint{3.601226in}{1.184309in}}%
\pgfpathlineto{\pgfqpoint{3.604560in}{1.214976in}}%
\pgfpathlineto{\pgfqpoint{3.605672in}{1.214976in}}%
\pgfpathlineto{\pgfqpoint{3.607895in}{1.246093in}}%
\pgfpathlineto{\pgfqpoint{3.609007in}{1.246093in}}%
\pgfpathlineto{\pgfqpoint{3.611230in}{1.276615in}}%
\pgfpathlineto{\pgfqpoint{3.616787in}{1.337849in}}%
\pgfpathlineto{\pgfqpoint{3.619010in}{1.337849in}}%
\pgfpathlineto{\pgfqpoint{3.621234in}{1.367690in}}%
\pgfpathlineto{\pgfqpoint{3.622345in}{1.367690in}}%
\pgfpathlineto{\pgfqpoint{3.624568in}{1.397768in}}%
\pgfpathlineto{\pgfqpoint{3.625680in}{1.397768in}}%
\pgfpathlineto{\pgfqpoint{3.627903in}{1.427330in}}%
\pgfpathlineto{\pgfqpoint{3.629014in}{1.427330in}}%
\pgfpathlineto{\pgfqpoint{3.631238in}{1.456539in}}%
\pgfpathlineto{\pgfqpoint{3.632349in}{1.456539in}}%
\pgfpathlineto{\pgfqpoint{3.635684in}{1.485028in}}%
\pgfpathlineto{\pgfqpoint{3.639018in}{1.513401in}}%
\pgfpathlineto{\pgfqpoint{3.643465in}{1.569182in}}%
\pgfpathlineto{\pgfqpoint{3.645688in}{1.569182in}}%
\pgfpathlineto{\pgfqpoint{3.647911in}{1.596425in}}%
\pgfpathlineto{\pgfqpoint{3.650134in}{1.596425in}}%
\pgfpathlineto{\pgfqpoint{3.652357in}{1.623596in}}%
\pgfpathlineto{\pgfqpoint{3.655692in}{1.650096in}}%
\pgfpathlineto{\pgfqpoint{3.659026in}{1.676190in}}%
\pgfpathlineto{\pgfqpoint{3.662361in}{1.701487in}}%
\pgfpathlineto{\pgfqpoint{3.663472in}{1.701487in}}%
\pgfpathlineto{\pgfqpoint{3.665696in}{1.726803in}}%
\pgfpathlineto{\pgfqpoint{3.666807in}{1.726803in}}%
\pgfpathlineto{\pgfqpoint{3.669030in}{1.751825in}}%
\pgfpathlineto{\pgfqpoint{3.670142in}{1.751825in}}%
\pgfpathlineto{\pgfqpoint{3.672365in}{1.775871in}}%
\pgfpathlineto{\pgfqpoint{3.673476in}{1.775871in}}%
\pgfpathlineto{\pgfqpoint{3.675699in}{1.800323in}}%
\pgfpathlineto{\pgfqpoint{3.676811in}{1.800323in}}%
\pgfpathlineto{\pgfqpoint{3.679034in}{1.823446in}}%
\pgfpathlineto{\pgfqpoint{3.680146in}{1.823446in}}%
\pgfpathlineto{\pgfqpoint{3.684592in}{1.869095in}}%
\pgfpathlineto{\pgfqpoint{3.686815in}{1.869095in}}%
\pgfpathlineto{\pgfqpoint{3.689038in}{1.891006in}}%
\pgfpathlineto{\pgfqpoint{3.690150in}{1.891006in}}%
\pgfpathlineto{\pgfqpoint{3.692373in}{1.912671in}}%
\pgfpathlineto{\pgfqpoint{3.694596in}{1.912671in}}%
\pgfpathlineto{\pgfqpoint{3.696819in}{1.933815in}}%
\pgfpathlineto{\pgfqpoint{3.699042in}{1.954925in}}%
\pgfpathlineto{\pgfqpoint{3.700154in}{1.954925in}}%
\pgfpathlineto{\pgfqpoint{3.702377in}{1.975233in}}%
\pgfpathlineto{\pgfqpoint{3.703488in}{1.975233in}}%
\pgfpathlineto{\pgfqpoint{3.705711in}{1.995044in}}%
\pgfpathlineto{\pgfqpoint{3.706823in}{1.995044in}}%
\pgfpathlineto{\pgfqpoint{3.709046in}{2.014347in}}%
\pgfpathlineto{\pgfqpoint{3.711269in}{2.014347in}}%
\pgfpathlineto{\pgfqpoint{3.713492in}{2.033308in}}%
\pgfpathlineto{\pgfqpoint{3.717938in}{2.060865in}}%
\pgfpathlineto{\pgfqpoint{3.720161in}{2.070022in}}%
\pgfpathlineto{\pgfqpoint{3.723496in}{2.087239in}}%
\pgfpathlineto{\pgfqpoint{3.724608in}{2.087239in}}%
\pgfpathlineto{\pgfqpoint{3.726831in}{2.104572in}}%
\pgfpathlineto{\pgfqpoint{3.727942in}{2.104572in}}%
\pgfpathlineto{\pgfqpoint{3.730165in}{2.121369in}}%
\pgfpathlineto{\pgfqpoint{3.731277in}{2.121369in}}%
\pgfpathlineto{\pgfqpoint{3.733500in}{2.137456in}}%
\pgfpathlineto{\pgfqpoint{3.734612in}{2.137456in}}%
\pgfpathlineto{\pgfqpoint{3.736835in}{2.153379in}}%
\pgfpathlineto{\pgfqpoint{3.737946in}{2.153379in}}%
\pgfpathlineto{\pgfqpoint{3.740169in}{2.168428in}}%
\pgfpathlineto{\pgfqpoint{3.742392in}{2.168428in}}%
\pgfpathlineto{\pgfqpoint{3.744616in}{2.183795in}}%
\pgfpathlineto{\pgfqpoint{3.746839in}{2.198163in}}%
\pgfpathlineto{\pgfqpoint{3.747950in}{2.198163in}}%
\pgfpathlineto{\pgfqpoint{3.750173in}{2.212458in}}%
\pgfpathlineto{\pgfqpoint{3.751285in}{2.212458in}}%
\pgfpathlineto{\pgfqpoint{3.753508in}{2.225938in}}%
\pgfpathlineto{\pgfqpoint{3.754619in}{2.225938in}}%
\pgfpathlineto{\pgfqpoint{3.756843in}{2.238866in}}%
\pgfpathlineto{\pgfqpoint{3.759066in}{2.238866in}}%
\pgfpathlineto{\pgfqpoint{3.761289in}{2.251578in}}%
\pgfpathlineto{\pgfqpoint{3.762400in}{2.251578in}}%
\pgfpathlineto{\pgfqpoint{3.764623in}{2.264004in}}%
\pgfpathlineto{\pgfqpoint{3.767958in}{2.276082in}}%
\pgfpathlineto{\pgfqpoint{3.771293in}{2.287842in}}%
\pgfpathlineto{\pgfqpoint{3.774627in}{2.298781in}}%
\pgfpathlineto{\pgfqpoint{3.775739in}{2.298781in}}%
\pgfpathlineto{\pgfqpoint{3.777962in}{2.309701in}}%
\pgfpathlineto{\pgfqpoint{3.779074in}{2.309701in}}%
\pgfpathlineto{\pgfqpoint{3.781297in}{2.320021in}}%
\pgfpathlineto{\pgfqpoint{3.782408in}{2.320021in}}%
\pgfpathlineto{\pgfqpoint{3.784631in}{2.330018in}}%
\pgfpathlineto{\pgfqpoint{3.785743in}{2.330018in}}%
\pgfpathlineto{\pgfqpoint{3.787966in}{2.339586in}}%
\pgfpathlineto{\pgfqpoint{3.789077in}{2.339586in}}%
\pgfpathlineto{\pgfqpoint{3.791301in}{2.348800in}}%
\pgfpathlineto{\pgfqpoint{3.792412in}{2.348800in}}%
\pgfpathlineto{\pgfqpoint{3.796858in}{2.361611in}}%
\pgfpathlineto{\pgfqpoint{3.801305in}{2.374049in}}%
\pgfpathlineto{\pgfqpoint{3.802416in}{2.374049in}}%
\pgfpathlineto{\pgfqpoint{3.804639in}{2.381501in}}%
\pgfpathlineto{\pgfqpoint{3.805751in}{2.381501in}}%
\pgfpathlineto{\pgfqpoint{3.809085in}{2.388895in}}%
\pgfpathlineto{\pgfqpoint{3.812420in}{2.395854in}}%
\pgfpathlineto{\pgfqpoint{3.816866in}{2.408710in}}%
\pgfpathlineto{\pgfqpoint{3.819089in}{2.408710in}}%
\pgfpathlineto{\pgfqpoint{3.821312in}{2.414443in}}%
\pgfpathlineto{\pgfqpoint{3.823535in}{2.414443in}}%
\pgfpathlineto{\pgfqpoint{3.825759in}{2.419862in}}%
\pgfpathlineto{\pgfqpoint{3.826870in}{2.419862in}}%
\pgfpathlineto{\pgfqpoint{3.829093in}{2.424986in}}%
\pgfpathlineto{\pgfqpoint{3.830205in}{2.424986in}}%
\pgfpathlineto{\pgfqpoint{3.832428in}{2.429994in}}%
\pgfpathlineto{\pgfqpoint{3.833539in}{2.429994in}}%
\pgfpathlineto{\pgfqpoint{3.835763in}{2.434312in}}%
\pgfpathlineto{\pgfqpoint{3.836874in}{2.434312in}}%
\pgfpathlineto{\pgfqpoint{3.839097in}{2.438528in}}%
\pgfpathlineto{\pgfqpoint{3.840209in}{2.438528in}}%
\pgfpathlineto{\pgfqpoint{3.842432in}{2.442285in}}%
\pgfpathlineto{\pgfqpoint{3.843543in}{2.442285in}}%
\pgfpathlineto{\pgfqpoint{3.845766in}{2.445801in}}%
\pgfpathlineto{\pgfqpoint{3.846878in}{2.445801in}}%
\pgfpathlineto{\pgfqpoint{3.849101in}{2.448829in}}%
\pgfpathlineto{\pgfqpoint{3.850213in}{2.448829in}}%
\pgfpathlineto{\pgfqpoint{3.852436in}{2.451616in}}%
\pgfpathlineto{\pgfqpoint{3.853547in}{2.451616in}}%
\pgfpathlineto{\pgfqpoint{3.859105in}{2.456160in}}%
\pgfpathlineto{\pgfqpoint{3.860217in}{2.456160in}}%
\pgfpathlineto{\pgfqpoint{3.862440in}{2.457860in}}%
\pgfpathlineto{\pgfqpoint{3.863551in}{2.457860in}}%
\pgfpathlineto{\pgfqpoint{3.865774in}{2.459333in}}%
\pgfpathlineto{\pgfqpoint{3.866886in}{2.459333in}}%
\pgfpathlineto{\pgfqpoint{3.869109in}{2.460517in}}%
\pgfpathlineto{\pgfqpoint{3.874667in}{2.461328in}}%
\pgfpathlineto{\pgfqpoint{3.878001in}{2.462033in}}%
\pgfpathlineto{\pgfqpoint{3.884671in}{2.461961in}}%
\pgfpathlineto{\pgfqpoint{3.889117in}{2.460512in}}%
\pgfpathlineto{\pgfqpoint{3.894675in}{2.459618in}}%
\pgfpathlineto{\pgfqpoint{3.896898in}{2.458295in}}%
\pgfpathlineto{\pgfqpoint{3.900232in}{2.456754in}}%
\pgfpathlineto{\pgfqpoint{3.902455in}{2.455718in}}%
\pgfpathlineto{\pgfqpoint{3.906902in}{2.452277in}}%
\pgfpathlineto{\pgfqpoint{3.908013in}{2.452277in}}%
\pgfpathlineto{\pgfqpoint{3.910236in}{2.449737in}}%
\pgfpathlineto{\pgfqpoint{3.911348in}{2.449737in}}%
\pgfpathlineto{\pgfqpoint{3.913571in}{2.446868in}}%
\pgfpathlineto{\pgfqpoint{3.914683in}{2.446868in}}%
\pgfpathlineto{\pgfqpoint{3.916906in}{2.443691in}}%
\pgfpathlineto{\pgfqpoint{3.919129in}{2.443691in}}%
\pgfpathlineto{\pgfqpoint{3.921352in}{2.440204in}}%
\pgfpathlineto{\pgfqpoint{3.926910in}{2.432235in}}%
\pgfpathlineto{\pgfqpoint{3.928021in}{2.432235in}}%
\pgfpathlineto{\pgfqpoint{3.931356in}{2.427961in}}%
\pgfpathlineto{\pgfqpoint{3.932467in}{2.427961in}}%
\pgfpathlineto{\pgfqpoint{3.934690in}{2.423112in}}%
\pgfpathlineto{\pgfqpoint{3.935802in}{2.423112in}}%
\pgfpathlineto{\pgfqpoint{3.938025in}{2.418312in}}%
\pgfpathlineto{\pgfqpoint{3.939137in}{2.418312in}}%
\pgfpathlineto{\pgfqpoint{3.941360in}{2.412859in}}%
\pgfpathlineto{\pgfqpoint{3.942471in}{2.412859in}}%
\pgfpathlineto{\pgfqpoint{3.944694in}{2.407344in}}%
\pgfpathlineto{\pgfqpoint{3.948029in}{2.401355in}}%
\pgfpathlineto{\pgfqpoint{3.949141in}{2.401355in}}%
\pgfpathlineto{\pgfqpoint{3.951364in}{2.394946in}}%
\pgfpathlineto{\pgfqpoint{3.952475in}{2.394946in}}%
\pgfpathlineto{\pgfqpoint{3.954698in}{2.388455in}}%
\pgfpathlineto{\pgfqpoint{3.955810in}{2.388455in}}%
\pgfpathlineto{\pgfqpoint{3.958033in}{2.381781in}}%
\pgfpathlineto{\pgfqpoint{3.959144in}{2.381781in}}%
\pgfpathlineto{\pgfqpoint{3.961368in}{2.374527in}}%
\pgfpathlineto{\pgfqpoint{3.962479in}{2.374527in}}%
\pgfpathlineto{\pgfqpoint{3.964702in}{2.367143in}}%
\pgfpathlineto{\pgfqpoint{3.965814in}{2.367143in}}%
\pgfpathlineto{\pgfqpoint{3.971372in}{2.351418in}}%
\pgfpathlineto{\pgfqpoint{3.972483in}{2.351418in}}%
\pgfpathlineto{\pgfqpoint{3.978041in}{2.334693in}}%
\pgfpathlineto{\pgfqpoint{3.979152in}{2.334693in}}%
\pgfpathlineto{\pgfqpoint{3.982487in}{2.325817in}}%
\pgfpathlineto{\pgfqpoint{3.983599in}{2.325817in}}%
\pgfpathlineto{\pgfqpoint{3.985822in}{2.316505in}}%
\pgfpathlineto{\pgfqpoint{3.991379in}{2.297163in}}%
\pgfpathlineto{\pgfqpoint{3.993602in}{2.297163in}}%
\pgfpathlineto{\pgfqpoint{3.994714in}{2.241228in}}%
\pgfpathlineto{\pgfqpoint{3.996937in}{2.241228in}}%
\pgfpathlineto{\pgfqpoint{4.000272in}{2.228584in}}%
\pgfpathlineto{\pgfqpoint{4.002495in}{2.222547in}}%
\pgfpathlineto{\pgfqpoint{4.003606in}{2.222547in}}%
\pgfpathlineto{\pgfqpoint{4.005830in}{2.214318in}}%
\pgfpathlineto{\pgfqpoint{4.006941in}{2.214318in}}%
\pgfpathlineto{\pgfqpoint{4.009164in}{2.190368in}}%
\pgfpathlineto{\pgfqpoint{4.010276in}{2.190368in}}%
\pgfpathlineto{\pgfqpoint{4.012499in}{2.178782in}}%
\pgfpathlineto{\pgfqpoint{4.013610in}{2.178782in}}%
\pgfpathlineto{\pgfqpoint{4.018057in}{2.155847in}}%
\pgfpathlineto{\pgfqpoint{4.020280in}{2.148250in}}%
\pgfpathlineto{\pgfqpoint{4.024726in}{2.137654in}}%
\pgfpathlineto{\pgfqpoint{4.026949in}{2.137654in}}%
\pgfpathlineto{\pgfqpoint{4.029172in}{2.130777in}}%
\pgfpathlineto{\pgfqpoint{4.031395in}{2.130777in}}%
\pgfpathlineto{\pgfqpoint{4.033618in}{2.123765in}}%
\pgfpathlineto{\pgfqpoint{4.034730in}{2.123765in}}%
\pgfpathlineto{\pgfqpoint{4.036953in}{2.125180in}}%
\pgfpathlineto{\pgfqpoint{4.038064in}{2.125180in}}%
\pgfpathlineto{\pgfqpoint{4.040288in}{2.114231in}}%
\pgfpathlineto{\pgfqpoint{4.042511in}{2.110783in}}%
\pgfpathlineto{\pgfqpoint{4.044734in}{2.110783in}}%
\pgfpathlineto{\pgfqpoint{4.046957in}{2.100873in}}%
\pgfpathlineto{\pgfqpoint{4.048068in}{2.100873in}}%
\pgfpathlineto{\pgfqpoint{4.050291in}{2.086191in}}%
\pgfpathlineto{\pgfqpoint{4.051403in}{2.086191in}}%
\pgfpathlineto{\pgfqpoint{4.053626in}{2.071147in}}%
\pgfpathlineto{\pgfqpoint{4.054738in}{2.071147in}}%
\pgfpathlineto{\pgfqpoint{4.056961in}{2.056132in}}%
\pgfpathlineto{\pgfqpoint{4.058072in}{2.056132in}}%
\pgfpathlineto{\pgfqpoint{4.060295in}{2.040625in}}%
\pgfpathlineto{\pgfqpoint{4.061407in}{2.040625in}}%
\pgfpathlineto{\pgfqpoint{4.063630in}{2.024610in}}%
\pgfpathlineto{\pgfqpoint{4.064742in}{2.024610in}}%
\pgfpathlineto{\pgfqpoint{4.066965in}{2.008663in}}%
\pgfpathlineto{\pgfqpoint{4.068076in}{2.008663in}}%
\pgfpathlineto{\pgfqpoint{4.070299in}{1.992537in}}%
\pgfpathlineto{\pgfqpoint{4.071411in}{1.992537in}}%
\pgfpathlineto{\pgfqpoint{4.073634in}{1.976006in}}%
\pgfpathlineto{\pgfqpoint{4.074746in}{1.976006in}}%
\pgfpathlineto{\pgfqpoint{4.076969in}{1.959469in}}%
\pgfpathlineto{\pgfqpoint{4.079192in}{1.959469in}}%
\pgfpathlineto{\pgfqpoint{4.081415in}{1.942470in}}%
\pgfpathlineto{\pgfqpoint{4.082526in}{1.942470in}}%
\pgfpathlineto{\pgfqpoint{4.084749in}{1.925136in}}%
\pgfpathlineto{\pgfqpoint{4.086973in}{1.907852in}}%
\pgfpathlineto{\pgfqpoint{4.088084in}{1.907852in}}%
\pgfpathlineto{\pgfqpoint{4.093642in}{1.872205in}}%
\pgfpathlineto{\pgfqpoint{4.095865in}{1.872205in}}%
\pgfpathlineto{\pgfqpoint{4.098088in}{1.854297in}}%
\pgfpathlineto{\pgfqpoint{4.099200in}{1.854297in}}%
\pgfpathlineto{\pgfqpoint{4.101423in}{1.835993in}}%
\pgfpathlineto{\pgfqpoint{4.102534in}{1.835993in}}%
\pgfpathlineto{\pgfqpoint{4.104757in}{1.817342in}}%
\pgfpathlineto{\pgfqpoint{4.105869in}{1.817342in}}%
\pgfpathlineto{\pgfqpoint{4.108092in}{1.798825in}}%
\pgfpathlineto{\pgfqpoint{4.109204in}{1.798825in}}%
\pgfpathlineto{\pgfqpoint{4.111427in}{1.779763in}}%
\pgfpathlineto{\pgfqpoint{4.112538in}{1.779763in}}%
\pgfpathlineto{\pgfqpoint{4.114761in}{1.760836in}}%
\pgfpathlineto{\pgfqpoint{4.115873in}{1.760836in}}%
\pgfpathlineto{\pgfqpoint{4.118096in}{1.741576in}}%
\pgfpathlineto{\pgfqpoint{4.119208in}{1.741576in}}%
\pgfpathlineto{\pgfqpoint{4.121431in}{1.722553in}}%
\pgfpathlineto{\pgfqpoint{4.122542in}{1.722553in}}%
\pgfpathlineto{\pgfqpoint{4.124765in}{1.703008in}}%
\pgfpathlineto{\pgfqpoint{4.125877in}{1.703008in}}%
\pgfpathlineto{\pgfqpoint{4.128100in}{1.683434in}}%
\pgfpathlineto{\pgfqpoint{4.129211in}{1.683434in}}%
\pgfpathlineto{\pgfqpoint{4.132546in}{1.663227in}}%
\pgfpathlineto{\pgfqpoint{4.138104in}{1.623307in}}%
\pgfpathlineto{\pgfqpoint{4.139215in}{1.623307in}}%
\pgfpathlineto{\pgfqpoint{4.141438in}{1.603013in}}%
\pgfpathlineto{\pgfqpoint{4.142550in}{1.603013in}}%
\pgfpathlineto{\pgfqpoint{4.145885in}{1.583082in}}%
\pgfpathlineto{\pgfqpoint{4.146996in}{1.583082in}}%
\pgfpathlineto{\pgfqpoint{4.149219in}{1.562532in}}%
\pgfpathlineto{\pgfqpoint{4.150331in}{1.562532in}}%
\pgfpathlineto{\pgfqpoint{4.152554in}{1.541891in}}%
\pgfpathlineto{\pgfqpoint{4.158112in}{1.500531in}}%
\pgfpathlineto{\pgfqpoint{4.160335in}{1.500531in}}%
\pgfpathlineto{\pgfqpoint{4.165893in}{1.459021in}}%
\pgfpathlineto{\pgfqpoint{4.167004in}{1.459021in}}%
\pgfpathlineto{\pgfqpoint{4.171450in}{1.417193in}}%
\pgfpathlineto{\pgfqpoint{4.173673in}{1.417193in}}%
\pgfpathlineto{\pgfqpoint{4.175897in}{1.396411in}}%
\pgfpathlineto{\pgfqpoint{4.177008in}{1.396411in}}%
\pgfpathlineto{\pgfqpoint{4.180343in}{1.375205in}}%
\pgfpathlineto{\pgfqpoint{4.185900in}{1.333826in}}%
\pgfpathlineto{\pgfqpoint{4.187012in}{1.333826in}}%
\pgfpathlineto{\pgfqpoint{4.189235in}{1.312837in}}%
\pgfpathlineto{\pgfqpoint{4.190347in}{1.312837in}}%
\pgfpathlineto{\pgfqpoint{4.192570in}{1.291944in}}%
\pgfpathlineto{\pgfqpoint{4.193681in}{1.291944in}}%
\pgfpathlineto{\pgfqpoint{4.195904in}{1.270776in}}%
\pgfpathlineto{\pgfqpoint{4.198127in}{1.270776in}}%
\pgfpathlineto{\pgfqpoint{4.200351in}{1.250212in}}%
\pgfpathlineto{\pgfqpoint{4.201462in}{1.250212in}}%
\pgfpathlineto{\pgfqpoint{4.203685in}{1.229687in}}%
\pgfpathlineto{\pgfqpoint{4.204797in}{1.229687in}}%
\pgfpathlineto{\pgfqpoint{4.207020in}{1.208766in}}%
\pgfpathlineto{\pgfqpoint{4.208131in}{1.208766in}}%
\pgfpathlineto{\pgfqpoint{4.210355in}{1.188100in}}%
\pgfpathlineto{\pgfqpoint{4.211466in}{1.188100in}}%
\pgfpathlineto{\pgfqpoint{4.213689in}{1.167777in}}%
\pgfpathlineto{\pgfqpoint{4.214801in}{1.167777in}}%
\pgfpathlineto{\pgfqpoint{4.217024in}{1.147228in}}%
\pgfpathlineto{\pgfqpoint{4.218135in}{1.147228in}}%
\pgfpathlineto{\pgfqpoint{4.220358in}{1.127002in}}%
\pgfpathlineto{\pgfqpoint{4.221470in}{1.127002in}}%
\pgfpathlineto{\pgfqpoint{4.223693in}{1.107046in}}%
\pgfpathlineto{\pgfqpoint{4.224805in}{1.107046in}}%
\pgfpathlineto{\pgfqpoint{4.227028in}{1.086632in}}%
\pgfpathlineto{\pgfqpoint{4.228139in}{1.086632in}}%
\pgfpathlineto{\pgfqpoint{4.232586in}{1.047054in}}%
\pgfpathlineto{\pgfqpoint{4.234809in}{1.047054in}}%
\pgfpathlineto{\pgfqpoint{4.237032in}{1.027538in}}%
\pgfpathlineto{\pgfqpoint{4.238143in}{1.027538in}}%
\pgfpathlineto{\pgfqpoint{4.240366in}{1.008297in}}%
\pgfpathlineto{\pgfqpoint{4.241478in}{1.008297in}}%
\pgfpathlineto{\pgfqpoint{4.243701in}{0.988767in}}%
\pgfpathlineto{\pgfqpoint{4.245924in}{0.988767in}}%
\pgfpathlineto{\pgfqpoint{4.248147in}{0.970168in}}%
\pgfpathlineto{\pgfqpoint{4.253705in}{0.932845in}}%
\pgfpathlineto{\pgfqpoint{4.254816in}{0.932845in}}%
\pgfpathlineto{\pgfqpoint{4.258151in}{0.914661in}}%
\pgfpathlineto{\pgfqpoint{4.259263in}{0.914661in}}%
\pgfpathlineto{\pgfqpoint{4.261486in}{0.896552in}}%
\pgfpathlineto{\pgfqpoint{4.262597in}{0.896552in}}%
\pgfpathlineto{\pgfqpoint{4.264820in}{0.878488in}}%
\pgfpathlineto{\pgfqpoint{4.270378in}{0.843841in}}%
\pgfpathlineto{\pgfqpoint{4.272601in}{0.843841in}}%
\pgfpathlineto{\pgfqpoint{4.274824in}{0.826520in}}%
\pgfpathlineto{\pgfqpoint{4.275936in}{0.826520in}}%
\pgfpathlineto{\pgfqpoint{4.278159in}{0.810122in}}%
\pgfpathlineto{\pgfqpoint{4.279271in}{0.810122in}}%
\pgfpathlineto{\pgfqpoint{4.281494in}{0.793767in}}%
\pgfpathlineto{\pgfqpoint{4.282605in}{0.793767in}}%
\pgfpathlineto{\pgfqpoint{4.284828in}{0.777447in}}%
\pgfpathlineto{\pgfqpoint{4.285940in}{0.777447in}}%
\pgfpathlineto{\pgfqpoint{4.288163in}{0.761509in}}%
\pgfpathlineto{\pgfqpoint{4.289275in}{0.761509in}}%
\pgfpathlineto{\pgfqpoint{4.292609in}{0.745930in}}%
\pgfpathlineto{\pgfqpoint{4.297055in}{0.723209in}}%
\pgfpathlineto{\pgfqpoint{4.299278in}{0.715685in}}%
\pgfpathlineto{\pgfqpoint{4.303725in}{0.686706in}}%
\pgfpathlineto{\pgfqpoint{4.305948in}{0.686706in}}%
\pgfpathlineto{\pgfqpoint{4.310394in}{0.666172in}}%
\pgfpathlineto{\pgfqpoint{4.312617in}{0.659367in}}%
\pgfpathlineto{\pgfqpoint{4.315952in}{0.646017in}}%
\pgfpathlineto{\pgfqpoint{4.319286in}{0.633261in}}%
\pgfpathlineto{\pgfqpoint{4.320398in}{0.633261in}}%
\pgfpathlineto{\pgfqpoint{4.322621in}{0.620776in}}%
\pgfpathlineto{\pgfqpoint{4.323733in}{0.620776in}}%
\pgfpathlineto{\pgfqpoint{4.325956in}{0.608757in}}%
\pgfpathlineto{\pgfqpoint{4.327067in}{0.608757in}}%
\pgfpathlineto{\pgfqpoint{4.329290in}{0.597634in}}%
\pgfpathlineto{\pgfqpoint{4.330402in}{0.597634in}}%
\pgfpathlineto{\pgfqpoint{4.332625in}{0.586754in}}%
\pgfpathlineto{\pgfqpoint{4.333736in}{0.586754in}}%
\pgfpathlineto{\pgfqpoint{4.335960in}{0.576262in}}%
\pgfpathlineto{\pgfqpoint{4.337071in}{0.576262in}}%
\pgfpathlineto{\pgfqpoint{4.339294in}{0.566374in}}%
\pgfpathlineto{\pgfqpoint{4.340406in}{0.566374in}}%
\pgfpathlineto{\pgfqpoint{4.342629in}{0.557038in}}%
\pgfpathlineto{\pgfqpoint{4.343740in}{0.557038in}}%
\pgfpathlineto{\pgfqpoint{4.345964in}{0.548354in}}%
\pgfpathlineto{\pgfqpoint{4.347075in}{0.548354in}}%
\pgfpathlineto{\pgfqpoint{4.349298in}{0.540106in}}%
\pgfpathlineto{\pgfqpoint{4.350410in}{0.540106in}}%
\pgfpathlineto{\pgfqpoint{4.352633in}{0.532476in}}%
\pgfpathlineto{\pgfqpoint{4.353744in}{0.532476in}}%
\pgfpathlineto{\pgfqpoint{4.357079in}{0.525414in}}%
\pgfpathlineto{\pgfqpoint{4.360414in}{0.519139in}}%
\pgfpathlineto{\pgfqpoint{4.363748in}{0.513433in}}%
\pgfpathlineto{\pgfqpoint{4.367083in}{0.508483in}}%
\pgfpathlineto{\pgfqpoint{4.368194in}{0.508483in}}%
\pgfpathlineto{\pgfqpoint{4.370418in}{0.503859in}}%
\pgfpathlineto{\pgfqpoint{4.371529in}{0.503859in}}%
\pgfpathlineto{\pgfqpoint{4.373752in}{0.499856in}}%
\pgfpathlineto{\pgfqpoint{4.374864in}{0.499856in}}%
\pgfpathlineto{\pgfqpoint{4.377087in}{0.496505in}}%
\pgfpathlineto{\pgfqpoint{4.381533in}{0.491535in}}%
\pgfpathlineto{\pgfqpoint{4.384868in}{0.491535in}}%
\pgfpathlineto{\pgfqpoint{4.387091in}{0.489943in}}%
\pgfpathlineto{\pgfqpoint{4.388202in}{0.489943in}}%
\pgfpathlineto{\pgfqpoint{4.390425in}{0.488602in}}%
\pgfpathlineto{\pgfqpoint{4.404876in}{0.489179in}}%
\pgfpathlineto{\pgfqpoint{4.407099in}{0.490531in}}%
\pgfpathlineto{\pgfqpoint{4.408210in}{0.490531in}}%
\pgfpathlineto{\pgfqpoint{4.410433in}{0.492649in}}%
\pgfpathlineto{\pgfqpoint{4.411545in}{0.492649in}}%
\pgfpathlineto{\pgfqpoint{4.413768in}{0.495363in}}%
\pgfpathlineto{\pgfqpoint{4.414880in}{0.495363in}}%
\pgfpathlineto{\pgfqpoint{4.417103in}{0.498683in}}%
\pgfpathlineto{\pgfqpoint{4.419326in}{0.498683in}}%
\pgfpathlineto{\pgfqpoint{4.421549in}{0.502691in}}%
\pgfpathlineto{\pgfqpoint{4.422660in}{0.502691in}}%
\pgfpathlineto{\pgfqpoint{4.424883in}{0.507542in}}%
\pgfpathlineto{\pgfqpoint{4.430441in}{0.519366in}}%
\pgfpathlineto{\pgfqpoint{4.432664in}{0.519366in}}%
\pgfpathlineto{\pgfqpoint{4.434887in}{0.526674in}}%
\pgfpathlineto{\pgfqpoint{4.435999in}{0.526674in}}%
\pgfpathlineto{\pgfqpoint{4.438222in}{0.534976in}}%
\pgfpathlineto{\pgfqpoint{4.439334in}{0.534976in}}%
\pgfpathlineto{\pgfqpoint{4.441557in}{0.544269in}}%
\pgfpathlineto{\pgfqpoint{4.442668in}{0.544269in}}%
\pgfpathlineto{\pgfqpoint{4.444891in}{0.554469in}}%
\pgfpathlineto{\pgfqpoint{4.448226in}{0.565961in}}%
\pgfpathlineto{\pgfqpoint{4.449338in}{0.565961in}}%
\pgfpathlineto{\pgfqpoint{4.454895in}{0.591972in}}%
\pgfpathlineto{\pgfqpoint{4.456007in}{0.591972in}}%
\pgfpathlineto{\pgfqpoint{4.458230in}{0.606661in}}%
\pgfpathlineto{\pgfqpoint{4.459342in}{0.606661in}}%
\pgfpathlineto{\pgfqpoint{4.462676in}{0.622488in}}%
\pgfpathlineto{\pgfqpoint{4.463788in}{0.622488in}}%
\pgfpathlineto{\pgfqpoint{4.466011in}{0.639701in}}%
\pgfpathlineto{\pgfqpoint{4.471569in}{0.677808in}}%
\pgfpathlineto{\pgfqpoint{4.472680in}{0.677808in}}%
\pgfpathlineto{\pgfqpoint{4.478238in}{0.719952in}}%
\pgfpathlineto{\pgfqpoint{4.480461in}{0.719952in}}%
\pgfpathlineto{\pgfqpoint{4.482684in}{0.742767in}}%
\pgfpathlineto{\pgfqpoint{4.483796in}{0.742767in}}%
\pgfpathlineto{\pgfqpoint{4.486019in}{0.766457in}}%
\pgfpathlineto{\pgfqpoint{4.487130in}{0.766457in}}%
\pgfpathlineto{\pgfqpoint{4.489353in}{0.791578in}}%
\pgfpathlineto{\pgfqpoint{4.492688in}{0.816914in}}%
\pgfpathlineto{\pgfqpoint{4.493800in}{0.816914in}}%
\pgfpathlineto{\pgfqpoint{4.496023in}{0.843838in}}%
\pgfpathlineto{\pgfqpoint{4.497134in}{0.843838in}}%
\pgfpathlineto{\pgfqpoint{4.499357in}{0.871186in}}%
\pgfpathlineto{\pgfqpoint{4.500469in}{0.871186in}}%
\pgfpathlineto{\pgfqpoint{4.502692in}{0.899489in}}%
\pgfpathlineto{\pgfqpoint{4.503803in}{0.899489in}}%
\pgfpathlineto{\pgfqpoint{4.506027in}{0.928526in}}%
\pgfpathlineto{\pgfqpoint{4.507138in}{0.928526in}}%
\pgfpathlineto{\pgfqpoint{4.509361in}{0.958375in}}%
\pgfpathlineto{\pgfqpoint{4.510473in}{0.958375in}}%
\pgfpathlineto{\pgfqpoint{4.512696in}{0.989211in}}%
\pgfpathlineto{\pgfqpoint{4.513807in}{0.989211in}}%
\pgfpathlineto{\pgfqpoint{4.519365in}{1.051613in}}%
\pgfpathlineto{\pgfqpoint{4.520477in}{1.051613in}}%
\pgfpathlineto{\pgfqpoint{4.522700in}{1.083474in}}%
\pgfpathlineto{\pgfqpoint{4.523811in}{1.083474in}}%
\pgfpathlineto{\pgfqpoint{4.526034in}{1.115372in}}%
\pgfpathlineto{\pgfqpoint{4.528258in}{1.115372in}}%
\pgfpathlineto{\pgfqpoint{4.530481in}{1.148252in}}%
\pgfpathlineto{\pgfqpoint{4.531592in}{1.148252in}}%
\pgfpathlineto{\pgfqpoint{4.533815in}{1.181382in}}%
\pgfpathlineto{\pgfqpoint{4.534927in}{1.181382in}}%
\pgfpathlineto{\pgfqpoint{4.537150in}{1.214368in}}%
\pgfpathlineto{\pgfqpoint{4.540485in}{1.248310in}}%
\pgfpathlineto{\pgfqpoint{4.543819in}{1.281595in}}%
\pgfpathlineto{\pgfqpoint{4.544931in}{1.281595in}}%
\pgfpathlineto{\pgfqpoint{4.547154in}{1.315609in}}%
\pgfpathlineto{\pgfqpoint{4.548265in}{1.315609in}}%
\pgfpathlineto{\pgfqpoint{4.550489in}{1.349816in}}%
\pgfpathlineto{\pgfqpoint{4.551600in}{1.349816in}}%
\pgfpathlineto{\pgfqpoint{4.553823in}{1.383570in}}%
\pgfpathlineto{\pgfqpoint{4.554935in}{1.383570in}}%
\pgfpathlineto{\pgfqpoint{4.557158in}{1.417589in}}%
\pgfpathlineto{\pgfqpoint{4.558269in}{1.417589in}}%
\pgfpathlineto{\pgfqpoint{4.560492in}{1.452187in}}%
\pgfpathlineto{\pgfqpoint{4.561604in}{1.452187in}}%
\pgfpathlineto{\pgfqpoint{4.563827in}{1.485950in}}%
\pgfpathlineto{\pgfqpoint{4.564939in}{1.485950in}}%
\pgfpathlineto{\pgfqpoint{4.567162in}{1.519390in}}%
\pgfpathlineto{\pgfqpoint{4.568273in}{1.519390in}}%
\pgfpathlineto{\pgfqpoint{4.572719in}{1.587134in}}%
\pgfpathlineto{\pgfqpoint{4.576054in}{1.587134in}}%
\pgfpathlineto{\pgfqpoint{4.578277in}{1.621109in}}%
\pgfpathlineto{\pgfqpoint{4.579389in}{1.621109in}}%
\pgfpathlineto{\pgfqpoint{4.581612in}{1.655379in}}%
\pgfpathlineto{\pgfqpoint{4.582723in}{1.655379in}}%
\pgfpathlineto{\pgfqpoint{4.584947in}{1.688022in}}%
\pgfpathlineto{\pgfqpoint{4.586058in}{1.688022in}}%
\pgfpathlineto{\pgfqpoint{4.588281in}{1.721452in}}%
\pgfpathlineto{\pgfqpoint{4.589393in}{1.721452in}}%
\pgfpathlineto{\pgfqpoint{4.591616in}{1.754906in}}%
\pgfpathlineto{\pgfqpoint{4.592727in}{1.754906in}}%
\pgfpathlineto{\pgfqpoint{4.594950in}{1.787370in}}%
\pgfpathlineto{\pgfqpoint{4.596062in}{1.787370in}}%
\pgfpathlineto{\pgfqpoint{4.598285in}{1.820104in}}%
\pgfpathlineto{\pgfqpoint{4.599397in}{1.820104in}}%
\pgfpathlineto{\pgfqpoint{4.601620in}{1.852636in}}%
\pgfpathlineto{\pgfqpoint{4.602731in}{1.852636in}}%
\pgfpathlineto{\pgfqpoint{4.604954in}{1.884684in}}%
\pgfpathlineto{\pgfqpoint{4.606066in}{1.884684in}}%
\pgfpathlineto{\pgfqpoint{4.608289in}{1.916719in}}%
\pgfpathlineto{\pgfqpoint{4.609401in}{1.916719in}}%
\pgfpathlineto{\pgfqpoint{4.611624in}{1.948448in}}%
\pgfpathlineto{\pgfqpoint{4.612735in}{1.948448in}}%
\pgfpathlineto{\pgfqpoint{4.614958in}{1.979454in}}%
\pgfpathlineto{\pgfqpoint{4.616070in}{1.979454in}}%
\pgfpathlineto{\pgfqpoint{4.618293in}{2.011029in}}%
\pgfpathlineto{\pgfqpoint{4.619405in}{2.011029in}}%
\pgfpathlineto{\pgfqpoint{4.621628in}{2.042440in}}%
\pgfpathlineto{\pgfqpoint{4.622739in}{2.042440in}}%
\pgfpathlineto{\pgfqpoint{4.624962in}{2.072775in}}%
\pgfpathlineto{\pgfqpoint{4.627185in}{2.072775in}}%
\pgfpathlineto{\pgfqpoint{4.629408in}{2.103625in}}%
\pgfpathlineto{\pgfqpoint{4.630520in}{2.103625in}}%
\pgfpathlineto{\pgfqpoint{4.632743in}{2.133926in}}%
\pgfpathlineto{\pgfqpoint{4.638301in}{2.193054in}}%
\pgfpathlineto{\pgfqpoint{4.639412in}{2.193054in}}%
\pgfpathlineto{\pgfqpoint{4.641636in}{2.222224in}}%
\pgfpathlineto{\pgfqpoint{4.643859in}{2.222224in}}%
\pgfpathlineto{\pgfqpoint{4.646082in}{2.250998in}}%
\pgfpathlineto{\pgfqpoint{4.647193in}{2.250998in}}%
\pgfpathlineto{\pgfqpoint{4.649416in}{2.279787in}}%
\pgfpathlineto{\pgfqpoint{4.650528in}{2.279787in}}%
\pgfpathlineto{\pgfqpoint{4.652751in}{2.308324in}}%
\pgfpathlineto{\pgfqpoint{4.658309in}{2.364009in}}%
\pgfpathlineto{\pgfqpoint{4.660532in}{2.364009in}}%
\pgfpathlineto{\pgfqpoint{4.662755in}{2.391793in}}%
\pgfpathlineto{\pgfqpoint{4.663867in}{2.391793in}}%
\pgfpathlineto{\pgfqpoint{4.666090in}{2.418920in}}%
\pgfpathlineto{\pgfqpoint{4.667201in}{2.418920in}}%
\pgfpathlineto{\pgfqpoint{4.669424in}{2.445661in}}%
\pgfpathlineto{\pgfqpoint{4.670536in}{2.445661in}}%
\pgfpathlineto{\pgfqpoint{4.672759in}{2.472267in}}%
\pgfpathlineto{\pgfqpoint{4.674982in}{2.472267in}}%
\pgfpathlineto{\pgfqpoint{4.677205in}{2.497960in}}%
\pgfpathlineto{\pgfqpoint{4.680540in}{2.524034in}}%
\pgfpathlineto{\pgfqpoint{4.683874in}{2.549611in}}%
\pgfpathlineto{\pgfqpoint{4.689432in}{2.599206in}}%
\pgfpathlineto{\pgfqpoint{4.690544in}{2.599206in}}%
\pgfpathlineto{\pgfqpoint{4.693878in}{2.624223in}}%
\pgfpathlineto{\pgfqpoint{4.697213in}{2.648022in}}%
\pgfpathlineto{\pgfqpoint{4.698325in}{2.648022in}}%
\pgfpathlineto{\pgfqpoint{4.700548in}{2.671841in}}%
\pgfpathlineto{\pgfqpoint{4.701659in}{2.671841in}}%
\pgfpathlineto{\pgfqpoint{4.703882in}{2.695255in}}%
\pgfpathlineto{\pgfqpoint{4.704994in}{2.695255in}}%
\pgfpathlineto{\pgfqpoint{4.710552in}{2.741478in}}%
\pgfpathlineto{\pgfqpoint{4.711663in}{2.741478in}}%
\pgfpathlineto{\pgfqpoint{4.713886in}{2.763882in}}%
\pgfpathlineto{\pgfqpoint{4.714998in}{2.763882in}}%
\pgfpathlineto{\pgfqpoint{4.717221in}{2.786113in}}%
\pgfpathlineto{\pgfqpoint{4.718332in}{2.786113in}}%
\pgfpathlineto{\pgfqpoint{4.720556in}{2.807860in}}%
\pgfpathlineto{\pgfqpoint{4.721667in}{2.807860in}}%
\pgfpathlineto{\pgfqpoint{4.727225in}{2.850176in}}%
\pgfpathlineto{\pgfqpoint{4.728336in}{2.850176in}}%
\pgfpathlineto{\pgfqpoint{4.732783in}{2.891739in}}%
\pgfpathlineto{\pgfqpoint{4.735006in}{2.891739in}}%
\pgfpathlineto{\pgfqpoint{4.737229in}{2.911158in}}%
\pgfpathlineto{\pgfqpoint{4.739452in}{2.911158in}}%
\pgfpathlineto{\pgfqpoint{4.741675in}{2.931432in}}%
\pgfpathlineto{\pgfqpoint{4.742786in}{2.931432in}}%
\pgfpathlineto{\pgfqpoint{4.745010in}{2.950784in}}%
\pgfpathlineto{\pgfqpoint{4.748344in}{2.970083in}}%
\pgfpathlineto{\pgfqpoint{4.749456in}{2.970083in}}%
\pgfpathlineto{\pgfqpoint{4.751679in}{2.989015in}}%
\pgfpathlineto{\pgfqpoint{4.755014in}{3.007372in}}%
\pgfpathlineto{\pgfqpoint{4.756125in}{3.007372in}}%
\pgfpathlineto{\pgfqpoint{4.758348in}{3.026057in}}%
\pgfpathlineto{\pgfqpoint{4.759460in}{3.026057in}}%
\pgfpathlineto{\pgfqpoint{4.761683in}{3.044028in}}%
\pgfpathlineto{\pgfqpoint{4.762794in}{3.044028in}}%
\pgfpathlineto{\pgfqpoint{4.765017in}{3.061795in}}%
\pgfpathlineto{\pgfqpoint{4.766129in}{3.061795in}}%
\pgfpathlineto{\pgfqpoint{4.768352in}{3.079259in}}%
\pgfpathlineto{\pgfqpoint{4.769464in}{3.079259in}}%
\pgfpathlineto{\pgfqpoint{4.773910in}{3.113152in}}%
\pgfpathlineto{\pgfqpoint{4.776133in}{3.113152in}}%
\pgfpathlineto{\pgfqpoint{4.778356in}{3.129838in}}%
\pgfpathlineto{\pgfqpoint{4.779468in}{3.129838in}}%
\pgfpathlineto{\pgfqpoint{4.781691in}{3.146012in}}%
\pgfpathlineto{\pgfqpoint{4.782802in}{3.146012in}}%
\pgfpathlineto{\pgfqpoint{4.785025in}{3.162022in}}%
\pgfpathlineto{\pgfqpoint{4.786137in}{3.162022in}}%
\pgfpathlineto{\pgfqpoint{4.789472in}{3.177636in}}%
\pgfpathlineto{\pgfqpoint{4.792806in}{3.192912in}}%
\pgfpathlineto{\pgfqpoint{4.793918in}{3.192912in}}%
\pgfpathlineto{\pgfqpoint{4.796141in}{3.208125in}}%
\pgfpathlineto{\pgfqpoint{4.799475in}{3.222990in}}%
\pgfpathlineto{\pgfqpoint{4.800587in}{3.222990in}}%
\pgfpathlineto{\pgfqpoint{4.802810in}{3.237314in}}%
\pgfpathlineto{\pgfqpoint{4.803922in}{3.237314in}}%
\pgfpathlineto{\pgfqpoint{4.806145in}{3.251721in}}%
\pgfpathlineto{\pgfqpoint{4.807256in}{3.251721in}}%
\pgfpathlineto{\pgfqpoint{4.811703in}{3.297307in}}%
\pgfpathlineto{\pgfqpoint{4.812814in}{3.297307in}}%
\pgfpathlineto{\pgfqpoint{4.816149in}{3.322671in}}%
\pgfpathlineto{\pgfqpoint{4.817260in}{3.322671in}}%
\pgfpathlineto{\pgfqpoint{4.819483in}{3.331509in}}%
\pgfpathlineto{\pgfqpoint{4.820595in}{3.331509in}}%
\pgfpathlineto{\pgfqpoint{4.822818in}{3.334112in}}%
\pgfpathlineto{\pgfqpoint{4.823930in}{3.334112in}}%
\pgfpathlineto{\pgfqpoint{4.826153in}{3.332837in}}%
\pgfpathlineto{\pgfqpoint{4.827264in}{3.332837in}}%
\pgfpathlineto{\pgfqpoint{4.829487in}{3.345520in}}%
\pgfpathlineto{\pgfqpoint{4.830599in}{3.345520in}}%
\pgfpathlineto{\pgfqpoint{4.832822in}{3.357598in}}%
\pgfpathlineto{\pgfqpoint{4.833934in}{3.357598in}}%
\pgfpathlineto{\pgfqpoint{4.837268in}{3.431412in}}%
\pgfpathlineto{\pgfqpoint{4.841714in}{3.455753in}}%
\pgfpathlineto{\pgfqpoint{4.843937in}{3.455753in}}%
\pgfpathlineto{\pgfqpoint{4.849495in}{3.485846in}}%
\pgfpathlineto{\pgfqpoint{4.851718in}{3.485846in}}%
\pgfpathlineto{\pgfqpoint{4.853941in}{3.488850in}}%
\pgfpathlineto{\pgfqpoint{4.855053in}{3.488850in}}%
\pgfpathlineto{\pgfqpoint{4.858388in}{3.501561in}}%
\pgfpathlineto{\pgfqpoint{4.860611in}{3.510457in}}%
\pgfpathlineto{\pgfqpoint{4.863945in}{3.507115in}}%
\pgfpathlineto{\pgfqpoint{4.868392in}{3.508400in}}%
\pgfpathlineto{\pgfqpoint{4.870615in}{3.516793in}}%
\pgfpathlineto{\pgfqpoint{4.871726in}{3.516793in}}%
\pgfpathlineto{\pgfqpoint{4.873949in}{3.531147in}}%
\pgfpathlineto{\pgfqpoint{4.875061in}{3.531147in}}%
\pgfpathlineto{\pgfqpoint{4.877284in}{3.535020in}}%
\pgfpathlineto{\pgfqpoint{4.878395in}{3.535020in}}%
\pgfpathlineto{\pgfqpoint{4.880619in}{3.542844in}}%
\pgfpathlineto{\pgfqpoint{4.881730in}{3.542844in}}%
\pgfpathlineto{\pgfqpoint{4.885065in}{3.520391in}}%
\pgfpathlineto{\pgfqpoint{4.888399in}{3.516305in}}%
\pgfpathlineto{\pgfqpoint{4.891734in}{3.559211in}}%
\pgfpathlineto{\pgfqpoint{4.893957in}{3.548436in}}%
\pgfpathlineto{\pgfqpoint{4.895069in}{3.548436in}}%
\pgfpathlineto{\pgfqpoint{4.897292in}{3.574960in}}%
\pgfpathlineto{\pgfqpoint{4.899515in}{3.574960in}}%
\pgfpathlineto{\pgfqpoint{4.901738in}{3.579529in}}%
\pgfpathlineto{\pgfqpoint{4.902850in}{3.579529in}}%
\pgfpathlineto{\pgfqpoint{4.905073in}{3.555777in}}%
\pgfpathlineto{\pgfqpoint{4.906184in}{3.555777in}}%
\pgfpathlineto{\pgfqpoint{4.908407in}{3.577974in}}%
\pgfpathlineto{\pgfqpoint{4.909519in}{3.577974in}}%
\pgfpathlineto{\pgfqpoint{4.910568in}{3.601168in}}%
\pgfpathlineto{\pgfqpoint{4.910568in}{3.601168in}}%
\pgfusepath{stroke}%
\end{pgfscope}%
\begin{pgfscope}%
\pgfsetrectcap%
\pgfsetmiterjoin%
\pgfsetlinewidth{0.803000pt}%
\definecolor{currentstroke}{rgb}{0.760784,0.211765,0.086275}%
\pgfsetstrokecolor{currentstroke}%
\pgfsetdash{}{0pt}%
\pgfpathmoveto{\pgfqpoint{0.592318in}{0.451986in}}%
\pgfpathlineto{\pgfqpoint{0.592318in}{3.591168in}}%
\pgfusepath{stroke}%
\end{pgfscope}%
\begin{pgfscope}%
\pgfsetrectcap%
\pgfsetmiterjoin%
\pgfsetlinewidth{0.803000pt}%
\definecolor{currentstroke}{rgb}{0.152941,0.235294,0.458824}%
\pgfsetstrokecolor{currentstroke}%
\pgfsetdash{}{0pt}%
\pgfpathmoveto{\pgfqpoint{5.481911in}{0.451986in}}%
\pgfpathlineto{\pgfqpoint{5.481911in}{3.591168in}}%
\pgfusepath{stroke}%
\end{pgfscope}%
\begin{pgfscope}%
\pgfsetrectcap%
\pgfsetmiterjoin%
\pgfsetlinewidth{0.803000pt}%
\definecolor{currentstroke}{rgb}{0.000000,0.000000,0.000000}%
\pgfsetstrokecolor{currentstroke}%
\pgfsetdash{}{0pt}%
\pgfpathmoveto{\pgfqpoint{0.592318in}{0.451986in}}%
\pgfpathlineto{\pgfqpoint{5.481911in}{0.451986in}}%
\pgfusepath{stroke}%
\end{pgfscope}%
\begin{pgfscope}%
\pgfsetrectcap%
\pgfsetmiterjoin%
\pgfsetlinewidth{0.000000pt}%
\definecolor{currentstroke}{rgb}{0.000000,0.000000,0.000000}%
\pgfsetstrokecolor{currentstroke}%
\pgfsetstrokeopacity{0.000000}%
\pgfsetdash{}{0pt}%
\pgfpathmoveto{\pgfqpoint{0.592318in}{3.591168in}}%
\pgfpathlineto{\pgfqpoint{5.481911in}{3.591168in}}%
\pgfusepath{}%
\end{pgfscope}%
\end{pgfpicture}%
\makeatother%
\endgroup%

	\caption{Messung von Längswiderstand und Hallwiderstand in Abhängigkeit des Magnetfelds für Probe P07 für die Kontaktpaare.... Markiert sind Hallplateaus mit korrespondierenden Minima der SdH-Oszillation und den jeweiligen Füllfaktoren}
	\label{abb:P07_Meth_1}
\end{figure}

Wir konnten so in jeder Messreihe für die Füllfaktoren $\nu = 6$, $\nu = 8$ und noch $\nu = 10 $ (relativ) klare Plateaus messen. Für die Faktoren $\nu = 12, 14, 16, 18$ lassen sich der SdH-Oszillation noch angedeutete Plateaus zuordnen. Diesen Bereich ordnen wir daher dem quantenmechanischen Regime zu und nutzen ihn für diese sowie für die in Abschnitt \ref{subsubsec:Meth_2} beschriebene Methode. \\
\\
Für jedes identifizierte Hallplateau berechnen wir nun die Elektronendichte nach Gl. ?\footnote{Die Werte für die Elementarladung und das Plank'sche Wirkungsquantum wurden dem Python-Package Scipy.constants entnommen.}. Die Werte sind für ..... exemplarisch in Tab. \ref{tab:P07_Meth_1} zu finden. 
$$ n_e = \frac{\nu B(\nu) e}{h}$$
Da die Elektronenkonzentration unabhängig vom Magnetfeld ist, mitteln wir über jedes berücksichtigte Plateau und erhalten Tabelle \ref{tab:Meth_1}. Die Std-Abweichung berschreibt die Abweichung innerhalb der Messreihe für jedes Plateau. \\ 

Die bestimmten Elektronenkonzentrationen liegen ausgesprochen nah beieinander, es gibt allerdings kleine Abweichungen, die nicht von der Standartabweichung in den einzelnen Messreihen erklärt wird. Auffällig ist, das bei beiden Messungen, bei denen der Längswiderstand zwischen direkt nebeneinander liegenden Kontakten gemessen wurde, eine leicht höhere Elektronenkonzentration festzustellen ist. Insgesamt erscheinen die Abweichungen aber insignifikant. \\

Die Elektronenmobilität lässt sich nun (und auch bei den anderen Ansätzen) mit Gl. ? aus der E-Konzentration und dem spezifischen Widerstand der Hallgeometrie berechnen. Dabei ist zu beachten, das bei den Messungen [....] mit einem Abstand von zwei Kontakten gemessen wurde. Der spezifische Widerstand ist hier also doppelt so groß. Die Ergebnisse sind ebenfalls Tabelle \ref{tab:Meth_1} zu entnehmen. Im Gegensatz zur E-Konzentration beobachten wir hier sig. Schwankungen. Da diese Schwankungen bei allen Ansätzen gleichermaßen auftreten, diskutieren wir sie in Abschnitt ? zusammenfassend.

\begin{table}
	\centering
	\caption{}
	\begin{tabular}{lrrrr}
\toprule
 $B_{min}[T]$ &    $R_{xy} [\Omega]$ 	&  $R_{xx}[\Omega]$ 	& $\nu$  &$n_e [1/m^2]$\\
\midrule
  1.069 &   1437.48 &   87.656 	&       17.957 &         4.642e+15 \\
  1.204 &   1617.87 &   81.681 	&       15.955 &         4.645e+15 \\
  1.379 &   1852.25 &   73.175 	&       13.936 &         4.647e+15 \\
  1.608 &  2156.23 	&   61.262 	&       11.971 &         4.655e+15 \\
  1.930 & 2585.12 	&    45.183 &        9.985 &         4.660e+15 \\
  2.414 &   3228.92 &    25.370 &        7.994 &         4.666e+15 \\
  3.222 &   4301.67 &     7.432 &        6.001 &         4.675e+15 \\
\bottomrule
\end{tabular}

	\label{tab:P07_Meth_1}
\end{table}
\begin{table}
	\centering
	\caption{}
	\begin{tabular}{lrrrr}
\toprule
        Messreihe &  $R_{xx} (B=0) [\Omega]$ &  E-Dichte$(\varnothing) [1/\si{m}^2]$  & Std-Ab. &  E-Mobilität \\
\midrule
 P07\_0611\_0708\_0109 &  106.743 &              4.656e+15 &          1.222e+13 &             15.70 \\
 P07\_0611\_0709\_0109 &  182.622 &              4.634e+15 &          1.501e+13 &             18.44 \\
 P07\_1115\_1213\_2012 &   77.843 &              4.657e+15 &          4.955e+12 &             21.52 \\
 P07\_1115\_2018\_1913 &  160.499 &              4.627e+15 &          1.033e+13 &             21.01 \\
\bottomrule
\end{tabular}

	\label{tab:Meth_1}
\end{table}

\newpage
\subsubsection{Methode 2: Bestimmung der reziproken Periode der SdH-Oszillation} \label{subsubsec:Meth_2}
Die Elektronenkonzentration lässt sich auch mit Gl. ?? aus der reziproken Periode der SdH-Oszillation berechnen. Dies ist besonders dann interessant, wenn sich den Hallplateaus keine eindeutigen Hallwiderstände mehr zuordnen lassen (was hier auf Füllfaktoren $\nu > 10$ zunehmend zutrifft.

Die rez. Abstände der schon bei Methode 1 genutzten Minima und die aus diesen berechnete Elektronenkonzentration wurden wie schon bei Methode 1 gemittelt und aus dem Mittelwert die Elektronenmobilität bestimmt. Diese Größen sind den Tabellen \ref{tab:P07_Meth_2} (repräsentativ für alle Messreihen) und Tab. \ref{tab:Meth_2} zu finden. 

Die beobachtete Elektronenkonzentration / -mobilität liegen mit dieser Methode noch etwas näher beieinander als zuvor, allerdings sind die Standartabweichungen größer geworden. Da die Werte auch mit denen aus Methode 1 in sehr guter Näherung übereinstimmen erhalten wir gleichermaßen unterschiedliche Werte für die Elektronenmobilität.
 
\begin{figure}[htbp]
	\centering
	%% Creator: Matplotlib, PGF backend
%%
%% To include the figure in your LaTeX document, write
%%   \input{<filename>.pgf}
%%
%% Make sure the required packages are loaded in your preamble
%%   \usepackage{pgf}
%%
%% Figures using additional raster images can only be included by \input if
%% they are in the same directory as the main LaTeX file. For loading figures
%% from other directories you can use the `import` package
%%   \usepackage{import}
%% and then include the figures with
%%   \import{<path to file>}{<filename>.pgf}
%%
%% Matplotlib used the following preamble
%%   \usepackage[utf8x]{inputenc}
%%   \usepackage[T1]{fontenc}
%%
\begingroup%
\makeatletter%
\begin{pgfpicture}%
\pgfpathrectangle{\pgfpointorigin}{\pgfqpoint{6.012886in}{2.477445in}}%
\pgfusepath{use as bounding box, clip}%
\begin{pgfscope}%
\pgfsetbuttcap%
\pgfsetmiterjoin%
\definecolor{currentfill}{rgb}{1.000000,1.000000,1.000000}%
\pgfsetfillcolor{currentfill}%
\pgfsetlinewidth{0.000000pt}%
\definecolor{currentstroke}{rgb}{1.000000,1.000000,1.000000}%
\pgfsetstrokecolor{currentstroke}%
\pgfsetdash{}{0pt}%
\pgfpathmoveto{\pgfqpoint{0.000000in}{0.000000in}}%
\pgfpathlineto{\pgfqpoint{6.012886in}{0.000000in}}%
\pgfpathlineto{\pgfqpoint{6.012886in}{2.477445in}}%
\pgfpathlineto{\pgfqpoint{0.000000in}{2.477445in}}%
\pgfpathclose%
\pgfusepath{fill}%
\end{pgfscope}%
\begin{pgfscope}%
\pgfsetbuttcap%
\pgfsetmiterjoin%
\definecolor{currentfill}{rgb}{1.000000,1.000000,1.000000}%
\pgfsetfillcolor{currentfill}%
\pgfsetlinewidth{0.000000pt}%
\definecolor{currentstroke}{rgb}{0.000000,0.000000,0.000000}%
\pgfsetstrokecolor{currentstroke}%
\pgfsetstrokeopacity{0.000000}%
\pgfsetdash{}{0pt}%
\pgfpathmoveto{\pgfqpoint{0.530948in}{0.451986in}}%
\pgfpathlineto{\pgfqpoint{5.887886in}{0.451986in}}%
\pgfpathlineto{\pgfqpoint{5.887886in}{2.352445in}}%
\pgfpathlineto{\pgfqpoint{0.530948in}{2.352445in}}%
\pgfpathclose%
\pgfusepath{fill}%
\end{pgfscope}%
\begin{pgfscope}%
\pgfsetroundcap%
\pgfsetroundjoin%
\pgfsetlinewidth{0.752812pt}%
\definecolor{currentstroke}{rgb}{0.000000,0.000000,0.000000}%
\pgfsetstrokecolor{currentstroke}%
\pgfsetdash{}{0pt}%
\pgfpathmoveto{\pgfqpoint{3.129330in}{1.636745in}}%
\pgfpathquadraticcurveto{\pgfqpoint{3.345439in}{1.636745in}}{\pgfqpoint{3.561548in}{1.636745in}}%
\pgfusepath{stroke}%
\end{pgfscope}%
\begin{pgfscope}%
\pgfsetroundcap%
\pgfsetroundjoin%
\pgfsetlinewidth{0.752812pt}%
\definecolor{currentstroke}{rgb}{0.000000,0.000000,0.000000}%
\pgfsetstrokecolor{currentstroke}%
\pgfsetdash{}{0pt}%
\pgfpathmoveto{\pgfqpoint{2.694999in}{1.636745in}}%
\pgfpathquadraticcurveto{\pgfqpoint{2.912164in}{1.636745in}}{\pgfqpoint{3.129330in}{1.636745in}}%
\pgfusepath{stroke}%
\end{pgfscope}%
\begin{pgfscope}%
\pgfsetroundcap%
\pgfsetroundjoin%
\pgfsetlinewidth{0.752812pt}%
\definecolor{currentstroke}{rgb}{0.000000,0.000000,0.000000}%
\pgfsetstrokecolor{currentstroke}%
\pgfsetdash{}{0pt}%
\pgfpathmoveto{\pgfqpoint{2.269441in}{1.636745in}}%
\pgfpathquadraticcurveto{\pgfqpoint{2.482220in}{1.636745in}}{\pgfqpoint{2.694999in}{1.636745in}}%
\pgfusepath{stroke}%
\end{pgfscope}%
\begin{pgfscope}%
\pgfsetroundcap%
\pgfsetroundjoin%
\pgfsetlinewidth{0.752812pt}%
\definecolor{currentstroke}{rgb}{0.000000,0.000000,0.000000}%
\pgfsetstrokecolor{currentstroke}%
\pgfsetdash{}{0pt}%
\pgfpathmoveto{\pgfqpoint{1.841892in}{1.636745in}}%
\pgfpathquadraticcurveto{\pgfqpoint{2.055667in}{1.636745in}}{\pgfqpoint{2.269441in}{1.636745in}}%
\pgfusepath{stroke}%
\end{pgfscope}%
\begin{pgfscope}%
\pgfsetroundcap%
\pgfsetroundjoin%
\pgfsetlinewidth{0.752812pt}%
\definecolor{currentstroke}{rgb}{0.000000,0.000000,0.000000}%
\pgfsetstrokecolor{currentstroke}%
\pgfsetdash{}{0pt}%
\pgfpathmoveto{\pgfqpoint{1.413813in}{1.636745in}}%
\pgfpathquadraticcurveto{\pgfqpoint{1.627853in}{1.636745in}}{\pgfqpoint{1.841892in}{1.636745in}}%
\pgfusepath{stroke}%
\end{pgfscope}%
\begin{pgfscope}%
\pgfsetroundcap%
\pgfsetroundjoin%
\pgfsetlinewidth{0.752812pt}%
\definecolor{currentstroke}{rgb}{0.000000,0.000000,0.000000}%
\pgfsetstrokecolor{currentstroke}%
\pgfsetdash{}{0pt}%
\pgfpathmoveto{\pgfqpoint{0.985736in}{1.636745in}}%
\pgfpathquadraticcurveto{\pgfqpoint{1.199775in}{1.636745in}}{\pgfqpoint{1.413813in}{1.636745in}}%
\pgfusepath{stroke}%
\end{pgfscope}%
\begin{pgfscope}%
\pgfpathrectangle{\pgfqpoint{0.530948in}{0.451986in}}{\pgfqpoint{5.356938in}{1.900459in}}%
\pgfusepath{clip}%
\pgfsetbuttcap%
\pgfsetroundjoin%
\pgfsetlinewidth{0.501875pt}%
\definecolor{currentstroke}{rgb}{0.690196,0.690196,0.690196}%
\pgfsetstrokecolor{currentstroke}%
\pgfsetdash{{1.850000pt}{0.800000pt}}{0.000000pt}%
\pgfpathmoveto{\pgfqpoint{0.530948in}{0.451986in}}%
\pgfpathlineto{\pgfqpoint{0.530948in}{2.352445in}}%
\pgfusepath{stroke}%
\end{pgfscope}%
\begin{pgfscope}%
\pgfsetbuttcap%
\pgfsetroundjoin%
\definecolor{currentfill}{rgb}{0.000000,0.000000,0.000000}%
\pgfsetfillcolor{currentfill}%
\pgfsetlinewidth{0.803000pt}%
\definecolor{currentstroke}{rgb}{0.000000,0.000000,0.000000}%
\pgfsetstrokecolor{currentstroke}%
\pgfsetdash{}{0pt}%
\pgfsys@defobject{currentmarker}{\pgfqpoint{0.000000in}{-0.048611in}}{\pgfqpoint{0.000000in}{0.000000in}}{%
\pgfpathmoveto{\pgfqpoint{0.000000in}{0.000000in}}%
\pgfpathlineto{\pgfqpoint{0.000000in}{-0.048611in}}%
\pgfusepath{stroke,fill}%
}%
\begin{pgfscope}%
\pgfsys@transformshift{0.530948in}{0.451986in}%
\pgfsys@useobject{currentmarker}{}%
\end{pgfscope}%
\end{pgfscope}%
\begin{pgfscope}%
\definecolor{textcolor}{rgb}{0.000000,0.000000,0.000000}%
\pgfsetstrokecolor{textcolor}%
\pgfsetfillcolor{textcolor}%
\pgftext[x=0.530948in,y=0.354764in,,top]{\color{textcolor}\rmfamily\fontsize{8.000000}{9.600000}\selectfont \(\displaystyle 0.2\)}%
\end{pgfscope}%
\begin{pgfscope}%
\pgfpathrectangle{\pgfqpoint{0.530948in}{0.451986in}}{\pgfqpoint{5.356938in}{1.900459in}}%
\pgfusepath{clip}%
\pgfsetbuttcap%
\pgfsetroundjoin%
\pgfsetlinewidth{0.501875pt}%
\definecolor{currentstroke}{rgb}{0.690196,0.690196,0.690196}%
\pgfsetstrokecolor{currentstroke}%
\pgfsetdash{{1.850000pt}{0.800000pt}}{0.000000pt}%
\pgfpathmoveto{\pgfqpoint{1.355092in}{0.451986in}}%
\pgfpathlineto{\pgfqpoint{1.355092in}{2.352445in}}%
\pgfusepath{stroke}%
\end{pgfscope}%
\begin{pgfscope}%
\pgfsetbuttcap%
\pgfsetroundjoin%
\definecolor{currentfill}{rgb}{0.000000,0.000000,0.000000}%
\pgfsetfillcolor{currentfill}%
\pgfsetlinewidth{0.803000pt}%
\definecolor{currentstroke}{rgb}{0.000000,0.000000,0.000000}%
\pgfsetstrokecolor{currentstroke}%
\pgfsetdash{}{0pt}%
\pgfsys@defobject{currentmarker}{\pgfqpoint{0.000000in}{-0.048611in}}{\pgfqpoint{0.000000in}{0.000000in}}{%
\pgfpathmoveto{\pgfqpoint{0.000000in}{0.000000in}}%
\pgfpathlineto{\pgfqpoint{0.000000in}{-0.048611in}}%
\pgfusepath{stroke,fill}%
}%
\begin{pgfscope}%
\pgfsys@transformshift{1.355092in}{0.451986in}%
\pgfsys@useobject{currentmarker}{}%
\end{pgfscope}%
\end{pgfscope}%
\begin{pgfscope}%
\definecolor{textcolor}{rgb}{0.000000,0.000000,0.000000}%
\pgfsetstrokecolor{textcolor}%
\pgfsetfillcolor{textcolor}%
\pgftext[x=1.355092in,y=0.354764in,,top]{\color{textcolor}\rmfamily\fontsize{8.000000}{9.600000}\selectfont \(\displaystyle 0.4\)}%
\end{pgfscope}%
\begin{pgfscope}%
\pgfpathrectangle{\pgfqpoint{0.530948in}{0.451986in}}{\pgfqpoint{5.356938in}{1.900459in}}%
\pgfusepath{clip}%
\pgfsetbuttcap%
\pgfsetroundjoin%
\pgfsetlinewidth{0.501875pt}%
\definecolor{currentstroke}{rgb}{0.690196,0.690196,0.690196}%
\pgfsetstrokecolor{currentstroke}%
\pgfsetdash{{1.850000pt}{0.800000pt}}{0.000000pt}%
\pgfpathmoveto{\pgfqpoint{2.179236in}{0.451986in}}%
\pgfpathlineto{\pgfqpoint{2.179236in}{2.352445in}}%
\pgfusepath{stroke}%
\end{pgfscope}%
\begin{pgfscope}%
\pgfsetbuttcap%
\pgfsetroundjoin%
\definecolor{currentfill}{rgb}{0.000000,0.000000,0.000000}%
\pgfsetfillcolor{currentfill}%
\pgfsetlinewidth{0.803000pt}%
\definecolor{currentstroke}{rgb}{0.000000,0.000000,0.000000}%
\pgfsetstrokecolor{currentstroke}%
\pgfsetdash{}{0pt}%
\pgfsys@defobject{currentmarker}{\pgfqpoint{0.000000in}{-0.048611in}}{\pgfqpoint{0.000000in}{0.000000in}}{%
\pgfpathmoveto{\pgfqpoint{0.000000in}{0.000000in}}%
\pgfpathlineto{\pgfqpoint{0.000000in}{-0.048611in}}%
\pgfusepath{stroke,fill}%
}%
\begin{pgfscope}%
\pgfsys@transformshift{2.179236in}{0.451986in}%
\pgfsys@useobject{currentmarker}{}%
\end{pgfscope}%
\end{pgfscope}%
\begin{pgfscope}%
\definecolor{textcolor}{rgb}{0.000000,0.000000,0.000000}%
\pgfsetstrokecolor{textcolor}%
\pgfsetfillcolor{textcolor}%
\pgftext[x=2.179236in,y=0.354764in,,top]{\color{textcolor}\rmfamily\fontsize{8.000000}{9.600000}\selectfont \(\displaystyle 0.6\)}%
\end{pgfscope}%
\begin{pgfscope}%
\pgfpathrectangle{\pgfqpoint{0.530948in}{0.451986in}}{\pgfqpoint{5.356938in}{1.900459in}}%
\pgfusepath{clip}%
\pgfsetbuttcap%
\pgfsetroundjoin%
\pgfsetlinewidth{0.501875pt}%
\definecolor{currentstroke}{rgb}{0.690196,0.690196,0.690196}%
\pgfsetstrokecolor{currentstroke}%
\pgfsetdash{{1.850000pt}{0.800000pt}}{0.000000pt}%
\pgfpathmoveto{\pgfqpoint{3.003381in}{0.451986in}}%
\pgfpathlineto{\pgfqpoint{3.003381in}{2.352445in}}%
\pgfusepath{stroke}%
\end{pgfscope}%
\begin{pgfscope}%
\pgfsetbuttcap%
\pgfsetroundjoin%
\definecolor{currentfill}{rgb}{0.000000,0.000000,0.000000}%
\pgfsetfillcolor{currentfill}%
\pgfsetlinewidth{0.803000pt}%
\definecolor{currentstroke}{rgb}{0.000000,0.000000,0.000000}%
\pgfsetstrokecolor{currentstroke}%
\pgfsetdash{}{0pt}%
\pgfsys@defobject{currentmarker}{\pgfqpoint{0.000000in}{-0.048611in}}{\pgfqpoint{0.000000in}{0.000000in}}{%
\pgfpathmoveto{\pgfqpoint{0.000000in}{0.000000in}}%
\pgfpathlineto{\pgfqpoint{0.000000in}{-0.048611in}}%
\pgfusepath{stroke,fill}%
}%
\begin{pgfscope}%
\pgfsys@transformshift{3.003381in}{0.451986in}%
\pgfsys@useobject{currentmarker}{}%
\end{pgfscope}%
\end{pgfscope}%
\begin{pgfscope}%
\definecolor{textcolor}{rgb}{0.000000,0.000000,0.000000}%
\pgfsetstrokecolor{textcolor}%
\pgfsetfillcolor{textcolor}%
\pgftext[x=3.003381in,y=0.354764in,,top]{\color{textcolor}\rmfamily\fontsize{8.000000}{9.600000}\selectfont \(\displaystyle 0.8\)}%
\end{pgfscope}%
\begin{pgfscope}%
\pgfpathrectangle{\pgfqpoint{0.530948in}{0.451986in}}{\pgfqpoint{5.356938in}{1.900459in}}%
\pgfusepath{clip}%
\pgfsetbuttcap%
\pgfsetroundjoin%
\pgfsetlinewidth{0.501875pt}%
\definecolor{currentstroke}{rgb}{0.690196,0.690196,0.690196}%
\pgfsetstrokecolor{currentstroke}%
\pgfsetdash{{1.850000pt}{0.800000pt}}{0.000000pt}%
\pgfpathmoveto{\pgfqpoint{3.827525in}{0.451986in}}%
\pgfpathlineto{\pgfqpoint{3.827525in}{2.352445in}}%
\pgfusepath{stroke}%
\end{pgfscope}%
\begin{pgfscope}%
\pgfsetbuttcap%
\pgfsetroundjoin%
\definecolor{currentfill}{rgb}{0.000000,0.000000,0.000000}%
\pgfsetfillcolor{currentfill}%
\pgfsetlinewidth{0.803000pt}%
\definecolor{currentstroke}{rgb}{0.000000,0.000000,0.000000}%
\pgfsetstrokecolor{currentstroke}%
\pgfsetdash{}{0pt}%
\pgfsys@defobject{currentmarker}{\pgfqpoint{0.000000in}{-0.048611in}}{\pgfqpoint{0.000000in}{0.000000in}}{%
\pgfpathmoveto{\pgfqpoint{0.000000in}{0.000000in}}%
\pgfpathlineto{\pgfqpoint{0.000000in}{-0.048611in}}%
\pgfusepath{stroke,fill}%
}%
\begin{pgfscope}%
\pgfsys@transformshift{3.827525in}{0.451986in}%
\pgfsys@useobject{currentmarker}{}%
\end{pgfscope}%
\end{pgfscope}%
\begin{pgfscope}%
\definecolor{textcolor}{rgb}{0.000000,0.000000,0.000000}%
\pgfsetstrokecolor{textcolor}%
\pgfsetfillcolor{textcolor}%
\pgftext[x=3.827525in,y=0.354764in,,top]{\color{textcolor}\rmfamily\fontsize{8.000000}{9.600000}\selectfont \(\displaystyle 1.0\)}%
\end{pgfscope}%
\begin{pgfscope}%
\pgfpathrectangle{\pgfqpoint{0.530948in}{0.451986in}}{\pgfqpoint{5.356938in}{1.900459in}}%
\pgfusepath{clip}%
\pgfsetbuttcap%
\pgfsetroundjoin%
\pgfsetlinewidth{0.501875pt}%
\definecolor{currentstroke}{rgb}{0.690196,0.690196,0.690196}%
\pgfsetstrokecolor{currentstroke}%
\pgfsetdash{{1.850000pt}{0.800000pt}}{0.000000pt}%
\pgfpathmoveto{\pgfqpoint{4.651669in}{0.451986in}}%
\pgfpathlineto{\pgfqpoint{4.651669in}{2.352445in}}%
\pgfusepath{stroke}%
\end{pgfscope}%
\begin{pgfscope}%
\pgfsetbuttcap%
\pgfsetroundjoin%
\definecolor{currentfill}{rgb}{0.000000,0.000000,0.000000}%
\pgfsetfillcolor{currentfill}%
\pgfsetlinewidth{0.803000pt}%
\definecolor{currentstroke}{rgb}{0.000000,0.000000,0.000000}%
\pgfsetstrokecolor{currentstroke}%
\pgfsetdash{}{0pt}%
\pgfsys@defobject{currentmarker}{\pgfqpoint{0.000000in}{-0.048611in}}{\pgfqpoint{0.000000in}{0.000000in}}{%
\pgfpathmoveto{\pgfqpoint{0.000000in}{0.000000in}}%
\pgfpathlineto{\pgfqpoint{0.000000in}{-0.048611in}}%
\pgfusepath{stroke,fill}%
}%
\begin{pgfscope}%
\pgfsys@transformshift{4.651669in}{0.451986in}%
\pgfsys@useobject{currentmarker}{}%
\end{pgfscope}%
\end{pgfscope}%
\begin{pgfscope}%
\definecolor{textcolor}{rgb}{0.000000,0.000000,0.000000}%
\pgfsetstrokecolor{textcolor}%
\pgfsetfillcolor{textcolor}%
\pgftext[x=4.651669in,y=0.354764in,,top]{\color{textcolor}\rmfamily\fontsize{8.000000}{9.600000}\selectfont \(\displaystyle 1.2\)}%
\end{pgfscope}%
\begin{pgfscope}%
\pgfpathrectangle{\pgfqpoint{0.530948in}{0.451986in}}{\pgfqpoint{5.356938in}{1.900459in}}%
\pgfusepath{clip}%
\pgfsetbuttcap%
\pgfsetroundjoin%
\pgfsetlinewidth{0.501875pt}%
\definecolor{currentstroke}{rgb}{0.690196,0.690196,0.690196}%
\pgfsetstrokecolor{currentstroke}%
\pgfsetdash{{1.850000pt}{0.800000pt}}{0.000000pt}%
\pgfpathmoveto{\pgfqpoint{5.475814in}{0.451986in}}%
\pgfpathlineto{\pgfqpoint{5.475814in}{2.352445in}}%
\pgfusepath{stroke}%
\end{pgfscope}%
\begin{pgfscope}%
\pgfsetbuttcap%
\pgfsetroundjoin%
\definecolor{currentfill}{rgb}{0.000000,0.000000,0.000000}%
\pgfsetfillcolor{currentfill}%
\pgfsetlinewidth{0.803000pt}%
\definecolor{currentstroke}{rgb}{0.000000,0.000000,0.000000}%
\pgfsetstrokecolor{currentstroke}%
\pgfsetdash{}{0pt}%
\pgfsys@defobject{currentmarker}{\pgfqpoint{0.000000in}{-0.048611in}}{\pgfqpoint{0.000000in}{0.000000in}}{%
\pgfpathmoveto{\pgfqpoint{0.000000in}{0.000000in}}%
\pgfpathlineto{\pgfqpoint{0.000000in}{-0.048611in}}%
\pgfusepath{stroke,fill}%
}%
\begin{pgfscope}%
\pgfsys@transformshift{5.475814in}{0.451986in}%
\pgfsys@useobject{currentmarker}{}%
\end{pgfscope}%
\end{pgfscope}%
\begin{pgfscope}%
\definecolor{textcolor}{rgb}{0.000000,0.000000,0.000000}%
\pgfsetstrokecolor{textcolor}%
\pgfsetfillcolor{textcolor}%
\pgftext[x=5.475814in,y=0.354764in,,top]{\color{textcolor}\rmfamily\fontsize{8.000000}{9.600000}\selectfont \(\displaystyle 1.4\)}%
\end{pgfscope}%
\begin{pgfscope}%
\pgfpathrectangle{\pgfqpoint{0.530948in}{0.451986in}}{\pgfqpoint{5.356938in}{1.900459in}}%
\pgfusepath{clip}%
\pgfsetbuttcap%
\pgfsetroundjoin%
\pgfsetlinewidth{0.250937pt}%
\definecolor{currentstroke}{rgb}{0.690196,0.690196,0.690196}%
\pgfsetstrokecolor{currentstroke}%
\pgfsetdash{{0.250000pt}{0.412500pt}}{0.000000pt}%
\pgfpathmoveto{\pgfqpoint{0.530948in}{0.451986in}}%
\pgfpathlineto{\pgfqpoint{0.530948in}{2.352445in}}%
\pgfusepath{stroke}%
\end{pgfscope}%
\begin{pgfscope}%
\pgfsetbuttcap%
\pgfsetroundjoin%
\definecolor{currentfill}{rgb}{0.000000,0.000000,0.000000}%
\pgfsetfillcolor{currentfill}%
\pgfsetlinewidth{0.602250pt}%
\definecolor{currentstroke}{rgb}{0.000000,0.000000,0.000000}%
\pgfsetstrokecolor{currentstroke}%
\pgfsetdash{}{0pt}%
\pgfsys@defobject{currentmarker}{\pgfqpoint{0.000000in}{-0.027778in}}{\pgfqpoint{0.000000in}{0.000000in}}{%
\pgfpathmoveto{\pgfqpoint{0.000000in}{0.000000in}}%
\pgfpathlineto{\pgfqpoint{0.000000in}{-0.027778in}}%
\pgfusepath{stroke,fill}%
}%
\begin{pgfscope}%
\pgfsys@transformshift{0.530948in}{0.451986in}%
\pgfsys@useobject{currentmarker}{}%
\end{pgfscope}%
\end{pgfscope}%
\begin{pgfscope}%
\pgfpathrectangle{\pgfqpoint{0.530948in}{0.451986in}}{\pgfqpoint{5.356938in}{1.900459in}}%
\pgfusepath{clip}%
\pgfsetbuttcap%
\pgfsetroundjoin%
\pgfsetlinewidth{0.250937pt}%
\definecolor{currentstroke}{rgb}{0.690196,0.690196,0.690196}%
\pgfsetstrokecolor{currentstroke}%
\pgfsetdash{{0.250000pt}{0.412500pt}}{0.000000pt}%
\pgfpathmoveto{\pgfqpoint{0.943020in}{0.451986in}}%
\pgfpathlineto{\pgfqpoint{0.943020in}{2.352445in}}%
\pgfusepath{stroke}%
\end{pgfscope}%
\begin{pgfscope}%
\pgfsetbuttcap%
\pgfsetroundjoin%
\definecolor{currentfill}{rgb}{0.000000,0.000000,0.000000}%
\pgfsetfillcolor{currentfill}%
\pgfsetlinewidth{0.602250pt}%
\definecolor{currentstroke}{rgb}{0.000000,0.000000,0.000000}%
\pgfsetstrokecolor{currentstroke}%
\pgfsetdash{}{0pt}%
\pgfsys@defobject{currentmarker}{\pgfqpoint{0.000000in}{-0.027778in}}{\pgfqpoint{0.000000in}{0.000000in}}{%
\pgfpathmoveto{\pgfqpoint{0.000000in}{0.000000in}}%
\pgfpathlineto{\pgfqpoint{0.000000in}{-0.027778in}}%
\pgfusepath{stroke,fill}%
}%
\begin{pgfscope}%
\pgfsys@transformshift{0.943020in}{0.451986in}%
\pgfsys@useobject{currentmarker}{}%
\end{pgfscope}%
\end{pgfscope}%
\begin{pgfscope}%
\pgfpathrectangle{\pgfqpoint{0.530948in}{0.451986in}}{\pgfqpoint{5.356938in}{1.900459in}}%
\pgfusepath{clip}%
\pgfsetbuttcap%
\pgfsetroundjoin%
\pgfsetlinewidth{0.250937pt}%
\definecolor{currentstroke}{rgb}{0.690196,0.690196,0.690196}%
\pgfsetstrokecolor{currentstroke}%
\pgfsetdash{{0.250000pt}{0.412500pt}}{0.000000pt}%
\pgfpathmoveto{\pgfqpoint{1.355092in}{0.451986in}}%
\pgfpathlineto{\pgfqpoint{1.355092in}{2.352445in}}%
\pgfusepath{stroke}%
\end{pgfscope}%
\begin{pgfscope}%
\pgfsetbuttcap%
\pgfsetroundjoin%
\definecolor{currentfill}{rgb}{0.000000,0.000000,0.000000}%
\pgfsetfillcolor{currentfill}%
\pgfsetlinewidth{0.602250pt}%
\definecolor{currentstroke}{rgb}{0.000000,0.000000,0.000000}%
\pgfsetstrokecolor{currentstroke}%
\pgfsetdash{}{0pt}%
\pgfsys@defobject{currentmarker}{\pgfqpoint{0.000000in}{-0.027778in}}{\pgfqpoint{0.000000in}{0.000000in}}{%
\pgfpathmoveto{\pgfqpoint{0.000000in}{0.000000in}}%
\pgfpathlineto{\pgfqpoint{0.000000in}{-0.027778in}}%
\pgfusepath{stroke,fill}%
}%
\begin{pgfscope}%
\pgfsys@transformshift{1.355092in}{0.451986in}%
\pgfsys@useobject{currentmarker}{}%
\end{pgfscope}%
\end{pgfscope}%
\begin{pgfscope}%
\pgfpathrectangle{\pgfqpoint{0.530948in}{0.451986in}}{\pgfqpoint{5.356938in}{1.900459in}}%
\pgfusepath{clip}%
\pgfsetbuttcap%
\pgfsetroundjoin%
\pgfsetlinewidth{0.250937pt}%
\definecolor{currentstroke}{rgb}{0.690196,0.690196,0.690196}%
\pgfsetstrokecolor{currentstroke}%
\pgfsetdash{{0.250000pt}{0.412500pt}}{0.000000pt}%
\pgfpathmoveto{\pgfqpoint{1.767164in}{0.451986in}}%
\pgfpathlineto{\pgfqpoint{1.767164in}{2.352445in}}%
\pgfusepath{stroke}%
\end{pgfscope}%
\begin{pgfscope}%
\pgfsetbuttcap%
\pgfsetroundjoin%
\definecolor{currentfill}{rgb}{0.000000,0.000000,0.000000}%
\pgfsetfillcolor{currentfill}%
\pgfsetlinewidth{0.602250pt}%
\definecolor{currentstroke}{rgb}{0.000000,0.000000,0.000000}%
\pgfsetstrokecolor{currentstroke}%
\pgfsetdash{}{0pt}%
\pgfsys@defobject{currentmarker}{\pgfqpoint{0.000000in}{-0.027778in}}{\pgfqpoint{0.000000in}{0.000000in}}{%
\pgfpathmoveto{\pgfqpoint{0.000000in}{0.000000in}}%
\pgfpathlineto{\pgfqpoint{0.000000in}{-0.027778in}}%
\pgfusepath{stroke,fill}%
}%
\begin{pgfscope}%
\pgfsys@transformshift{1.767164in}{0.451986in}%
\pgfsys@useobject{currentmarker}{}%
\end{pgfscope}%
\end{pgfscope}%
\begin{pgfscope}%
\pgfpathrectangle{\pgfqpoint{0.530948in}{0.451986in}}{\pgfqpoint{5.356938in}{1.900459in}}%
\pgfusepath{clip}%
\pgfsetbuttcap%
\pgfsetroundjoin%
\pgfsetlinewidth{0.250937pt}%
\definecolor{currentstroke}{rgb}{0.690196,0.690196,0.690196}%
\pgfsetstrokecolor{currentstroke}%
\pgfsetdash{{0.250000pt}{0.412500pt}}{0.000000pt}%
\pgfpathmoveto{\pgfqpoint{2.179236in}{0.451986in}}%
\pgfpathlineto{\pgfqpoint{2.179236in}{2.352445in}}%
\pgfusepath{stroke}%
\end{pgfscope}%
\begin{pgfscope}%
\pgfsetbuttcap%
\pgfsetroundjoin%
\definecolor{currentfill}{rgb}{0.000000,0.000000,0.000000}%
\pgfsetfillcolor{currentfill}%
\pgfsetlinewidth{0.602250pt}%
\definecolor{currentstroke}{rgb}{0.000000,0.000000,0.000000}%
\pgfsetstrokecolor{currentstroke}%
\pgfsetdash{}{0pt}%
\pgfsys@defobject{currentmarker}{\pgfqpoint{0.000000in}{-0.027778in}}{\pgfqpoint{0.000000in}{0.000000in}}{%
\pgfpathmoveto{\pgfqpoint{0.000000in}{0.000000in}}%
\pgfpathlineto{\pgfqpoint{0.000000in}{-0.027778in}}%
\pgfusepath{stroke,fill}%
}%
\begin{pgfscope}%
\pgfsys@transformshift{2.179236in}{0.451986in}%
\pgfsys@useobject{currentmarker}{}%
\end{pgfscope}%
\end{pgfscope}%
\begin{pgfscope}%
\pgfpathrectangle{\pgfqpoint{0.530948in}{0.451986in}}{\pgfqpoint{5.356938in}{1.900459in}}%
\pgfusepath{clip}%
\pgfsetbuttcap%
\pgfsetroundjoin%
\pgfsetlinewidth{0.250937pt}%
\definecolor{currentstroke}{rgb}{0.690196,0.690196,0.690196}%
\pgfsetstrokecolor{currentstroke}%
\pgfsetdash{{0.250000pt}{0.412500pt}}{0.000000pt}%
\pgfpathmoveto{\pgfqpoint{2.591308in}{0.451986in}}%
\pgfpathlineto{\pgfqpoint{2.591308in}{2.352445in}}%
\pgfusepath{stroke}%
\end{pgfscope}%
\begin{pgfscope}%
\pgfsetbuttcap%
\pgfsetroundjoin%
\definecolor{currentfill}{rgb}{0.000000,0.000000,0.000000}%
\pgfsetfillcolor{currentfill}%
\pgfsetlinewidth{0.602250pt}%
\definecolor{currentstroke}{rgb}{0.000000,0.000000,0.000000}%
\pgfsetstrokecolor{currentstroke}%
\pgfsetdash{}{0pt}%
\pgfsys@defobject{currentmarker}{\pgfqpoint{0.000000in}{-0.027778in}}{\pgfqpoint{0.000000in}{0.000000in}}{%
\pgfpathmoveto{\pgfqpoint{0.000000in}{0.000000in}}%
\pgfpathlineto{\pgfqpoint{0.000000in}{-0.027778in}}%
\pgfusepath{stroke,fill}%
}%
\begin{pgfscope}%
\pgfsys@transformshift{2.591308in}{0.451986in}%
\pgfsys@useobject{currentmarker}{}%
\end{pgfscope}%
\end{pgfscope}%
\begin{pgfscope}%
\pgfpathrectangle{\pgfqpoint{0.530948in}{0.451986in}}{\pgfqpoint{5.356938in}{1.900459in}}%
\pgfusepath{clip}%
\pgfsetbuttcap%
\pgfsetroundjoin%
\pgfsetlinewidth{0.250937pt}%
\definecolor{currentstroke}{rgb}{0.690196,0.690196,0.690196}%
\pgfsetstrokecolor{currentstroke}%
\pgfsetdash{{0.250000pt}{0.412500pt}}{0.000000pt}%
\pgfpathmoveto{\pgfqpoint{3.003381in}{0.451986in}}%
\pgfpathlineto{\pgfqpoint{3.003381in}{2.352445in}}%
\pgfusepath{stroke}%
\end{pgfscope}%
\begin{pgfscope}%
\pgfsetbuttcap%
\pgfsetroundjoin%
\definecolor{currentfill}{rgb}{0.000000,0.000000,0.000000}%
\pgfsetfillcolor{currentfill}%
\pgfsetlinewidth{0.602250pt}%
\definecolor{currentstroke}{rgb}{0.000000,0.000000,0.000000}%
\pgfsetstrokecolor{currentstroke}%
\pgfsetdash{}{0pt}%
\pgfsys@defobject{currentmarker}{\pgfqpoint{0.000000in}{-0.027778in}}{\pgfqpoint{0.000000in}{0.000000in}}{%
\pgfpathmoveto{\pgfqpoint{0.000000in}{0.000000in}}%
\pgfpathlineto{\pgfqpoint{0.000000in}{-0.027778in}}%
\pgfusepath{stroke,fill}%
}%
\begin{pgfscope}%
\pgfsys@transformshift{3.003381in}{0.451986in}%
\pgfsys@useobject{currentmarker}{}%
\end{pgfscope}%
\end{pgfscope}%
\begin{pgfscope}%
\pgfpathrectangle{\pgfqpoint{0.530948in}{0.451986in}}{\pgfqpoint{5.356938in}{1.900459in}}%
\pgfusepath{clip}%
\pgfsetbuttcap%
\pgfsetroundjoin%
\pgfsetlinewidth{0.250937pt}%
\definecolor{currentstroke}{rgb}{0.690196,0.690196,0.690196}%
\pgfsetstrokecolor{currentstroke}%
\pgfsetdash{{0.250000pt}{0.412500pt}}{0.000000pt}%
\pgfpathmoveto{\pgfqpoint{3.415453in}{0.451986in}}%
\pgfpathlineto{\pgfqpoint{3.415453in}{2.352445in}}%
\pgfusepath{stroke}%
\end{pgfscope}%
\begin{pgfscope}%
\pgfsetbuttcap%
\pgfsetroundjoin%
\definecolor{currentfill}{rgb}{0.000000,0.000000,0.000000}%
\pgfsetfillcolor{currentfill}%
\pgfsetlinewidth{0.602250pt}%
\definecolor{currentstroke}{rgb}{0.000000,0.000000,0.000000}%
\pgfsetstrokecolor{currentstroke}%
\pgfsetdash{}{0pt}%
\pgfsys@defobject{currentmarker}{\pgfqpoint{0.000000in}{-0.027778in}}{\pgfqpoint{0.000000in}{0.000000in}}{%
\pgfpathmoveto{\pgfqpoint{0.000000in}{0.000000in}}%
\pgfpathlineto{\pgfqpoint{0.000000in}{-0.027778in}}%
\pgfusepath{stroke,fill}%
}%
\begin{pgfscope}%
\pgfsys@transformshift{3.415453in}{0.451986in}%
\pgfsys@useobject{currentmarker}{}%
\end{pgfscope}%
\end{pgfscope}%
\begin{pgfscope}%
\pgfpathrectangle{\pgfqpoint{0.530948in}{0.451986in}}{\pgfqpoint{5.356938in}{1.900459in}}%
\pgfusepath{clip}%
\pgfsetbuttcap%
\pgfsetroundjoin%
\pgfsetlinewidth{0.250937pt}%
\definecolor{currentstroke}{rgb}{0.690196,0.690196,0.690196}%
\pgfsetstrokecolor{currentstroke}%
\pgfsetdash{{0.250000pt}{0.412500pt}}{0.000000pt}%
\pgfpathmoveto{\pgfqpoint{3.827525in}{0.451986in}}%
\pgfpathlineto{\pgfqpoint{3.827525in}{2.352445in}}%
\pgfusepath{stroke}%
\end{pgfscope}%
\begin{pgfscope}%
\pgfsetbuttcap%
\pgfsetroundjoin%
\definecolor{currentfill}{rgb}{0.000000,0.000000,0.000000}%
\pgfsetfillcolor{currentfill}%
\pgfsetlinewidth{0.602250pt}%
\definecolor{currentstroke}{rgb}{0.000000,0.000000,0.000000}%
\pgfsetstrokecolor{currentstroke}%
\pgfsetdash{}{0pt}%
\pgfsys@defobject{currentmarker}{\pgfqpoint{0.000000in}{-0.027778in}}{\pgfqpoint{0.000000in}{0.000000in}}{%
\pgfpathmoveto{\pgfqpoint{0.000000in}{0.000000in}}%
\pgfpathlineto{\pgfqpoint{0.000000in}{-0.027778in}}%
\pgfusepath{stroke,fill}%
}%
\begin{pgfscope}%
\pgfsys@transformshift{3.827525in}{0.451986in}%
\pgfsys@useobject{currentmarker}{}%
\end{pgfscope}%
\end{pgfscope}%
\begin{pgfscope}%
\pgfpathrectangle{\pgfqpoint{0.530948in}{0.451986in}}{\pgfqpoint{5.356938in}{1.900459in}}%
\pgfusepath{clip}%
\pgfsetbuttcap%
\pgfsetroundjoin%
\pgfsetlinewidth{0.250937pt}%
\definecolor{currentstroke}{rgb}{0.690196,0.690196,0.690196}%
\pgfsetstrokecolor{currentstroke}%
\pgfsetdash{{0.250000pt}{0.412500pt}}{0.000000pt}%
\pgfpathmoveto{\pgfqpoint{4.239597in}{0.451986in}}%
\pgfpathlineto{\pgfqpoint{4.239597in}{2.352445in}}%
\pgfusepath{stroke}%
\end{pgfscope}%
\begin{pgfscope}%
\pgfsetbuttcap%
\pgfsetroundjoin%
\definecolor{currentfill}{rgb}{0.000000,0.000000,0.000000}%
\pgfsetfillcolor{currentfill}%
\pgfsetlinewidth{0.602250pt}%
\definecolor{currentstroke}{rgb}{0.000000,0.000000,0.000000}%
\pgfsetstrokecolor{currentstroke}%
\pgfsetdash{}{0pt}%
\pgfsys@defobject{currentmarker}{\pgfqpoint{0.000000in}{-0.027778in}}{\pgfqpoint{0.000000in}{0.000000in}}{%
\pgfpathmoveto{\pgfqpoint{0.000000in}{0.000000in}}%
\pgfpathlineto{\pgfqpoint{0.000000in}{-0.027778in}}%
\pgfusepath{stroke,fill}%
}%
\begin{pgfscope}%
\pgfsys@transformshift{4.239597in}{0.451986in}%
\pgfsys@useobject{currentmarker}{}%
\end{pgfscope}%
\end{pgfscope}%
\begin{pgfscope}%
\pgfpathrectangle{\pgfqpoint{0.530948in}{0.451986in}}{\pgfqpoint{5.356938in}{1.900459in}}%
\pgfusepath{clip}%
\pgfsetbuttcap%
\pgfsetroundjoin%
\pgfsetlinewidth{0.250937pt}%
\definecolor{currentstroke}{rgb}{0.690196,0.690196,0.690196}%
\pgfsetstrokecolor{currentstroke}%
\pgfsetdash{{0.250000pt}{0.412500pt}}{0.000000pt}%
\pgfpathmoveto{\pgfqpoint{4.651669in}{0.451986in}}%
\pgfpathlineto{\pgfqpoint{4.651669in}{2.352445in}}%
\pgfusepath{stroke}%
\end{pgfscope}%
\begin{pgfscope}%
\pgfsetbuttcap%
\pgfsetroundjoin%
\definecolor{currentfill}{rgb}{0.000000,0.000000,0.000000}%
\pgfsetfillcolor{currentfill}%
\pgfsetlinewidth{0.602250pt}%
\definecolor{currentstroke}{rgb}{0.000000,0.000000,0.000000}%
\pgfsetstrokecolor{currentstroke}%
\pgfsetdash{}{0pt}%
\pgfsys@defobject{currentmarker}{\pgfqpoint{0.000000in}{-0.027778in}}{\pgfqpoint{0.000000in}{0.000000in}}{%
\pgfpathmoveto{\pgfqpoint{0.000000in}{0.000000in}}%
\pgfpathlineto{\pgfqpoint{0.000000in}{-0.027778in}}%
\pgfusepath{stroke,fill}%
}%
\begin{pgfscope}%
\pgfsys@transformshift{4.651669in}{0.451986in}%
\pgfsys@useobject{currentmarker}{}%
\end{pgfscope}%
\end{pgfscope}%
\begin{pgfscope}%
\pgfpathrectangle{\pgfqpoint{0.530948in}{0.451986in}}{\pgfqpoint{5.356938in}{1.900459in}}%
\pgfusepath{clip}%
\pgfsetbuttcap%
\pgfsetroundjoin%
\pgfsetlinewidth{0.250937pt}%
\definecolor{currentstroke}{rgb}{0.690196,0.690196,0.690196}%
\pgfsetstrokecolor{currentstroke}%
\pgfsetdash{{0.250000pt}{0.412500pt}}{0.000000pt}%
\pgfpathmoveto{\pgfqpoint{5.063741in}{0.451986in}}%
\pgfpathlineto{\pgfqpoint{5.063741in}{2.352445in}}%
\pgfusepath{stroke}%
\end{pgfscope}%
\begin{pgfscope}%
\pgfsetbuttcap%
\pgfsetroundjoin%
\definecolor{currentfill}{rgb}{0.000000,0.000000,0.000000}%
\pgfsetfillcolor{currentfill}%
\pgfsetlinewidth{0.602250pt}%
\definecolor{currentstroke}{rgb}{0.000000,0.000000,0.000000}%
\pgfsetstrokecolor{currentstroke}%
\pgfsetdash{}{0pt}%
\pgfsys@defobject{currentmarker}{\pgfqpoint{0.000000in}{-0.027778in}}{\pgfqpoint{0.000000in}{0.000000in}}{%
\pgfpathmoveto{\pgfqpoint{0.000000in}{0.000000in}}%
\pgfpathlineto{\pgfqpoint{0.000000in}{-0.027778in}}%
\pgfusepath{stroke,fill}%
}%
\begin{pgfscope}%
\pgfsys@transformshift{5.063741in}{0.451986in}%
\pgfsys@useobject{currentmarker}{}%
\end{pgfscope}%
\end{pgfscope}%
\begin{pgfscope}%
\pgfpathrectangle{\pgfqpoint{0.530948in}{0.451986in}}{\pgfqpoint{5.356938in}{1.900459in}}%
\pgfusepath{clip}%
\pgfsetbuttcap%
\pgfsetroundjoin%
\pgfsetlinewidth{0.250937pt}%
\definecolor{currentstroke}{rgb}{0.690196,0.690196,0.690196}%
\pgfsetstrokecolor{currentstroke}%
\pgfsetdash{{0.250000pt}{0.412500pt}}{0.000000pt}%
\pgfpathmoveto{\pgfqpoint{5.475814in}{0.451986in}}%
\pgfpathlineto{\pgfqpoint{5.475814in}{2.352445in}}%
\pgfusepath{stroke}%
\end{pgfscope}%
\begin{pgfscope}%
\pgfsetbuttcap%
\pgfsetroundjoin%
\definecolor{currentfill}{rgb}{0.000000,0.000000,0.000000}%
\pgfsetfillcolor{currentfill}%
\pgfsetlinewidth{0.602250pt}%
\definecolor{currentstroke}{rgb}{0.000000,0.000000,0.000000}%
\pgfsetstrokecolor{currentstroke}%
\pgfsetdash{}{0pt}%
\pgfsys@defobject{currentmarker}{\pgfqpoint{0.000000in}{-0.027778in}}{\pgfqpoint{0.000000in}{0.000000in}}{%
\pgfpathmoveto{\pgfqpoint{0.000000in}{0.000000in}}%
\pgfpathlineto{\pgfqpoint{0.000000in}{-0.027778in}}%
\pgfusepath{stroke,fill}%
}%
\begin{pgfscope}%
\pgfsys@transformshift{5.475814in}{0.451986in}%
\pgfsys@useobject{currentmarker}{}%
\end{pgfscope}%
\end{pgfscope}%
\begin{pgfscope}%
\pgfpathrectangle{\pgfqpoint{0.530948in}{0.451986in}}{\pgfqpoint{5.356938in}{1.900459in}}%
\pgfusepath{clip}%
\pgfsetbuttcap%
\pgfsetroundjoin%
\pgfsetlinewidth{0.250937pt}%
\definecolor{currentstroke}{rgb}{0.690196,0.690196,0.690196}%
\pgfsetstrokecolor{currentstroke}%
\pgfsetdash{{0.250000pt}{0.412500pt}}{0.000000pt}%
\pgfpathmoveto{\pgfqpoint{5.887886in}{0.451986in}}%
\pgfpathlineto{\pgfqpoint{5.887886in}{2.352445in}}%
\pgfusepath{stroke}%
\end{pgfscope}%
\begin{pgfscope}%
\pgfsetbuttcap%
\pgfsetroundjoin%
\definecolor{currentfill}{rgb}{0.000000,0.000000,0.000000}%
\pgfsetfillcolor{currentfill}%
\pgfsetlinewidth{0.602250pt}%
\definecolor{currentstroke}{rgb}{0.000000,0.000000,0.000000}%
\pgfsetstrokecolor{currentstroke}%
\pgfsetdash{}{0pt}%
\pgfsys@defobject{currentmarker}{\pgfqpoint{0.000000in}{-0.027778in}}{\pgfqpoint{0.000000in}{0.000000in}}{%
\pgfpathmoveto{\pgfqpoint{0.000000in}{0.000000in}}%
\pgfpathlineto{\pgfqpoint{0.000000in}{-0.027778in}}%
\pgfusepath{stroke,fill}%
}%
\begin{pgfscope}%
\pgfsys@transformshift{5.887886in}{0.451986in}%
\pgfsys@useobject{currentmarker}{}%
\end{pgfscope}%
\end{pgfscope}%
\begin{pgfscope}%
\definecolor{textcolor}{rgb}{0.000000,0.000000,0.000000}%
\pgfsetstrokecolor{textcolor}%
\pgfsetfillcolor{textcolor}%
\pgftext[x=3.209417in,y=0.201084in,,top]{\color{textcolor}\rmfamily\fontsize{8.000000}{9.600000}\selectfont \textbf{rez. Magnetfeld 1/B (1/T)}}%
\end{pgfscope}%
\begin{pgfscope}%
\pgfpathrectangle{\pgfqpoint{0.530948in}{0.451986in}}{\pgfqpoint{5.356938in}{1.900459in}}%
\pgfusepath{clip}%
\pgfsetbuttcap%
\pgfsetroundjoin%
\pgfsetlinewidth{0.501875pt}%
\definecolor{currentstroke}{rgb}{0.690196,0.690196,0.690196}%
\pgfsetstrokecolor{currentstroke}%
\pgfsetdash{{1.850000pt}{0.800000pt}}{0.000000pt}%
\pgfpathmoveto{\pgfqpoint{0.530948in}{0.519500in}}%
\pgfpathlineto{\pgfqpoint{5.887886in}{0.519500in}}%
\pgfusepath{stroke}%
\end{pgfscope}%
\begin{pgfscope}%
\pgfsetbuttcap%
\pgfsetroundjoin%
\definecolor{currentfill}{rgb}{0.000000,0.000000,0.000000}%
\pgfsetfillcolor{currentfill}%
\pgfsetlinewidth{0.803000pt}%
\definecolor{currentstroke}{rgb}{0.000000,0.000000,0.000000}%
\pgfsetstrokecolor{currentstroke}%
\pgfsetdash{}{0pt}%
\pgfsys@defobject{currentmarker}{\pgfqpoint{-0.048611in}{0.000000in}}{\pgfqpoint{0.000000in}{0.000000in}}{%
\pgfpathmoveto{\pgfqpoint{0.000000in}{0.000000in}}%
\pgfpathlineto{\pgfqpoint{-0.048611in}{0.000000in}}%
\pgfusepath{stroke,fill}%
}%
\begin{pgfscope}%
\pgfsys@transformshift{0.530948in}{0.519500in}%
\pgfsys@useobject{currentmarker}{}%
\end{pgfscope}%
\end{pgfscope}%
\begin{pgfscope}%
\definecolor{textcolor}{rgb}{0.000000,0.000000,0.000000}%
\pgfsetstrokecolor{textcolor}%
\pgfsetfillcolor{textcolor}%
\pgftext[x=0.374697in,y=0.481237in,left,base]{\color{textcolor}\rmfamily\fontsize{8.000000}{9.600000}\selectfont \(\displaystyle 0\)}%
\end{pgfscope}%
\begin{pgfscope}%
\pgfpathrectangle{\pgfqpoint{0.530948in}{0.451986in}}{\pgfqpoint{5.356938in}{1.900459in}}%
\pgfusepath{clip}%
\pgfsetbuttcap%
\pgfsetroundjoin%
\pgfsetlinewidth{0.501875pt}%
\definecolor{currentstroke}{rgb}{0.690196,0.690196,0.690196}%
\pgfsetstrokecolor{currentstroke}%
\pgfsetdash{{1.850000pt}{0.800000pt}}{0.000000pt}%
\pgfpathmoveto{\pgfqpoint{0.530948in}{1.027338in}}%
\pgfpathlineto{\pgfqpoint{5.887886in}{1.027338in}}%
\pgfusepath{stroke}%
\end{pgfscope}%
\begin{pgfscope}%
\pgfsetbuttcap%
\pgfsetroundjoin%
\definecolor{currentfill}{rgb}{0.000000,0.000000,0.000000}%
\pgfsetfillcolor{currentfill}%
\pgfsetlinewidth{0.803000pt}%
\definecolor{currentstroke}{rgb}{0.000000,0.000000,0.000000}%
\pgfsetstrokecolor{currentstroke}%
\pgfsetdash{}{0pt}%
\pgfsys@defobject{currentmarker}{\pgfqpoint{-0.048611in}{0.000000in}}{\pgfqpoint{0.000000in}{0.000000in}}{%
\pgfpathmoveto{\pgfqpoint{0.000000in}{0.000000in}}%
\pgfpathlineto{\pgfqpoint{-0.048611in}{0.000000in}}%
\pgfusepath{stroke,fill}%
}%
\begin{pgfscope}%
\pgfsys@transformshift{0.530948in}{1.027338in}%
\pgfsys@useobject{currentmarker}{}%
\end{pgfscope}%
\end{pgfscope}%
\begin{pgfscope}%
\definecolor{textcolor}{rgb}{0.000000,0.000000,0.000000}%
\pgfsetstrokecolor{textcolor}%
\pgfsetfillcolor{textcolor}%
\pgftext[x=0.256640in,y=0.989076in,left,base]{\color{textcolor}\rmfamily\fontsize{8.000000}{9.600000}\selectfont \(\displaystyle 200\)}%
\end{pgfscope}%
\begin{pgfscope}%
\pgfpathrectangle{\pgfqpoint{0.530948in}{0.451986in}}{\pgfqpoint{5.356938in}{1.900459in}}%
\pgfusepath{clip}%
\pgfsetbuttcap%
\pgfsetroundjoin%
\pgfsetlinewidth{0.501875pt}%
\definecolor{currentstroke}{rgb}{0.690196,0.690196,0.690196}%
\pgfsetstrokecolor{currentstroke}%
\pgfsetdash{{1.850000pt}{0.800000pt}}{0.000000pt}%
\pgfpathmoveto{\pgfqpoint{0.530948in}{1.535177in}}%
\pgfpathlineto{\pgfqpoint{5.887886in}{1.535177in}}%
\pgfusepath{stroke}%
\end{pgfscope}%
\begin{pgfscope}%
\pgfsetbuttcap%
\pgfsetroundjoin%
\definecolor{currentfill}{rgb}{0.000000,0.000000,0.000000}%
\pgfsetfillcolor{currentfill}%
\pgfsetlinewidth{0.803000pt}%
\definecolor{currentstroke}{rgb}{0.000000,0.000000,0.000000}%
\pgfsetstrokecolor{currentstroke}%
\pgfsetdash{}{0pt}%
\pgfsys@defobject{currentmarker}{\pgfqpoint{-0.048611in}{0.000000in}}{\pgfqpoint{0.000000in}{0.000000in}}{%
\pgfpathmoveto{\pgfqpoint{0.000000in}{0.000000in}}%
\pgfpathlineto{\pgfqpoint{-0.048611in}{0.000000in}}%
\pgfusepath{stroke,fill}%
}%
\begin{pgfscope}%
\pgfsys@transformshift{0.530948in}{1.535177in}%
\pgfsys@useobject{currentmarker}{}%
\end{pgfscope}%
\end{pgfscope}%
\begin{pgfscope}%
\definecolor{textcolor}{rgb}{0.000000,0.000000,0.000000}%
\pgfsetstrokecolor{textcolor}%
\pgfsetfillcolor{textcolor}%
\pgftext[x=0.256640in,y=1.496915in,left,base]{\color{textcolor}\rmfamily\fontsize{8.000000}{9.600000}\selectfont \(\displaystyle 400\)}%
\end{pgfscope}%
\begin{pgfscope}%
\pgfpathrectangle{\pgfqpoint{0.530948in}{0.451986in}}{\pgfqpoint{5.356938in}{1.900459in}}%
\pgfusepath{clip}%
\pgfsetbuttcap%
\pgfsetroundjoin%
\pgfsetlinewidth{0.501875pt}%
\definecolor{currentstroke}{rgb}{0.690196,0.690196,0.690196}%
\pgfsetstrokecolor{currentstroke}%
\pgfsetdash{{1.850000pt}{0.800000pt}}{0.000000pt}%
\pgfpathmoveto{\pgfqpoint{0.530948in}{2.043015in}}%
\pgfpathlineto{\pgfqpoint{5.887886in}{2.043015in}}%
\pgfusepath{stroke}%
\end{pgfscope}%
\begin{pgfscope}%
\pgfsetbuttcap%
\pgfsetroundjoin%
\definecolor{currentfill}{rgb}{0.000000,0.000000,0.000000}%
\pgfsetfillcolor{currentfill}%
\pgfsetlinewidth{0.803000pt}%
\definecolor{currentstroke}{rgb}{0.000000,0.000000,0.000000}%
\pgfsetstrokecolor{currentstroke}%
\pgfsetdash{}{0pt}%
\pgfsys@defobject{currentmarker}{\pgfqpoint{-0.048611in}{0.000000in}}{\pgfqpoint{0.000000in}{0.000000in}}{%
\pgfpathmoveto{\pgfqpoint{0.000000in}{0.000000in}}%
\pgfpathlineto{\pgfqpoint{-0.048611in}{0.000000in}}%
\pgfusepath{stroke,fill}%
}%
\begin{pgfscope}%
\pgfsys@transformshift{0.530948in}{2.043015in}%
\pgfsys@useobject{currentmarker}{}%
\end{pgfscope}%
\end{pgfscope}%
\begin{pgfscope}%
\definecolor{textcolor}{rgb}{0.000000,0.000000,0.000000}%
\pgfsetstrokecolor{textcolor}%
\pgfsetfillcolor{textcolor}%
\pgftext[x=0.256640in,y=2.004753in,left,base]{\color{textcolor}\rmfamily\fontsize{8.000000}{9.600000}\selectfont \(\displaystyle 600\)}%
\end{pgfscope}%
\begin{pgfscope}%
\pgfpathrectangle{\pgfqpoint{0.530948in}{0.451986in}}{\pgfqpoint{5.356938in}{1.900459in}}%
\pgfusepath{clip}%
\pgfsetbuttcap%
\pgfsetroundjoin%
\pgfsetlinewidth{0.250937pt}%
\definecolor{currentstroke}{rgb}{0.690196,0.690196,0.690196}%
\pgfsetstrokecolor{currentstroke}%
\pgfsetdash{{0.250000pt}{0.412500pt}}{0.000000pt}%
\pgfpathmoveto{\pgfqpoint{0.530948in}{0.519500in}}%
\pgfpathlineto{\pgfqpoint{5.887886in}{0.519500in}}%
\pgfusepath{stroke}%
\end{pgfscope}%
\begin{pgfscope}%
\pgfsetbuttcap%
\pgfsetroundjoin%
\definecolor{currentfill}{rgb}{0.000000,0.000000,0.000000}%
\pgfsetfillcolor{currentfill}%
\pgfsetlinewidth{0.602250pt}%
\definecolor{currentstroke}{rgb}{0.000000,0.000000,0.000000}%
\pgfsetstrokecolor{currentstroke}%
\pgfsetdash{}{0pt}%
\pgfsys@defobject{currentmarker}{\pgfqpoint{-0.027778in}{0.000000in}}{\pgfqpoint{0.000000in}{0.000000in}}{%
\pgfpathmoveto{\pgfqpoint{0.000000in}{0.000000in}}%
\pgfpathlineto{\pgfqpoint{-0.027778in}{0.000000in}}%
\pgfusepath{stroke,fill}%
}%
\begin{pgfscope}%
\pgfsys@transformshift{0.530948in}{0.519500in}%
\pgfsys@useobject{currentmarker}{}%
\end{pgfscope}%
\end{pgfscope}%
\begin{pgfscope}%
\pgfpathrectangle{\pgfqpoint{0.530948in}{0.451986in}}{\pgfqpoint{5.356938in}{1.900459in}}%
\pgfusepath{clip}%
\pgfsetbuttcap%
\pgfsetroundjoin%
\pgfsetlinewidth{0.250937pt}%
\definecolor{currentstroke}{rgb}{0.690196,0.690196,0.690196}%
\pgfsetstrokecolor{currentstroke}%
\pgfsetdash{{0.250000pt}{0.412500pt}}{0.000000pt}%
\pgfpathmoveto{\pgfqpoint{0.530948in}{0.773419in}}%
\pgfpathlineto{\pgfqpoint{5.887886in}{0.773419in}}%
\pgfusepath{stroke}%
\end{pgfscope}%
\begin{pgfscope}%
\pgfsetbuttcap%
\pgfsetroundjoin%
\definecolor{currentfill}{rgb}{0.000000,0.000000,0.000000}%
\pgfsetfillcolor{currentfill}%
\pgfsetlinewidth{0.602250pt}%
\definecolor{currentstroke}{rgb}{0.000000,0.000000,0.000000}%
\pgfsetstrokecolor{currentstroke}%
\pgfsetdash{}{0pt}%
\pgfsys@defobject{currentmarker}{\pgfqpoint{-0.027778in}{0.000000in}}{\pgfqpoint{0.000000in}{0.000000in}}{%
\pgfpathmoveto{\pgfqpoint{0.000000in}{0.000000in}}%
\pgfpathlineto{\pgfqpoint{-0.027778in}{0.000000in}}%
\pgfusepath{stroke,fill}%
}%
\begin{pgfscope}%
\pgfsys@transformshift{0.530948in}{0.773419in}%
\pgfsys@useobject{currentmarker}{}%
\end{pgfscope}%
\end{pgfscope}%
\begin{pgfscope}%
\pgfpathrectangle{\pgfqpoint{0.530948in}{0.451986in}}{\pgfqpoint{5.356938in}{1.900459in}}%
\pgfusepath{clip}%
\pgfsetbuttcap%
\pgfsetroundjoin%
\pgfsetlinewidth{0.250937pt}%
\definecolor{currentstroke}{rgb}{0.690196,0.690196,0.690196}%
\pgfsetstrokecolor{currentstroke}%
\pgfsetdash{{0.250000pt}{0.412500pt}}{0.000000pt}%
\pgfpathmoveto{\pgfqpoint{0.530948in}{1.027338in}}%
\pgfpathlineto{\pgfqpoint{5.887886in}{1.027338in}}%
\pgfusepath{stroke}%
\end{pgfscope}%
\begin{pgfscope}%
\pgfsetbuttcap%
\pgfsetroundjoin%
\definecolor{currentfill}{rgb}{0.000000,0.000000,0.000000}%
\pgfsetfillcolor{currentfill}%
\pgfsetlinewidth{0.602250pt}%
\definecolor{currentstroke}{rgb}{0.000000,0.000000,0.000000}%
\pgfsetstrokecolor{currentstroke}%
\pgfsetdash{}{0pt}%
\pgfsys@defobject{currentmarker}{\pgfqpoint{-0.027778in}{0.000000in}}{\pgfqpoint{0.000000in}{0.000000in}}{%
\pgfpathmoveto{\pgfqpoint{0.000000in}{0.000000in}}%
\pgfpathlineto{\pgfqpoint{-0.027778in}{0.000000in}}%
\pgfusepath{stroke,fill}%
}%
\begin{pgfscope}%
\pgfsys@transformshift{0.530948in}{1.027338in}%
\pgfsys@useobject{currentmarker}{}%
\end{pgfscope}%
\end{pgfscope}%
\begin{pgfscope}%
\pgfpathrectangle{\pgfqpoint{0.530948in}{0.451986in}}{\pgfqpoint{5.356938in}{1.900459in}}%
\pgfusepath{clip}%
\pgfsetbuttcap%
\pgfsetroundjoin%
\pgfsetlinewidth{0.250937pt}%
\definecolor{currentstroke}{rgb}{0.690196,0.690196,0.690196}%
\pgfsetstrokecolor{currentstroke}%
\pgfsetdash{{0.250000pt}{0.412500pt}}{0.000000pt}%
\pgfpathmoveto{\pgfqpoint{0.530948in}{1.281258in}}%
\pgfpathlineto{\pgfqpoint{5.887886in}{1.281258in}}%
\pgfusepath{stroke}%
\end{pgfscope}%
\begin{pgfscope}%
\pgfsetbuttcap%
\pgfsetroundjoin%
\definecolor{currentfill}{rgb}{0.000000,0.000000,0.000000}%
\pgfsetfillcolor{currentfill}%
\pgfsetlinewidth{0.602250pt}%
\definecolor{currentstroke}{rgb}{0.000000,0.000000,0.000000}%
\pgfsetstrokecolor{currentstroke}%
\pgfsetdash{}{0pt}%
\pgfsys@defobject{currentmarker}{\pgfqpoint{-0.027778in}{0.000000in}}{\pgfqpoint{0.000000in}{0.000000in}}{%
\pgfpathmoveto{\pgfqpoint{0.000000in}{0.000000in}}%
\pgfpathlineto{\pgfqpoint{-0.027778in}{0.000000in}}%
\pgfusepath{stroke,fill}%
}%
\begin{pgfscope}%
\pgfsys@transformshift{0.530948in}{1.281258in}%
\pgfsys@useobject{currentmarker}{}%
\end{pgfscope}%
\end{pgfscope}%
\begin{pgfscope}%
\pgfpathrectangle{\pgfqpoint{0.530948in}{0.451986in}}{\pgfqpoint{5.356938in}{1.900459in}}%
\pgfusepath{clip}%
\pgfsetbuttcap%
\pgfsetroundjoin%
\pgfsetlinewidth{0.250937pt}%
\definecolor{currentstroke}{rgb}{0.690196,0.690196,0.690196}%
\pgfsetstrokecolor{currentstroke}%
\pgfsetdash{{0.250000pt}{0.412500pt}}{0.000000pt}%
\pgfpathmoveto{\pgfqpoint{0.530948in}{1.535177in}}%
\pgfpathlineto{\pgfqpoint{5.887886in}{1.535177in}}%
\pgfusepath{stroke}%
\end{pgfscope}%
\begin{pgfscope}%
\pgfsetbuttcap%
\pgfsetroundjoin%
\definecolor{currentfill}{rgb}{0.000000,0.000000,0.000000}%
\pgfsetfillcolor{currentfill}%
\pgfsetlinewidth{0.602250pt}%
\definecolor{currentstroke}{rgb}{0.000000,0.000000,0.000000}%
\pgfsetstrokecolor{currentstroke}%
\pgfsetdash{}{0pt}%
\pgfsys@defobject{currentmarker}{\pgfqpoint{-0.027778in}{0.000000in}}{\pgfqpoint{0.000000in}{0.000000in}}{%
\pgfpathmoveto{\pgfqpoint{0.000000in}{0.000000in}}%
\pgfpathlineto{\pgfqpoint{-0.027778in}{0.000000in}}%
\pgfusepath{stroke,fill}%
}%
\begin{pgfscope}%
\pgfsys@transformshift{0.530948in}{1.535177in}%
\pgfsys@useobject{currentmarker}{}%
\end{pgfscope}%
\end{pgfscope}%
\begin{pgfscope}%
\pgfpathrectangle{\pgfqpoint{0.530948in}{0.451986in}}{\pgfqpoint{5.356938in}{1.900459in}}%
\pgfusepath{clip}%
\pgfsetbuttcap%
\pgfsetroundjoin%
\pgfsetlinewidth{0.250937pt}%
\definecolor{currentstroke}{rgb}{0.690196,0.690196,0.690196}%
\pgfsetstrokecolor{currentstroke}%
\pgfsetdash{{0.250000pt}{0.412500pt}}{0.000000pt}%
\pgfpathmoveto{\pgfqpoint{0.530948in}{1.789096in}}%
\pgfpathlineto{\pgfqpoint{5.887886in}{1.789096in}}%
\pgfusepath{stroke}%
\end{pgfscope}%
\begin{pgfscope}%
\pgfsetbuttcap%
\pgfsetroundjoin%
\definecolor{currentfill}{rgb}{0.000000,0.000000,0.000000}%
\pgfsetfillcolor{currentfill}%
\pgfsetlinewidth{0.602250pt}%
\definecolor{currentstroke}{rgb}{0.000000,0.000000,0.000000}%
\pgfsetstrokecolor{currentstroke}%
\pgfsetdash{}{0pt}%
\pgfsys@defobject{currentmarker}{\pgfqpoint{-0.027778in}{0.000000in}}{\pgfqpoint{0.000000in}{0.000000in}}{%
\pgfpathmoveto{\pgfqpoint{0.000000in}{0.000000in}}%
\pgfpathlineto{\pgfqpoint{-0.027778in}{0.000000in}}%
\pgfusepath{stroke,fill}%
}%
\begin{pgfscope}%
\pgfsys@transformshift{0.530948in}{1.789096in}%
\pgfsys@useobject{currentmarker}{}%
\end{pgfscope}%
\end{pgfscope}%
\begin{pgfscope}%
\pgfpathrectangle{\pgfqpoint{0.530948in}{0.451986in}}{\pgfqpoint{5.356938in}{1.900459in}}%
\pgfusepath{clip}%
\pgfsetbuttcap%
\pgfsetroundjoin%
\pgfsetlinewidth{0.250937pt}%
\definecolor{currentstroke}{rgb}{0.690196,0.690196,0.690196}%
\pgfsetstrokecolor{currentstroke}%
\pgfsetdash{{0.250000pt}{0.412500pt}}{0.000000pt}%
\pgfpathmoveto{\pgfqpoint{0.530948in}{2.043015in}}%
\pgfpathlineto{\pgfqpoint{5.887886in}{2.043015in}}%
\pgfusepath{stroke}%
\end{pgfscope}%
\begin{pgfscope}%
\pgfsetbuttcap%
\pgfsetroundjoin%
\definecolor{currentfill}{rgb}{0.000000,0.000000,0.000000}%
\pgfsetfillcolor{currentfill}%
\pgfsetlinewidth{0.602250pt}%
\definecolor{currentstroke}{rgb}{0.000000,0.000000,0.000000}%
\pgfsetstrokecolor{currentstroke}%
\pgfsetdash{}{0pt}%
\pgfsys@defobject{currentmarker}{\pgfqpoint{-0.027778in}{0.000000in}}{\pgfqpoint{0.000000in}{0.000000in}}{%
\pgfpathmoveto{\pgfqpoint{0.000000in}{0.000000in}}%
\pgfpathlineto{\pgfqpoint{-0.027778in}{0.000000in}}%
\pgfusepath{stroke,fill}%
}%
\begin{pgfscope}%
\pgfsys@transformshift{0.530948in}{2.043015in}%
\pgfsys@useobject{currentmarker}{}%
\end{pgfscope}%
\end{pgfscope}%
\begin{pgfscope}%
\pgfpathrectangle{\pgfqpoint{0.530948in}{0.451986in}}{\pgfqpoint{5.356938in}{1.900459in}}%
\pgfusepath{clip}%
\pgfsetbuttcap%
\pgfsetroundjoin%
\pgfsetlinewidth{0.250937pt}%
\definecolor{currentstroke}{rgb}{0.690196,0.690196,0.690196}%
\pgfsetstrokecolor{currentstroke}%
\pgfsetdash{{0.250000pt}{0.412500pt}}{0.000000pt}%
\pgfpathmoveto{\pgfqpoint{0.530948in}{2.296935in}}%
\pgfpathlineto{\pgfqpoint{5.887886in}{2.296935in}}%
\pgfusepath{stroke}%
\end{pgfscope}%
\begin{pgfscope}%
\pgfsetbuttcap%
\pgfsetroundjoin%
\definecolor{currentfill}{rgb}{0.000000,0.000000,0.000000}%
\pgfsetfillcolor{currentfill}%
\pgfsetlinewidth{0.602250pt}%
\definecolor{currentstroke}{rgb}{0.000000,0.000000,0.000000}%
\pgfsetstrokecolor{currentstroke}%
\pgfsetdash{}{0pt}%
\pgfsys@defobject{currentmarker}{\pgfqpoint{-0.027778in}{0.000000in}}{\pgfqpoint{0.000000in}{0.000000in}}{%
\pgfpathmoveto{\pgfqpoint{0.000000in}{0.000000in}}%
\pgfpathlineto{\pgfqpoint{-0.027778in}{0.000000in}}%
\pgfusepath{stroke,fill}%
}%
\begin{pgfscope}%
\pgfsys@transformshift{0.530948in}{2.296935in}%
\pgfsys@useobject{currentmarker}{}%
\end{pgfscope}%
\end{pgfscope}%
\begin{pgfscope}%
\definecolor{textcolor}{rgb}{0.000000,0.000000,0.000000}%
\pgfsetstrokecolor{textcolor}%
\pgfsetfillcolor{textcolor}%
\pgftext[x=0.201084in,y=1.402216in,,bottom,rotate=90.000000]{\color{textcolor}\rmfamily\fontsize{8.000000}{9.600000}\selectfont \textbf{Längswiderstand \boldmath\(\displaystyle R_{xx}\) (}\boldmath\(\displaystyle \Omega\)\textbf{)}}%
\end{pgfscope}%
\begin{pgfscope}%
\pgfpathrectangle{\pgfqpoint{0.530948in}{0.451986in}}{\pgfqpoint{5.356938in}{1.900459in}}%
\pgfusepath{clip}%
\pgfsetrectcap%
\pgfsetroundjoin%
\pgfsetlinewidth{1.003750pt}%
\definecolor{currentstroke}{rgb}{0.760784,0.211765,0.086275}%
\pgfsetstrokecolor{currentstroke}%
\pgfsetdash{}{0pt}%
\pgfpathmoveto{\pgfqpoint{5.897886in}{0.775125in}}%
\pgfpathlineto{\pgfqpoint{5.847968in}{0.774402in}}%
\pgfpathlineto{\pgfqpoint{5.731250in}{0.770026in}}%
\pgfpathlineto{\pgfqpoint{5.618886in}{0.771046in}}%
\pgfpathlineto{\pgfqpoint{5.610416in}{0.772251in}}%
\pgfpathlineto{\pgfqpoint{5.585151in}{0.772251in}}%
\pgfpathlineto{\pgfqpoint{5.576777in}{0.773439in}}%
\pgfpathlineto{\pgfqpoint{5.560101in}{0.773439in}}%
\pgfpathlineto{\pgfqpoint{5.551798in}{0.774600in}}%
\pgfpathlineto{\pgfqpoint{5.510637in}{0.775537in}}%
\pgfpathlineto{\pgfqpoint{5.486216in}{0.776089in}}%
\pgfpathlineto{\pgfqpoint{5.437988in}{0.775750in}}%
\pgfpathlineto{\pgfqpoint{5.422090in}{0.774818in}}%
\pgfpathlineto{\pgfqpoint{5.414174in}{0.774818in}}%
\pgfpathlineto{\pgfqpoint{5.406280in}{0.773526in}}%
\pgfpathlineto{\pgfqpoint{5.390557in}{0.773526in}}%
\pgfpathlineto{\pgfqpoint{5.382728in}{0.772040in}}%
\pgfpathlineto{\pgfqpoint{5.359371in}{0.771218in}}%
\pgfpathlineto{\pgfqpoint{5.328524in}{0.769052in}}%
\pgfpathlineto{\pgfqpoint{5.320865in}{0.769052in}}%
\pgfpathlineto{\pgfqpoint{5.313227in}{0.767846in}}%
\pgfpathlineto{\pgfqpoint{5.201099in}{0.768590in}}%
\pgfpathlineto{\pgfqpoint{5.193783in}{0.770054in}}%
\pgfpathlineto{\pgfqpoint{5.179209in}{0.770054in}}%
\pgfpathlineto{\pgfqpoint{5.171951in}{0.771770in}}%
\pgfpathlineto{\pgfqpoint{5.157493in}{0.771770in}}%
\pgfpathlineto{\pgfqpoint{5.150293in}{0.773660in}}%
\pgfpathlineto{\pgfqpoint{5.135949in}{0.773660in}}%
\pgfpathlineto{\pgfqpoint{5.128805in}{0.775349in}}%
\pgfpathlineto{\pgfqpoint{5.114574in}{0.775349in}}%
\pgfpathlineto{\pgfqpoint{5.107487in}{0.776829in}}%
\pgfpathlineto{\pgfqpoint{5.072326in}{0.777814in}}%
\pgfpathlineto{\pgfqpoint{5.051449in}{0.778345in}}%
\pgfpathlineto{\pgfqpoint{5.010177in}{0.777464in}}%
\pgfpathlineto{\pgfqpoint{5.003361in}{0.776164in}}%
\pgfpathlineto{\pgfqpoint{4.989780in}{0.776164in}}%
\pgfpathlineto{\pgfqpoint{4.983015in}{0.774442in}}%
\pgfpathlineto{\pgfqpoint{4.962826in}{0.774442in}}%
\pgfpathlineto{\pgfqpoint{4.956130in}{0.772341in}}%
\pgfpathlineto{\pgfqpoint{4.942790in}{0.771248in}}%
\pgfpathlineto{\pgfqpoint{4.922907in}{0.769095in}}%
\pgfpathlineto{\pgfqpoint{4.909735in}{0.768034in}}%
\pgfpathlineto{\pgfqpoint{4.890101in}{0.766157in}}%
\pgfpathlineto{\pgfqpoint{4.883589in}{0.766157in}}%
\pgfpathlineto{\pgfqpoint{4.877094in}{0.764728in}}%
\pgfpathlineto{\pgfqpoint{4.787841in}{0.765298in}}%
\pgfpathlineto{\pgfqpoint{4.781584in}{0.767043in}}%
\pgfpathlineto{\pgfqpoint{4.769115in}{0.767043in}}%
\pgfpathlineto{\pgfqpoint{4.738209in}{0.771766in}}%
\pgfpathlineto{\pgfqpoint{4.732074in}{0.771766in}}%
\pgfpathlineto{\pgfqpoint{4.725953in}{0.774343in}}%
\pgfpathlineto{\pgfqpoint{4.713755in}{0.774343in}}%
\pgfpathlineto{\pgfqpoint{4.701617in}{0.776959in}}%
\pgfpathlineto{\pgfqpoint{4.695570in}{0.776959in}}%
\pgfpathlineto{\pgfqpoint{4.677517in}{0.779208in}}%
\pgfpathlineto{\pgfqpoint{4.653648in}{0.782276in}}%
\pgfpathlineto{\pgfqpoint{4.589175in}{0.781770in}}%
\pgfpathlineto{\pgfqpoint{4.571882in}{0.780270in}}%
\pgfpathlineto{\pgfqpoint{4.566145in}{0.780270in}}%
\pgfpathlineto{\pgfqpoint{4.560421in}{0.778111in}}%
\pgfpathlineto{\pgfqpoint{4.549014in}{0.778111in}}%
\pgfpathlineto{\pgfqpoint{4.543331in}{0.775544in}}%
\pgfpathlineto{\pgfqpoint{4.532004in}{0.775544in}}%
\pgfpathlineto{\pgfqpoint{4.526361in}{0.772649in}}%
\pgfpathlineto{\pgfqpoint{4.515113in}{0.772649in}}%
\pgfpathlineto{\pgfqpoint{4.509509in}{0.769625in}}%
\pgfpathlineto{\pgfqpoint{4.498340in}{0.769625in}}%
\pgfpathlineto{\pgfqpoint{4.492775in}{0.766715in}}%
\pgfpathlineto{\pgfqpoint{4.481683in}{0.766715in}}%
\pgfpathlineto{\pgfqpoint{4.470643in}{0.763968in}}%
\pgfpathlineto{\pgfqpoint{4.465142in}{0.763968in}}%
\pgfpathlineto{\pgfqpoint{4.459654in}{0.761760in}}%
\pgfpathlineto{\pgfqpoint{4.443265in}{0.760882in}}%
\pgfpathlineto{\pgfqpoint{4.426988in}{0.759071in}}%
\pgfpathlineto{\pgfqpoint{4.384126in}{0.759666in}}%
\pgfpathlineto{\pgfqpoint{4.378823in}{0.761092in}}%
\pgfpathlineto{\pgfqpoint{4.368253in}{0.761092in}}%
\pgfpathlineto{\pgfqpoint{4.342036in}{0.766255in}}%
\pgfpathlineto{\pgfqpoint{4.331631in}{0.766255in}}%
\pgfpathlineto{\pgfqpoint{4.326446in}{0.769666in}}%
\pgfpathlineto{\pgfqpoint{4.316112in}{0.769666in}}%
\pgfpathlineto{\pgfqpoint{4.310962in}{0.773219in}}%
\pgfpathlineto{\pgfqpoint{4.300696in}{0.773219in}}%
\pgfpathlineto{\pgfqpoint{4.295580in}{0.776865in}}%
\pgfpathlineto{\pgfqpoint{4.290476in}{0.776865in}}%
\pgfpathlineto{\pgfqpoint{4.280301in}{0.780407in}}%
\pgfpathlineto{\pgfqpoint{4.275231in}{0.780407in}}%
\pgfpathlineto{\pgfqpoint{4.265124in}{0.783754in}}%
\pgfpathlineto{\pgfqpoint{4.255061in}{0.783754in}}%
\pgfpathlineto{\pgfqpoint{4.245043in}{0.786470in}}%
\pgfpathlineto{\pgfqpoint{4.240050in}{0.786470in}}%
\pgfpathlineto{\pgfqpoint{4.235069in}{0.788766in}}%
\pgfpathlineto{\pgfqpoint{4.225138in}{0.788766in}}%
\pgfpathlineto{\pgfqpoint{4.220190in}{0.790213in}}%
\pgfpathlineto{\pgfqpoint{4.166459in}{0.790096in}}%
\pgfpathlineto{\pgfqpoint{4.161637in}{0.788370in}}%
\pgfpathlineto{\pgfqpoint{4.152026in}{0.788370in}}%
\pgfpathlineto{\pgfqpoint{4.147236in}{0.786110in}}%
\pgfpathlineto{\pgfqpoint{4.132927in}{0.786110in}}%
\pgfpathlineto{\pgfqpoint{4.128178in}{0.783083in}}%
\pgfpathlineto{\pgfqpoint{4.118711in}{0.783083in}}%
\pgfpathlineto{\pgfqpoint{4.113992in}{0.779612in}}%
\pgfpathlineto{\pgfqpoint{4.109284in}{0.779612in}}%
\pgfpathlineto{\pgfqpoint{4.099897in}{0.775773in}}%
\pgfpathlineto{\pgfqpoint{4.095218in}{0.775773in}}%
\pgfpathlineto{\pgfqpoint{4.085891in}{0.771824in}}%
\pgfpathlineto{\pgfqpoint{4.081242in}{0.771824in}}%
\pgfpathlineto{\pgfqpoint{4.071974in}{0.767698in}}%
\pgfpathlineto{\pgfqpoint{4.062746in}{0.767698in}}%
\pgfpathlineto{\pgfqpoint{4.058146in}{0.763699in}}%
\pgfpathlineto{\pgfqpoint{4.048976in}{0.763699in}}%
\pgfpathlineto{\pgfqpoint{4.044405in}{0.760175in}}%
\pgfpathlineto{\pgfqpoint{4.035292in}{0.760175in}}%
\pgfpathlineto{\pgfqpoint{4.030750in}{0.757070in}}%
\pgfpathlineto{\pgfqpoint{4.026218in}{0.757070in}}%
\pgfpathlineto{\pgfqpoint{4.017182in}{0.754656in}}%
\pgfpathlineto{\pgfqpoint{4.008183in}{0.754656in}}%
\pgfpathlineto{\pgfqpoint{4.003698in}{0.752908in}}%
\pgfpathlineto{\pgfqpoint{3.954970in}{0.753595in}}%
\pgfpathlineto{\pgfqpoint{3.950595in}{0.755615in}}%
\pgfpathlineto{\pgfqpoint{3.937524in}{0.755615in}}%
\pgfpathlineto{\pgfqpoint{3.933184in}{0.758553in}}%
\pgfpathlineto{\pgfqpoint{3.928854in}{0.758553in}}%
\pgfpathlineto{\pgfqpoint{3.920220in}{0.762327in}}%
\pgfpathlineto{\pgfqpoint{3.915916in}{0.762327in}}%
\pgfpathlineto{\pgfqpoint{3.907335in}{0.766560in}}%
\pgfpathlineto{\pgfqpoint{3.903057in}{0.766560in}}%
\pgfpathlineto{\pgfqpoint{3.894529in}{0.771342in}}%
\pgfpathlineto{\pgfqpoint{3.890277in}{0.771342in}}%
\pgfpathlineto{\pgfqpoint{3.886034in}{0.776306in}}%
\pgfpathlineto{\pgfqpoint{3.873357in}{0.776306in}}%
\pgfpathlineto{\pgfqpoint{3.869148in}{0.781318in}}%
\pgfpathlineto{\pgfqpoint{3.860757in}{0.781318in}}%
\pgfpathlineto{\pgfqpoint{3.856573in}{0.786204in}}%
\pgfpathlineto{\pgfqpoint{3.852398in}{0.786204in}}%
\pgfpathlineto{\pgfqpoint{3.844074in}{0.790632in}}%
\pgfpathlineto{\pgfqpoint{3.835783in}{0.790632in}}%
\pgfpathlineto{\pgfqpoint{3.831650in}{0.794781in}}%
\pgfpathlineto{\pgfqpoint{3.823408in}{0.794781in}}%
\pgfpathlineto{\pgfqpoint{3.815200in}{0.798405in}}%
\pgfpathlineto{\pgfqpoint{3.811108in}{0.798405in}}%
\pgfpathlineto{\pgfqpoint{3.807024in}{0.801269in}}%
\pgfpathlineto{\pgfqpoint{3.798880in}{0.801269in}}%
\pgfpathlineto{\pgfqpoint{3.794821in}{0.803369in}}%
\pgfpathlineto{\pgfqpoint{3.786726in}{0.803369in}}%
\pgfpathlineto{\pgfqpoint{3.782690in}{0.804742in}}%
\pgfpathlineto{\pgfqpoint{3.746726in}{0.804183in}}%
\pgfpathlineto{\pgfqpoint{3.727020in}{0.801508in}}%
\pgfpathlineto{\pgfqpoint{3.723101in}{0.798661in}}%
\pgfpathlineto{\pgfqpoint{3.715287in}{0.798661in}}%
\pgfpathlineto{\pgfqpoint{3.711392in}{0.795165in}}%
\pgfpathlineto{\pgfqpoint{3.699751in}{0.795165in}}%
\pgfpathlineto{\pgfqpoint{3.695885in}{0.791061in}}%
\pgfpathlineto{\pgfqpoint{3.688177in}{0.791061in}}%
\pgfpathlineto{\pgfqpoint{3.684334in}{0.786503in}}%
\pgfpathlineto{\pgfqpoint{3.676670in}{0.786503in}}%
\pgfpathlineto{\pgfqpoint{3.672849in}{0.781458in}}%
\pgfpathlineto{\pgfqpoint{3.665229in}{0.781458in}}%
\pgfpathlineto{\pgfqpoint{3.661431in}{0.776184in}}%
\pgfpathlineto{\pgfqpoint{3.653855in}{0.776184in}}%
\pgfpathlineto{\pgfqpoint{3.650078in}{0.770837in}}%
\pgfpathlineto{\pgfqpoint{3.642545in}{0.770837in}}%
\pgfpathlineto{\pgfqpoint{3.638790in}{0.765542in}}%
\pgfpathlineto{\pgfqpoint{3.631300in}{0.765542in}}%
\pgfpathlineto{\pgfqpoint{3.627566in}{0.760391in}}%
\pgfpathlineto{\pgfqpoint{3.620119in}{0.760391in}}%
\pgfpathlineto{\pgfqpoint{3.616406in}{0.755585in}}%
\pgfpathlineto{\pgfqpoint{3.609002in}{0.755585in}}%
\pgfpathlineto{\pgfqpoint{3.605310in}{0.751233in}}%
\pgfpathlineto{\pgfqpoint{3.597947in}{0.751233in}}%
\pgfpathlineto{\pgfqpoint{3.590613in}{0.747619in}}%
\pgfpathlineto{\pgfqpoint{3.586955in}{0.747619in}}%
\pgfpathlineto{\pgfqpoint{3.583305in}{0.744792in}}%
\pgfpathlineto{\pgfqpoint{3.576025in}{0.744792in}}%
\pgfpathlineto{\pgfqpoint{3.572396in}{0.742975in}}%
\pgfpathlineto{\pgfqpoint{3.532914in}{0.743397in}}%
\pgfpathlineto{\pgfqpoint{3.529365in}{0.745716in}}%
\pgfpathlineto{\pgfqpoint{3.522286in}{0.745716in}}%
\pgfpathlineto{\pgfqpoint{3.518757in}{0.748992in}}%
\pgfpathlineto{\pgfqpoint{3.511717in}{0.748992in}}%
\pgfpathlineto{\pgfqpoint{3.508207in}{0.753179in}}%
\pgfpathlineto{\pgfqpoint{3.497716in}{0.753179in}}%
\pgfpathlineto{\pgfqpoint{3.494231in}{0.758203in}}%
\pgfpathlineto{\pgfqpoint{3.487282in}{0.758203in}}%
\pgfpathlineto{\pgfqpoint{3.483817in}{0.763910in}}%
\pgfpathlineto{\pgfqpoint{3.480358in}{0.763910in}}%
\pgfpathlineto{\pgfqpoint{3.473459in}{0.770037in}}%
\pgfpathlineto{\pgfqpoint{3.470019in}{0.770037in}}%
\pgfpathlineto{\pgfqpoint{3.463158in}{0.776512in}}%
\pgfpathlineto{\pgfqpoint{3.456323in}{0.776512in}}%
\pgfpathlineto{\pgfqpoint{3.452914in}{0.783058in}}%
\pgfpathlineto{\pgfqpoint{3.446115in}{0.783058in}}%
\pgfpathlineto{\pgfqpoint{3.442725in}{0.789614in}}%
\pgfpathlineto{\pgfqpoint{3.435963in}{0.789614in}}%
\pgfpathlineto{\pgfqpoint{3.432591in}{0.796043in}}%
\pgfpathlineto{\pgfqpoint{3.425866in}{0.796043in}}%
\pgfpathlineto{\pgfqpoint{3.422513in}{0.802186in}}%
\pgfpathlineto{\pgfqpoint{3.415824in}{0.802186in}}%
\pgfpathlineto{\pgfqpoint{3.412488in}{0.807787in}}%
\pgfpathlineto{\pgfqpoint{3.405835in}{0.807787in}}%
\pgfpathlineto{\pgfqpoint{3.399206in}{0.812990in}}%
\pgfpathlineto{\pgfqpoint{3.395901in}{0.812990in}}%
\pgfpathlineto{\pgfqpoint{3.389307in}{0.817464in}}%
\pgfpathlineto{\pgfqpoint{3.386019in}{0.817464in}}%
\pgfpathlineto{\pgfqpoint{3.379461in}{0.821257in}}%
\pgfpathlineto{\pgfqpoint{3.376190in}{0.821257in}}%
\pgfpathlineto{\pgfqpoint{3.369667in}{0.824358in}}%
\pgfpathlineto{\pgfqpoint{3.363167in}{0.824358in}}%
\pgfpathlineto{\pgfqpoint{3.359925in}{0.826516in}}%
\pgfpathlineto{\pgfqpoint{3.353460in}{0.827240in}}%
\pgfpathlineto{\pgfqpoint{3.347017in}{0.827963in}}%
\pgfpathlineto{\pgfqpoint{3.334199in}{0.828318in}}%
\pgfpathlineto{\pgfqpoint{3.315141in}{0.826965in}}%
\pgfpathlineto{\pgfqpoint{3.311984in}{0.824965in}}%
\pgfpathlineto{\pgfqpoint{3.305687in}{0.824965in}}%
\pgfpathlineto{\pgfqpoint{3.302546in}{0.822151in}}%
\pgfpathlineto{\pgfqpoint{3.296282in}{0.822151in}}%
\pgfpathlineto{\pgfqpoint{3.293158in}{0.818680in}}%
\pgfpathlineto{\pgfqpoint{3.286926in}{0.818680in}}%
\pgfpathlineto{\pgfqpoint{3.283819in}{0.814422in}}%
\pgfpathlineto{\pgfqpoint{3.277619in}{0.814422in}}%
\pgfpathlineto{\pgfqpoint{3.271441in}{0.809486in}}%
\pgfpathlineto{\pgfqpoint{3.268360in}{0.809486in}}%
\pgfpathlineto{\pgfqpoint{3.262215in}{0.804219in}}%
\pgfpathlineto{\pgfqpoint{3.259150in}{0.804219in}}%
\pgfpathlineto{\pgfqpoint{3.256090in}{0.798252in}}%
\pgfpathlineto{\pgfqpoint{3.249986in}{0.798252in}}%
\pgfpathlineto{\pgfqpoint{3.246942in}{0.792047in}}%
\pgfpathlineto{\pgfqpoint{3.240870in}{0.792047in}}%
\pgfpathlineto{\pgfqpoint{3.237842in}{0.785333in}}%
\pgfpathlineto{\pgfqpoint{3.228788in}{0.785333in}}%
\pgfpathlineto{\pgfqpoint{3.225780in}{0.778457in}}%
\pgfpathlineto{\pgfqpoint{3.222777in}{0.778457in}}%
\pgfpathlineto{\pgfqpoint{3.216788in}{0.771650in}}%
\pgfpathlineto{\pgfqpoint{3.213800in}{0.771650in}}%
\pgfpathlineto{\pgfqpoint{3.207841in}{0.764907in}}%
\pgfpathlineto{\pgfqpoint{3.204869in}{0.764907in}}%
\pgfpathlineto{\pgfqpoint{3.198940in}{0.758244in}}%
\pgfpathlineto{\pgfqpoint{3.195983in}{0.758244in}}%
\pgfpathlineto{\pgfqpoint{3.190084in}{0.752004in}}%
\pgfpathlineto{\pgfqpoint{3.184206in}{0.752004in}}%
\pgfpathlineto{\pgfqpoint{3.181273in}{0.746086in}}%
\pgfpathlineto{\pgfqpoint{3.175424in}{0.746086in}}%
\pgfpathlineto{\pgfqpoint{3.172507in}{0.740796in}}%
\pgfpathlineto{\pgfqpoint{3.166687in}{0.740796in}}%
\pgfpathlineto{\pgfqpoint{3.163785in}{0.736062in}}%
\pgfpathlineto{\pgfqpoint{3.157994in}{0.736062in}}%
\pgfpathlineto{\pgfqpoint{3.155106in}{0.732294in}}%
\pgfpathlineto{\pgfqpoint{3.149344in}{0.732294in}}%
\pgfpathlineto{\pgfqpoint{3.146471in}{0.729509in}}%
\pgfpathlineto{\pgfqpoint{3.137879in}{0.728557in}}%
\pgfpathlineto{\pgfqpoint{3.132175in}{0.727605in}}%
\pgfpathlineto{\pgfqpoint{3.126489in}{0.726903in}}%
\pgfpathlineto{\pgfqpoint{3.115175in}{0.727305in}}%
\pgfpathlineto{\pgfqpoint{3.112358in}{0.728912in}}%
\pgfpathlineto{\pgfqpoint{3.106739in}{0.728912in}}%
\pgfpathlineto{\pgfqpoint{3.101137in}{0.731651in}}%
\pgfpathlineto{\pgfqpoint{3.095555in}{0.731651in}}%
\pgfpathlineto{\pgfqpoint{3.092770in}{0.735420in}}%
\pgfpathlineto{\pgfqpoint{3.089990in}{0.735420in}}%
\pgfpathlineto{\pgfqpoint{3.087215in}{0.740272in}}%
\pgfpathlineto{\pgfqpoint{3.081678in}{0.740272in}}%
\pgfpathlineto{\pgfqpoint{3.078916in}{0.746083in}}%
\pgfpathlineto{\pgfqpoint{3.073406in}{0.746083in}}%
\pgfpathlineto{\pgfqpoint{3.070658in}{0.752602in}}%
\pgfpathlineto{\pgfqpoint{3.062440in}{0.752602in}}%
\pgfpathlineto{\pgfqpoint{3.059709in}{0.759899in}}%
\pgfpathlineto{\pgfqpoint{3.054262in}{0.759899in}}%
\pgfpathlineto{\pgfqpoint{3.051545in}{0.767784in}}%
\pgfpathlineto{\pgfqpoint{3.048832in}{0.767784in}}%
\pgfpathlineto{\pgfqpoint{3.043420in}{0.775951in}}%
\pgfpathlineto{\pgfqpoint{3.038025in}{0.775951in}}%
\pgfpathlineto{\pgfqpoint{3.035334in}{0.784284in}}%
\pgfpathlineto{\pgfqpoint{3.029966in}{0.784284in}}%
\pgfpathlineto{\pgfqpoint{3.027288in}{0.792824in}}%
\pgfpathlineto{\pgfqpoint{3.021945in}{0.792824in}}%
\pgfpathlineto{\pgfqpoint{3.019280in}{0.801287in}}%
\pgfpathlineto{\pgfqpoint{3.013964in}{0.801287in}}%
\pgfpathlineto{\pgfqpoint{3.011311in}{0.809519in}}%
\pgfpathlineto{\pgfqpoint{3.006020in}{0.809519in}}%
\pgfpathlineto{\pgfqpoint{3.003381in}{0.817502in}}%
\pgfpathlineto{\pgfqpoint{2.998114in}{0.817502in}}%
\pgfpathlineto{\pgfqpoint{2.995488in}{0.825165in}}%
\pgfpathlineto{\pgfqpoint{2.990247in}{0.825165in}}%
\pgfpathlineto{\pgfqpoint{2.987633in}{0.832404in}}%
\pgfpathlineto{\pgfqpoint{2.979815in}{0.832404in}}%
\pgfpathlineto{\pgfqpoint{2.977217in}{0.838945in}}%
\pgfpathlineto{\pgfqpoint{2.974624in}{0.838945in}}%
\pgfpathlineto{\pgfqpoint{2.972034in}{0.844920in}}%
\pgfpathlineto{\pgfqpoint{2.966868in}{0.844920in}}%
\pgfpathlineto{\pgfqpoint{2.961718in}{0.850232in}}%
\pgfpathlineto{\pgfqpoint{2.959149in}{0.850232in}}%
\pgfpathlineto{\pgfqpoint{2.956584in}{0.854826in}}%
\pgfpathlineto{\pgfqpoint{2.948913in}{0.854826in}}%
\pgfpathlineto{\pgfqpoint{2.946364in}{0.858777in}}%
\pgfpathlineto{\pgfqpoint{2.943819in}{0.858777in}}%
\pgfpathlineto{\pgfqpoint{2.938742in}{0.861940in}}%
\pgfpathlineto{\pgfqpoint{2.933680in}{0.861940in}}%
\pgfpathlineto{\pgfqpoint{2.931155in}{0.864353in}}%
\pgfpathlineto{\pgfqpoint{2.926117in}{0.865193in}}%
\pgfpathlineto{\pgfqpoint{2.921095in}{0.866034in}}%
\pgfpathlineto{\pgfqpoint{2.911097in}{0.866871in}}%
\pgfpathlineto{\pgfqpoint{2.896216in}{0.866145in}}%
\pgfpathlineto{\pgfqpoint{2.893750in}{0.864655in}}%
\pgfpathlineto{\pgfqpoint{2.888828in}{0.863558in}}%
\pgfpathlineto{\pgfqpoint{2.883921in}{0.862461in}}%
\pgfpathlineto{\pgfqpoint{2.881473in}{0.862461in}}%
\pgfpathlineto{\pgfqpoint{2.879029in}{0.859505in}}%
\pgfpathlineto{\pgfqpoint{2.874153in}{0.859505in}}%
\pgfpathlineto{\pgfqpoint{2.869291in}{0.855994in}}%
\pgfpathlineto{\pgfqpoint{2.866866in}{0.855994in}}%
\pgfpathlineto{\pgfqpoint{2.862027in}{0.851563in}}%
\pgfpathlineto{\pgfqpoint{2.859612in}{0.851563in}}%
\pgfpathlineto{\pgfqpoint{2.857202in}{0.846725in}}%
\pgfpathlineto{\pgfqpoint{2.852392in}{0.846725in}}%
\pgfpathlineto{\pgfqpoint{2.849993in}{0.841175in}}%
\pgfpathlineto{\pgfqpoint{2.845205in}{0.841175in}}%
\pgfpathlineto{\pgfqpoint{2.842817in}{0.835083in}}%
\pgfpathlineto{\pgfqpoint{2.838051in}{0.835083in}}%
\pgfpathlineto{\pgfqpoint{2.833299in}{0.828540in}}%
\pgfpathlineto{\pgfqpoint{2.830929in}{0.828540in}}%
\pgfpathlineto{\pgfqpoint{2.826199in}{0.821514in}}%
\pgfpathlineto{\pgfqpoint{2.823839in}{0.821514in}}%
\pgfpathlineto{\pgfqpoint{2.819131in}{0.814214in}}%
\pgfpathlineto{\pgfqpoint{2.816782in}{0.814214in}}%
\pgfpathlineto{\pgfqpoint{2.812095in}{0.806512in}}%
\pgfpathlineto{\pgfqpoint{2.807422in}{0.806512in}}%
\pgfpathlineto{\pgfqpoint{2.805090in}{0.798473in}}%
\pgfpathlineto{\pgfqpoint{2.800438in}{0.798473in}}%
\pgfpathlineto{\pgfqpoint{2.798117in}{0.790124in}}%
\pgfpathlineto{\pgfqpoint{2.793486in}{0.790124in}}%
\pgfpathlineto{\pgfqpoint{2.791176in}{0.781867in}}%
\pgfpathlineto{\pgfqpoint{2.786565in}{0.781867in}}%
\pgfpathlineto{\pgfqpoint{2.784265in}{0.773462in}}%
\pgfpathlineto{\pgfqpoint{2.779675in}{0.773462in}}%
\pgfpathlineto{\pgfqpoint{2.777386in}{0.765028in}}%
\pgfpathlineto{\pgfqpoint{2.772816in}{0.765028in}}%
\pgfpathlineto{\pgfqpoint{2.768261in}{0.756779in}}%
\pgfpathlineto{\pgfqpoint{2.765988in}{0.756779in}}%
\pgfpathlineto{\pgfqpoint{2.763718in}{0.748983in}}%
\pgfpathlineto{\pgfqpoint{2.759190in}{0.748983in}}%
\pgfpathlineto{\pgfqpoint{2.756930in}{0.741333in}}%
\pgfpathlineto{\pgfqpoint{2.752422in}{0.741333in}}%
\pgfpathlineto{\pgfqpoint{2.750172in}{0.734183in}}%
\pgfpathlineto{\pgfqpoint{2.745684in}{0.734183in}}%
\pgfpathlineto{\pgfqpoint{2.741208in}{0.727497in}}%
\pgfpathlineto{\pgfqpoint{2.738975in}{0.727497in}}%
\pgfpathlineto{\pgfqpoint{2.734519in}{0.721607in}}%
\pgfpathlineto{\pgfqpoint{2.732296in}{0.721607in}}%
\pgfpathlineto{\pgfqpoint{2.727860in}{0.716453in}}%
\pgfpathlineto{\pgfqpoint{2.725647in}{0.716453in}}%
\pgfpathlineto{\pgfqpoint{2.723437in}{0.712201in}}%
\pgfpathlineto{\pgfqpoint{2.719027in}{0.712201in}}%
\pgfpathlineto{\pgfqpoint{2.716826in}{0.708834in}}%
\pgfpathlineto{\pgfqpoint{2.710245in}{0.708834in}}%
\pgfpathlineto{\pgfqpoint{2.708057in}{0.706516in}}%
\pgfpathlineto{\pgfqpoint{2.703692in}{0.706516in}}%
\pgfpathlineto{\pgfqpoint{2.699339in}{0.705357in}}%
\pgfpathlineto{\pgfqpoint{2.690671in}{0.705887in}}%
\pgfpathlineto{\pgfqpoint{2.677763in}{0.708956in}}%
\pgfpathlineto{\pgfqpoint{2.675623in}{0.712538in}}%
\pgfpathlineto{\pgfqpoint{2.671351in}{0.712538in}}%
\pgfpathlineto{\pgfqpoint{2.669220in}{0.717161in}}%
\pgfpathlineto{\pgfqpoint{2.664967in}{0.717161in}}%
\pgfpathlineto{\pgfqpoint{2.662845in}{0.723104in}}%
\pgfpathlineto{\pgfqpoint{2.660726in}{0.723104in}}%
\pgfpathlineto{\pgfqpoint{2.656497in}{0.729852in}}%
\pgfpathlineto{\pgfqpoint{2.654387in}{0.729852in}}%
\pgfpathlineto{\pgfqpoint{2.650176in}{0.737649in}}%
\pgfpathlineto{\pgfqpoint{2.645977in}{0.737649in}}%
\pgfpathlineto{\pgfqpoint{2.643882in}{0.746064in}}%
\pgfpathlineto{\pgfqpoint{2.639701in}{0.746064in}}%
\pgfpathlineto{\pgfqpoint{2.637615in}{0.755212in}}%
\pgfpathlineto{\pgfqpoint{2.633452in}{0.755212in}}%
\pgfpathlineto{\pgfqpoint{2.631375in}{0.764850in}}%
\pgfpathlineto{\pgfqpoint{2.627230in}{0.764850in}}%
\pgfpathlineto{\pgfqpoint{2.625161in}{0.775021in}}%
\pgfpathlineto{\pgfqpoint{2.618974in}{0.775021in}}%
\pgfpathlineto{\pgfqpoint{2.616917in}{0.785404in}}%
\pgfpathlineto{\pgfqpoint{2.614864in}{0.785404in}}%
\pgfpathlineto{\pgfqpoint{2.610765in}{0.795865in}}%
\pgfpathlineto{\pgfqpoint{2.608720in}{0.795865in}}%
\pgfpathlineto{\pgfqpoint{2.604638in}{0.806512in}}%
\pgfpathlineto{\pgfqpoint{2.602602in}{0.806512in}}%
\pgfpathlineto{\pgfqpoint{2.598538in}{0.816966in}}%
\pgfpathlineto{\pgfqpoint{2.596510in}{0.816966in}}%
\pgfpathlineto{\pgfqpoint{2.592463in}{0.827529in}}%
\pgfpathlineto{\pgfqpoint{2.588427in}{0.827529in}}%
\pgfpathlineto{\pgfqpoint{2.586413in}{0.837478in}}%
\pgfpathlineto{\pgfqpoint{2.582394in}{0.837478in}}%
\pgfpathlineto{\pgfqpoint{2.580389in}{0.847411in}}%
\pgfpathlineto{\pgfqpoint{2.576387in}{0.847411in}}%
\pgfpathlineto{\pgfqpoint{2.574390in}{0.856885in}}%
\pgfpathlineto{\pgfqpoint{2.570404in}{0.856885in}}%
\pgfpathlineto{\pgfqpoint{2.568416in}{0.865988in}}%
\pgfpathlineto{\pgfqpoint{2.564447in}{0.865988in}}%
\pgfpathlineto{\pgfqpoint{2.562466in}{0.874547in}}%
\pgfpathlineto{\pgfqpoint{2.558514in}{0.874547in}}%
\pgfpathlineto{\pgfqpoint{2.556542in}{0.882640in}}%
\pgfpathlineto{\pgfqpoint{2.552606in}{0.882640in}}%
\pgfpathlineto{\pgfqpoint{2.550642in}{0.890133in}}%
\pgfpathlineto{\pgfqpoint{2.546722in}{0.890133in}}%
\pgfpathlineto{\pgfqpoint{2.544766in}{0.897131in}}%
\pgfpathlineto{\pgfqpoint{2.540862in}{0.897131in}}%
\pgfpathlineto{\pgfqpoint{2.538914in}{0.903464in}}%
\pgfpathlineto{\pgfqpoint{2.535027in}{0.903464in}}%
\pgfpathlineto{\pgfqpoint{2.531150in}{0.909273in}}%
\pgfpathlineto{\pgfqpoint{2.529215in}{0.909273in}}%
\pgfpathlineto{\pgfqpoint{2.527284in}{0.914334in}}%
\pgfpathlineto{\pgfqpoint{2.523428in}{0.914334in}}%
\pgfpathlineto{\pgfqpoint{2.521504in}{0.918714in}}%
\pgfpathlineto{\pgfqpoint{2.517664in}{0.918714in}}%
\pgfpathlineto{\pgfqpoint{2.515748in}{0.922569in}}%
\pgfpathlineto{\pgfqpoint{2.511924in}{0.922569in}}%
\pgfpathlineto{\pgfqpoint{2.510015in}{0.925687in}}%
\pgfpathlineto{\pgfqpoint{2.504306in}{0.925687in}}%
\pgfpathlineto{\pgfqpoint{2.502408in}{0.928162in}}%
\pgfpathlineto{\pgfqpoint{2.498620in}{0.928162in}}%
\pgfpathlineto{\pgfqpoint{2.496730in}{0.929920in}}%
\pgfpathlineto{\pgfqpoint{2.492957in}{0.930510in}}%
\pgfpathlineto{\pgfqpoint{2.487317in}{0.931356in}}%
\pgfpathlineto{\pgfqpoint{2.477968in}{0.931616in}}%
\pgfpathlineto{\pgfqpoint{2.470533in}{0.930671in}}%
\pgfpathlineto{\pgfqpoint{2.466831in}{0.929300in}}%
\pgfpathlineto{\pgfqpoint{2.457619in}{0.924549in}}%
\pgfpathlineto{\pgfqpoint{2.453951in}{0.924549in}}%
\pgfpathlineto{\pgfqpoint{2.450293in}{0.921309in}}%
\pgfpathlineto{\pgfqpoint{2.448468in}{0.921309in}}%
\pgfpathlineto{\pgfqpoint{2.446645in}{0.917536in}}%
\pgfpathlineto{\pgfqpoint{2.443006in}{0.917536in}}%
\pgfpathlineto{\pgfqpoint{2.441191in}{0.913064in}}%
\pgfpathlineto{\pgfqpoint{2.437566in}{0.913064in}}%
\pgfpathlineto{\pgfqpoint{2.435758in}{0.908126in}}%
\pgfpathlineto{\pgfqpoint{2.430347in}{0.908126in}}%
\pgfpathlineto{\pgfqpoint{2.428548in}{0.902616in}}%
\pgfpathlineto{\pgfqpoint{2.424957in}{0.902616in}}%
\pgfpathlineto{\pgfqpoint{2.423166in}{0.896689in}}%
\pgfpathlineto{\pgfqpoint{2.421376in}{0.896689in}}%
\pgfpathlineto{\pgfqpoint{2.419589in}{0.890235in}}%
\pgfpathlineto{\pgfqpoint{2.416022in}{0.890235in}}%
\pgfpathlineto{\pgfqpoint{2.414242in}{0.883295in}}%
\pgfpathlineto{\pgfqpoint{2.410689in}{0.883295in}}%
\pgfpathlineto{\pgfqpoint{2.408916in}{0.875947in}}%
\pgfpathlineto{\pgfqpoint{2.403611in}{0.875947in}}%
\pgfpathlineto{\pgfqpoint{2.401847in}{0.868149in}}%
\pgfpathlineto{\pgfqpoint{2.400085in}{0.868149in}}%
\pgfpathlineto{\pgfqpoint{2.396569in}{0.860176in}}%
\pgfpathlineto{\pgfqpoint{2.394815in}{0.860176in}}%
\pgfpathlineto{\pgfqpoint{2.391313in}{0.851570in}}%
\pgfpathlineto{\pgfqpoint{2.387819in}{0.851570in}}%
\pgfpathlineto{\pgfqpoint{2.386076in}{0.842780in}}%
\pgfpathlineto{\pgfqpoint{2.382597in}{0.842780in}}%
\pgfpathlineto{\pgfqpoint{2.380860in}{0.833793in}}%
\pgfpathlineto{\pgfqpoint{2.377394in}{0.833793in}}%
\pgfpathlineto{\pgfqpoint{2.375664in}{0.824650in}}%
\pgfpathlineto{\pgfqpoint{2.372212in}{0.824650in}}%
\pgfpathlineto{\pgfqpoint{2.370489in}{0.815199in}}%
\pgfpathlineto{\pgfqpoint{2.368768in}{0.815199in}}%
\pgfpathlineto{\pgfqpoint{2.365333in}{0.805568in}}%
\pgfpathlineto{\pgfqpoint{2.361907in}{0.805568in}}%
\pgfpathlineto{\pgfqpoint{2.360198in}{0.795812in}}%
\pgfpathlineto{\pgfqpoint{2.356785in}{0.795812in}}%
\pgfpathlineto{\pgfqpoint{2.353381in}{0.785917in}}%
\pgfpathlineto{\pgfqpoint{2.351682in}{0.785917in}}%
\pgfpathlineto{\pgfqpoint{2.349986in}{0.776248in}}%
\pgfpathlineto{\pgfqpoint{2.346599in}{0.776248in}}%
\pgfpathlineto{\pgfqpoint{2.344909in}{0.766395in}}%
\pgfpathlineto{\pgfqpoint{2.341536in}{0.766395in}}%
\pgfpathlineto{\pgfqpoint{2.339852in}{0.756959in}}%
\pgfpathlineto{\pgfqpoint{2.334815in}{0.756959in}}%
\pgfpathlineto{\pgfqpoint{2.333140in}{0.747518in}}%
\pgfpathlineto{\pgfqpoint{2.331467in}{0.747518in}}%
\pgfpathlineto{\pgfqpoint{2.328127in}{0.738257in}}%
\pgfpathlineto{\pgfqpoint{2.324797in}{0.738257in}}%
\pgfpathlineto{\pgfqpoint{2.323134in}{0.729431in}}%
\pgfpathlineto{\pgfqpoint{2.321474in}{0.729431in}}%
\pgfpathlineto{\pgfqpoint{2.319816in}{0.720984in}}%
\pgfpathlineto{\pgfqpoint{2.316507in}{0.720984in}}%
\pgfpathlineto{\pgfqpoint{2.313205in}{0.712825in}}%
\pgfpathlineto{\pgfqpoint{2.309912in}{0.712825in}}%
\pgfpathlineto{\pgfqpoint{2.308269in}{0.705403in}}%
\pgfpathlineto{\pgfqpoint{2.304988in}{0.705403in}}%
\pgfpathlineto{\pgfqpoint{2.303351in}{0.698566in}}%
\pgfpathlineto{\pgfqpoint{2.301716in}{0.698566in}}%
\pgfpathlineto{\pgfqpoint{2.298452in}{0.692488in}}%
\pgfpathlineto{\pgfqpoint{2.296823in}{0.692488in}}%
\pgfpathlineto{\pgfqpoint{2.293571in}{0.687256in}}%
\pgfpathlineto{\pgfqpoint{2.290328in}{0.687256in}}%
\pgfpathlineto{\pgfqpoint{2.288709in}{0.682636in}}%
\pgfpathlineto{\pgfqpoint{2.285478in}{0.682636in}}%
\pgfpathlineto{\pgfqpoint{2.283865in}{0.679231in}}%
\pgfpathlineto{\pgfqpoint{2.280646in}{0.679231in}}%
\pgfpathlineto{\pgfqpoint{2.279039in}{0.676690in}}%
\pgfpathlineto{\pgfqpoint{2.275832in}{0.676690in}}%
\pgfpathlineto{\pgfqpoint{2.272632in}{0.675336in}}%
\pgfpathlineto{\pgfqpoint{2.261497in}{0.675872in}}%
\pgfpathlineto{\pgfqpoint{2.252030in}{0.681322in}}%
\pgfpathlineto{\pgfqpoint{2.250459in}{0.685881in}}%
\pgfpathlineto{\pgfqpoint{2.247322in}{0.685881in}}%
\pgfpathlineto{\pgfqpoint{2.245757in}{0.691502in}}%
\pgfpathlineto{\pgfqpoint{2.241072in}{0.691502in}}%
\pgfpathlineto{\pgfqpoint{2.239515in}{0.698236in}}%
\pgfpathlineto{\pgfqpoint{2.237959in}{0.698236in}}%
\pgfpathlineto{\pgfqpoint{2.234853in}{0.705955in}}%
\pgfpathlineto{\pgfqpoint{2.233303in}{0.705955in}}%
\pgfpathlineto{\pgfqpoint{2.230209in}{0.714751in}}%
\pgfpathlineto{\pgfqpoint{2.228665in}{0.714751in}}%
\pgfpathlineto{\pgfqpoint{2.227122in}{0.724261in}}%
\pgfpathlineto{\pgfqpoint{2.222506in}{0.724261in}}%
\pgfpathlineto{\pgfqpoint{2.220971in}{0.734740in}}%
\pgfpathlineto{\pgfqpoint{2.217907in}{0.734740in}}%
\pgfpathlineto{\pgfqpoint{2.216378in}{0.745684in}}%
\pgfpathlineto{\pgfqpoint{2.213325in}{0.745684in}}%
\pgfpathlineto{\pgfqpoint{2.211801in}{0.757070in}}%
\pgfpathlineto{\pgfqpoint{2.208759in}{0.757070in}}%
\pgfpathlineto{\pgfqpoint{2.207241in}{0.769031in}}%
\pgfpathlineto{\pgfqpoint{2.205725in}{0.769031in}}%
\pgfpathlineto{\pgfqpoint{2.202698in}{0.781412in}}%
\pgfpathlineto{\pgfqpoint{2.199678in}{0.781412in}}%
\pgfpathlineto{\pgfqpoint{2.198171in}{0.793910in}}%
\pgfpathlineto{\pgfqpoint{2.195162in}{0.793910in}}%
\pgfpathlineto{\pgfqpoint{2.193660in}{0.806479in}}%
\pgfpathlineto{\pgfqpoint{2.190662in}{0.806479in}}%
\pgfpathlineto{\pgfqpoint{2.189166in}{0.819358in}}%
\pgfpathlineto{\pgfqpoint{2.186179in}{0.819358in}}%
\pgfpathlineto{\pgfqpoint{2.184688in}{0.831942in}}%
\pgfpathlineto{\pgfqpoint{2.180226in}{0.831942in}}%
\pgfpathlineto{\pgfqpoint{2.178742in}{0.844704in}}%
\pgfpathlineto{\pgfqpoint{2.177260in}{0.844704in}}%
\pgfpathlineto{\pgfqpoint{2.174301in}{0.857060in}}%
\pgfpathlineto{\pgfqpoint{2.172825in}{0.857060in}}%
\pgfpathlineto{\pgfqpoint{2.171350in}{0.869301in}}%
\pgfpathlineto{\pgfqpoint{2.168405in}{0.869301in}}%
\pgfpathlineto{\pgfqpoint{2.166936in}{0.881228in}}%
\pgfpathlineto{\pgfqpoint{2.162537in}{0.881228in}}%
\pgfpathlineto{\pgfqpoint{2.161075in}{0.893289in}}%
\pgfpathlineto{\pgfqpoint{2.158155in}{0.893289in}}%
\pgfpathlineto{\pgfqpoint{2.156697in}{0.904622in}}%
\pgfpathlineto{\pgfqpoint{2.153788in}{0.904622in}}%
\pgfpathlineto{\pgfqpoint{2.152335in}{0.915703in}}%
\pgfpathlineto{\pgfqpoint{2.150885in}{0.915703in}}%
\pgfpathlineto{\pgfqpoint{2.147989in}{0.926342in}}%
\pgfpathlineto{\pgfqpoint{2.145100in}{0.926342in}}%
\pgfpathlineto{\pgfqpoint{2.143658in}{0.936717in}}%
\pgfpathlineto{\pgfqpoint{2.140779in}{0.936717in}}%
\pgfpathlineto{\pgfqpoint{2.139343in}{0.946561in}}%
\pgfpathlineto{\pgfqpoint{2.136474in}{0.946561in}}%
\pgfpathlineto{\pgfqpoint{2.135042in}{0.955916in}}%
\pgfpathlineto{\pgfqpoint{2.132184in}{0.955916in}}%
\pgfpathlineto{\pgfqpoint{2.130757in}{0.964828in}}%
\pgfpathlineto{\pgfqpoint{2.127909in}{0.964828in}}%
\pgfpathlineto{\pgfqpoint{2.126487in}{0.973264in}}%
\pgfpathlineto{\pgfqpoint{2.123649in}{0.973264in}}%
\pgfpathlineto{\pgfqpoint{2.122232in}{0.981128in}}%
\pgfpathlineto{\pgfqpoint{2.117992in}{0.981128in}}%
\pgfpathlineto{\pgfqpoint{2.116582in}{0.988552in}}%
\pgfpathlineto{\pgfqpoint{2.115174in}{0.988552in}}%
\pgfpathlineto{\pgfqpoint{2.113767in}{0.995502in}}%
\pgfpathlineto{\pgfqpoint{2.110958in}{0.995502in}}%
\pgfpathlineto{\pgfqpoint{2.109556in}{1.001984in}}%
\pgfpathlineto{\pgfqpoint{2.106758in}{1.001984in}}%
\pgfpathlineto{\pgfqpoint{2.105361in}{1.008040in}}%
\pgfpathlineto{\pgfqpoint{2.101180in}{1.008040in}}%
\pgfpathlineto{\pgfqpoint{2.099789in}{1.013479in}}%
\pgfpathlineto{\pgfqpoint{2.097013in}{1.013479in}}%
\pgfpathlineto{\pgfqpoint{2.095627in}{1.018438in}}%
\pgfpathlineto{\pgfqpoint{2.092861in}{1.018438in}}%
\pgfpathlineto{\pgfqpoint{2.091480in}{1.022879in}}%
\pgfpathlineto{\pgfqpoint{2.090101in}{1.022879in}}%
\pgfpathlineto{\pgfqpoint{2.087347in}{1.026739in}}%
\pgfpathlineto{\pgfqpoint{2.085973in}{1.026739in}}%
\pgfpathlineto{\pgfqpoint{2.081859in}{1.030144in}}%
\pgfpathlineto{\pgfqpoint{2.080491in}{1.030144in}}%
\pgfpathlineto{\pgfqpoint{2.079124in}{1.033046in}}%
\pgfpathlineto{\pgfqpoint{2.076396in}{1.033046in}}%
\pgfpathlineto{\pgfqpoint{2.075034in}{1.035522in}}%
\pgfpathlineto{\pgfqpoint{2.072315in}{1.036172in}}%
\pgfpathlineto{\pgfqpoint{2.069602in}{1.037472in}}%
\pgfpathlineto{\pgfqpoint{2.065545in}{1.038780in}}%
\pgfpathlineto{\pgfqpoint{2.060157in}{1.039737in}}%
\pgfpathlineto{\pgfqpoint{2.052120in}{1.040108in}}%
\pgfpathlineto{\pgfqpoint{2.046793in}{1.038546in}}%
\pgfpathlineto{\pgfqpoint{2.042813in}{1.037854in}}%
\pgfpathlineto{\pgfqpoint{2.028337in}{1.029921in}}%
\pgfpathlineto{\pgfqpoint{2.027029in}{1.026698in}}%
\pgfpathlineto{\pgfqpoint{2.023117in}{1.026698in}}%
\pgfpathlineto{\pgfqpoint{2.021815in}{1.022978in}}%
\pgfpathlineto{\pgfqpoint{2.019217in}{1.022978in}}%
\pgfpathlineto{\pgfqpoint{2.017920in}{1.018885in}}%
\pgfpathlineto{\pgfqpoint{2.015331in}{1.018885in}}%
\pgfpathlineto{\pgfqpoint{2.014038in}{1.014226in}}%
\pgfpathlineto{\pgfqpoint{2.011457in}{1.014226in}}%
\pgfpathlineto{\pgfqpoint{2.010169in}{1.009259in}}%
\pgfpathlineto{\pgfqpoint{2.007597in}{1.009259in}}%
\pgfpathlineto{\pgfqpoint{2.005031in}{1.003884in}}%
\pgfpathlineto{\pgfqpoint{2.003750in}{1.003884in}}%
\pgfpathlineto{\pgfqpoint{2.002470in}{0.998127in}}%
\pgfpathlineto{\pgfqpoint{1.999915in}{0.998127in}}%
\pgfpathlineto{\pgfqpoint{1.998640in}{0.991886in}}%
\pgfpathlineto{\pgfqpoint{1.996093in}{0.991886in}}%
\pgfpathlineto{\pgfqpoint{1.994822in}{0.985373in}}%
\pgfpathlineto{\pgfqpoint{1.992284in}{0.985373in}}%
\pgfpathlineto{\pgfqpoint{1.991017in}{0.978499in}}%
\pgfpathlineto{\pgfqpoint{1.987225in}{0.978499in}}%
\pgfpathlineto{\pgfqpoint{1.985963in}{0.971164in}}%
\pgfpathlineto{\pgfqpoint{1.984704in}{0.971164in}}%
\pgfpathlineto{\pgfqpoint{1.982188in}{0.963582in}}%
\pgfpathlineto{\pgfqpoint{1.980932in}{0.963582in}}%
\pgfpathlineto{\pgfqpoint{1.978425in}{0.955687in}}%
\pgfpathlineto{\pgfqpoint{1.977173in}{0.955687in}}%
\pgfpathlineto{\pgfqpoint{1.975923in}{0.947521in}}%
\pgfpathlineto{\pgfqpoint{1.973427in}{0.947521in}}%
\pgfpathlineto{\pgfqpoint{1.972181in}{0.938997in}}%
\pgfpathlineto{\pgfqpoint{1.968451in}{0.938997in}}%
\pgfpathlineto{\pgfqpoint{1.967210in}{0.930085in}}%
\pgfpathlineto{\pgfqpoint{1.964733in}{0.930085in}}%
\pgfpathlineto{\pgfqpoint{1.963496in}{0.920969in}}%
\pgfpathlineto{\pgfqpoint{1.962261in}{0.920969in}}%
\pgfpathlineto{\pgfqpoint{1.959795in}{0.911673in}}%
\pgfpathlineto{\pgfqpoint{1.958564in}{0.911673in}}%
\pgfpathlineto{\pgfqpoint{1.956105in}{0.902171in}}%
\pgfpathlineto{\pgfqpoint{1.953653in}{0.902171in}}%
\pgfpathlineto{\pgfqpoint{1.952428in}{0.892413in}}%
\pgfpathlineto{\pgfqpoint{1.949983in}{0.892413in}}%
\pgfpathlineto{\pgfqpoint{1.948763in}{0.882394in}}%
\pgfpathlineto{\pgfqpoint{1.946326in}{0.882394in}}%
\pgfpathlineto{\pgfqpoint{1.945109in}{0.872171in}}%
\pgfpathlineto{\pgfqpoint{1.942680in}{0.872171in}}%
\pgfpathlineto{\pgfqpoint{1.941468in}{0.861859in}}%
\pgfpathlineto{\pgfqpoint{1.940257in}{0.861859in}}%
\pgfpathlineto{\pgfqpoint{1.937838in}{0.851476in}}%
\pgfpathlineto{\pgfqpoint{1.935425in}{0.851476in}}%
\pgfpathlineto{\pgfqpoint{1.934220in}{0.840880in}}%
\pgfpathlineto{\pgfqpoint{1.931815in}{0.840880in}}%
\pgfpathlineto{\pgfqpoint{1.929415in}{0.830188in}}%
\pgfpathlineto{\pgfqpoint{1.928217in}{0.830188in}}%
\pgfpathlineto{\pgfqpoint{1.927020in}{0.819503in}}%
\pgfpathlineto{\pgfqpoint{1.924630in}{0.819503in}}%
\pgfpathlineto{\pgfqpoint{1.923437in}{0.808757in}}%
\pgfpathlineto{\pgfqpoint{1.921055in}{0.808757in}}%
\pgfpathlineto{\pgfqpoint{1.919865in}{0.797856in}}%
\pgfpathlineto{\pgfqpoint{1.916306in}{0.797856in}}%
\pgfpathlineto{\pgfqpoint{1.915121in}{0.787151in}}%
\pgfpathlineto{\pgfqpoint{1.913939in}{0.787151in}}%
\pgfpathlineto{\pgfqpoint{1.911577in}{0.776403in}}%
\pgfpathlineto{\pgfqpoint{1.910398in}{0.776403in}}%
\pgfpathlineto{\pgfqpoint{1.908043in}{0.765769in}}%
\pgfpathlineto{\pgfqpoint{1.906868in}{0.765769in}}%
\pgfpathlineto{\pgfqpoint{1.905694in}{0.755413in}}%
\pgfpathlineto{\pgfqpoint{1.902180in}{0.755413in}}%
\pgfpathlineto{\pgfqpoint{1.901011in}{0.744950in}}%
\pgfpathlineto{\pgfqpoint{1.898677in}{0.744950in}}%
\pgfpathlineto{\pgfqpoint{1.897511in}{0.734738in}}%
\pgfpathlineto{\pgfqpoint{1.895184in}{0.734738in}}%
\pgfpathlineto{\pgfqpoint{1.894023in}{0.724507in}}%
\pgfpathlineto{\pgfqpoint{1.891703in}{0.724507in}}%
\pgfpathlineto{\pgfqpoint{1.890546in}{0.714791in}}%
\pgfpathlineto{\pgfqpoint{1.888233in}{0.714791in}}%
\pgfpathlineto{\pgfqpoint{1.887079in}{0.705427in}}%
\pgfpathlineto{\pgfqpoint{1.884775in}{0.705427in}}%
\pgfpathlineto{\pgfqpoint{1.883624in}{0.696383in}}%
\pgfpathlineto{\pgfqpoint{1.881327in}{0.696383in}}%
\pgfpathlineto{\pgfqpoint{1.880180in}{0.687665in}}%
\pgfpathlineto{\pgfqpoint{1.877890in}{0.687665in}}%
\pgfpathlineto{\pgfqpoint{1.876746in}{0.679411in}}%
\pgfpathlineto{\pgfqpoint{1.874463in}{0.679411in}}%
\pgfpathlineto{\pgfqpoint{1.873324in}{0.671707in}}%
\pgfpathlineto{\pgfqpoint{1.869912in}{0.671707in}}%
\pgfpathlineto{\pgfqpoint{1.868777in}{0.664642in}}%
\pgfpathlineto{\pgfqpoint{1.867643in}{0.664642in}}%
\pgfpathlineto{\pgfqpoint{1.863120in}{0.652071in}}%
\pgfpathlineto{\pgfqpoint{1.860866in}{0.652071in}}%
\pgfpathlineto{\pgfqpoint{1.858616in}{0.646975in}}%
\pgfpathlineto{\pgfqpoint{1.857493in}{0.646975in}}%
\pgfpathlineto{\pgfqpoint{1.855250in}{0.642610in}}%
\pgfpathlineto{\pgfqpoint{1.853012in}{0.642610in}}%
\pgfpathlineto{\pgfqpoint{1.850779in}{0.638972in}}%
\pgfpathlineto{\pgfqpoint{1.847438in}{0.636451in}}%
\pgfpathlineto{\pgfqpoint{1.842999in}{0.634549in}}%
\pgfpathlineto{\pgfqpoint{1.839682in}{0.634229in}}%
\pgfpathlineto{\pgfqpoint{1.833079in}{0.636335in}}%
\pgfpathlineto{\pgfqpoint{1.830887in}{0.638948in}}%
\pgfpathlineto{\pgfqpoint{1.829792in}{0.638948in}}%
\pgfpathlineto{\pgfqpoint{1.827607in}{0.642772in}}%
\pgfpathlineto{\pgfqpoint{1.826516in}{0.642772in}}%
\pgfpathlineto{\pgfqpoint{1.824338in}{0.647601in}}%
\pgfpathlineto{\pgfqpoint{1.823250in}{0.647601in}}%
\pgfpathlineto{\pgfqpoint{1.817829in}{0.660923in}}%
\pgfpathlineto{\pgfqpoint{1.816748in}{0.660923in}}%
\pgfpathlineto{\pgfqpoint{1.814589in}{0.669015in}}%
\pgfpathlineto{\pgfqpoint{1.813512in}{0.669015in}}%
\pgfpathlineto{\pgfqpoint{1.811360in}{0.678018in}}%
\pgfpathlineto{\pgfqpoint{1.810286in}{0.678018in}}%
\pgfpathlineto{\pgfqpoint{1.808140in}{0.688073in}}%
\pgfpathlineto{\pgfqpoint{1.805999in}{0.688073in}}%
\pgfpathlineto{\pgfqpoint{1.803863in}{0.698909in}}%
\pgfpathlineto{\pgfqpoint{1.802796in}{0.698909in}}%
\pgfpathlineto{\pgfqpoint{1.800666in}{0.710607in}}%
\pgfpathlineto{\pgfqpoint{1.799603in}{0.710607in}}%
\pgfpathlineto{\pgfqpoint{1.797479in}{0.722608in}}%
\pgfpathlineto{\pgfqpoint{1.794302in}{0.735461in}}%
\pgfpathlineto{\pgfqpoint{1.793245in}{0.735461in}}%
\pgfpathlineto{\pgfqpoint{1.791134in}{0.748631in}}%
\pgfpathlineto{\pgfqpoint{1.790080in}{0.748631in}}%
\pgfpathlineto{\pgfqpoint{1.787976in}{0.762372in}}%
\pgfpathlineto{\pgfqpoint{1.786925in}{0.762372in}}%
\pgfpathlineto{\pgfqpoint{1.784827in}{0.776339in}}%
\pgfpathlineto{\pgfqpoint{1.783780in}{0.776339in}}%
\pgfpathlineto{\pgfqpoint{1.781688in}{0.790561in}}%
\pgfpathlineto{\pgfqpoint{1.780644in}{0.790561in}}%
\pgfpathlineto{\pgfqpoint{1.778559in}{0.805207in}}%
\pgfpathlineto{\pgfqpoint{1.777518in}{0.805207in}}%
\pgfpathlineto{\pgfqpoint{1.772328in}{0.834441in}}%
\pgfpathlineto{\pgfqpoint{1.771293in}{0.834441in}}%
\pgfpathlineto{\pgfqpoint{1.769227in}{0.849115in}}%
\pgfpathlineto{\pgfqpoint{1.768195in}{0.849115in}}%
\pgfpathlineto{\pgfqpoint{1.766134in}{0.863796in}}%
\pgfpathlineto{\pgfqpoint{1.764078in}{0.863796in}}%
\pgfpathlineto{\pgfqpoint{1.761002in}{0.893033in}}%
\pgfpathlineto{\pgfqpoint{1.758956in}{0.893033in}}%
\pgfpathlineto{\pgfqpoint{1.756914in}{0.907138in}}%
\pgfpathlineto{\pgfqpoint{1.755894in}{0.907138in}}%
\pgfpathlineto{\pgfqpoint{1.753858in}{0.921355in}}%
\pgfpathlineto{\pgfqpoint{1.752842in}{0.921355in}}%
\pgfpathlineto{\pgfqpoint{1.750812in}{0.935105in}}%
\pgfpathlineto{\pgfqpoint{1.748786in}{0.935105in}}%
\pgfpathlineto{\pgfqpoint{1.746765in}{0.949058in}}%
\pgfpathlineto{\pgfqpoint{1.741728in}{0.975887in}}%
\pgfpathlineto{\pgfqpoint{1.740723in}{0.975887in}}%
\pgfpathlineto{\pgfqpoint{1.735716in}{1.001207in}}%
\pgfpathlineto{\pgfqpoint{1.733720in}{1.001207in}}%
\pgfpathlineto{\pgfqpoint{1.731728in}{1.013505in}}%
\pgfpathlineto{\pgfqpoint{1.730733in}{1.013505in}}%
\pgfpathlineto{\pgfqpoint{1.728747in}{1.025604in}}%
\pgfpathlineto{\pgfqpoint{1.727756in}{1.025604in}}%
\pgfpathlineto{\pgfqpoint{1.725775in}{1.037152in}}%
\pgfpathlineto{\pgfqpoint{1.724786in}{1.037152in}}%
\pgfpathlineto{\pgfqpoint{1.722812in}{1.048581in}}%
\pgfpathlineto{\pgfqpoint{1.721826in}{1.048581in}}%
\pgfpathlineto{\pgfqpoint{1.716911in}{1.070289in}}%
\pgfpathlineto{\pgfqpoint{1.715931in}{1.070289in}}%
\pgfpathlineto{\pgfqpoint{1.713974in}{1.080666in}}%
\pgfpathlineto{\pgfqpoint{1.712997in}{1.080666in}}%
\pgfpathlineto{\pgfqpoint{1.711045in}{1.090491in}}%
\pgfpathlineto{\pgfqpoint{1.710071in}{1.090491in}}%
\pgfpathlineto{\pgfqpoint{1.708125in}{1.100137in}}%
\pgfpathlineto{\pgfqpoint{1.707154in}{1.100137in}}%
\pgfpathlineto{\pgfqpoint{1.702310in}{1.118533in}}%
\pgfpathlineto{\pgfqpoint{1.701344in}{1.118533in}}%
\pgfpathlineto{\pgfqpoint{1.696529in}{1.135399in}}%
\pgfpathlineto{\pgfqpoint{1.695569in}{1.135399in}}%
\pgfpathlineto{\pgfqpoint{1.690781in}{1.150784in}}%
\pgfpathlineto{\pgfqpoint{1.689826in}{1.150784in}}%
\pgfpathlineto{\pgfqpoint{1.687919in}{1.157891in}}%
\pgfpathlineto{\pgfqpoint{1.686016in}{1.157891in}}%
\pgfpathlineto{\pgfqpoint{1.684117in}{1.164830in}}%
\pgfpathlineto{\pgfqpoint{1.683169in}{1.164830in}}%
\pgfpathlineto{\pgfqpoint{1.681275in}{1.171239in}}%
\pgfpathlineto{\pgfqpoint{1.680329in}{1.171239in}}%
\pgfpathlineto{\pgfqpoint{1.678440in}{1.177534in}}%
\pgfpathlineto{\pgfqpoint{1.677498in}{1.177534in}}%
\pgfpathlineto{\pgfqpoint{1.675614in}{1.183199in}}%
\pgfpathlineto{\pgfqpoint{1.674674in}{1.183199in}}%
\pgfpathlineto{\pgfqpoint{1.672796in}{1.188516in}}%
\pgfpathlineto{\pgfqpoint{1.671859in}{1.188516in}}%
\pgfpathlineto{\pgfqpoint{1.669987in}{1.193694in}}%
\pgfpathlineto{\pgfqpoint{1.669052in}{1.193694in}}%
\pgfpathlineto{\pgfqpoint{1.667185in}{1.198411in}}%
\pgfpathlineto{\pgfqpoint{1.666252in}{1.198411in}}%
\pgfpathlineto{\pgfqpoint{1.664391in}{1.202903in}}%
\pgfpathlineto{\pgfqpoint{1.663461in}{1.202903in}}%
\pgfpathlineto{\pgfqpoint{1.661605in}{1.206867in}}%
\pgfpathlineto{\pgfqpoint{1.660678in}{1.206867in}}%
\pgfpathlineto{\pgfqpoint{1.658827in}{1.210528in}}%
\pgfpathlineto{\pgfqpoint{1.657903in}{1.210528in}}%
\pgfpathlineto{\pgfqpoint{1.655135in}{1.213860in}}%
\pgfpathlineto{\pgfqpoint{1.654214in}{1.213860in}}%
\pgfpathlineto{\pgfqpoint{1.651457in}{1.219748in}}%
\pgfpathlineto{\pgfqpoint{1.649623in}{1.219748in}}%
\pgfpathlineto{\pgfqpoint{1.647793in}{1.222031in}}%
\pgfpathlineto{\pgfqpoint{1.646879in}{1.222031in}}%
\pgfpathlineto{\pgfqpoint{1.645054in}{1.224019in}}%
\pgfpathlineto{\pgfqpoint{1.643233in}{1.224019in}}%
\pgfpathlineto{\pgfqpoint{1.641414in}{1.225705in}}%
\pgfpathlineto{\pgfqpoint{1.640507in}{1.225705in}}%
\pgfpathlineto{\pgfqpoint{1.638693in}{1.227112in}}%
\pgfpathlineto{\pgfqpoint{1.635077in}{1.228183in}}%
\pgfpathlineto{\pgfqpoint{1.632374in}{1.229118in}}%
\pgfpathlineto{\pgfqpoint{1.626990in}{1.229435in}}%
\pgfpathlineto{\pgfqpoint{1.613662in}{1.225223in}}%
\pgfpathlineto{\pgfqpoint{1.611899in}{1.223491in}}%
\pgfpathlineto{\pgfqpoint{1.610139in}{1.222405in}}%
\pgfpathlineto{\pgfqpoint{1.606629in}{1.218956in}}%
\pgfpathlineto{\pgfqpoint{1.604879in}{1.218956in}}%
\pgfpathlineto{\pgfqpoint{1.603132in}{1.216267in}}%
\pgfpathlineto{\pgfqpoint{1.600517in}{1.213311in}}%
\pgfpathlineto{\pgfqpoint{1.599647in}{1.213311in}}%
\pgfpathlineto{\pgfqpoint{1.597910in}{1.210033in}}%
\pgfpathlineto{\pgfqpoint{1.593581in}{1.202652in}}%
\pgfpathlineto{\pgfqpoint{1.591854in}{1.202652in}}%
\pgfpathlineto{\pgfqpoint{1.590131in}{1.198485in}}%
\pgfpathlineto{\pgfqpoint{1.589271in}{1.198485in}}%
\pgfpathlineto{\pgfqpoint{1.587553in}{1.194026in}}%
\pgfpathlineto{\pgfqpoint{1.586695in}{1.194026in}}%
\pgfpathlineto{\pgfqpoint{1.584981in}{1.189410in}}%
\pgfpathlineto{\pgfqpoint{1.584125in}{1.189410in}}%
\pgfpathlineto{\pgfqpoint{1.582416in}{1.184497in}}%
\pgfpathlineto{\pgfqpoint{1.581563in}{1.184497in}}%
\pgfpathlineto{\pgfqpoint{1.579859in}{1.179065in}}%
\pgfpathlineto{\pgfqpoint{1.579008in}{1.179065in}}%
\pgfpathlineto{\pgfqpoint{1.574764in}{1.167941in}}%
\pgfpathlineto{\pgfqpoint{1.573918in}{1.167941in}}%
\pgfpathlineto{\pgfqpoint{1.569697in}{1.155519in}}%
\pgfpathlineto{\pgfqpoint{1.568015in}{1.155519in}}%
\pgfpathlineto{\pgfqpoint{1.566335in}{1.148801in}}%
\pgfpathlineto{\pgfqpoint{1.562148in}{1.135056in}}%
\pgfpathlineto{\pgfqpoint{1.561313in}{1.135056in}}%
\pgfpathlineto{\pgfqpoint{1.557150in}{1.120016in}}%
\pgfpathlineto{\pgfqpoint{1.555490in}{1.120016in}}%
\pgfpathlineto{\pgfqpoint{1.551352in}{1.104421in}}%
\pgfpathlineto{\pgfqpoint{1.550527in}{1.104421in}}%
\pgfpathlineto{\pgfqpoint{1.548878in}{1.096275in}}%
\pgfpathlineto{\pgfqpoint{1.548055in}{1.096275in}}%
\pgfpathlineto{\pgfqpoint{1.546411in}{1.087934in}}%
\pgfpathlineto{\pgfqpoint{1.545590in}{1.087934in}}%
\pgfpathlineto{\pgfqpoint{1.541497in}{1.070418in}}%
\pgfpathlineto{\pgfqpoint{1.540680in}{1.070418in}}%
\pgfpathlineto{\pgfqpoint{1.536609in}{1.052098in}}%
\pgfpathlineto{\pgfqpoint{1.535796in}{1.052098in}}%
\pgfpathlineto{\pgfqpoint{1.534174in}{1.042708in}}%
\pgfpathlineto{\pgfqpoint{1.533364in}{1.042708in}}%
\pgfpathlineto{\pgfqpoint{1.529325in}{1.023563in}}%
\pgfpathlineto{\pgfqpoint{1.527714in}{1.023563in}}%
\pgfpathlineto{\pgfqpoint{1.526106in}{1.013660in}}%
\pgfpathlineto{\pgfqpoint{1.522099in}{0.993323in}}%
\pgfpathlineto{\pgfqpoint{1.520501in}{0.993323in}}%
\pgfpathlineto{\pgfqpoint{1.518906in}{0.983136in}}%
\pgfpathlineto{\pgfqpoint{1.518110in}{0.983136in}}%
\pgfpathlineto{\pgfqpoint{1.516519in}{0.972761in}}%
\pgfpathlineto{\pgfqpoint{1.515724in}{0.972761in}}%
\pgfpathlineto{\pgfqpoint{1.514137in}{0.962238in}}%
\pgfpathlineto{\pgfqpoint{1.513345in}{0.962238in}}%
\pgfpathlineto{\pgfqpoint{1.511762in}{0.951551in}}%
\pgfpathlineto{\pgfqpoint{1.510972in}{0.951551in}}%
\pgfpathlineto{\pgfqpoint{1.509394in}{0.940853in}}%
\pgfpathlineto{\pgfqpoint{1.507031in}{0.929884in}}%
\pgfpathlineto{\pgfqpoint{1.506245in}{0.929884in}}%
\pgfpathlineto{\pgfqpoint{1.504675in}{0.918948in}}%
\pgfpathlineto{\pgfqpoint{1.503891in}{0.918948in}}%
\pgfpathlineto{\pgfqpoint{1.502325in}{0.908126in}}%
\pgfpathlineto{\pgfqpoint{1.501543in}{0.908126in}}%
\pgfpathlineto{\pgfqpoint{1.499981in}{0.897050in}}%
\pgfpathlineto{\pgfqpoint{1.498421in}{0.897050in}}%
\pgfpathlineto{\pgfqpoint{1.496865in}{0.885974in}}%
\pgfpathlineto{\pgfqpoint{1.492985in}{0.863723in}}%
\pgfpathlineto{\pgfqpoint{1.492211in}{0.863723in}}%
\pgfpathlineto{\pgfqpoint{1.488352in}{0.841304in}}%
\pgfpathlineto{\pgfqpoint{1.486813in}{0.841304in}}%
\pgfpathlineto{\pgfqpoint{1.485276in}{0.830305in}}%
\pgfpathlineto{\pgfqpoint{1.484509in}{0.830305in}}%
\pgfpathlineto{\pgfqpoint{1.482976in}{0.819211in}}%
\pgfpathlineto{\pgfqpoint{1.482211in}{0.819211in}}%
\pgfpathlineto{\pgfqpoint{1.480683in}{0.808244in}}%
\pgfpathlineto{\pgfqpoint{1.479919in}{0.808244in}}%
\pgfpathlineto{\pgfqpoint{1.478395in}{0.797305in}}%
\pgfpathlineto{\pgfqpoint{1.477633in}{0.797305in}}%
\pgfpathlineto{\pgfqpoint{1.473837in}{0.775379in}}%
\pgfpathlineto{\pgfqpoint{1.473079in}{0.775379in}}%
\pgfpathlineto{\pgfqpoint{1.471566in}{0.764715in}}%
\pgfpathlineto{\pgfqpoint{1.470811in}{0.764715in}}%
\pgfpathlineto{\pgfqpoint{1.469302in}{0.754018in}}%
\pgfpathlineto{\pgfqpoint{1.468548in}{0.754018in}}%
\pgfpathlineto{\pgfqpoint{1.467043in}{0.743545in}}%
\pgfpathlineto{\pgfqpoint{1.466292in}{0.743545in}}%
\pgfpathlineto{\pgfqpoint{1.464790in}{0.733340in}}%
\pgfpathlineto{\pgfqpoint{1.464041in}{0.733340in}}%
\pgfpathlineto{\pgfqpoint{1.460302in}{0.713001in}}%
\pgfpathlineto{\pgfqpoint{1.459556in}{0.713001in}}%
\pgfpathlineto{\pgfqpoint{1.458066in}{0.703202in}}%
\pgfpathlineto{\pgfqpoint{1.456579in}{0.703202in}}%
\pgfpathlineto{\pgfqpoint{1.455094in}{0.693663in}}%
\pgfpathlineto{\pgfqpoint{1.454353in}{0.693663in}}%
\pgfpathlineto{\pgfqpoint{1.452872in}{0.684280in}}%
\pgfpathlineto{\pgfqpoint{1.449180in}{0.666347in}}%
\pgfpathlineto{\pgfqpoint{1.447708in}{0.666347in}}%
\pgfpathlineto{\pgfqpoint{1.446238in}{0.657751in}}%
\pgfpathlineto{\pgfqpoint{1.445504in}{0.657751in}}%
\pgfpathlineto{\pgfqpoint{1.444038in}{0.649566in}}%
\pgfpathlineto{\pgfqpoint{1.443306in}{0.649566in}}%
\pgfpathlineto{\pgfqpoint{1.441114in}{0.637810in}}%
\pgfpathlineto{\pgfqpoint{1.439655in}{0.634025in}}%
\pgfpathlineto{\pgfqpoint{1.438927in}{0.634025in}}%
\pgfpathlineto{\pgfqpoint{1.437472in}{0.626765in}}%
\pgfpathlineto{\pgfqpoint{1.436745in}{0.626765in}}%
\pgfpathlineto{\pgfqpoint{1.433122in}{0.613779in}}%
\pgfpathlineto{\pgfqpoint{1.432399in}{0.613779in}}%
\pgfpathlineto{\pgfqpoint{1.430955in}{0.608078in}}%
\pgfpathlineto{\pgfqpoint{1.430234in}{0.608078in}}%
\pgfpathlineto{\pgfqpoint{1.428793in}{0.602779in}}%
\pgfpathlineto{\pgfqpoint{1.427355in}{0.602779in}}%
\pgfpathlineto{\pgfqpoint{1.425203in}{0.594124in}}%
\pgfpathlineto{\pgfqpoint{1.423771in}{0.594124in}}%
\pgfpathlineto{\pgfqpoint{1.420201in}{0.587983in}}%
\pgfpathlineto{\pgfqpoint{1.419489in}{0.587983in}}%
\pgfpathlineto{\pgfqpoint{1.418066in}{0.585856in}}%
\pgfpathlineto{\pgfqpoint{1.416646in}{0.585205in}}%
\pgfpathlineto{\pgfqpoint{1.413813in}{0.583919in}}%
\pgfpathlineto{\pgfqpoint{1.410285in}{0.584608in}}%
\pgfpathlineto{\pgfqpoint{1.406070in}{0.586952in}}%
\pgfpathlineto{\pgfqpoint{1.404670in}{0.589567in}}%
\pgfpathlineto{\pgfqpoint{1.403970in}{0.589567in}}%
\pgfpathlineto{\pgfqpoint{1.402574in}{0.593128in}}%
\pgfpathlineto{\pgfqpoint{1.401876in}{0.593128in}}%
\pgfpathlineto{\pgfqpoint{1.400483in}{0.597607in}}%
\pgfpathlineto{\pgfqpoint{1.399787in}{0.597607in}}%
\pgfpathlineto{\pgfqpoint{1.398397in}{0.602916in}}%
\pgfpathlineto{\pgfqpoint{1.397703in}{0.602916in}}%
\pgfpathlineto{\pgfqpoint{1.394240in}{0.616127in}}%
\pgfpathlineto{\pgfqpoint{1.393550in}{0.616127in}}%
\pgfpathlineto{\pgfqpoint{1.392170in}{0.624172in}}%
\pgfpathlineto{\pgfqpoint{1.391481in}{0.624172in}}%
\pgfpathlineto{\pgfqpoint{1.390105in}{0.633081in}}%
\pgfpathlineto{\pgfqpoint{1.389417in}{0.633081in}}%
\pgfpathlineto{\pgfqpoint{1.388044in}{0.642741in}}%
\pgfpathlineto{\pgfqpoint{1.386674in}{0.642741in}}%
\pgfpathlineto{\pgfqpoint{1.384622in}{0.659028in}}%
\pgfpathlineto{\pgfqpoint{1.381893in}{0.676675in}}%
\pgfpathlineto{\pgfqpoint{1.381213in}{0.676675in}}%
\pgfpathlineto{\pgfqpoint{1.377818in}{0.702469in}}%
\pgfpathlineto{\pgfqpoint{1.376464in}{0.702469in}}%
\pgfpathlineto{\pgfqpoint{1.375112in}{0.716506in}}%
\pgfpathlineto{\pgfqpoint{1.374437in}{0.716506in}}%
\pgfpathlineto{\pgfqpoint{1.373088in}{0.730608in}}%
\pgfpathlineto{\pgfqpoint{1.372414in}{0.730608in}}%
\pgfpathlineto{\pgfqpoint{1.371069in}{0.745342in}}%
\pgfpathlineto{\pgfqpoint{1.370397in}{0.745342in}}%
\pgfpathlineto{\pgfqpoint{1.369055in}{0.760298in}}%
\pgfpathlineto{\pgfqpoint{1.368385in}{0.760298in}}%
\pgfpathlineto{\pgfqpoint{1.365041in}{0.792727in}}%
\pgfpathlineto{\pgfqpoint{1.364374in}{0.792727in}}%
\pgfpathlineto{\pgfqpoint{1.363042in}{0.807396in}}%
\pgfpathlineto{\pgfqpoint{1.362376in}{0.807396in}}%
\pgfpathlineto{\pgfqpoint{1.361047in}{0.823274in}}%
\pgfpathlineto{\pgfqpoint{1.360383in}{0.823274in}}%
\pgfpathlineto{\pgfqpoint{1.359057in}{0.839288in}}%
\pgfpathlineto{\pgfqpoint{1.358395in}{0.839288in}}%
\pgfpathlineto{\pgfqpoint{1.355092in}{0.871782in}}%
\pgfpathlineto{\pgfqpoint{1.354433in}{0.871782in}}%
\pgfpathlineto{\pgfqpoint{1.351146in}{0.904530in}}%
\pgfpathlineto{\pgfqpoint{1.350490in}{0.904530in}}%
\pgfpathlineto{\pgfqpoint{1.348525in}{0.920654in}}%
\pgfpathlineto{\pgfqpoint{1.347871in}{0.920654in}}%
\pgfpathlineto{\pgfqpoint{1.346565in}{0.937014in}}%
\pgfpathlineto{\pgfqpoint{1.345913in}{0.937014in}}%
\pgfpathlineto{\pgfqpoint{1.343959in}{0.961101in}}%
\pgfpathlineto{\pgfqpoint{1.341362in}{0.985256in}}%
\pgfpathlineto{\pgfqpoint{1.340066in}{0.985256in}}%
\pgfpathlineto{\pgfqpoint{1.338772in}{1.000946in}}%
\pgfpathlineto{\pgfqpoint{1.338126in}{1.000946in}}%
\pgfpathlineto{\pgfqpoint{1.336836in}{1.016760in}}%
\pgfpathlineto{\pgfqpoint{1.336191in}{1.016760in}}%
\pgfpathlineto{\pgfqpoint{1.334904in}{1.032302in}}%
\pgfpathlineto{\pgfqpoint{1.334261in}{1.032302in}}%
\pgfpathlineto{\pgfqpoint{1.332976in}{1.047659in}}%
\pgfpathlineto{\pgfqpoint{1.332335in}{1.047659in}}%
\pgfpathlineto{\pgfqpoint{1.329135in}{1.077556in}}%
\pgfpathlineto{\pgfqpoint{1.328496in}{1.077556in}}%
\pgfpathlineto{\pgfqpoint{1.325311in}{1.106884in}}%
\pgfpathlineto{\pgfqpoint{1.324676in}{1.106884in}}%
\pgfpathlineto{\pgfqpoint{1.322773in}{1.121207in}}%
\pgfpathlineto{\pgfqpoint{1.322139in}{1.121207in}}%
\pgfpathlineto{\pgfqpoint{1.320242in}{1.142459in}}%
\pgfpathlineto{\pgfqpoint{1.317719in}{1.163144in}}%
\pgfpathlineto{\pgfqpoint{1.317089in}{1.163144in}}%
\pgfpathlineto{\pgfqpoint{1.315204in}{1.176445in}}%
\pgfpathlineto{\pgfqpoint{1.314576in}{1.176445in}}%
\pgfpathlineto{\pgfqpoint{1.313322in}{1.189755in}}%
\pgfpathlineto{\pgfqpoint{1.312696in}{1.189755in}}%
\pgfpathlineto{\pgfqpoint{1.311446in}{1.202911in}}%
\pgfpathlineto{\pgfqpoint{1.310821in}{1.202911in}}%
\pgfpathlineto{\pgfqpoint{1.309573in}{1.215553in}}%
\pgfpathlineto{\pgfqpoint{1.308950in}{1.215553in}}%
\pgfpathlineto{\pgfqpoint{1.307705in}{1.228409in}}%
\pgfpathlineto{\pgfqpoint{1.307084in}{1.228409in}}%
\pgfpathlineto{\pgfqpoint{1.305842in}{1.240567in}}%
\pgfpathlineto{\pgfqpoint{1.305221in}{1.240567in}}%
\pgfpathlineto{\pgfqpoint{1.302127in}{1.264567in}}%
\pgfpathlineto{\pgfqpoint{1.301510in}{1.264567in}}%
\pgfpathlineto{\pgfqpoint{1.300276in}{1.276088in}}%
\pgfpathlineto{\pgfqpoint{1.299661in}{1.276088in}}%
\pgfpathlineto{\pgfqpoint{1.298430in}{1.287479in}}%
\pgfpathlineto{\pgfqpoint{1.297201in}{1.287479in}}%
\pgfpathlineto{\pgfqpoint{1.295362in}{1.309694in}}%
\pgfpathlineto{\pgfqpoint{1.294138in}{1.309694in}}%
\pgfpathlineto{\pgfqpoint{1.292916in}{1.320371in}}%
\pgfpathlineto{\pgfqpoint{1.292306in}{1.320371in}}%
\pgfpathlineto{\pgfqpoint{1.291087in}{1.330787in}}%
\pgfpathlineto{\pgfqpoint{1.290478in}{1.330787in}}%
\pgfpathlineto{\pgfqpoint{1.289262in}{1.340936in}}%
\pgfpathlineto{\pgfqpoint{1.288047in}{1.340936in}}%
\pgfpathlineto{\pgfqpoint{1.286229in}{1.355742in}}%
\pgfpathlineto{\pgfqpoint{1.283811in}{1.370208in}}%
\pgfpathlineto{\pgfqpoint{1.283208in}{1.370208in}}%
\pgfpathlineto{\pgfqpoint{1.281401in}{1.379260in}}%
\pgfpathlineto{\pgfqpoint{1.280799in}{1.379260in}}%
\pgfpathlineto{\pgfqpoint{1.279598in}{1.388373in}}%
\pgfpathlineto{\pgfqpoint{1.278998in}{1.388373in}}%
\pgfpathlineto{\pgfqpoint{1.277799in}{1.397205in}}%
\pgfpathlineto{\pgfqpoint{1.277200in}{1.397205in}}%
\pgfpathlineto{\pgfqpoint{1.276004in}{1.405663in}}%
\pgfpathlineto{\pgfqpoint{1.275407in}{1.405663in}}%
\pgfpathlineto{\pgfqpoint{1.274214in}{1.414034in}}%
\pgfpathlineto{\pgfqpoint{1.273618in}{1.414034in}}%
\pgfpathlineto{\pgfqpoint{1.272427in}{1.421947in}}%
\pgfpathlineto{\pgfqpoint{1.271238in}{1.421947in}}%
\pgfpathlineto{\pgfqpoint{1.269458in}{1.437580in}}%
\pgfpathlineto{\pgfqpoint{1.268274in}{1.437580in}}%
\pgfpathlineto{\pgfqpoint{1.267092in}{1.445096in}}%
\pgfpathlineto{\pgfqpoint{1.266501in}{1.445096in}}%
\pgfpathlineto{\pgfqpoint{1.265321in}{1.452183in}}%
\pgfpathlineto{\pgfqpoint{1.264732in}{1.452183in}}%
\pgfpathlineto{\pgfqpoint{1.263555in}{1.458981in}}%
\pgfpathlineto{\pgfqpoint{1.262380in}{1.458981in}}%
\pgfpathlineto{\pgfqpoint{1.261206in}{1.465664in}}%
\pgfpathlineto{\pgfqpoint{1.260620in}{1.465664in}}%
\pgfpathlineto{\pgfqpoint{1.258864in}{1.475373in}}%
\pgfpathlineto{\pgfqpoint{1.256529in}{1.484731in}}%
\pgfpathlineto{\pgfqpoint{1.255947in}{1.484731in}}%
\pgfpathlineto{\pgfqpoint{1.254202in}{1.490482in}}%
\pgfpathlineto{\pgfqpoint{1.253621in}{1.490482in}}%
\pgfpathlineto{\pgfqpoint{1.252460in}{1.496223in}}%
\pgfpathlineto{\pgfqpoint{1.251881in}{1.496223in}}%
\pgfpathlineto{\pgfqpoint{1.250723in}{1.501649in}}%
\pgfpathlineto{\pgfqpoint{1.250145in}{1.501649in}}%
\pgfpathlineto{\pgfqpoint{1.248990in}{1.506905in}}%
\pgfpathlineto{\pgfqpoint{1.248413in}{1.506905in}}%
\pgfpathlineto{\pgfqpoint{1.247260in}{1.511936in}}%
\pgfpathlineto{\pgfqpoint{1.246684in}{1.511936in}}%
\pgfpathlineto{\pgfqpoint{1.245534in}{1.516780in}}%
\pgfpathlineto{\pgfqpoint{1.244960in}{1.516780in}}%
\pgfpathlineto{\pgfqpoint{1.242095in}{1.525718in}}%
\pgfpathlineto{\pgfqpoint{1.241523in}{1.525718in}}%
\pgfpathlineto{\pgfqpoint{1.240380in}{1.530055in}}%
\pgfpathlineto{\pgfqpoint{1.239810in}{1.530055in}}%
\pgfpathlineto{\pgfqpoint{1.238670in}{1.533973in}}%
\pgfpathlineto{\pgfqpoint{1.238101in}{1.533973in}}%
\pgfpathlineto{\pgfqpoint{1.235261in}{1.541520in}}%
\pgfpathlineto{\pgfqpoint{1.234694in}{1.541520in}}%
\pgfpathlineto{\pgfqpoint{1.230739in}{1.551293in}}%
\pgfpathlineto{\pgfqpoint{1.229050in}{1.551293in}}%
\pgfpathlineto{\pgfqpoint{1.227365in}{1.554142in}}%
\pgfpathlineto{\pgfqpoint{1.225124in}{1.559469in}}%
\pgfpathlineto{\pgfqpoint{1.224006in}{1.559469in}}%
\pgfpathlineto{\pgfqpoint{1.221774in}{1.563956in}}%
\pgfpathlineto{\pgfqpoint{1.220661in}{1.563956in}}%
\pgfpathlineto{\pgfqpoint{1.218440in}{1.567780in}}%
\pgfpathlineto{\pgfqpoint{1.217332in}{1.567780in}}%
\pgfpathlineto{\pgfqpoint{1.215120in}{1.570837in}}%
\pgfpathlineto{\pgfqpoint{1.213466in}{1.571490in}}%
\pgfpathlineto{\pgfqpoint{1.210168in}{1.574120in}}%
\pgfpathlineto{\pgfqpoint{1.207430in}{1.574895in}}%
\pgfpathlineto{\pgfqpoint{1.205793in}{1.575517in}}%
\pgfpathlineto{\pgfqpoint{1.193892in}{1.575045in}}%
\pgfpathlineto{\pgfqpoint{1.191748in}{1.573539in}}%
\pgfpathlineto{\pgfqpoint{1.189077in}{1.572450in}}%
\pgfpathlineto{\pgfqpoint{1.183235in}{1.566670in}}%
\pgfpathlineto{\pgfqpoint{1.182178in}{1.566670in}}%
\pgfpathlineto{\pgfqpoint{1.180068in}{1.562824in}}%
\pgfpathlineto{\pgfqpoint{1.179015in}{1.561736in}}%
\pgfpathlineto{\pgfqpoint{1.176390in}{1.558400in}}%
\pgfpathlineto{\pgfqpoint{1.175866in}{1.558400in}}%
\pgfpathlineto{\pgfqpoint{1.173774in}{1.553327in}}%
\pgfpathlineto{\pgfqpoint{1.172731in}{1.553327in}}%
\pgfpathlineto{\pgfqpoint{1.170648in}{1.547560in}}%
\pgfpathlineto{\pgfqpoint{1.170128in}{1.547560in}}%
\pgfpathlineto{\pgfqpoint{1.167016in}{1.541042in}}%
\pgfpathlineto{\pgfqpoint{1.166499in}{1.541042in}}%
\pgfpathlineto{\pgfqpoint{1.164434in}{1.534121in}}%
\pgfpathlineto{\pgfqpoint{1.163403in}{1.534121in}}%
\pgfpathlineto{\pgfqpoint{1.161347in}{1.526424in}}%
\pgfpathlineto{\pgfqpoint{1.160321in}{1.526424in}}%
\pgfpathlineto{\pgfqpoint{1.156231in}{1.513815in}}%
\pgfpathlineto{\pgfqpoint{1.155721in}{1.513815in}}%
\pgfpathlineto{\pgfqpoint{1.152671in}{1.504696in}}%
\pgfpathlineto{\pgfqpoint{1.152163in}{1.504696in}}%
\pgfpathlineto{\pgfqpoint{1.150138in}{1.494804in}}%
\pgfpathlineto{\pgfqpoint{1.149633in}{1.494804in}}%
\pgfpathlineto{\pgfqpoint{1.147615in}{1.489631in}}%
\pgfpathlineto{\pgfqpoint{1.146105in}{1.460222in}}%
\pgfpathlineto{\pgfqpoint{1.143596in}{1.450401in}}%
\pgfpathlineto{\pgfqpoint{1.143095in}{1.450401in}}%
\pgfpathlineto{\pgfqpoint{1.139599in}{1.427391in}}%
\pgfpathlineto{\pgfqpoint{1.138604in}{1.427391in}}%
\pgfpathlineto{\pgfqpoint{1.135625in}{1.411338in}}%
\pgfpathlineto{\pgfqpoint{1.131672in}{1.402151in}}%
\pgfpathlineto{\pgfqpoint{1.130688in}{1.402151in}}%
\pgfpathlineto{\pgfqpoint{1.129213in}{1.398464in}}%
\pgfpathlineto{\pgfqpoint{1.127742in}{1.399208in}}%
\pgfpathlineto{\pgfqpoint{1.126273in}{1.391639in}}%
\pgfpathlineto{\pgfqpoint{1.124808in}{1.391639in}}%
\pgfpathlineto{\pgfqpoint{1.122859in}{1.378709in}}%
\pgfpathlineto{\pgfqpoint{1.121886in}{1.378709in}}%
\pgfpathlineto{\pgfqpoint{1.119945in}{1.362905in}}%
\pgfpathlineto{\pgfqpoint{1.118977in}{1.362905in}}%
\pgfpathlineto{\pgfqpoint{1.117043in}{1.346332in}}%
\pgfpathlineto{\pgfqpoint{1.116079in}{1.346332in}}%
\pgfpathlineto{\pgfqpoint{1.114154in}{1.329469in}}%
\pgfpathlineto{\pgfqpoint{1.113193in}{1.329469in}}%
\pgfpathlineto{\pgfqpoint{1.111276in}{1.312083in}}%
\pgfpathlineto{\pgfqpoint{1.109841in}{1.312083in}}%
\pgfpathlineto{\pgfqpoint{1.107933in}{1.294032in}}%
\pgfpathlineto{\pgfqpoint{1.107456in}{1.294032in}}%
\pgfpathlineto{\pgfqpoint{1.106030in}{1.284944in}}%
\pgfpathlineto{\pgfqpoint{1.103185in}{1.266203in}}%
\pgfpathlineto{\pgfqpoint{1.102712in}{1.266203in}}%
\pgfpathlineto{\pgfqpoint{1.100823in}{1.247164in}}%
\pgfpathlineto{\pgfqpoint{1.099880in}{1.247164in}}%
\pgfpathlineto{\pgfqpoint{1.097999in}{1.227622in}}%
\pgfpathlineto{\pgfqpoint{1.097060in}{1.227622in}}%
\pgfpathlineto{\pgfqpoint{1.095187in}{1.207649in}}%
\pgfpathlineto{\pgfqpoint{1.094252in}{1.207649in}}%
\pgfpathlineto{\pgfqpoint{1.092386in}{1.187521in}}%
\pgfpathlineto{\pgfqpoint{1.091454in}{1.187521in}}%
\pgfpathlineto{\pgfqpoint{1.089596in}{1.166953in}}%
\pgfpathlineto{\pgfqpoint{1.088668in}{1.166953in}}%
\pgfpathlineto{\pgfqpoint{1.084971in}{1.135340in}}%
\pgfpathlineto{\pgfqpoint{1.084510in}{1.135340in}}%
\pgfpathlineto{\pgfqpoint{1.083130in}{1.124671in}}%
\pgfpathlineto{\pgfqpoint{1.080377in}{1.103387in}}%
\pgfpathlineto{\pgfqpoint{1.079919in}{1.103387in}}%
\pgfpathlineto{\pgfqpoint{1.078092in}{1.081743in}}%
\pgfpathlineto{\pgfqpoint{1.077636in}{1.081743in}}%
\pgfpathlineto{\pgfqpoint{1.074905in}{1.059962in}}%
\pgfpathlineto{\pgfqpoint{1.074451in}{1.059962in}}%
\pgfpathlineto{\pgfqpoint{1.073090in}{1.048965in}}%
\pgfpathlineto{\pgfqpoint{1.069476in}{1.016046in}}%
\pgfpathlineto{\pgfqpoint{1.069025in}{1.016046in}}%
\pgfpathlineto{\pgfqpoint{1.065432in}{0.983141in}}%
\pgfpathlineto{\pgfqpoint{1.064984in}{0.983141in}}%
\pgfpathlineto{\pgfqpoint{1.063196in}{0.961121in}}%
\pgfpathlineto{\pgfqpoint{1.062304in}{0.961121in}}%
\pgfpathlineto{\pgfqpoint{1.060967in}{0.949992in}}%
\pgfpathlineto{\pgfqpoint{1.060523in}{0.949992in}}%
\pgfpathlineto{\pgfqpoint{1.058746in}{0.928388in}}%
\pgfpathlineto{\pgfqpoint{1.057860in}{0.928388in}}%
\pgfpathlineto{\pgfqpoint{1.056090in}{0.906523in}}%
\pgfpathlineto{\pgfqpoint{1.055207in}{0.906523in}}%
\pgfpathlineto{\pgfqpoint{1.053444in}{0.885034in}}%
\pgfpathlineto{\pgfqpoint{1.052565in}{0.885034in}}%
\pgfpathlineto{\pgfqpoint{1.050809in}{0.863908in}}%
\pgfpathlineto{\pgfqpoint{1.049933in}{0.863908in}}%
\pgfpathlineto{\pgfqpoint{1.048184in}{0.847881in}}%
\pgfpathlineto{\pgfqpoint{1.045134in}{0.822105in}}%
\pgfpathlineto{\pgfqpoint{1.044700in}{0.822105in}}%
\pgfpathlineto{\pgfqpoint{1.042965in}{0.801721in}}%
\pgfpathlineto{\pgfqpoint{1.041666in}{0.801721in}}%
\pgfpathlineto{\pgfqpoint{1.039939in}{0.782116in}}%
\pgfpathlineto{\pgfqpoint{1.039508in}{0.782116in}}%
\pgfpathlineto{\pgfqpoint{1.036926in}{0.762759in}}%
\pgfpathlineto{\pgfqpoint{1.036497in}{0.762759in}}%
\pgfpathlineto{\pgfqpoint{1.034783in}{0.743740in}}%
\pgfpathlineto{\pgfqpoint{1.034355in}{0.743740in}}%
\pgfpathlineto{\pgfqpoint{1.031794in}{0.725524in}}%
\pgfpathlineto{\pgfqpoint{1.031368in}{0.725524in}}%
\pgfpathlineto{\pgfqpoint{1.029667in}{0.707796in}}%
\pgfpathlineto{\pgfqpoint{1.028819in}{0.707796in}}%
\pgfpathlineto{\pgfqpoint{1.027124in}{0.690616in}}%
\pgfpathlineto{\pgfqpoint{1.026279in}{0.690616in}}%
\pgfpathlineto{\pgfqpoint{1.024591in}{0.678141in}}%
\pgfpathlineto{\pgfqpoint{1.021647in}{0.658144in}}%
\pgfpathlineto{\pgfqpoint{1.021228in}{0.658144in}}%
\pgfpathlineto{\pgfqpoint{1.017882in}{0.635690in}}%
\pgfpathlineto{\pgfqpoint{1.017465in}{0.635690in}}%
\pgfpathlineto{\pgfqpoint{1.014139in}{0.614808in}}%
\pgfpathlineto{\pgfqpoint{1.013310in}{0.614808in}}%
\pgfpathlineto{\pgfqpoint{1.011655in}{0.601924in}}%
\pgfpathlineto{\pgfqpoint{1.010829in}{0.601924in}}%
\pgfpathlineto{\pgfqpoint{1.009181in}{0.590356in}}%
\pgfpathlineto{\pgfqpoint{1.008358in}{0.590356in}}%
\pgfpathlineto{\pgfqpoint{1.006715in}{0.579641in}}%
\pgfpathlineto{\pgfqpoint{1.005896in}{0.579641in}}%
\pgfpathlineto{\pgfqpoint{1.004260in}{0.570167in}}%
\pgfpathlineto{\pgfqpoint{1.003443in}{0.570167in}}%
\pgfpathlineto{\pgfqpoint{1.001813in}{0.561819in}}%
\pgfpathlineto{\pgfqpoint{1.001000in}{0.561819in}}%
\pgfpathlineto{\pgfqpoint{0.997756in}{0.551806in}}%
\pgfpathlineto{\pgfqpoint{0.997352in}{0.551806in}}%
\pgfpathlineto{\pgfqpoint{0.994931in}{0.546773in}}%
\pgfpathlineto{\pgfqpoint{0.994529in}{0.546773in}}%
\pgfpathlineto{\pgfqpoint{0.992921in}{0.542906in}}%
\pgfpathlineto{\pgfqpoint{0.992119in}{0.542168in}}%
\pgfpathlineto{\pgfqpoint{0.990117in}{0.540293in}}%
\pgfpathlineto{\pgfqpoint{0.988521in}{0.539456in}}%
\pgfpathlineto{\pgfqpoint{0.986928in}{0.538401in}}%
\pgfpathlineto{\pgfqpoint{0.982569in}{0.539055in}}%
\pgfpathlineto{\pgfqpoint{0.980597in}{0.540879in}}%
\pgfpathlineto{\pgfqpoint{0.980203in}{0.540879in}}%
\pgfpathlineto{\pgfqpoint{0.978239in}{0.544051in}}%
\pgfpathlineto{\pgfqpoint{0.977454in}{0.544051in}}%
\pgfpathlineto{\pgfqpoint{0.975498in}{0.548709in}}%
\pgfpathlineto{\pgfqpoint{0.971993in}{0.558768in}}%
\pgfpathlineto{\pgfqpoint{0.971604in}{0.558768in}}%
\pgfpathlineto{\pgfqpoint{0.970053in}{0.568019in}}%
\pgfpathlineto{\pgfqpoint{0.969279in}{0.568019in}}%
\pgfpathlineto{\pgfqpoint{0.967348in}{0.579424in}}%
\pgfpathlineto{\pgfqpoint{0.966963in}{0.579424in}}%
\pgfpathlineto{\pgfqpoint{0.962738in}{0.609144in}}%
\pgfpathlineto{\pgfqpoint{0.961973in}{0.609144in}}%
\pgfpathlineto{\pgfqpoint{0.960445in}{0.627852in}}%
\pgfpathlineto{\pgfqpoint{0.960064in}{0.627852in}}%
\pgfpathlineto{\pgfqpoint{0.956643in}{0.660387in}}%
\pgfpathlineto{\pgfqpoint{0.956264in}{0.660387in}}%
\pgfpathlineto{\pgfqpoint{0.954751in}{0.684838in}}%
\pgfpathlineto{\pgfqpoint{0.953995in}{0.684838in}}%
\pgfpathlineto{\pgfqpoint{0.952487in}{0.711366in}}%
\pgfpathlineto{\pgfqpoint{0.951734in}{0.711366in}}%
\pgfpathlineto{\pgfqpoint{0.950232in}{0.739901in}}%
\pgfpathlineto{\pgfqpoint{0.949482in}{0.739901in}}%
\pgfpathlineto{\pgfqpoint{0.947985in}{0.770048in}}%
\pgfpathlineto{\pgfqpoint{0.947237in}{0.770048in}}%
\pgfpathlineto{\pgfqpoint{0.945373in}{0.801954in}}%
\pgfpathlineto{\pgfqpoint{0.945001in}{0.801954in}}%
\pgfpathlineto{\pgfqpoint{0.940922in}{0.868286in}}%
\pgfpathlineto{\pgfqpoint{0.940183in}{0.868286in}}%
\pgfpathlineto{\pgfqpoint{0.938340in}{0.902991in}}%
\pgfpathlineto{\pgfqpoint{0.937972in}{0.902991in}}%
\pgfpathlineto{\pgfqpoint{0.936136in}{0.938180in}}%
\pgfpathlineto{\pgfqpoint{0.932844in}{0.991548in}}%
\pgfpathlineto{\pgfqpoint{0.932479in}{0.991548in}}%
\pgfpathlineto{\pgfqpoint{0.930659in}{1.027181in}}%
\pgfpathlineto{\pgfqpoint{0.930295in}{1.027181in}}%
\pgfpathlineto{\pgfqpoint{0.928482in}{1.063123in}}%
\pgfpathlineto{\pgfqpoint{0.928120in}{1.063123in}}%
\pgfpathlineto{\pgfqpoint{0.926313in}{1.098814in}}%
\pgfpathlineto{\pgfqpoint{0.925952in}{1.098814in}}%
\pgfpathlineto{\pgfqpoint{0.924151in}{1.134185in}}%
\pgfpathlineto{\pgfqpoint{0.923432in}{1.134185in}}%
\pgfpathlineto{\pgfqpoint{0.921639in}{1.169365in}}%
\pgfpathlineto{\pgfqpoint{0.921281in}{1.169365in}}%
\pgfpathlineto{\pgfqpoint{0.919494in}{1.204531in}}%
\pgfpathlineto{\pgfqpoint{0.919137in}{1.204531in}}%
\pgfpathlineto{\pgfqpoint{0.917356in}{1.238810in}}%
\pgfpathlineto{\pgfqpoint{0.917001in}{1.238810in}}%
\pgfpathlineto{\pgfqpoint{0.915226in}{1.272764in}}%
\pgfpathlineto{\pgfqpoint{0.914872in}{1.272764in}}%
\pgfpathlineto{\pgfqpoint{0.913104in}{1.306289in}}%
\pgfpathlineto{\pgfqpoint{0.912751in}{1.306289in}}%
\pgfpathlineto{\pgfqpoint{0.910988in}{1.339192in}}%
\pgfpathlineto{\pgfqpoint{0.910637in}{1.339192in}}%
\pgfpathlineto{\pgfqpoint{0.908881in}{1.371655in}}%
\pgfpathlineto{\pgfqpoint{0.908180in}{1.371655in}}%
\pgfpathlineto{\pgfqpoint{0.906431in}{1.403807in}}%
\pgfpathlineto{\pgfqpoint{0.903296in}{1.450231in}}%
\pgfpathlineto{\pgfqpoint{0.902949in}{1.450231in}}%
\pgfpathlineto{\pgfqpoint{0.901215in}{1.480495in}}%
\pgfpathlineto{\pgfqpoint{0.900869in}{1.480495in}}%
\pgfpathlineto{\pgfqpoint{0.899142in}{1.510125in}}%
\pgfpathlineto{\pgfqpoint{0.896045in}{1.553647in}}%
\pgfpathlineto{\pgfqpoint{0.895702in}{1.553647in}}%
\pgfpathlineto{\pgfqpoint{0.893989in}{1.581695in}}%
\pgfpathlineto{\pgfqpoint{0.893305in}{1.581695in}}%
\pgfpathlineto{\pgfqpoint{0.891600in}{1.608912in}}%
\pgfpathlineto{\pgfqpoint{0.887864in}{1.661588in}}%
\pgfpathlineto{\pgfqpoint{0.887526in}{1.661588in}}%
\pgfpathlineto{\pgfqpoint{0.884489in}{1.698934in}}%
\pgfpathlineto{\pgfqpoint{0.884152in}{1.698934in}}%
\pgfpathlineto{\pgfqpoint{0.880464in}{1.746704in}}%
\pgfpathlineto{\pgfqpoint{0.880130in}{1.746704in}}%
\pgfpathlineto{\pgfqpoint{0.878461in}{1.769265in}}%
\pgfpathlineto{\pgfqpoint{0.878128in}{1.769265in}}%
\pgfpathlineto{\pgfqpoint{0.874477in}{1.812449in}}%
\pgfpathlineto{\pgfqpoint{0.873816in}{1.812449in}}%
\pgfpathlineto{\pgfqpoint{0.872166in}{1.833283in}}%
\pgfpathlineto{\pgfqpoint{0.869207in}{1.863035in}}%
\pgfpathlineto{\pgfqpoint{0.868879in}{1.863035in}}%
\pgfpathlineto{\pgfqpoint{0.867243in}{1.882307in}}%
\pgfpathlineto{\pgfqpoint{0.866916in}{1.882307in}}%
\pgfpathlineto{\pgfqpoint{0.865286in}{1.900831in}}%
\pgfpathlineto{\pgfqpoint{0.864960in}{1.900831in}}%
\pgfpathlineto{\pgfqpoint{0.861390in}{1.935927in}}%
\pgfpathlineto{\pgfqpoint{0.861067in}{1.935927in}}%
\pgfpathlineto{\pgfqpoint{0.859453in}{1.952554in}}%
\pgfpathlineto{\pgfqpoint{0.859130in}{1.952554in}}%
\pgfpathlineto{\pgfqpoint{0.856238in}{1.976400in}}%
\pgfpathlineto{\pgfqpoint{0.855917in}{1.976400in}}%
\pgfpathlineto{\pgfqpoint{0.854317in}{1.991505in}}%
\pgfpathlineto{\pgfqpoint{0.853998in}{1.991505in}}%
\pgfpathlineto{\pgfqpoint{0.851448in}{2.028809in}}%
\pgfpathlineto{\pgfqpoint{0.851130in}{2.028809in}}%
\pgfpathlineto{\pgfqpoint{0.849861in}{2.034824in}}%
\pgfpathlineto{\pgfqpoint{0.848277in}{2.034154in}}%
\pgfpathlineto{\pgfqpoint{0.846699in}{2.047172in}}%
\pgfpathlineto{\pgfqpoint{0.846383in}{2.047172in}}%
\pgfpathlineto{\pgfqpoint{0.844810in}{2.093535in}}%
\pgfpathlineto{\pgfqpoint{0.844496in}{2.093535in}}%
\pgfpathlineto{\pgfqpoint{0.842927in}{2.105383in}}%
\pgfpathlineto{\pgfqpoint{0.842614in}{2.105383in}}%
\pgfpathlineto{\pgfqpoint{0.841363in}{2.114600in}}%
\pgfpathlineto{\pgfqpoint{0.838870in}{2.127540in}}%
\pgfpathlineto{\pgfqpoint{0.838248in}{2.125783in}}%
\pgfpathlineto{\pgfqpoint{0.836697in}{2.126458in}}%
\pgfpathlineto{\pgfqpoint{0.834841in}{2.138418in}}%
\pgfpathlineto{\pgfqpoint{0.832991in}{2.144568in}}%
\pgfpathlineto{\pgfqpoint{0.831454in}{2.130615in}}%
\pgfpathlineto{\pgfqpoint{0.831148in}{2.130615in}}%
\pgfpathlineto{\pgfqpoint{0.830228in}{2.153173in}}%
\pgfpathlineto{\pgfqpoint{0.829616in}{2.147508in}}%
\pgfpathlineto{\pgfqpoint{0.829310in}{2.147508in}}%
\pgfpathlineto{\pgfqpoint{0.828088in}{2.161454in}}%
\pgfpathlineto{\pgfqpoint{0.827174in}{2.163856in}}%
\pgfpathlineto{\pgfqpoint{0.826260in}{2.151368in}}%
\pgfpathlineto{\pgfqpoint{0.824742in}{2.175957in}}%
\pgfpathlineto{\pgfqpoint{0.822321in}{2.185693in}}%
\pgfpathlineto{\pgfqpoint{0.821717in}{2.185693in}}%
\pgfpathlineto{\pgfqpoint{0.820211in}{2.193496in}}%
\pgfpathlineto{\pgfqpoint{0.819910in}{2.193496in}}%
\pgfpathlineto{\pgfqpoint{0.818409in}{2.199214in}}%
\pgfpathlineto{\pgfqpoint{0.818109in}{2.199214in}}%
\pgfpathlineto{\pgfqpoint{0.816314in}{2.204211in}}%
\pgfpathlineto{\pgfqpoint{0.814524in}{2.207164in}}%
\pgfpathlineto{\pgfqpoint{0.813632in}{2.197424in}}%
\pgfpathlineto{\pgfqpoint{0.813037in}{2.200176in}}%
\pgfpathlineto{\pgfqpoint{0.812740in}{2.200176in}}%
\pgfpathlineto{\pgfqpoint{0.808601in}{2.217834in}}%
\pgfpathlineto{\pgfqpoint{0.808012in}{2.217834in}}%
\pgfpathlineto{\pgfqpoint{0.807130in}{2.215259in}}%
\pgfpathlineto{\pgfqpoint{0.805662in}{2.220802in}}%
\pgfpathlineto{\pgfqpoint{0.805369in}{2.220802in}}%
\pgfpathlineto{\pgfqpoint{0.803907in}{2.223529in}}%
\pgfpathlineto{\pgfqpoint{0.803615in}{2.223529in}}%
\pgfpathlineto{\pgfqpoint{0.802157in}{2.243635in}}%
\pgfpathlineto{\pgfqpoint{0.801575in}{2.243635in}}%
\pgfpathlineto{\pgfqpoint{0.799833in}{2.246842in}}%
\pgfpathlineto{\pgfqpoint{0.797807in}{2.250315in}}%
\pgfpathlineto{\pgfqpoint{0.796365in}{2.249856in}}%
\pgfpathlineto{\pgfqpoint{0.794926in}{2.252456in}}%
\pgfpathlineto{\pgfqpoint{0.792919in}{2.252329in}}%
\pgfpathlineto{\pgfqpoint{0.792061in}{2.254667in}}%
\pgfpathlineto{\pgfqpoint{0.791204in}{2.246197in}}%
\pgfpathlineto{\pgfqpoint{0.790918in}{2.247664in}}%
\pgfpathlineto{\pgfqpoint{0.790063in}{2.247664in}}%
\pgfpathlineto{\pgfqpoint{0.789494in}{2.251211in}}%
\pgfpathlineto{\pgfqpoint{0.788926in}{2.250031in}}%
\pgfpathlineto{\pgfqpoint{0.788641in}{2.250031in}}%
\pgfpathlineto{\pgfqpoint{0.787790in}{2.254647in}}%
\pgfpathlineto{\pgfqpoint{0.787507in}{2.253452in}}%
\pgfpathlineto{\pgfqpoint{0.786940in}{2.252258in}}%
\pgfpathlineto{\pgfqpoint{0.786657in}{2.253908in}}%
\pgfpathlineto{\pgfqpoint{0.784962in}{2.254934in}}%
\pgfpathlineto{\pgfqpoint{0.783553in}{2.256300in}}%
\pgfpathlineto{\pgfqpoint{0.780187in}{2.259601in}}%
\pgfpathlineto{\pgfqpoint{0.779070in}{2.260215in}}%
\pgfpathlineto{\pgfqpoint{0.777121in}{2.262074in}}%
\pgfpathlineto{\pgfqpoint{0.774901in}{2.263123in}}%
\pgfpathlineto{\pgfqpoint{0.772691in}{2.264446in}}%
\pgfpathlineto{\pgfqpoint{0.770764in}{2.265225in}}%
\pgfpathlineto{\pgfqpoint{0.769393in}{2.266015in}}%
\pgfpathlineto{\pgfqpoint{0.764755in}{2.265086in}}%
\pgfpathlineto{\pgfqpoint{0.762046in}{2.263831in}}%
\pgfpathlineto{\pgfqpoint{0.759888in}{2.263024in}}%
\pgfpathlineto{\pgfqpoint{0.745815in}{2.244869in}}%
\pgfpathlineto{\pgfqpoint{0.743202in}{2.240923in}}%
\pgfpathlineto{\pgfqpoint{0.741900in}{2.241697in}}%
\pgfpathlineto{\pgfqpoint{0.740602in}{2.238648in}}%
\pgfpathlineto{\pgfqpoint{0.740343in}{2.238648in}}%
\pgfpathlineto{\pgfqpoint{0.739048in}{2.236593in}}%
\pgfpathlineto{\pgfqpoint{0.738790in}{2.236593in}}%
\pgfpathlineto{\pgfqpoint{0.737241in}{2.240011in}}%
\pgfpathlineto{\pgfqpoint{0.737241in}{2.240011in}}%
\pgfusepath{stroke}%
\end{pgfscope}%
\begin{pgfscope}%
\pgfsetrectcap%
\pgfsetmiterjoin%
\pgfsetlinewidth{0.803000pt}%
\definecolor{currentstroke}{rgb}{0.000000,0.000000,0.000000}%
\pgfsetstrokecolor{currentstroke}%
\pgfsetdash{}{0pt}%
\pgfpathmoveto{\pgfqpoint{0.530948in}{0.451986in}}%
\pgfpathlineto{\pgfqpoint{0.530948in}{2.352445in}}%
\pgfusepath{stroke}%
\end{pgfscope}%
\begin{pgfscope}%
\pgfsetrectcap%
\pgfsetmiterjoin%
\pgfsetlinewidth{0.000000pt}%
\definecolor{currentstroke}{rgb}{0.000000,0.000000,0.000000}%
\pgfsetstrokecolor{currentstroke}%
\pgfsetstrokeopacity{0.000000}%
\pgfsetdash{}{0pt}%
\pgfpathmoveto{\pgfqpoint{5.887886in}{0.451986in}}%
\pgfpathlineto{\pgfqpoint{5.887886in}{2.352445in}}%
\pgfusepath{}%
\end{pgfscope}%
\begin{pgfscope}%
\pgfsetrectcap%
\pgfsetmiterjoin%
\pgfsetlinewidth{0.803000pt}%
\definecolor{currentstroke}{rgb}{0.000000,0.000000,0.000000}%
\pgfsetstrokecolor{currentstroke}%
\pgfsetdash{}{0pt}%
\pgfpathmoveto{\pgfqpoint{0.530948in}{0.451986in}}%
\pgfpathlineto{\pgfqpoint{5.887886in}{0.451986in}}%
\pgfusepath{stroke}%
\end{pgfscope}%
\begin{pgfscope}%
\pgfsetrectcap%
\pgfsetmiterjoin%
\pgfsetlinewidth{0.000000pt}%
\definecolor{currentstroke}{rgb}{0.000000,0.000000,0.000000}%
\pgfsetstrokecolor{currentstroke}%
\pgfsetstrokeopacity{0.000000}%
\pgfsetdash{}{0pt}%
\pgfpathmoveto{\pgfqpoint{0.530948in}{2.352445in}}%
\pgfpathlineto{\pgfqpoint{5.887886in}{2.352445in}}%
\pgfusepath{}%
\end{pgfscope}%
\begin{pgfscope}%
\pgfsetroundcap%
\pgfsetroundjoin%
\pgfsetlinewidth{0.501875pt}%
\definecolor{currentstroke}{rgb}{0.000000,0.000000,0.000000}%
\pgfsetstrokecolor{currentstroke}%
\pgfsetdash{{0.500000pt}{0.825000pt}}{0.000000pt}%
\pgfpathmoveto{\pgfqpoint{3.561548in}{1.710555in}}%
\pgfpathquadraticcurveto{\pgfqpoint{3.561548in}{1.240191in}}{\pgfqpoint{3.561548in}{0.769827in}}%
\pgfusepath{stroke}%
\end{pgfscope}%
\begin{pgfscope}%
\pgfsetroundcap%
\pgfsetroundjoin%
\pgfsetlinewidth{0.501875pt}%
\definecolor{currentstroke}{rgb}{0.000000,0.000000,0.000000}%
\pgfsetstrokecolor{currentstroke}%
\pgfsetdash{{0.500000pt}{0.825000pt}}{0.000000pt}%
\pgfpathmoveto{\pgfqpoint{3.129330in}{1.710555in}}%
\pgfpathquadraticcurveto{\pgfqpoint{3.129330in}{1.240191in}}{\pgfqpoint{3.129330in}{0.769827in}}%
\pgfusepath{stroke}%
\end{pgfscope}%
\begin{pgfscope}%
\definecolor{textcolor}{rgb}{0.000000,0.000000,0.000000}%
\pgfsetstrokecolor{textcolor}%
\pgfsetfillcolor{textcolor}%
\pgftext[x=3.170966in,y=1.656408in,left,base]{\color{textcolor}\rmfamily\fontsize{6.000000}{7.200000}\selectfont \(\displaystyle \Delta(1/B)\)}%
\end{pgfscope}%
\begin{pgfscope}%
\definecolor{textcolor}{rgb}{0.000000,0.000000,0.000000}%
\pgfsetstrokecolor{textcolor}%
\pgfsetfillcolor{textcolor}%
\pgftext[x=3.227771in,y=1.566147in,left,base]{\color{textcolor}\rmfamily\fontsize{6.000000}{7.200000}\selectfont \(\displaystyle 0.105\)}%
\end{pgfscope}%
\begin{pgfscope}%
\pgfsetroundcap%
\pgfsetroundjoin%
\pgfsetlinewidth{0.501875pt}%
\definecolor{currentstroke}{rgb}{0.000000,0.000000,0.000000}%
\pgfsetstrokecolor{currentstroke}%
\pgfsetdash{{0.500000pt}{0.825000pt}}{0.000000pt}%
\pgfpathmoveto{\pgfqpoint{3.129330in}{1.710564in}}%
\pgfpathquadraticcurveto{\pgfqpoint{3.129330in}{1.232617in}}{\pgfqpoint{3.129330in}{0.754670in}}%
\pgfusepath{stroke}%
\end{pgfscope}%
\begin{pgfscope}%
\pgfsetroundcap%
\pgfsetroundjoin%
\pgfsetlinewidth{0.501875pt}%
\definecolor{currentstroke}{rgb}{0.000000,0.000000,0.000000}%
\pgfsetstrokecolor{currentstroke}%
\pgfsetdash{{0.500000pt}{0.825000pt}}{0.000000pt}%
\pgfpathmoveto{\pgfqpoint{2.694999in}{1.710564in}}%
\pgfpathquadraticcurveto{\pgfqpoint{2.694999in}{1.232617in}}{\pgfqpoint{2.694999in}{0.754670in}}%
\pgfusepath{stroke}%
\end{pgfscope}%
\begin{pgfscope}%
\definecolor{textcolor}{rgb}{0.000000,0.000000,0.000000}%
\pgfsetstrokecolor{textcolor}%
\pgfsetfillcolor{textcolor}%
\pgftext[x=2.737692in,y=1.656408in,left,base]{\color{textcolor}\rmfamily\fontsize{6.000000}{7.200000}\selectfont \(\displaystyle \Delta(1/B)\)}%
\end{pgfscope}%
\begin{pgfscope}%
\definecolor{textcolor}{rgb}{0.000000,0.000000,0.000000}%
\pgfsetstrokecolor{textcolor}%
\pgfsetfillcolor{textcolor}%
\pgftext[x=2.794496in,y=1.566147in,left,base]{\color{textcolor}\rmfamily\fontsize{6.000000}{7.200000}\selectfont \(\displaystyle 0.105\)}%
\end{pgfscope}%
\begin{pgfscope}%
\pgfsetroundcap%
\pgfsetroundjoin%
\pgfsetlinewidth{0.501875pt}%
\definecolor{currentstroke}{rgb}{0.000000,0.000000,0.000000}%
\pgfsetstrokecolor{currentstroke}%
\pgfsetdash{{0.500000pt}{0.825000pt}}{0.000000pt}%
\pgfpathmoveto{\pgfqpoint{2.694999in}{1.710539in}}%
\pgfpathquadraticcurveto{\pgfqpoint{2.694999in}{1.221803in}}{\pgfqpoint{2.694999in}{0.733067in}}%
\pgfusepath{stroke}%
\end{pgfscope}%
\begin{pgfscope}%
\pgfsetroundcap%
\pgfsetroundjoin%
\pgfsetlinewidth{0.501875pt}%
\definecolor{currentstroke}{rgb}{0.000000,0.000000,0.000000}%
\pgfsetstrokecolor{currentstroke}%
\pgfsetdash{{0.500000pt}{0.825000pt}}{0.000000pt}%
\pgfpathmoveto{\pgfqpoint{2.269441in}{1.710539in}}%
\pgfpathquadraticcurveto{\pgfqpoint{2.269441in}{1.221803in}}{\pgfqpoint{2.269441in}{0.733067in}}%
\pgfusepath{stroke}%
\end{pgfscope}%
\begin{pgfscope}%
\definecolor{textcolor}{rgb}{0.000000,0.000000,0.000000}%
\pgfsetstrokecolor{textcolor}%
\pgfsetfillcolor{textcolor}%
\pgftext[x=2.307747in,y=1.656408in,left,base]{\color{textcolor}\rmfamily\fontsize{6.000000}{7.200000}\selectfont \(\displaystyle \Delta(1/B)\)}%
\end{pgfscope}%
\begin{pgfscope}%
\definecolor{textcolor}{rgb}{0.000000,0.000000,0.000000}%
\pgfsetstrokecolor{textcolor}%
\pgfsetfillcolor{textcolor}%
\pgftext[x=2.364552in,y=1.566147in,left,base]{\color{textcolor}\rmfamily\fontsize{6.000000}{7.200000}\selectfont \(\displaystyle 0.103\)}%
\end{pgfscope}%
\begin{pgfscope}%
\pgfsetroundcap%
\pgfsetroundjoin%
\pgfsetlinewidth{0.501875pt}%
\definecolor{currentstroke}{rgb}{0.000000,0.000000,0.000000}%
\pgfsetstrokecolor{currentstroke}%
\pgfsetdash{{0.500000pt}{0.825000pt}}{0.000000pt}%
\pgfpathmoveto{\pgfqpoint{2.269441in}{1.710504in}}%
\pgfpathquadraticcurveto{\pgfqpoint{2.269441in}{1.206668in}}{\pgfqpoint{2.269441in}{0.702832in}}%
\pgfusepath{stroke}%
\end{pgfscope}%
\begin{pgfscope}%
\pgfsetroundcap%
\pgfsetroundjoin%
\pgfsetlinewidth{0.501875pt}%
\definecolor{currentstroke}{rgb}{0.000000,0.000000,0.000000}%
\pgfsetstrokecolor{currentstroke}%
\pgfsetdash{{0.500000pt}{0.825000pt}}{0.000000pt}%
\pgfpathmoveto{\pgfqpoint{1.841892in}{1.710504in}}%
\pgfpathquadraticcurveto{\pgfqpoint{1.841892in}{1.206668in}}{\pgfqpoint{1.841892in}{0.702832in}}%
\pgfusepath{stroke}%
\end{pgfscope}%
\begin{pgfscope}%
\definecolor{textcolor}{rgb}{0.000000,0.000000,0.000000}%
\pgfsetstrokecolor{textcolor}%
\pgfsetfillcolor{textcolor}%
\pgftext[x=1.881194in,y=1.656408in,left,base]{\color{textcolor}\rmfamily\fontsize{6.000000}{7.200000}\selectfont \(\displaystyle \Delta(1/B)\)}%
\end{pgfscope}%
\begin{pgfscope}%
\definecolor{textcolor}{rgb}{0.000000,0.000000,0.000000}%
\pgfsetstrokecolor{textcolor}%
\pgfsetfillcolor{textcolor}%
\pgftext[x=1.937999in,y=1.566147in,left,base]{\color{textcolor}\rmfamily\fontsize{6.000000}{7.200000}\selectfont \(\displaystyle 0.104\)}%
\end{pgfscope}%
\begin{pgfscope}%
\pgfsetroundcap%
\pgfsetroundjoin%
\pgfsetlinewidth{0.501875pt}%
\definecolor{currentstroke}{rgb}{0.000000,0.000000,0.000000}%
\pgfsetstrokecolor{currentstroke}%
\pgfsetdash{{0.500000pt}{0.825000pt}}{0.000000pt}%
\pgfpathmoveto{\pgfqpoint{1.841892in}{1.710515in}}%
\pgfpathquadraticcurveto{\pgfqpoint{1.841892in}{1.186249in}}{\pgfqpoint{1.841892in}{0.661983in}}%
\pgfusepath{stroke}%
\end{pgfscope}%
\begin{pgfscope}%
\pgfsetroundcap%
\pgfsetroundjoin%
\pgfsetlinewidth{0.501875pt}%
\definecolor{currentstroke}{rgb}{0.000000,0.000000,0.000000}%
\pgfsetstrokecolor{currentstroke}%
\pgfsetdash{{0.500000pt}{0.825000pt}}{0.000000pt}%
\pgfpathmoveto{\pgfqpoint{1.413813in}{1.710515in}}%
\pgfpathquadraticcurveto{\pgfqpoint{1.413813in}{1.186249in}}{\pgfqpoint{1.413813in}{0.661983in}}%
\pgfusepath{stroke}%
\end{pgfscope}%
\begin{pgfscope}%
\definecolor{textcolor}{rgb}{0.000000,0.000000,0.000000}%
\pgfsetstrokecolor{textcolor}%
\pgfsetfillcolor{textcolor}%
\pgftext[x=1.453380in,y=1.656408in,left,base]{\color{textcolor}\rmfamily\fontsize{6.000000}{7.200000}\selectfont \(\displaystyle \Delta(1/B)\)}%
\end{pgfscope}%
\begin{pgfscope}%
\definecolor{textcolor}{rgb}{0.000000,0.000000,0.000000}%
\pgfsetstrokecolor{textcolor}%
\pgfsetfillcolor{textcolor}%
\pgftext[x=1.510185in,y=1.566147in,left,base]{\color{textcolor}\rmfamily\fontsize{6.000000}{7.200000}\selectfont \(\displaystyle 0.104\)}%
\end{pgfscope}%
\begin{pgfscope}%
\pgfsetroundcap%
\pgfsetroundjoin%
\pgfsetlinewidth{0.501875pt}%
\definecolor{currentstroke}{rgb}{0.000000,0.000000,0.000000}%
\pgfsetstrokecolor{currentstroke}%
\pgfsetdash{{0.500000pt}{0.825000pt}}{0.000000pt}%
\pgfpathmoveto{\pgfqpoint{1.413813in}{1.710516in}}%
\pgfpathquadraticcurveto{\pgfqpoint{1.413813in}{1.161090in}}{\pgfqpoint{1.413813in}{0.611664in}}%
\pgfusepath{stroke}%
\end{pgfscope}%
\begin{pgfscope}%
\pgfsetroundcap%
\pgfsetroundjoin%
\pgfsetlinewidth{0.501875pt}%
\definecolor{currentstroke}{rgb}{0.000000,0.000000,0.000000}%
\pgfsetstrokecolor{currentstroke}%
\pgfsetdash{{0.500000pt}{0.825000pt}}{0.000000pt}%
\pgfpathmoveto{\pgfqpoint{0.985736in}{1.710516in}}%
\pgfpathquadraticcurveto{\pgfqpoint{0.985736in}{1.161090in}}{\pgfqpoint{0.985736in}{0.611664in}}%
\pgfusepath{stroke}%
\end{pgfscope}%
\begin{pgfscope}%
\definecolor{textcolor}{rgb}{0.000000,0.000000,0.000000}%
\pgfsetstrokecolor{textcolor}%
\pgfsetfillcolor{textcolor}%
\pgftext[x=1.025302in,y=1.656408in,left,base]{\color{textcolor}\rmfamily\fontsize{6.000000}{7.200000}\selectfont \(\displaystyle \Delta(1/B)\)}%
\end{pgfscope}%
\begin{pgfscope}%
\definecolor{textcolor}{rgb}{0.000000,0.000000,0.000000}%
\pgfsetstrokecolor{textcolor}%
\pgfsetfillcolor{textcolor}%
\pgftext[x=1.082107in,y=1.566147in,left,base]{\color{textcolor}\rmfamily\fontsize{6.000000}{7.200000}\selectfont \(\displaystyle 0.104\)}%
\end{pgfscope}%
\begin{pgfscope}%
\pgfsetbuttcap%
\pgfsetmiterjoin%
\definecolor{currentfill}{rgb}{1.000000,1.000000,1.000000}%
\pgfsetfillcolor{currentfill}%
\pgfsetfillopacity{0.800000}%
\pgfsetlinewidth{1.003750pt}%
\definecolor{currentstroke}{rgb}{0.800000,0.800000,0.800000}%
\pgfsetstrokecolor{currentstroke}%
\pgfsetstrokeopacity{0.800000}%
\pgfsetdash{}{0pt}%
\pgfpathmoveto{\pgfqpoint{4.334514in}{2.108623in}}%
\pgfpathlineto{\pgfqpoint{5.810108in}{2.108623in}}%
\pgfpathquadraticcurveto{\pgfqpoint{5.832330in}{2.108623in}}{\pgfqpoint{5.832330in}{2.130845in}}%
\pgfpathlineto{\pgfqpoint{5.832330in}{2.274667in}}%
\pgfpathquadraticcurveto{\pgfqpoint{5.832330in}{2.296890in}}{\pgfqpoint{5.810108in}{2.296890in}}%
\pgfpathlineto{\pgfqpoint{4.334514in}{2.296890in}}%
\pgfpathquadraticcurveto{\pgfqpoint{4.312292in}{2.296890in}}{\pgfqpoint{4.312292in}{2.274667in}}%
\pgfpathlineto{\pgfqpoint{4.312292in}{2.130845in}}%
\pgfpathquadraticcurveto{\pgfqpoint{4.312292in}{2.108623in}}{\pgfqpoint{4.334514in}{2.108623in}}%
\pgfpathclose%
\pgfusepath{stroke,fill}%
\end{pgfscope}%
\begin{pgfscope}%
\pgfsetrectcap%
\pgfsetroundjoin%
\pgfsetlinewidth{1.003750pt}%
\definecolor{currentstroke}{rgb}{0.760784,0.211765,0.086275}%
\pgfsetstrokecolor{currentstroke}%
\pgfsetdash{}{0pt}%
\pgfpathmoveto{\pgfqpoint{4.356736in}{2.213556in}}%
\pgfpathlineto{\pgfqpoint{4.578958in}{2.213556in}}%
\pgfusepath{stroke}%
\end{pgfscope}%
\begin{pgfscope}%
\definecolor{textcolor}{rgb}{0.000000,0.000000,0.000000}%
\pgfsetstrokecolor{textcolor}%
\pgfsetfillcolor{textcolor}%
\pgftext[x=4.667847in,y=2.174667in,left,base]{\color{textcolor}\rmfamily\fontsize{8.000000}{9.600000}\selectfont Längswiderstand \(\displaystyle R_{xx}\)}%
\end{pgfscope}%
\end{pgfpicture}%
\makeatother%
\endgroup%

	\caption{}
	\label{abb:P07_Meth_2}
\end{figure}

\begin{table}[h!]
	\centering
	\caption{}
	\begin{tabular}{ccrr}
\toprule
 $1/B_{min, l}[1/\si{T}] $&  $1/B_{min, r}[1/\si{T}] $ & $ \Delta_{1/B}[1/\si{T}]$ &  E-dichte $[1/\si{m}^2]$\\
\midrule
       0.831 &         0.935 &         0.105 &  4.611e+15 \\
       0.725 &         0.831 &         0.105 &  4.588e+15 \\
       0.622 &         0.725 &         0.103 &  4.683e+15 \\
       0.518 &         0.622 &         0.104 &  4.661e+15 \\
       0.414 &         0.518 &         0.104 &  4.655e+15 \\
       0.310 &         0.414 &         0.104 &  4.655e+15 \\
\bottomrule
\end{tabular}

	\label{tab:P07_Meth_2}
\end{table}

\begin{table}[h!]
	\centering
	\caption{}
	\begin{tabular}{lrrrr}
\toprule
        Messreihe &  $R_{xx} (B=0) [\Omega]$ &  E-Dichte$(\varnothing) [1/\si{m}^2]$  & Std-Ab. &  E-Mobilität \\
\midrule
 P07\_0611\_0708\_0109 &  106.743 &              4.642e+15 &          3.536e+13 &             15.75 \\
 P07\_0611\_0709\_0109 &  182.622 &              4.637e+15 &           8.97e+13 &             18.43 \\
 P07\_1115\_1213\_2012 &   77.843 &              4.630e+15 &          1.846e+13 &             21.65 \\
 P07\_1115\_2018\_1913 &  160.499 &              4.620e+15 &           7.29e+13 &             21.04 \\
\bottomrule
\end{tabular}

	\label{tab:Meth_2}
\end{table}
\newpage
\subsubsection{Methode 3:}
Für die letzte Herangehensweise betrachten wir das klassische Regime des Hall-Effekts - also niedrige Magnetfelder. Als Grenze wählen wir dafür das niedrigste Hallplateau, dass wir identifizieren konnten.

Im klassischen Bereich lässt sich mit Gl. ? aus der Steigung des Hallwiderstands die Elektronendichte bestimmen. Diese Steigung erhalten wir mittels eines linearen Fits (s. Abb. \ref{abb:P07_Meth_3}). Hier ist bereits gut zu erkennen, dass eine Extrapolation dieses Fits gerade die Hallplateaus schneidet. Die Elektronenkonzentration und -mobilität sind Tab. \ref{tab:Meth_3} zu entnehmen. Die Werte stimmen wiederum in guter Näherung mit den zuvor ermittelten überein. Auch das Bild der stark schwankenden Mobilität zeigt sich.
\begin{figure}[htbp]
	\centering
	%% Creator: Matplotlib, PGF backend
%%
%% To include the figure in your LaTeX document, write
%%   \input{<filename>.pgf}
%%
%% Make sure the required packages are loaded in your preamble
%%   \usepackage{pgf}
%%
%% Figures using additional raster images can only be included by \input if
%% they are in the same directory as the main LaTeX file. For loading figures
%% from other directories you can use the `import` package
%%   \usepackage{import}
%% and then include the figures with
%%   \import{<path to file>}{<filename>.pgf}
%%
%% Matplotlib used the following preamble
%%   \usepackage[utf8x]{inputenc}
%%   \usepackage[T1]{fontenc}
%%
\begingroup%
\makeatletter%
\begin{pgfpicture}%
\pgfpathrectangle{\pgfpointorigin}{\pgfqpoint{4.803151in}{1.979007in}}%
\pgfusepath{use as bounding box, clip}%
\begin{pgfscope}%
\pgfsetbuttcap%
\pgfsetmiterjoin%
\definecolor{currentfill}{rgb}{1.000000,1.000000,1.000000}%
\pgfsetfillcolor{currentfill}%
\pgfsetlinewidth{0.000000pt}%
\definecolor{currentstroke}{rgb}{1.000000,1.000000,1.000000}%
\pgfsetstrokecolor{currentstroke}%
\pgfsetdash{}{0pt}%
\pgfpathmoveto{\pgfqpoint{0.000000in}{0.000000in}}%
\pgfpathlineto{\pgfqpoint{4.803151in}{0.000000in}}%
\pgfpathlineto{\pgfqpoint{4.803151in}{1.979007in}}%
\pgfpathlineto{\pgfqpoint{0.000000in}{1.979007in}}%
\pgfpathclose%
\pgfusepath{fill}%
\end{pgfscope}%
\begin{pgfscope}%
\pgfsetbuttcap%
\pgfsetmiterjoin%
\definecolor{currentfill}{rgb}{1.000000,1.000000,1.000000}%
\pgfsetfillcolor{currentfill}%
\pgfsetlinewidth{0.000000pt}%
\definecolor{currentstroke}{rgb}{0.000000,0.000000,0.000000}%
\pgfsetstrokecolor{currentstroke}%
\pgfsetstrokeopacity{0.000000}%
\pgfsetdash{}{0pt}%
\pgfpathmoveto{\pgfqpoint{0.592318in}{0.451986in}}%
\pgfpathlineto{\pgfqpoint{4.678151in}{0.451986in}}%
\pgfpathlineto{\pgfqpoint{4.678151in}{1.854007in}}%
\pgfpathlineto{\pgfqpoint{0.592318in}{1.854007in}}%
\pgfpathclose%
\pgfusepath{fill}%
\end{pgfscope}%
\begin{pgfscope}%
\pgfpathrectangle{\pgfqpoint{0.592318in}{0.451986in}}{\pgfqpoint{4.085832in}{1.402021in}}%
\pgfusepath{clip}%
\pgfsetbuttcap%
\pgfsetroundjoin%
\pgfsetlinewidth{0.501875pt}%
\definecolor{currentstroke}{rgb}{0.690196,0.690196,0.690196}%
\pgfsetstrokecolor{currentstroke}%
\pgfsetdash{{1.850000pt}{0.800000pt}}{0.000000pt}%
\pgfpathmoveto{\pgfqpoint{0.778038in}{0.451986in}}%
\pgfpathlineto{\pgfqpoint{0.778038in}{1.854007in}}%
\pgfusepath{stroke}%
\end{pgfscope}%
\begin{pgfscope}%
\pgfsetbuttcap%
\pgfsetroundjoin%
\definecolor{currentfill}{rgb}{0.000000,0.000000,0.000000}%
\pgfsetfillcolor{currentfill}%
\pgfsetlinewidth{0.803000pt}%
\definecolor{currentstroke}{rgb}{0.000000,0.000000,0.000000}%
\pgfsetstrokecolor{currentstroke}%
\pgfsetdash{}{0pt}%
\pgfsys@defobject{currentmarker}{\pgfqpoint{0.000000in}{-0.048611in}}{\pgfqpoint{0.000000in}{0.000000in}}{%
\pgfpathmoveto{\pgfqpoint{0.000000in}{0.000000in}}%
\pgfpathlineto{\pgfqpoint{0.000000in}{-0.048611in}}%
\pgfusepath{stroke,fill}%
}%
\begin{pgfscope}%
\pgfsys@transformshift{0.778038in}{0.451986in}%
\pgfsys@useobject{currentmarker}{}%
\end{pgfscope}%
\end{pgfscope}%
\begin{pgfscope}%
\definecolor{textcolor}{rgb}{0.000000,0.000000,0.000000}%
\pgfsetstrokecolor{textcolor}%
\pgfsetfillcolor{textcolor}%
\pgftext[x=0.778038in,y=0.354764in,,top]{\color{textcolor}\rmfamily\fontsize{8.000000}{9.600000}\selectfont \(\displaystyle 0.0\)}%
\end{pgfscope}%
\begin{pgfscope}%
\pgfpathrectangle{\pgfqpoint{0.592318in}{0.451986in}}{\pgfqpoint{4.085832in}{1.402021in}}%
\pgfusepath{clip}%
\pgfsetbuttcap%
\pgfsetroundjoin%
\pgfsetlinewidth{0.501875pt}%
\definecolor{currentstroke}{rgb}{0.690196,0.690196,0.690196}%
\pgfsetstrokecolor{currentstroke}%
\pgfsetdash{{1.850000pt}{0.800000pt}}{0.000000pt}%
\pgfpathmoveto{\pgfqpoint{1.242337in}{0.451986in}}%
\pgfpathlineto{\pgfqpoint{1.242337in}{1.854007in}}%
\pgfusepath{stroke}%
\end{pgfscope}%
\begin{pgfscope}%
\pgfsetbuttcap%
\pgfsetroundjoin%
\definecolor{currentfill}{rgb}{0.000000,0.000000,0.000000}%
\pgfsetfillcolor{currentfill}%
\pgfsetlinewidth{0.803000pt}%
\definecolor{currentstroke}{rgb}{0.000000,0.000000,0.000000}%
\pgfsetstrokecolor{currentstroke}%
\pgfsetdash{}{0pt}%
\pgfsys@defobject{currentmarker}{\pgfqpoint{0.000000in}{-0.048611in}}{\pgfqpoint{0.000000in}{0.000000in}}{%
\pgfpathmoveto{\pgfqpoint{0.000000in}{0.000000in}}%
\pgfpathlineto{\pgfqpoint{0.000000in}{-0.048611in}}%
\pgfusepath{stroke,fill}%
}%
\begin{pgfscope}%
\pgfsys@transformshift{1.242337in}{0.451986in}%
\pgfsys@useobject{currentmarker}{}%
\end{pgfscope}%
\end{pgfscope}%
\begin{pgfscope}%
\definecolor{textcolor}{rgb}{0.000000,0.000000,0.000000}%
\pgfsetstrokecolor{textcolor}%
\pgfsetfillcolor{textcolor}%
\pgftext[x=1.242337in,y=0.354764in,,top]{\color{textcolor}\rmfamily\fontsize{8.000000}{9.600000}\selectfont \(\displaystyle 0.5\)}%
\end{pgfscope}%
\begin{pgfscope}%
\pgfpathrectangle{\pgfqpoint{0.592318in}{0.451986in}}{\pgfqpoint{4.085832in}{1.402021in}}%
\pgfusepath{clip}%
\pgfsetbuttcap%
\pgfsetroundjoin%
\pgfsetlinewidth{0.501875pt}%
\definecolor{currentstroke}{rgb}{0.690196,0.690196,0.690196}%
\pgfsetstrokecolor{currentstroke}%
\pgfsetdash{{1.850000pt}{0.800000pt}}{0.000000pt}%
\pgfpathmoveto{\pgfqpoint{1.706636in}{0.451986in}}%
\pgfpathlineto{\pgfqpoint{1.706636in}{1.854007in}}%
\pgfusepath{stroke}%
\end{pgfscope}%
\begin{pgfscope}%
\pgfsetbuttcap%
\pgfsetroundjoin%
\definecolor{currentfill}{rgb}{0.000000,0.000000,0.000000}%
\pgfsetfillcolor{currentfill}%
\pgfsetlinewidth{0.803000pt}%
\definecolor{currentstroke}{rgb}{0.000000,0.000000,0.000000}%
\pgfsetstrokecolor{currentstroke}%
\pgfsetdash{}{0pt}%
\pgfsys@defobject{currentmarker}{\pgfqpoint{0.000000in}{-0.048611in}}{\pgfqpoint{0.000000in}{0.000000in}}{%
\pgfpathmoveto{\pgfqpoint{0.000000in}{0.000000in}}%
\pgfpathlineto{\pgfqpoint{0.000000in}{-0.048611in}}%
\pgfusepath{stroke,fill}%
}%
\begin{pgfscope}%
\pgfsys@transformshift{1.706636in}{0.451986in}%
\pgfsys@useobject{currentmarker}{}%
\end{pgfscope}%
\end{pgfscope}%
\begin{pgfscope}%
\definecolor{textcolor}{rgb}{0.000000,0.000000,0.000000}%
\pgfsetstrokecolor{textcolor}%
\pgfsetfillcolor{textcolor}%
\pgftext[x=1.706636in,y=0.354764in,,top]{\color{textcolor}\rmfamily\fontsize{8.000000}{9.600000}\selectfont \(\displaystyle 1.0\)}%
\end{pgfscope}%
\begin{pgfscope}%
\pgfpathrectangle{\pgfqpoint{0.592318in}{0.451986in}}{\pgfqpoint{4.085832in}{1.402021in}}%
\pgfusepath{clip}%
\pgfsetbuttcap%
\pgfsetroundjoin%
\pgfsetlinewidth{0.501875pt}%
\definecolor{currentstroke}{rgb}{0.690196,0.690196,0.690196}%
\pgfsetstrokecolor{currentstroke}%
\pgfsetdash{{1.850000pt}{0.800000pt}}{0.000000pt}%
\pgfpathmoveto{\pgfqpoint{2.170935in}{0.451986in}}%
\pgfpathlineto{\pgfqpoint{2.170935in}{1.854007in}}%
\pgfusepath{stroke}%
\end{pgfscope}%
\begin{pgfscope}%
\pgfsetbuttcap%
\pgfsetroundjoin%
\definecolor{currentfill}{rgb}{0.000000,0.000000,0.000000}%
\pgfsetfillcolor{currentfill}%
\pgfsetlinewidth{0.803000pt}%
\definecolor{currentstroke}{rgb}{0.000000,0.000000,0.000000}%
\pgfsetstrokecolor{currentstroke}%
\pgfsetdash{}{0pt}%
\pgfsys@defobject{currentmarker}{\pgfqpoint{0.000000in}{-0.048611in}}{\pgfqpoint{0.000000in}{0.000000in}}{%
\pgfpathmoveto{\pgfqpoint{0.000000in}{0.000000in}}%
\pgfpathlineto{\pgfqpoint{0.000000in}{-0.048611in}}%
\pgfusepath{stroke,fill}%
}%
\begin{pgfscope}%
\pgfsys@transformshift{2.170935in}{0.451986in}%
\pgfsys@useobject{currentmarker}{}%
\end{pgfscope}%
\end{pgfscope}%
\begin{pgfscope}%
\definecolor{textcolor}{rgb}{0.000000,0.000000,0.000000}%
\pgfsetstrokecolor{textcolor}%
\pgfsetfillcolor{textcolor}%
\pgftext[x=2.170935in,y=0.354764in,,top]{\color{textcolor}\rmfamily\fontsize{8.000000}{9.600000}\selectfont \(\displaystyle 1.5\)}%
\end{pgfscope}%
\begin{pgfscope}%
\pgfpathrectangle{\pgfqpoint{0.592318in}{0.451986in}}{\pgfqpoint{4.085832in}{1.402021in}}%
\pgfusepath{clip}%
\pgfsetbuttcap%
\pgfsetroundjoin%
\pgfsetlinewidth{0.501875pt}%
\definecolor{currentstroke}{rgb}{0.690196,0.690196,0.690196}%
\pgfsetstrokecolor{currentstroke}%
\pgfsetdash{{1.850000pt}{0.800000pt}}{0.000000pt}%
\pgfpathmoveto{\pgfqpoint{2.635234in}{0.451986in}}%
\pgfpathlineto{\pgfqpoint{2.635234in}{1.854007in}}%
\pgfusepath{stroke}%
\end{pgfscope}%
\begin{pgfscope}%
\pgfsetbuttcap%
\pgfsetroundjoin%
\definecolor{currentfill}{rgb}{0.000000,0.000000,0.000000}%
\pgfsetfillcolor{currentfill}%
\pgfsetlinewidth{0.803000pt}%
\definecolor{currentstroke}{rgb}{0.000000,0.000000,0.000000}%
\pgfsetstrokecolor{currentstroke}%
\pgfsetdash{}{0pt}%
\pgfsys@defobject{currentmarker}{\pgfqpoint{0.000000in}{-0.048611in}}{\pgfqpoint{0.000000in}{0.000000in}}{%
\pgfpathmoveto{\pgfqpoint{0.000000in}{0.000000in}}%
\pgfpathlineto{\pgfqpoint{0.000000in}{-0.048611in}}%
\pgfusepath{stroke,fill}%
}%
\begin{pgfscope}%
\pgfsys@transformshift{2.635234in}{0.451986in}%
\pgfsys@useobject{currentmarker}{}%
\end{pgfscope}%
\end{pgfscope}%
\begin{pgfscope}%
\definecolor{textcolor}{rgb}{0.000000,0.000000,0.000000}%
\pgfsetstrokecolor{textcolor}%
\pgfsetfillcolor{textcolor}%
\pgftext[x=2.635234in,y=0.354764in,,top]{\color{textcolor}\rmfamily\fontsize{8.000000}{9.600000}\selectfont \(\displaystyle 2.0\)}%
\end{pgfscope}%
\begin{pgfscope}%
\pgfpathrectangle{\pgfqpoint{0.592318in}{0.451986in}}{\pgfqpoint{4.085832in}{1.402021in}}%
\pgfusepath{clip}%
\pgfsetbuttcap%
\pgfsetroundjoin%
\pgfsetlinewidth{0.501875pt}%
\definecolor{currentstroke}{rgb}{0.690196,0.690196,0.690196}%
\pgfsetstrokecolor{currentstroke}%
\pgfsetdash{{1.850000pt}{0.800000pt}}{0.000000pt}%
\pgfpathmoveto{\pgfqpoint{3.099534in}{0.451986in}}%
\pgfpathlineto{\pgfqpoint{3.099534in}{1.854007in}}%
\pgfusepath{stroke}%
\end{pgfscope}%
\begin{pgfscope}%
\pgfsetbuttcap%
\pgfsetroundjoin%
\definecolor{currentfill}{rgb}{0.000000,0.000000,0.000000}%
\pgfsetfillcolor{currentfill}%
\pgfsetlinewidth{0.803000pt}%
\definecolor{currentstroke}{rgb}{0.000000,0.000000,0.000000}%
\pgfsetstrokecolor{currentstroke}%
\pgfsetdash{}{0pt}%
\pgfsys@defobject{currentmarker}{\pgfqpoint{0.000000in}{-0.048611in}}{\pgfqpoint{0.000000in}{0.000000in}}{%
\pgfpathmoveto{\pgfqpoint{0.000000in}{0.000000in}}%
\pgfpathlineto{\pgfqpoint{0.000000in}{-0.048611in}}%
\pgfusepath{stroke,fill}%
}%
\begin{pgfscope}%
\pgfsys@transformshift{3.099534in}{0.451986in}%
\pgfsys@useobject{currentmarker}{}%
\end{pgfscope}%
\end{pgfscope}%
\begin{pgfscope}%
\definecolor{textcolor}{rgb}{0.000000,0.000000,0.000000}%
\pgfsetstrokecolor{textcolor}%
\pgfsetfillcolor{textcolor}%
\pgftext[x=3.099534in,y=0.354764in,,top]{\color{textcolor}\rmfamily\fontsize{8.000000}{9.600000}\selectfont \(\displaystyle 2.5\)}%
\end{pgfscope}%
\begin{pgfscope}%
\pgfpathrectangle{\pgfqpoint{0.592318in}{0.451986in}}{\pgfqpoint{4.085832in}{1.402021in}}%
\pgfusepath{clip}%
\pgfsetbuttcap%
\pgfsetroundjoin%
\pgfsetlinewidth{0.501875pt}%
\definecolor{currentstroke}{rgb}{0.690196,0.690196,0.690196}%
\pgfsetstrokecolor{currentstroke}%
\pgfsetdash{{1.850000pt}{0.800000pt}}{0.000000pt}%
\pgfpathmoveto{\pgfqpoint{3.563833in}{0.451986in}}%
\pgfpathlineto{\pgfqpoint{3.563833in}{1.854007in}}%
\pgfusepath{stroke}%
\end{pgfscope}%
\begin{pgfscope}%
\pgfsetbuttcap%
\pgfsetroundjoin%
\definecolor{currentfill}{rgb}{0.000000,0.000000,0.000000}%
\pgfsetfillcolor{currentfill}%
\pgfsetlinewidth{0.803000pt}%
\definecolor{currentstroke}{rgb}{0.000000,0.000000,0.000000}%
\pgfsetstrokecolor{currentstroke}%
\pgfsetdash{}{0pt}%
\pgfsys@defobject{currentmarker}{\pgfqpoint{0.000000in}{-0.048611in}}{\pgfqpoint{0.000000in}{0.000000in}}{%
\pgfpathmoveto{\pgfqpoint{0.000000in}{0.000000in}}%
\pgfpathlineto{\pgfqpoint{0.000000in}{-0.048611in}}%
\pgfusepath{stroke,fill}%
}%
\begin{pgfscope}%
\pgfsys@transformshift{3.563833in}{0.451986in}%
\pgfsys@useobject{currentmarker}{}%
\end{pgfscope}%
\end{pgfscope}%
\begin{pgfscope}%
\definecolor{textcolor}{rgb}{0.000000,0.000000,0.000000}%
\pgfsetstrokecolor{textcolor}%
\pgfsetfillcolor{textcolor}%
\pgftext[x=3.563833in,y=0.354764in,,top]{\color{textcolor}\rmfamily\fontsize{8.000000}{9.600000}\selectfont \(\displaystyle 3.0\)}%
\end{pgfscope}%
\begin{pgfscope}%
\pgfpathrectangle{\pgfqpoint{0.592318in}{0.451986in}}{\pgfqpoint{4.085832in}{1.402021in}}%
\pgfusepath{clip}%
\pgfsetbuttcap%
\pgfsetroundjoin%
\pgfsetlinewidth{0.501875pt}%
\definecolor{currentstroke}{rgb}{0.690196,0.690196,0.690196}%
\pgfsetstrokecolor{currentstroke}%
\pgfsetdash{{1.850000pt}{0.800000pt}}{0.000000pt}%
\pgfpathmoveto{\pgfqpoint{4.028132in}{0.451986in}}%
\pgfpathlineto{\pgfqpoint{4.028132in}{1.854007in}}%
\pgfusepath{stroke}%
\end{pgfscope}%
\begin{pgfscope}%
\pgfsetbuttcap%
\pgfsetroundjoin%
\definecolor{currentfill}{rgb}{0.000000,0.000000,0.000000}%
\pgfsetfillcolor{currentfill}%
\pgfsetlinewidth{0.803000pt}%
\definecolor{currentstroke}{rgb}{0.000000,0.000000,0.000000}%
\pgfsetstrokecolor{currentstroke}%
\pgfsetdash{}{0pt}%
\pgfsys@defobject{currentmarker}{\pgfqpoint{0.000000in}{-0.048611in}}{\pgfqpoint{0.000000in}{0.000000in}}{%
\pgfpathmoveto{\pgfqpoint{0.000000in}{0.000000in}}%
\pgfpathlineto{\pgfqpoint{0.000000in}{-0.048611in}}%
\pgfusepath{stroke,fill}%
}%
\begin{pgfscope}%
\pgfsys@transformshift{4.028132in}{0.451986in}%
\pgfsys@useobject{currentmarker}{}%
\end{pgfscope}%
\end{pgfscope}%
\begin{pgfscope}%
\definecolor{textcolor}{rgb}{0.000000,0.000000,0.000000}%
\pgfsetstrokecolor{textcolor}%
\pgfsetfillcolor{textcolor}%
\pgftext[x=4.028132in,y=0.354764in,,top]{\color{textcolor}\rmfamily\fontsize{8.000000}{9.600000}\selectfont \(\displaystyle 3.5\)}%
\end{pgfscope}%
\begin{pgfscope}%
\pgfpathrectangle{\pgfqpoint{0.592318in}{0.451986in}}{\pgfqpoint{4.085832in}{1.402021in}}%
\pgfusepath{clip}%
\pgfsetbuttcap%
\pgfsetroundjoin%
\pgfsetlinewidth{0.501875pt}%
\definecolor{currentstroke}{rgb}{0.690196,0.690196,0.690196}%
\pgfsetstrokecolor{currentstroke}%
\pgfsetdash{{1.850000pt}{0.800000pt}}{0.000000pt}%
\pgfpathmoveto{\pgfqpoint{4.492431in}{0.451986in}}%
\pgfpathlineto{\pgfqpoint{4.492431in}{1.854007in}}%
\pgfusepath{stroke}%
\end{pgfscope}%
\begin{pgfscope}%
\pgfsetbuttcap%
\pgfsetroundjoin%
\definecolor{currentfill}{rgb}{0.000000,0.000000,0.000000}%
\pgfsetfillcolor{currentfill}%
\pgfsetlinewidth{0.803000pt}%
\definecolor{currentstroke}{rgb}{0.000000,0.000000,0.000000}%
\pgfsetstrokecolor{currentstroke}%
\pgfsetdash{}{0pt}%
\pgfsys@defobject{currentmarker}{\pgfqpoint{0.000000in}{-0.048611in}}{\pgfqpoint{0.000000in}{0.000000in}}{%
\pgfpathmoveto{\pgfqpoint{0.000000in}{0.000000in}}%
\pgfpathlineto{\pgfqpoint{0.000000in}{-0.048611in}}%
\pgfusepath{stroke,fill}%
}%
\begin{pgfscope}%
\pgfsys@transformshift{4.492431in}{0.451986in}%
\pgfsys@useobject{currentmarker}{}%
\end{pgfscope}%
\end{pgfscope}%
\begin{pgfscope}%
\definecolor{textcolor}{rgb}{0.000000,0.000000,0.000000}%
\pgfsetstrokecolor{textcolor}%
\pgfsetfillcolor{textcolor}%
\pgftext[x=4.492431in,y=0.354764in,,top]{\color{textcolor}\rmfamily\fontsize{8.000000}{9.600000}\selectfont \(\displaystyle 4.0\)}%
\end{pgfscope}%
\begin{pgfscope}%
\definecolor{textcolor}{rgb}{0.000000,0.000000,0.000000}%
\pgfsetstrokecolor{textcolor}%
\pgfsetfillcolor{textcolor}%
\pgftext[x=2.635234in,y=0.201084in,,top]{\color{textcolor}\rmfamily\fontsize{8.000000}{9.600000}\selectfont Magnetfeld B [T]}%
\end{pgfscope}%
\begin{pgfscope}%
\pgfpathrectangle{\pgfqpoint{0.592318in}{0.451986in}}{\pgfqpoint{4.085832in}{1.402021in}}%
\pgfusepath{clip}%
\pgfsetbuttcap%
\pgfsetroundjoin%
\pgfsetlinewidth{0.501875pt}%
\definecolor{currentstroke}{rgb}{0.690196,0.690196,0.690196}%
\pgfsetstrokecolor{currentstroke}%
\pgfsetdash{{1.850000pt}{0.800000pt}}{0.000000pt}%
\pgfpathmoveto{\pgfqpoint{0.592318in}{0.515714in}}%
\pgfpathlineto{\pgfqpoint{4.678151in}{0.515714in}}%
\pgfusepath{stroke}%
\end{pgfscope}%
\begin{pgfscope}%
\pgfsetbuttcap%
\pgfsetroundjoin%
\definecolor{currentfill}{rgb}{0.000000,0.000000,0.000000}%
\pgfsetfillcolor{currentfill}%
\pgfsetlinewidth{0.803000pt}%
\definecolor{currentstroke}{rgb}{0.000000,0.000000,0.000000}%
\pgfsetstrokecolor{currentstroke}%
\pgfsetdash{}{0pt}%
\pgfsys@defobject{currentmarker}{\pgfqpoint{-0.048611in}{0.000000in}}{\pgfqpoint{0.000000in}{0.000000in}}{%
\pgfpathmoveto{\pgfqpoint{0.000000in}{0.000000in}}%
\pgfpathlineto{\pgfqpoint{-0.048611in}{0.000000in}}%
\pgfusepath{stroke,fill}%
}%
\begin{pgfscope}%
\pgfsys@transformshift{0.592318in}{0.515714in}%
\pgfsys@useobject{currentmarker}{}%
\end{pgfscope}%
\end{pgfscope}%
\begin{pgfscope}%
\definecolor{textcolor}{rgb}{0.000000,0.000000,0.000000}%
\pgfsetstrokecolor{textcolor}%
\pgfsetfillcolor{textcolor}%
\pgftext[x=0.436067in,y=0.477452in,left,base]{\color{textcolor}\rmfamily\fontsize{8.000000}{9.600000}\selectfont \(\displaystyle 0\)}%
\end{pgfscope}%
\begin{pgfscope}%
\pgfpathrectangle{\pgfqpoint{0.592318in}{0.451986in}}{\pgfqpoint{4.085832in}{1.402021in}}%
\pgfusepath{clip}%
\pgfsetbuttcap%
\pgfsetroundjoin%
\pgfsetlinewidth{0.501875pt}%
\definecolor{currentstroke}{rgb}{0.690196,0.690196,0.690196}%
\pgfsetstrokecolor{currentstroke}%
\pgfsetdash{{1.850000pt}{0.800000pt}}{0.000000pt}%
\pgfpathmoveto{\pgfqpoint{0.592318in}{0.982189in}}%
\pgfpathlineto{\pgfqpoint{4.678151in}{0.982189in}}%
\pgfusepath{stroke}%
\end{pgfscope}%
\begin{pgfscope}%
\pgfsetbuttcap%
\pgfsetroundjoin%
\definecolor{currentfill}{rgb}{0.000000,0.000000,0.000000}%
\pgfsetfillcolor{currentfill}%
\pgfsetlinewidth{0.803000pt}%
\definecolor{currentstroke}{rgb}{0.000000,0.000000,0.000000}%
\pgfsetstrokecolor{currentstroke}%
\pgfsetdash{}{0pt}%
\pgfsys@defobject{currentmarker}{\pgfqpoint{-0.048611in}{0.000000in}}{\pgfqpoint{0.000000in}{0.000000in}}{%
\pgfpathmoveto{\pgfqpoint{0.000000in}{0.000000in}}%
\pgfpathlineto{\pgfqpoint{-0.048611in}{0.000000in}}%
\pgfusepath{stroke,fill}%
}%
\begin{pgfscope}%
\pgfsys@transformshift{0.592318in}{0.982189in}%
\pgfsys@useobject{currentmarker}{}%
\end{pgfscope}%
\end{pgfscope}%
\begin{pgfscope}%
\definecolor{textcolor}{rgb}{0.000000,0.000000,0.000000}%
\pgfsetstrokecolor{textcolor}%
\pgfsetfillcolor{textcolor}%
\pgftext[x=0.258982in,y=0.943926in,left,base]{\color{textcolor}\rmfamily\fontsize{8.000000}{9.600000}\selectfont \(\displaystyle 2000\)}%
\end{pgfscope}%
\begin{pgfscope}%
\pgfpathrectangle{\pgfqpoint{0.592318in}{0.451986in}}{\pgfqpoint{4.085832in}{1.402021in}}%
\pgfusepath{clip}%
\pgfsetbuttcap%
\pgfsetroundjoin%
\pgfsetlinewidth{0.501875pt}%
\definecolor{currentstroke}{rgb}{0.690196,0.690196,0.690196}%
\pgfsetstrokecolor{currentstroke}%
\pgfsetdash{{1.850000pt}{0.800000pt}}{0.000000pt}%
\pgfpathmoveto{\pgfqpoint{0.592318in}{1.448663in}}%
\pgfpathlineto{\pgfqpoint{4.678151in}{1.448663in}}%
\pgfusepath{stroke}%
\end{pgfscope}%
\begin{pgfscope}%
\pgfsetbuttcap%
\pgfsetroundjoin%
\definecolor{currentfill}{rgb}{0.000000,0.000000,0.000000}%
\pgfsetfillcolor{currentfill}%
\pgfsetlinewidth{0.803000pt}%
\definecolor{currentstroke}{rgb}{0.000000,0.000000,0.000000}%
\pgfsetstrokecolor{currentstroke}%
\pgfsetdash{}{0pt}%
\pgfsys@defobject{currentmarker}{\pgfqpoint{-0.048611in}{0.000000in}}{\pgfqpoint{0.000000in}{0.000000in}}{%
\pgfpathmoveto{\pgfqpoint{0.000000in}{0.000000in}}%
\pgfpathlineto{\pgfqpoint{-0.048611in}{0.000000in}}%
\pgfusepath{stroke,fill}%
}%
\begin{pgfscope}%
\pgfsys@transformshift{0.592318in}{1.448663in}%
\pgfsys@useobject{currentmarker}{}%
\end{pgfscope}%
\end{pgfscope}%
\begin{pgfscope}%
\definecolor{textcolor}{rgb}{0.000000,0.000000,0.000000}%
\pgfsetstrokecolor{textcolor}%
\pgfsetfillcolor{textcolor}%
\pgftext[x=0.258982in,y=1.410401in,left,base]{\color{textcolor}\rmfamily\fontsize{8.000000}{9.600000}\selectfont \(\displaystyle 4000\)}%
\end{pgfscope}%
\begin{pgfscope}%
\definecolor{textcolor}{rgb}{0.000000,0.000000,0.000000}%
\pgfsetstrokecolor{textcolor}%
\pgfsetfillcolor{textcolor}%
\pgftext[x=0.203426in,y=1.152996in,,bottom,rotate=90.000000]{\color{textcolor}\rmfamily\fontsize{8.000000}{9.600000}\selectfont Hallwiderstand \(\displaystyle R_{xy} [\Omega]\)}%
\end{pgfscope}%
\begin{pgfscope}%
\pgfpathrectangle{\pgfqpoint{0.592318in}{0.451986in}}{\pgfqpoint{4.085832in}{1.402021in}}%
\pgfusepath{clip}%
\pgfsetrectcap%
\pgfsetroundjoin%
\pgfsetlinewidth{1.003750pt}%
\definecolor{currentstroke}{rgb}{0.152941,0.235294,0.458824}%
\pgfsetstrokecolor{currentstroke}%
\pgfsetdash{}{0pt}%
\pgfpathmoveto{\pgfqpoint{0.778038in}{0.516459in}}%
\pgfpathlineto{\pgfqpoint{0.784538in}{0.517155in}}%
\pgfpathlineto{\pgfqpoint{0.787324in}{0.518597in}}%
\pgfpathlineto{\pgfqpoint{0.791967in}{0.520048in}}%
\pgfpathlineto{\pgfqpoint{0.795681in}{0.521012in}}%
\pgfpathlineto{\pgfqpoint{0.797538in}{0.521977in}}%
\pgfpathlineto{\pgfqpoint{0.801253in}{0.522948in}}%
\pgfpathlineto{\pgfqpoint{0.803110in}{0.523915in}}%
\pgfpathlineto{\pgfqpoint{0.806824in}{0.524869in}}%
\pgfpathlineto{\pgfqpoint{0.808682in}{0.525846in}}%
\pgfpathlineto{\pgfqpoint{0.812396in}{0.526816in}}%
\pgfpathlineto{\pgfqpoint{0.819825in}{0.529711in}}%
\pgfpathlineto{\pgfqpoint{0.824468in}{0.530673in}}%
\pgfpathlineto{\pgfqpoint{0.826325in}{0.531640in}}%
\pgfpathlineto{\pgfqpoint{0.829111in}{0.532599in}}%
\pgfpathlineto{\pgfqpoint{0.834682in}{0.534529in}}%
\pgfpathlineto{\pgfqpoint{0.838397in}{0.535494in}}%
\pgfpathlineto{\pgfqpoint{0.840254in}{0.536444in}}%
\pgfpathlineto{\pgfqpoint{0.843968in}{0.537409in}}%
\pgfpathlineto{\pgfqpoint{0.845826in}{0.538363in}}%
\pgfpathlineto{\pgfqpoint{0.849540in}{0.539319in}}%
\pgfpathlineto{\pgfqpoint{0.856969in}{0.542214in}}%
\pgfpathlineto{\pgfqpoint{0.861612in}{0.543170in}}%
\pgfpathlineto{\pgfqpoint{0.863469in}{0.544133in}}%
\pgfpathlineto{\pgfqpoint{0.866255in}{0.545096in}}%
\pgfpathlineto{\pgfqpoint{0.868112in}{0.546052in}}%
\pgfpathlineto{\pgfqpoint{0.871826in}{0.547005in}}%
\pgfpathlineto{\pgfqpoint{0.877398in}{0.548935in}}%
\pgfpathlineto{\pgfqpoint{0.880184in}{0.549903in}}%
\pgfpathlineto{\pgfqpoint{0.885755in}{0.551809in}}%
\pgfpathlineto{\pgfqpoint{0.889470in}{0.552778in}}%
\pgfpathlineto{\pgfqpoint{0.891327in}{0.553746in}}%
\pgfpathlineto{\pgfqpoint{0.895041in}{0.554716in}}%
\pgfpathlineto{\pgfqpoint{0.896898in}{0.555665in}}%
\pgfpathlineto{\pgfqpoint{0.900613in}{0.556644in}}%
\pgfpathlineto{\pgfqpoint{0.908042in}{0.559543in}}%
\pgfpathlineto{\pgfqpoint{0.911756in}{0.560503in}}%
\pgfpathlineto{\pgfqpoint{0.913613in}{0.561422in}}%
\pgfpathlineto{\pgfqpoint{0.917328in}{0.562458in}}%
\pgfpathlineto{\pgfqpoint{0.924756in}{0.565294in}}%
\pgfpathlineto{\pgfqpoint{0.929399in}{0.566273in}}%
\pgfpathlineto{\pgfqpoint{0.931257in}{0.567233in}}%
\pgfpathlineto{\pgfqpoint{0.934971in}{0.568183in}}%
\pgfpathlineto{\pgfqpoint{0.936828in}{0.569153in}}%
\pgfpathlineto{\pgfqpoint{0.940543in}{0.570119in}}%
\pgfpathlineto{\pgfqpoint{0.947971in}{0.572999in}}%
\pgfpathlineto{\pgfqpoint{0.951686in}{0.573981in}}%
\pgfpathlineto{\pgfqpoint{0.953543in}{0.574931in}}%
\pgfpathlineto{\pgfqpoint{0.957257in}{0.575899in}}%
\pgfpathlineto{\pgfqpoint{0.964686in}{0.578781in}}%
\pgfpathlineto{\pgfqpoint{0.969329in}{0.579748in}}%
\pgfpathlineto{\pgfqpoint{0.971186in}{0.580712in}}%
\pgfpathlineto{\pgfqpoint{0.974901in}{0.581667in}}%
\pgfpathlineto{\pgfqpoint{0.976758in}{0.582642in}}%
\pgfpathlineto{\pgfqpoint{0.980472in}{0.583609in}}%
\pgfpathlineto{\pgfqpoint{0.987901in}{0.586497in}}%
\pgfpathlineto{\pgfqpoint{0.991615in}{0.587463in}}%
\pgfpathlineto{\pgfqpoint{0.999044in}{0.590352in}}%
\pgfpathlineto{\pgfqpoint{1.002759in}{0.591321in}}%
\pgfpathlineto{\pgfqpoint{1.004616in}{0.592288in}}%
\pgfpathlineto{\pgfqpoint{1.009259in}{0.593227in}}%
\pgfpathlineto{\pgfqpoint{1.011116in}{0.594196in}}%
\pgfpathlineto{\pgfqpoint{1.014830in}{0.595156in}}%
\pgfpathlineto{\pgfqpoint{1.017616in}{0.596613in}}%
\pgfpathlineto{\pgfqpoint{1.022259in}{0.598060in}}%
\pgfpathlineto{\pgfqpoint{1.025974in}{0.599030in}}%
\pgfpathlineto{\pgfqpoint{1.027831in}{0.600004in}}%
\pgfpathlineto{\pgfqpoint{1.031545in}{0.600975in}}%
\pgfpathlineto{\pgfqpoint{1.033402in}{0.601942in}}%
\pgfpathlineto{\pgfqpoint{1.037117in}{0.602904in}}%
\pgfpathlineto{\pgfqpoint{1.038974in}{0.603872in}}%
\pgfpathlineto{\pgfqpoint{1.042688in}{0.604847in}}%
\pgfpathlineto{\pgfqpoint{1.044546in}{0.605819in}}%
\pgfpathlineto{\pgfqpoint{1.048260in}{0.606781in}}%
\pgfpathlineto{\pgfqpoint{1.055689in}{0.609671in}}%
\pgfpathlineto{\pgfqpoint{1.059403in}{0.610630in}}%
\pgfpathlineto{\pgfqpoint{1.061260in}{0.611588in}}%
\pgfpathlineto{\pgfqpoint{1.065903in}{0.612561in}}%
\pgfpathlineto{\pgfqpoint{1.067761in}{0.613509in}}%
\pgfpathlineto{\pgfqpoint{1.071475in}{0.614482in}}%
\pgfpathlineto{\pgfqpoint{1.073332in}{0.615431in}}%
\pgfpathlineto{\pgfqpoint{1.077046in}{0.616398in}}%
\pgfpathlineto{\pgfqpoint{1.078904in}{0.617370in}}%
\pgfpathlineto{\pgfqpoint{1.082618in}{0.618333in}}%
\pgfpathlineto{\pgfqpoint{1.084475in}{0.619298in}}%
\pgfpathlineto{\pgfqpoint{1.088190in}{0.620252in}}%
\pgfpathlineto{\pgfqpoint{1.095618in}{0.623154in}}%
\pgfpathlineto{\pgfqpoint{1.100261in}{0.624148in}}%
\pgfpathlineto{\pgfqpoint{1.102119in}{0.625109in}}%
\pgfpathlineto{\pgfqpoint{1.105833in}{0.626046in}}%
\pgfpathlineto{\pgfqpoint{1.108619in}{0.627990in}}%
\pgfpathlineto{\pgfqpoint{1.114190in}{0.628956in}}%
\pgfpathlineto{\pgfqpoint{1.116048in}{0.629934in}}%
\pgfpathlineto{\pgfqpoint{1.119762in}{0.630897in}}%
\pgfpathlineto{\pgfqpoint{1.121619in}{0.631858in}}%
\pgfpathlineto{\pgfqpoint{1.125334in}{0.632824in}}%
\pgfpathlineto{\pgfqpoint{1.127191in}{0.633780in}}%
\pgfpathlineto{\pgfqpoint{1.130905in}{0.634736in}}%
\pgfpathlineto{\pgfqpoint{1.138334in}{0.637641in}}%
\pgfpathlineto{\pgfqpoint{1.142048in}{0.638615in}}%
\pgfpathlineto{\pgfqpoint{1.143906in}{0.639577in}}%
\pgfpathlineto{\pgfqpoint{1.148549in}{0.640549in}}%
\pgfpathlineto{\pgfqpoint{1.150406in}{0.641510in}}%
\pgfpathlineto{\pgfqpoint{1.154120in}{0.642481in}}%
\pgfpathlineto{\pgfqpoint{1.155977in}{0.643442in}}%
\pgfpathlineto{\pgfqpoint{1.159692in}{0.644411in}}%
\pgfpathlineto{\pgfqpoint{1.161549in}{0.645377in}}%
\pgfpathlineto{\pgfqpoint{1.165263in}{0.646337in}}%
\pgfpathlineto{\pgfqpoint{1.167121in}{0.647299in}}%
\pgfpathlineto{\pgfqpoint{1.171764in}{0.648254in}}%
\pgfpathlineto{\pgfqpoint{1.174549in}{0.649726in}}%
\pgfpathlineto{\pgfqpoint{1.181049in}{0.652159in}}%
\pgfpathlineto{\pgfqpoint{1.184764in}{0.653102in}}%
\pgfpathlineto{\pgfqpoint{1.192193in}{0.655985in}}%
\pgfpathlineto{\pgfqpoint{1.196836in}{0.656950in}}%
\pgfpathlineto{\pgfqpoint{1.198693in}{0.657903in}}%
\pgfpathlineto{\pgfqpoint{1.202407in}{0.658875in}}%
\pgfpathlineto{\pgfqpoint{1.204264in}{0.659820in}}%
\pgfpathlineto{\pgfqpoint{1.207050in}{0.660762in}}%
\pgfpathlineto{\pgfqpoint{1.212622in}{0.662715in}}%
\pgfpathlineto{\pgfqpoint{1.216336in}{0.663703in}}%
\pgfpathlineto{\pgfqpoint{1.218193in}{0.664663in}}%
\pgfpathlineto{\pgfqpoint{1.221908in}{0.665645in}}%
\pgfpathlineto{\pgfqpoint{1.223765in}{0.666622in}}%
\pgfpathlineto{\pgfqpoint{1.227479in}{0.667576in}}%
\pgfpathlineto{\pgfqpoint{1.234908in}{0.670474in}}%
\pgfpathlineto{\pgfqpoint{1.238623in}{0.671424in}}%
\pgfpathlineto{\pgfqpoint{1.244194in}{0.673350in}}%
\pgfpathlineto{\pgfqpoint{1.246980in}{0.674302in}}%
\pgfpathlineto{\pgfqpoint{1.252552in}{0.676244in}}%
\pgfpathlineto{\pgfqpoint{1.256266in}{0.677202in}}%
\pgfpathlineto{\pgfqpoint{1.258123in}{0.678161in}}%
\pgfpathlineto{\pgfqpoint{1.261838in}{0.679142in}}%
\pgfpathlineto{\pgfqpoint{1.269266in}{0.682043in}}%
\pgfpathlineto{\pgfqpoint{1.272981in}{0.683009in}}%
\pgfpathlineto{\pgfqpoint{1.274838in}{0.683963in}}%
\pgfpathlineto{\pgfqpoint{1.279481in}{0.684915in}}%
\pgfpathlineto{\pgfqpoint{1.281338in}{0.685867in}}%
\pgfpathlineto{\pgfqpoint{1.284124in}{0.686831in}}%
\pgfpathlineto{\pgfqpoint{1.289695in}{0.688793in}}%
\pgfpathlineto{\pgfqpoint{1.293410in}{0.689758in}}%
\pgfpathlineto{\pgfqpoint{1.295267in}{0.690728in}}%
\pgfpathlineto{\pgfqpoint{1.298981in}{0.691686in}}%
\pgfpathlineto{\pgfqpoint{1.300839in}{0.692644in}}%
\pgfpathlineto{\pgfqpoint{1.304553in}{0.693598in}}%
\pgfpathlineto{\pgfqpoint{1.306410in}{0.694563in}}%
\pgfpathlineto{\pgfqpoint{1.310125in}{0.695505in}}%
\pgfpathlineto{\pgfqpoint{1.311982in}{0.696471in}}%
\pgfpathlineto{\pgfqpoint{1.315696in}{0.697432in}}%
\pgfpathlineto{\pgfqpoint{1.323125in}{0.700357in}}%
\pgfpathlineto{\pgfqpoint{1.327768in}{0.701328in}}%
\pgfpathlineto{\pgfqpoint{1.330554in}{0.702752in}}%
\pgfpathlineto{\pgfqpoint{1.337054in}{0.705121in}}%
\pgfpathlineto{\pgfqpoint{1.341697in}{0.706062in}}%
\pgfpathlineto{\pgfqpoint{1.343554in}{0.707018in}}%
\pgfpathlineto{\pgfqpoint{1.347269in}{0.707969in}}%
\pgfpathlineto{\pgfqpoint{1.349126in}{0.708952in}}%
\pgfpathlineto{\pgfqpoint{1.352840in}{0.709925in}}%
\pgfpathlineto{\pgfqpoint{1.354697in}{0.710920in}}%
\pgfpathlineto{\pgfqpoint{1.358412in}{0.711900in}}%
\pgfpathlineto{\pgfqpoint{1.360269in}{0.712901in}}%
\pgfpathlineto{\pgfqpoint{1.364912in}{0.713863in}}%
\pgfpathlineto{\pgfqpoint{1.367698in}{0.715299in}}%
\pgfpathlineto{\pgfqpoint{1.374198in}{0.717647in}}%
\pgfpathlineto{\pgfqpoint{1.378841in}{0.718577in}}%
\pgfpathlineto{\pgfqpoint{1.380698in}{0.719531in}}%
\pgfpathlineto{\pgfqpoint{1.384413in}{0.720487in}}%
\pgfpathlineto{\pgfqpoint{1.386270in}{0.721457in}}%
\pgfpathlineto{\pgfqpoint{1.389056in}{0.722432in}}%
\pgfpathlineto{\pgfqpoint{1.394627in}{0.724406in}}%
\pgfpathlineto{\pgfqpoint{1.398341in}{0.725402in}}%
\pgfpathlineto{\pgfqpoint{1.400199in}{0.726394in}}%
\pgfpathlineto{\pgfqpoint{1.403913in}{0.727372in}}%
\pgfpathlineto{\pgfqpoint{1.405770in}{0.728328in}}%
\pgfpathlineto{\pgfqpoint{1.409485in}{0.729267in}}%
\pgfpathlineto{\pgfqpoint{1.416913in}{0.732068in}}%
\pgfpathlineto{\pgfqpoint{1.420628in}{0.732993in}}%
\pgfpathlineto{\pgfqpoint{1.422485in}{0.733911in}}%
\pgfpathlineto{\pgfqpoint{1.427128in}{0.734859in}}%
\pgfpathlineto{\pgfqpoint{1.428985in}{0.735821in}}%
\pgfpathlineto{\pgfqpoint{1.431771in}{0.736793in}}%
\pgfpathlineto{\pgfqpoint{1.437343in}{0.738793in}}%
\pgfpathlineto{\pgfqpoint{1.441057in}{0.739803in}}%
\pgfpathlineto{\pgfqpoint{1.442914in}{0.740804in}}%
\pgfpathlineto{\pgfqpoint{1.446629in}{0.741805in}}%
\pgfpathlineto{\pgfqpoint{1.448486in}{0.742792in}}%
\pgfpathlineto{\pgfqpoint{1.452200in}{0.743770in}}%
\pgfpathlineto{\pgfqpoint{1.459629in}{0.746604in}}%
\pgfpathlineto{\pgfqpoint{1.464272in}{0.747536in}}%
\pgfpathlineto{\pgfqpoint{1.466129in}{0.748449in}}%
\pgfpathlineto{\pgfqpoint{1.468915in}{0.749346in}}%
\pgfpathlineto{\pgfqpoint{1.476344in}{0.752147in}}%
\pgfpathlineto{\pgfqpoint{1.480987in}{0.753087in}}%
\pgfpathlineto{\pgfqpoint{1.483773in}{0.754556in}}%
\pgfpathlineto{\pgfqpoint{1.488416in}{0.756049in}}%
\pgfpathlineto{\pgfqpoint{1.492130in}{0.757071in}}%
\pgfpathlineto{\pgfqpoint{1.493987in}{0.758083in}}%
\pgfpathlineto{\pgfqpoint{1.497701in}{0.759097in}}%
\pgfpathlineto{\pgfqpoint{1.499559in}{0.760093in}}%
\pgfpathlineto{\pgfqpoint{1.503273in}{0.761113in}}%
\pgfpathlineto{\pgfqpoint{1.510702in}{0.764000in}}%
\pgfpathlineto{\pgfqpoint{1.515345in}{0.764921in}}%
\pgfpathlineto{\pgfqpoint{1.517202in}{0.765833in}}%
\pgfpathlineto{\pgfqpoint{1.519988in}{0.766715in}}%
\pgfpathlineto{\pgfqpoint{1.525559in}{0.768480in}}%
\pgfpathlineto{\pgfqpoint{1.529274in}{0.769378in}}%
\pgfpathlineto{\pgfqpoint{1.531131in}{0.770295in}}%
\pgfpathlineto{\pgfqpoint{1.534845in}{0.771202in}}%
\pgfpathlineto{\pgfqpoint{1.536703in}{0.772138in}}%
\pgfpathlineto{\pgfqpoint{1.540417in}{0.773106in}}%
\pgfpathlineto{\pgfqpoint{1.547846in}{0.776093in}}%
\pgfpathlineto{\pgfqpoint{1.552489in}{0.777131in}}%
\pgfpathlineto{\pgfqpoint{1.554346in}{0.778178in}}%
\pgfpathlineto{\pgfqpoint{1.557132in}{0.779221in}}%
\pgfpathlineto{\pgfqpoint{1.562703in}{0.781297in}}%
\pgfpathlineto{\pgfqpoint{1.566418in}{0.782330in}}%
\pgfpathlineto{\pgfqpoint{1.568275in}{0.783352in}}%
\pgfpathlineto{\pgfqpoint{1.571989in}{0.784343in}}%
\pgfpathlineto{\pgfqpoint{1.573847in}{0.785304in}}%
\pgfpathlineto{\pgfqpoint{1.577561in}{0.786241in}}%
\pgfpathlineto{\pgfqpoint{1.579418in}{0.787158in}}%
\pgfpathlineto{\pgfqpoint{1.583133in}{0.788044in}}%
\pgfpathlineto{\pgfqpoint{1.584990in}{0.788924in}}%
\pgfpathlineto{\pgfqpoint{1.588704in}{0.789787in}}%
\pgfpathlineto{\pgfqpoint{1.596133in}{0.792357in}}%
\pgfpathlineto{\pgfqpoint{1.599847in}{0.793208in}}%
\pgfpathlineto{\pgfqpoint{1.601705in}{0.794111in}}%
\pgfpathlineto{\pgfqpoint{1.605419in}{0.795028in}}%
\pgfpathlineto{\pgfqpoint{1.612848in}{0.797929in}}%
\pgfpathlineto{\pgfqpoint{1.617491in}{0.798925in}}%
\pgfpathlineto{\pgfqpoint{1.619348in}{0.799956in}}%
\pgfpathlineto{\pgfqpoint{1.623062in}{0.801001in}}%
\pgfpathlineto{\pgfqpoint{1.624919in}{0.802050in}}%
\pgfpathlineto{\pgfqpoint{1.628634in}{0.803128in}}%
\pgfpathlineto{\pgfqpoint{1.630491in}{0.804201in}}%
\pgfpathlineto{\pgfqpoint{1.634205in}{0.805278in}}%
\pgfpathlineto{\pgfqpoint{1.636063in}{0.806342in}}%
\pgfpathlineto{\pgfqpoint{1.639777in}{0.807394in}}%
\pgfpathlineto{\pgfqpoint{1.647206in}{0.810477in}}%
\pgfpathlineto{\pgfqpoint{1.651849in}{0.811440in}}%
\pgfpathlineto{\pgfqpoint{1.652777in}{0.812387in}}%
\pgfpathlineto{\pgfqpoint{1.657420in}{0.813290in}}%
\pgfpathlineto{\pgfqpoint{1.659278in}{0.814176in}}%
\pgfpathlineto{\pgfqpoint{1.662992in}{0.815039in}}%
\pgfpathlineto{\pgfqpoint{1.664849in}{0.815860in}}%
\pgfpathlineto{\pgfqpoint{1.668564in}{0.816688in}}%
\pgfpathlineto{\pgfqpoint{1.670421in}{0.817505in}}%
\pgfpathlineto{\pgfqpoint{1.674135in}{0.818305in}}%
\pgfpathlineto{\pgfqpoint{1.675992in}{0.819114in}}%
\pgfpathlineto{\pgfqpoint{1.679707in}{0.819940in}}%
\pgfpathlineto{\pgfqpoint{1.687136in}{0.822545in}}%
\pgfpathlineto{\pgfqpoint{1.691779in}{0.823452in}}%
\pgfpathlineto{\pgfqpoint{1.693636in}{0.824401in}}%
\pgfpathlineto{\pgfqpoint{1.697350in}{0.825372in}}%
\pgfpathlineto{\pgfqpoint{1.700136in}{0.826888in}}%
\pgfpathlineto{\pgfqpoint{1.704779in}{0.828434in}}%
\pgfpathlineto{\pgfqpoint{1.708493in}{0.829502in}}%
\pgfpathlineto{\pgfqpoint{1.710351in}{0.830587in}}%
\pgfpathlineto{\pgfqpoint{1.714065in}{0.831692in}}%
\pgfpathlineto{\pgfqpoint{1.715922in}{0.832784in}}%
\pgfpathlineto{\pgfqpoint{1.717779in}{0.833353in}}%
\pgfpathlineto{\pgfqpoint{1.724279in}{0.836133in}}%
\pgfpathlineto{\pgfqpoint{1.727994in}{0.837225in}}%
\pgfpathlineto{\pgfqpoint{1.729851in}{0.838319in}}%
\pgfpathlineto{\pgfqpoint{1.733565in}{0.839399in}}%
\pgfpathlineto{\pgfqpoint{1.740994in}{0.842515in}}%
\pgfpathlineto{\pgfqpoint{1.745637in}{0.843492in}}%
\pgfpathlineto{\pgfqpoint{1.747494in}{0.844439in}}%
\pgfpathlineto{\pgfqpoint{1.751209in}{0.845353in}}%
\pgfpathlineto{\pgfqpoint{1.753066in}{0.846237in}}%
\pgfpathlineto{\pgfqpoint{1.756780in}{0.847100in}}%
\pgfpathlineto{\pgfqpoint{1.758638in}{0.847902in}}%
\pgfpathlineto{\pgfqpoint{1.762352in}{0.848714in}}%
\pgfpathlineto{\pgfqpoint{1.764209in}{0.849477in}}%
\pgfpathlineto{\pgfqpoint{1.767924in}{0.850239in}}%
\pgfpathlineto{\pgfqpoint{1.769781in}{0.850988in}}%
\pgfpathlineto{\pgfqpoint{1.773495in}{0.851730in}}%
\pgfpathlineto{\pgfqpoint{1.781853in}{0.854039in}}%
\pgfpathlineto{\pgfqpoint{1.785567in}{0.854846in}}%
\pgfpathlineto{\pgfqpoint{1.788353in}{0.856115in}}%
\pgfpathlineto{\pgfqpoint{1.795782in}{0.858384in}}%
\pgfpathlineto{\pgfqpoint{1.799496in}{0.859354in}}%
\pgfpathlineto{\pgfqpoint{1.801353in}{0.860339in}}%
\pgfpathlineto{\pgfqpoint{1.805068in}{0.861346in}}%
\pgfpathlineto{\pgfqpoint{1.806925in}{0.862391in}}%
\pgfpathlineto{\pgfqpoint{1.810639in}{0.863455in}}%
\pgfpathlineto{\pgfqpoint{1.812496in}{0.864549in}}%
\pgfpathlineto{\pgfqpoint{1.816211in}{0.865649in}}%
\pgfpathlineto{\pgfqpoint{1.818068in}{0.866785in}}%
\pgfpathlineto{\pgfqpoint{1.818996in}{0.866785in}}%
\pgfpathlineto{\pgfqpoint{1.820854in}{0.867900in}}%
\pgfpathlineto{\pgfqpoint{1.822711in}{0.868481in}}%
\pgfpathlineto{\pgfqpoint{1.829211in}{0.871373in}}%
\pgfpathlineto{\pgfqpoint{1.830140in}{0.871373in}}%
\pgfpathlineto{\pgfqpoint{1.831997in}{0.872516in}}%
\pgfpathlineto{\pgfqpoint{1.833854in}{0.872516in}}%
\pgfpathlineto{\pgfqpoint{1.835711in}{0.873687in}}%
\pgfpathlineto{\pgfqpoint{1.836640in}{0.873687in}}%
\pgfpathlineto{\pgfqpoint{1.838497in}{0.874839in}}%
\pgfpathlineto{\pgfqpoint{1.839426in}{0.874839in}}%
\pgfpathlineto{\pgfqpoint{1.841283in}{0.875984in}}%
\pgfpathlineto{\pgfqpoint{1.842211in}{0.875984in}}%
\pgfpathlineto{\pgfqpoint{1.844069in}{0.877122in}}%
\pgfpathlineto{\pgfqpoint{1.844997in}{0.877122in}}%
\pgfpathlineto{\pgfqpoint{1.846854in}{0.878237in}}%
\pgfpathlineto{\pgfqpoint{1.850569in}{0.879340in}}%
\pgfpathlineto{\pgfqpoint{1.852426in}{0.880444in}}%
\pgfpathlineto{\pgfqpoint{1.856140in}{0.881510in}}%
\pgfpathlineto{\pgfqpoint{1.857998in}{0.882536in}}%
\pgfpathlineto{\pgfqpoint{1.862641in}{0.883541in}}%
\pgfpathlineto{\pgfqpoint{1.864498in}{0.884530in}}%
\pgfpathlineto{\pgfqpoint{1.867284in}{0.885461in}}%
\pgfpathlineto{\pgfqpoint{1.869141in}{0.886359in}}%
\pgfpathlineto{\pgfqpoint{1.872855in}{0.887233in}}%
\pgfpathlineto{\pgfqpoint{1.874712in}{0.888068in}}%
\pgfpathlineto{\pgfqpoint{1.878427in}{0.888864in}}%
\pgfpathlineto{\pgfqpoint{1.886784in}{0.891049in}}%
\pgfpathlineto{\pgfqpoint{1.890499in}{0.891730in}}%
\pgfpathlineto{\pgfqpoint{1.894213in}{0.893062in}}%
\pgfpathlineto{\pgfqpoint{1.898856in}{0.893729in}}%
\pgfpathlineto{\pgfqpoint{1.902570in}{0.895093in}}%
\pgfpathlineto{\pgfqpoint{1.908142in}{0.896170in}}%
\pgfpathlineto{\pgfqpoint{1.917428in}{0.898960in}}%
\pgfpathlineto{\pgfqpoint{1.921142in}{0.899835in}}%
\pgfpathlineto{\pgfqpoint{1.929500in}{0.902667in}}%
\pgfpathlineto{\pgfqpoint{1.933214in}{0.903672in}}%
\pgfpathlineto{\pgfqpoint{1.935071in}{0.904703in}}%
\pgfpathlineto{\pgfqpoint{1.938786in}{0.905764in}}%
\pgfpathlineto{\pgfqpoint{1.940643in}{0.906839in}}%
\pgfpathlineto{\pgfqpoint{1.941571in}{0.906839in}}%
\pgfpathlineto{\pgfqpoint{1.943429in}{0.907961in}}%
\pgfpathlineto{\pgfqpoint{1.944357in}{0.907961in}}%
\pgfpathlineto{\pgfqpoint{1.946214in}{0.909081in}}%
\pgfpathlineto{\pgfqpoint{1.948072in}{0.909657in}}%
\pgfpathlineto{\pgfqpoint{1.954572in}{0.912556in}}%
\pgfpathlineto{\pgfqpoint{1.956429in}{0.913138in}}%
\pgfpathlineto{\pgfqpoint{1.962929in}{0.916099in}}%
\pgfpathlineto{\pgfqpoint{1.964786in}{0.916702in}}%
\pgfpathlineto{\pgfqpoint{1.969429in}{0.918503in}}%
\pgfpathlineto{\pgfqpoint{1.970358in}{0.918503in}}%
\pgfpathlineto{\pgfqpoint{1.972215in}{0.919704in}}%
\pgfpathlineto{\pgfqpoint{1.973144in}{0.919704in}}%
\pgfpathlineto{\pgfqpoint{1.975001in}{0.920920in}}%
\pgfpathlineto{\pgfqpoint{1.975930in}{0.920920in}}%
\pgfpathlineto{\pgfqpoint{1.977787in}{0.922111in}}%
\pgfpathlineto{\pgfqpoint{1.978715in}{0.922111in}}%
\pgfpathlineto{\pgfqpoint{1.981501in}{0.924495in}}%
\pgfpathlineto{\pgfqpoint{1.984287in}{0.924495in}}%
\pgfpathlineto{\pgfqpoint{1.986144in}{0.925694in}}%
\pgfpathlineto{\pgfqpoint{1.987073in}{0.925694in}}%
\pgfpathlineto{\pgfqpoint{1.988930in}{0.926863in}}%
\pgfpathlineto{\pgfqpoint{1.989859in}{0.926863in}}%
\pgfpathlineto{\pgfqpoint{1.991716in}{0.928038in}}%
\pgfpathlineto{\pgfqpoint{1.992644in}{0.928038in}}%
\pgfpathlineto{\pgfqpoint{1.994502in}{0.929172in}}%
\pgfpathlineto{\pgfqpoint{1.996359in}{0.929747in}}%
\pgfpathlineto{\pgfqpoint{2.002859in}{0.932544in}}%
\pgfpathlineto{\pgfqpoint{2.006573in}{0.933594in}}%
\pgfpathlineto{\pgfqpoint{2.008431in}{0.934632in}}%
\pgfpathlineto{\pgfqpoint{2.012145in}{0.935646in}}%
\pgfpathlineto{\pgfqpoint{2.019574in}{0.938492in}}%
\pgfpathlineto{\pgfqpoint{2.024217in}{0.939371in}}%
\pgfpathlineto{\pgfqpoint{2.026074in}{0.940206in}}%
\pgfpathlineto{\pgfqpoint{2.028860in}{0.940994in}}%
\pgfpathlineto{\pgfqpoint{2.036288in}{0.943140in}}%
\pgfpathlineto{\pgfqpoint{2.040931in}{0.943775in}}%
\pgfpathlineto{\pgfqpoint{2.044646in}{0.944969in}}%
\pgfpathlineto{\pgfqpoint{2.050217in}{0.945811in}}%
\pgfpathlineto{\pgfqpoint{2.062289in}{0.948309in}}%
\pgfpathlineto{\pgfqpoint{2.066932in}{0.948901in}}%
\pgfpathlineto{\pgfqpoint{2.070647in}{0.950177in}}%
\pgfpathlineto{\pgfqpoint{2.074361in}{0.950863in}}%
\pgfpathlineto{\pgfqpoint{2.076218in}{0.951590in}}%
\pgfpathlineto{\pgfqpoint{2.080861in}{0.952372in}}%
\pgfpathlineto{\pgfqpoint{2.082718in}{0.953162in}}%
\pgfpathlineto{\pgfqpoint{2.086433in}{0.954021in}}%
\pgfpathlineto{\pgfqpoint{2.088290in}{0.954898in}}%
\pgfpathlineto{\pgfqpoint{2.092004in}{0.955826in}}%
\pgfpathlineto{\pgfqpoint{2.099433in}{0.958818in}}%
\pgfpathlineto{\pgfqpoint{2.103148in}{0.959868in}}%
\pgfpathlineto{\pgfqpoint{2.108719in}{0.962039in}}%
\pgfpathlineto{\pgfqpoint{2.112434in}{0.963140in}}%
\pgfpathlineto{\pgfqpoint{2.114291in}{0.964295in}}%
\pgfpathlineto{\pgfqpoint{2.115219in}{0.964295in}}%
\pgfpathlineto{\pgfqpoint{2.117077in}{0.965461in}}%
\pgfpathlineto{\pgfqpoint{2.118005in}{0.965461in}}%
\pgfpathlineto{\pgfqpoint{2.119862in}{0.966611in}}%
\pgfpathlineto{\pgfqpoint{2.120791in}{0.966611in}}%
\pgfpathlineto{\pgfqpoint{2.122648in}{0.967782in}}%
\pgfpathlineto{\pgfqpoint{2.123577in}{0.967782in}}%
\pgfpathlineto{\pgfqpoint{2.125434in}{0.968974in}}%
\pgfpathlineto{\pgfqpoint{2.126363in}{0.968974in}}%
\pgfpathlineto{\pgfqpoint{2.128220in}{0.970182in}}%
\pgfpathlineto{\pgfqpoint{2.130077in}{0.970788in}}%
\pgfpathlineto{\pgfqpoint{2.136577in}{0.973827in}}%
\pgfpathlineto{\pgfqpoint{2.137506in}{0.973827in}}%
\pgfpathlineto{\pgfqpoint{2.139363in}{0.975066in}}%
\pgfpathlineto{\pgfqpoint{2.141220in}{0.975676in}}%
\pgfpathlineto{\pgfqpoint{2.147720in}{0.978746in}}%
\pgfpathlineto{\pgfqpoint{2.149577in}{0.979370in}}%
\pgfpathlineto{\pgfqpoint{2.154220in}{0.981237in}}%
\pgfpathlineto{\pgfqpoint{2.155149in}{0.981237in}}%
\pgfpathlineto{\pgfqpoint{2.157006in}{0.982501in}}%
\pgfpathlineto{\pgfqpoint{2.157935in}{0.982501in}}%
\pgfpathlineto{\pgfqpoint{2.159792in}{0.983740in}}%
\pgfpathlineto{\pgfqpoint{2.160721in}{0.983740in}}%
\pgfpathlineto{\pgfqpoint{2.162578in}{0.984978in}}%
\pgfpathlineto{\pgfqpoint{2.163506in}{0.984978in}}%
\pgfpathlineto{\pgfqpoint{2.165364in}{0.986219in}}%
\pgfpathlineto{\pgfqpoint{2.166292in}{0.986219in}}%
\pgfpathlineto{\pgfqpoint{2.168149in}{0.987448in}}%
\pgfpathlineto{\pgfqpoint{2.169078in}{0.987448in}}%
\pgfpathlineto{\pgfqpoint{2.170935in}{0.988689in}}%
\pgfpathlineto{\pgfqpoint{2.171864in}{0.988689in}}%
\pgfpathlineto{\pgfqpoint{2.173721in}{0.989921in}}%
\pgfpathlineto{\pgfqpoint{2.174650in}{0.989921in}}%
\pgfpathlineto{\pgfqpoint{2.176507in}{0.991138in}}%
\pgfpathlineto{\pgfqpoint{2.178364in}{0.991750in}}%
\pgfpathlineto{\pgfqpoint{2.184864in}{0.994772in}}%
\pgfpathlineto{\pgfqpoint{2.186721in}{0.995364in}}%
\pgfpathlineto{\pgfqpoint{2.191364in}{0.997144in}}%
\pgfpathlineto{\pgfqpoint{2.192293in}{0.997144in}}%
\pgfpathlineto{\pgfqpoint{2.194150in}{0.998331in}}%
\pgfpathlineto{\pgfqpoint{2.201579in}{1.001715in}}%
\pgfpathlineto{\pgfqpoint{2.206222in}{1.002805in}}%
\pgfpathlineto{\pgfqpoint{2.208079in}{1.003884in}}%
\pgfpathlineto{\pgfqpoint{2.210865in}{1.004932in}}%
\pgfpathlineto{\pgfqpoint{2.216437in}{1.006919in}}%
\pgfpathlineto{\pgfqpoint{2.220151in}{1.007887in}}%
\pgfpathlineto{\pgfqpoint{2.222008in}{1.008808in}}%
\pgfpathlineto{\pgfqpoint{2.225723in}{1.009683in}}%
\pgfpathlineto{\pgfqpoint{2.227580in}{1.010527in}}%
\pgfpathlineto{\pgfqpoint{2.231294in}{1.011343in}}%
\pgfpathlineto{\pgfqpoint{2.238723in}{1.013487in}}%
\pgfpathlineto{\pgfqpoint{2.243366in}{1.014112in}}%
\pgfpathlineto{\pgfqpoint{2.247080in}{1.015243in}}%
\pgfpathlineto{\pgfqpoint{2.253580in}{1.016223in}}%
\pgfpathlineto{\pgfqpoint{2.256366in}{1.016666in}}%
\pgfpathlineto{\pgfqpoint{2.262866in}{1.017478in}}%
\pgfpathlineto{\pgfqpoint{2.267509in}{1.018245in}}%
\pgfpathlineto{\pgfqpoint{2.274010in}{1.019015in}}%
\pgfpathlineto{\pgfqpoint{2.277724in}{1.019857in}}%
\pgfpathlineto{\pgfqpoint{2.286081in}{1.020806in}}%
\pgfpathlineto{\pgfqpoint{2.289796in}{1.021921in}}%
\pgfpathlineto{\pgfqpoint{2.294439in}{1.022529in}}%
\pgfpathlineto{\pgfqpoint{2.298153in}{1.023885in}}%
\pgfpathlineto{\pgfqpoint{2.302796in}{1.024631in}}%
\pgfpathlineto{\pgfqpoint{2.304653in}{1.025415in}}%
\pgfpathlineto{\pgfqpoint{2.308368in}{1.026261in}}%
\pgfpathlineto{\pgfqpoint{2.310225in}{1.027124in}}%
\pgfpathlineto{\pgfqpoint{2.313939in}{1.028041in}}%
\pgfpathlineto{\pgfqpoint{2.315797in}{1.028962in}}%
\pgfpathlineto{\pgfqpoint{2.319511in}{1.029937in}}%
\pgfpathlineto{\pgfqpoint{2.321368in}{1.030942in}}%
\pgfpathlineto{\pgfqpoint{2.325083in}{1.031990in}}%
\pgfpathlineto{\pgfqpoint{2.332511in}{1.035220in}}%
\pgfpathlineto{\pgfqpoint{2.334369in}{1.035220in}}%
\pgfpathlineto{\pgfqpoint{2.336226in}{1.036335in}}%
\pgfpathlineto{\pgfqpoint{2.337154in}{1.036335in}}%
\pgfpathlineto{\pgfqpoint{2.339012in}{1.037475in}}%
\pgfpathlineto{\pgfqpoint{2.339940in}{1.037475in}}%
\pgfpathlineto{\pgfqpoint{2.342726in}{1.039213in}}%
\pgfpathlineto{\pgfqpoint{2.347369in}{1.040967in}}%
\pgfpathlineto{\pgfqpoint{2.348298in}{1.040967in}}%
\pgfpathlineto{\pgfqpoint{2.350155in}{1.042175in}}%
\pgfpathlineto{\pgfqpoint{2.351083in}{1.042175in}}%
\pgfpathlineto{\pgfqpoint{2.352940in}{1.043339in}}%
\pgfpathlineto{\pgfqpoint{2.353869in}{1.043339in}}%
\pgfpathlineto{\pgfqpoint{2.355726in}{1.044570in}}%
\pgfpathlineto{\pgfqpoint{2.356655in}{1.044570in}}%
\pgfpathlineto{\pgfqpoint{2.358512in}{1.045783in}}%
\pgfpathlineto{\pgfqpoint{2.360369in}{1.046398in}}%
\pgfpathlineto{\pgfqpoint{2.366869in}{1.049459in}}%
\pgfpathlineto{\pgfqpoint{2.367798in}{1.049459in}}%
\pgfpathlineto{\pgfqpoint{2.369655in}{1.050707in}}%
\pgfpathlineto{\pgfqpoint{2.370584in}{1.050707in}}%
\pgfpathlineto{\pgfqpoint{2.372441in}{1.051955in}}%
\pgfpathlineto{\pgfqpoint{2.374298in}{1.052577in}}%
\pgfpathlineto{\pgfqpoint{2.380798in}{1.055731in}}%
\pgfpathlineto{\pgfqpoint{2.382656in}{1.056370in}}%
\pgfpathlineto{\pgfqpoint{2.387299in}{1.058252in}}%
\pgfpathlineto{\pgfqpoint{2.388227in}{1.058252in}}%
\pgfpathlineto{\pgfqpoint{2.390084in}{1.059537in}}%
\pgfpathlineto{\pgfqpoint{2.391013in}{1.059537in}}%
\pgfpathlineto{\pgfqpoint{2.392870in}{1.060792in}}%
\pgfpathlineto{\pgfqpoint{2.393799in}{1.060792in}}%
\pgfpathlineto{\pgfqpoint{2.396585in}{1.063362in}}%
\pgfpathlineto{\pgfqpoint{2.399370in}{1.063362in}}%
\pgfpathlineto{\pgfqpoint{2.401228in}{1.064650in}}%
\pgfpathlineto{\pgfqpoint{2.402156in}{1.064650in}}%
\pgfpathlineto{\pgfqpoint{2.404013in}{1.065930in}}%
\pgfpathlineto{\pgfqpoint{2.404942in}{1.065930in}}%
\pgfpathlineto{\pgfqpoint{2.406799in}{1.067208in}}%
\pgfpathlineto{\pgfqpoint{2.407728in}{1.067208in}}%
\pgfpathlineto{\pgfqpoint{2.409585in}{1.068484in}}%
\pgfpathlineto{\pgfqpoint{2.410514in}{1.068484in}}%
\pgfpathlineto{\pgfqpoint{2.412371in}{1.069769in}}%
\pgfpathlineto{\pgfqpoint{2.414228in}{1.069769in}}%
\pgfpathlineto{\pgfqpoint{2.417014in}{1.072272in}}%
\pgfpathlineto{\pgfqpoint{2.418871in}{1.072272in}}%
\pgfpathlineto{\pgfqpoint{2.420728in}{1.073564in}}%
\pgfpathlineto{\pgfqpoint{2.421657in}{1.073564in}}%
\pgfpathlineto{\pgfqpoint{2.423514in}{1.074840in}}%
\pgfpathlineto{\pgfqpoint{2.424443in}{1.074840in}}%
\pgfpathlineto{\pgfqpoint{2.426300in}{1.076113in}}%
\pgfpathlineto{\pgfqpoint{2.428157in}{1.076113in}}%
\pgfpathlineto{\pgfqpoint{2.430014in}{1.077380in}}%
\pgfpathlineto{\pgfqpoint{2.430943in}{1.077380in}}%
\pgfpathlineto{\pgfqpoint{2.432800in}{1.078665in}}%
\pgfpathlineto{\pgfqpoint{2.433729in}{1.078665in}}%
\pgfpathlineto{\pgfqpoint{2.435586in}{1.079943in}}%
\pgfpathlineto{\pgfqpoint{2.441157in}{1.082481in}}%
\pgfpathlineto{\pgfqpoint{2.442086in}{1.082481in}}%
\pgfpathlineto{\pgfqpoint{2.443943in}{1.083754in}}%
\pgfpathlineto{\pgfqpoint{2.444872in}{1.083754in}}%
\pgfpathlineto{\pgfqpoint{2.446729in}{1.084979in}}%
\pgfpathlineto{\pgfqpoint{2.447658in}{1.084979in}}%
\pgfpathlineto{\pgfqpoint{2.449515in}{1.086245in}}%
\pgfpathlineto{\pgfqpoint{2.455086in}{1.088729in}}%
\pgfpathlineto{\pgfqpoint{2.456944in}{1.089344in}}%
\pgfpathlineto{\pgfqpoint{2.463444in}{1.092363in}}%
\pgfpathlineto{\pgfqpoint{2.464372in}{1.092363in}}%
\pgfpathlineto{\pgfqpoint{2.466229in}{1.093580in}}%
\pgfpathlineto{\pgfqpoint{2.468087in}{1.093580in}}%
\pgfpathlineto{\pgfqpoint{2.470872in}{1.095332in}}%
\pgfpathlineto{\pgfqpoint{2.477373in}{1.098238in}}%
\pgfpathlineto{\pgfqpoint{2.479230in}{1.098238in}}%
\pgfpathlineto{\pgfqpoint{2.481087in}{1.099365in}}%
\pgfpathlineto{\pgfqpoint{2.482016in}{1.099365in}}%
\pgfpathlineto{\pgfqpoint{2.483873in}{1.100482in}}%
\pgfpathlineto{\pgfqpoint{2.486659in}{1.101569in}}%
\pgfpathlineto{\pgfqpoint{2.492230in}{1.103689in}}%
\pgfpathlineto{\pgfqpoint{2.495945in}{1.104697in}}%
\pgfpathlineto{\pgfqpoint{2.497802in}{1.105695in}}%
\pgfpathlineto{\pgfqpoint{2.501516in}{1.106667in}}%
\pgfpathlineto{\pgfqpoint{2.503373in}{1.107607in}}%
\pgfpathlineto{\pgfqpoint{2.507088in}{1.108510in}}%
\pgfpathlineto{\pgfqpoint{2.514517in}{1.111045in}}%
\pgfpathlineto{\pgfqpoint{2.519160in}{1.111787in}}%
\pgfpathlineto{\pgfqpoint{2.521017in}{1.112505in}}%
\pgfpathlineto{\pgfqpoint{2.524731in}{1.113485in}}%
\pgfpathlineto{\pgfqpoint{2.532160in}{1.114922in}}%
\pgfpathlineto{\pgfqpoint{2.538660in}{1.115855in}}%
\pgfpathlineto{\pgfqpoint{2.542375in}{1.116615in}}%
\pgfpathlineto{\pgfqpoint{2.555375in}{1.117714in}}%
\pgfpathlineto{\pgfqpoint{2.562804in}{1.118294in}}%
\pgfpathlineto{\pgfqpoint{2.578590in}{1.119272in}}%
\pgfpathlineto{\pgfqpoint{2.585090in}{1.119968in}}%
\pgfpathlineto{\pgfqpoint{2.617591in}{1.126451in}}%
\pgfpathlineto{\pgfqpoint{2.625948in}{1.129033in}}%
\pgfpathlineto{\pgfqpoint{2.629663in}{1.129972in}}%
\pgfpathlineto{\pgfqpoint{2.631520in}{1.130912in}}%
\pgfpathlineto{\pgfqpoint{2.635234in}{1.131901in}}%
\pgfpathlineto{\pgfqpoint{2.642663in}{1.135003in}}%
\pgfpathlineto{\pgfqpoint{2.647306in}{1.136090in}}%
\pgfpathlineto{\pgfqpoint{2.650092in}{1.137729in}}%
\pgfpathlineto{\pgfqpoint{2.656592in}{1.140559in}}%
\pgfpathlineto{\pgfqpoint{2.657521in}{1.140559in}}%
\pgfpathlineto{\pgfqpoint{2.659378in}{1.141702in}}%
\pgfpathlineto{\pgfqpoint{2.661235in}{1.141702in}}%
\pgfpathlineto{\pgfqpoint{2.663092in}{1.142863in}}%
\pgfpathlineto{\pgfqpoint{2.664021in}{1.142863in}}%
\pgfpathlineto{\pgfqpoint{2.665878in}{1.144044in}}%
\pgfpathlineto{\pgfqpoint{2.666807in}{1.144044in}}%
\pgfpathlineto{\pgfqpoint{2.668664in}{1.145196in}}%
\pgfpathlineto{\pgfqpoint{2.669593in}{1.145196in}}%
\pgfpathlineto{\pgfqpoint{2.671450in}{1.146402in}}%
\pgfpathlineto{\pgfqpoint{2.677021in}{1.148797in}}%
\pgfpathlineto{\pgfqpoint{2.677950in}{1.148797in}}%
\pgfpathlineto{\pgfqpoint{2.679807in}{1.149996in}}%
\pgfpathlineto{\pgfqpoint{2.680736in}{1.149996in}}%
\pgfpathlineto{\pgfqpoint{2.682593in}{1.151223in}}%
\pgfpathlineto{\pgfqpoint{2.683521in}{1.151223in}}%
\pgfpathlineto{\pgfqpoint{2.685379in}{1.152436in}}%
\pgfpathlineto{\pgfqpoint{2.686307in}{1.152436in}}%
\pgfpathlineto{\pgfqpoint{2.688164in}{1.153686in}}%
\pgfpathlineto{\pgfqpoint{2.690022in}{1.154304in}}%
\pgfpathlineto{\pgfqpoint{2.696522in}{1.157404in}}%
\pgfpathlineto{\pgfqpoint{2.697450in}{1.157404in}}%
\pgfpathlineto{\pgfqpoint{2.699308in}{1.158640in}}%
\pgfpathlineto{\pgfqpoint{2.700236in}{1.158640in}}%
\pgfpathlineto{\pgfqpoint{2.702093in}{1.159892in}}%
\pgfpathlineto{\pgfqpoint{2.703022in}{1.159892in}}%
\pgfpathlineto{\pgfqpoint{2.704879in}{1.161133in}}%
\pgfpathlineto{\pgfqpoint{2.706736in}{1.161133in}}%
\pgfpathlineto{\pgfqpoint{2.708594in}{1.162402in}}%
\pgfpathlineto{\pgfqpoint{2.709522in}{1.162402in}}%
\pgfpathlineto{\pgfqpoint{2.711379in}{1.163640in}}%
\pgfpathlineto{\pgfqpoint{2.712308in}{1.163640in}}%
\pgfpathlineto{\pgfqpoint{2.714165in}{1.164918in}}%
\pgfpathlineto{\pgfqpoint{2.715094in}{1.164918in}}%
\pgfpathlineto{\pgfqpoint{2.717880in}{1.166811in}}%
\pgfpathlineto{\pgfqpoint{2.722523in}{1.168711in}}%
\pgfpathlineto{\pgfqpoint{2.723451in}{1.168711in}}%
\pgfpathlineto{\pgfqpoint{2.725308in}{1.169991in}}%
\pgfpathlineto{\pgfqpoint{2.726237in}{1.169991in}}%
\pgfpathlineto{\pgfqpoint{2.728094in}{1.171253in}}%
\pgfpathlineto{\pgfqpoint{2.729023in}{1.171253in}}%
\pgfpathlineto{\pgfqpoint{2.730880in}{1.172531in}}%
\pgfpathlineto{\pgfqpoint{2.731809in}{1.172531in}}%
\pgfpathlineto{\pgfqpoint{2.733666in}{1.173795in}}%
\pgfpathlineto{\pgfqpoint{2.735523in}{1.174425in}}%
\pgfpathlineto{\pgfqpoint{2.742023in}{1.177632in}}%
\pgfpathlineto{\pgfqpoint{2.743880in}{1.178283in}}%
\pgfpathlineto{\pgfqpoint{2.748523in}{1.180202in}}%
\pgfpathlineto{\pgfqpoint{2.749452in}{1.180202in}}%
\pgfpathlineto{\pgfqpoint{2.751309in}{1.181506in}}%
\pgfpathlineto{\pgfqpoint{2.752238in}{1.181506in}}%
\pgfpathlineto{\pgfqpoint{2.754095in}{1.182787in}}%
\pgfpathlineto{\pgfqpoint{2.755024in}{1.182787in}}%
\pgfpathlineto{\pgfqpoint{2.756881in}{1.184102in}}%
\pgfpathlineto{\pgfqpoint{2.757809in}{1.184102in}}%
\pgfpathlineto{\pgfqpoint{2.759667in}{1.185401in}}%
\pgfpathlineto{\pgfqpoint{2.760595in}{1.185401in}}%
\pgfpathlineto{\pgfqpoint{2.762452in}{1.186710in}}%
\pgfpathlineto{\pgfqpoint{2.763381in}{1.186710in}}%
\pgfpathlineto{\pgfqpoint{2.765238in}{1.188004in}}%
\pgfpathlineto{\pgfqpoint{2.766167in}{1.188004in}}%
\pgfpathlineto{\pgfqpoint{2.768024in}{1.189310in}}%
\pgfpathlineto{\pgfqpoint{2.768953in}{1.189310in}}%
\pgfpathlineto{\pgfqpoint{2.770810in}{1.190633in}}%
\pgfpathlineto{\pgfqpoint{2.771738in}{1.190633in}}%
\pgfpathlineto{\pgfqpoint{2.773596in}{1.191923in}}%
\pgfpathlineto{\pgfqpoint{2.775453in}{1.191923in}}%
\pgfpathlineto{\pgfqpoint{2.777310in}{1.193250in}}%
\pgfpathlineto{\pgfqpoint{2.784739in}{1.197189in}}%
\pgfpathlineto{\pgfqpoint{2.785667in}{1.197189in}}%
\pgfpathlineto{\pgfqpoint{2.787524in}{1.198505in}}%
\pgfpathlineto{\pgfqpoint{2.789382in}{1.199175in}}%
\pgfpathlineto{\pgfqpoint{2.793096in}{1.201152in}}%
\pgfpathlineto{\pgfqpoint{2.794953in}{1.201152in}}%
\pgfpathlineto{\pgfqpoint{2.797739in}{1.203804in}}%
\pgfpathlineto{\pgfqpoint{2.800525in}{1.204468in}}%
\pgfpathlineto{\pgfqpoint{2.805168in}{1.206446in}}%
\pgfpathlineto{\pgfqpoint{2.806096in}{1.206446in}}%
\pgfpathlineto{\pgfqpoint{2.807954in}{1.207759in}}%
\pgfpathlineto{\pgfqpoint{2.808882in}{1.207759in}}%
\pgfpathlineto{\pgfqpoint{2.810739in}{1.209094in}}%
\pgfpathlineto{\pgfqpoint{2.811668in}{1.209094in}}%
\pgfpathlineto{\pgfqpoint{2.813525in}{1.210402in}}%
\pgfpathlineto{\pgfqpoint{2.814454in}{1.210402in}}%
\pgfpathlineto{\pgfqpoint{2.816311in}{1.211736in}}%
\pgfpathlineto{\pgfqpoint{2.818168in}{1.212392in}}%
\pgfpathlineto{\pgfqpoint{2.824668in}{1.215678in}}%
\pgfpathlineto{\pgfqpoint{2.825597in}{1.215678in}}%
\pgfpathlineto{\pgfqpoint{2.827454in}{1.216977in}}%
\pgfpathlineto{\pgfqpoint{2.829311in}{1.216977in}}%
\pgfpathlineto{\pgfqpoint{2.831169in}{1.218290in}}%
\pgfpathlineto{\pgfqpoint{2.832097in}{1.218290in}}%
\pgfpathlineto{\pgfqpoint{2.833954in}{1.219610in}}%
\pgfpathlineto{\pgfqpoint{2.840455in}{1.223512in}}%
\pgfpathlineto{\pgfqpoint{2.843240in}{1.223512in}}%
\pgfpathlineto{\pgfqpoint{2.845098in}{1.224797in}}%
\pgfpathlineto{\pgfqpoint{2.846026in}{1.224797in}}%
\pgfpathlineto{\pgfqpoint{2.847883in}{1.226085in}}%
\pgfpathlineto{\pgfqpoint{2.848812in}{1.226085in}}%
\pgfpathlineto{\pgfqpoint{2.850669in}{1.227368in}}%
\pgfpathlineto{\pgfqpoint{2.856241in}{1.229891in}}%
\pgfpathlineto{\pgfqpoint{2.857169in}{1.229891in}}%
\pgfpathlineto{\pgfqpoint{2.859027in}{1.231144in}}%
\pgfpathlineto{\pgfqpoint{2.859955in}{1.231144in}}%
\pgfpathlineto{\pgfqpoint{2.861812in}{1.232382in}}%
\pgfpathlineto{\pgfqpoint{2.862741in}{1.232382in}}%
\pgfpathlineto{\pgfqpoint{2.864598in}{1.233614in}}%
\pgfpathlineto{\pgfqpoint{2.865527in}{1.233614in}}%
\pgfpathlineto{\pgfqpoint{2.867384in}{1.234848in}}%
\pgfpathlineto{\pgfqpoint{2.869241in}{1.235457in}}%
\pgfpathlineto{\pgfqpoint{2.875741in}{1.238491in}}%
\pgfpathlineto{\pgfqpoint{2.876670in}{1.238491in}}%
\pgfpathlineto{\pgfqpoint{2.878527in}{1.239680in}}%
\pgfpathlineto{\pgfqpoint{2.880384in}{1.240276in}}%
\pgfpathlineto{\pgfqpoint{2.885027in}{1.242027in}}%
\pgfpathlineto{\pgfqpoint{2.885956in}{1.242027in}}%
\pgfpathlineto{\pgfqpoint{2.888742in}{1.243758in}}%
\pgfpathlineto{\pgfqpoint{2.893385in}{1.245476in}}%
\pgfpathlineto{\pgfqpoint{2.897099in}{1.246586in}}%
\pgfpathlineto{\pgfqpoint{2.898956in}{1.247694in}}%
\pgfpathlineto{\pgfqpoint{2.902671in}{1.248770in}}%
\pgfpathlineto{\pgfqpoint{2.904528in}{1.249847in}}%
\pgfpathlineto{\pgfqpoint{2.909171in}{1.250897in}}%
\pgfpathlineto{\pgfqpoint{2.911957in}{1.252956in}}%
\pgfpathlineto{\pgfqpoint{2.916600in}{1.253950in}}%
\pgfpathlineto{\pgfqpoint{2.918457in}{1.254927in}}%
\pgfpathlineto{\pgfqpoint{2.922171in}{1.255872in}}%
\pgfpathlineto{\pgfqpoint{2.929600in}{1.258582in}}%
\pgfpathlineto{\pgfqpoint{2.933314in}{1.259414in}}%
\pgfpathlineto{\pgfqpoint{2.938886in}{1.261028in}}%
\pgfpathlineto{\pgfqpoint{2.942600in}{1.261772in}}%
\pgfpathlineto{\pgfqpoint{2.944458in}{1.262517in}}%
\pgfpathlineto{\pgfqpoint{2.948172in}{1.263191in}}%
\pgfpathlineto{\pgfqpoint{2.951886in}{1.264448in}}%
\pgfpathlineto{\pgfqpoint{2.959315in}{1.265530in}}%
\pgfpathlineto{\pgfqpoint{2.963030in}{1.266241in}}%
\pgfpathlineto{\pgfqpoint{2.972316in}{1.267515in}}%
\pgfpathlineto{\pgfqpoint{2.990888in}{1.268604in}}%
\pgfpathlineto{\pgfqpoint{3.006674in}{1.268807in}}%
\pgfpathlineto{\pgfqpoint{3.047532in}{1.269782in}}%
\pgfpathlineto{\pgfqpoint{3.054961in}{1.270633in}}%
\pgfpathlineto{\pgfqpoint{3.061461in}{1.271401in}}%
\pgfpathlineto{\pgfqpoint{3.065175in}{1.272333in}}%
\pgfpathlineto{\pgfqpoint{3.073533in}{1.273432in}}%
\pgfpathlineto{\pgfqpoint{3.077247in}{1.274675in}}%
\pgfpathlineto{\pgfqpoint{3.081890in}{1.275342in}}%
\pgfpathlineto{\pgfqpoint{3.085605in}{1.276856in}}%
\pgfpathlineto{\pgfqpoint{3.091176in}{1.277607in}}%
\pgfpathlineto{\pgfqpoint{3.093962in}{1.278816in}}%
\pgfpathlineto{\pgfqpoint{3.103248in}{1.281924in}}%
\pgfpathlineto{\pgfqpoint{3.106962in}{1.282864in}}%
\pgfpathlineto{\pgfqpoint{3.114391in}{1.285803in}}%
\pgfpathlineto{\pgfqpoint{3.119034in}{1.286813in}}%
\pgfpathlineto{\pgfqpoint{3.120891in}{1.287848in}}%
\pgfpathlineto{\pgfqpoint{3.124606in}{1.288891in}}%
\pgfpathlineto{\pgfqpoint{3.126463in}{1.289964in}}%
\pgfpathlineto{\pgfqpoint{3.130177in}{1.291030in}}%
\pgfpathlineto{\pgfqpoint{3.132034in}{1.292121in}}%
\pgfpathlineto{\pgfqpoint{3.135749in}{1.293229in}}%
\pgfpathlineto{\pgfqpoint{3.137606in}{1.294318in}}%
\pgfpathlineto{\pgfqpoint{3.138535in}{1.294318in}}%
\pgfpathlineto{\pgfqpoint{3.140392in}{1.295457in}}%
\pgfpathlineto{\pgfqpoint{3.144106in}{1.296567in}}%
\pgfpathlineto{\pgfqpoint{3.145963in}{1.297712in}}%
\pgfpathlineto{\pgfqpoint{3.147821in}{1.298283in}}%
\pgfpathlineto{\pgfqpoint{3.154321in}{1.301162in}}%
\pgfpathlineto{\pgfqpoint{3.156178in}{1.301162in}}%
\pgfpathlineto{\pgfqpoint{3.158035in}{1.302332in}}%
\pgfpathlineto{\pgfqpoint{3.158964in}{1.302332in}}%
\pgfpathlineto{\pgfqpoint{3.161750in}{1.304076in}}%
\pgfpathlineto{\pgfqpoint{3.168250in}{1.307018in}}%
\pgfpathlineto{\pgfqpoint{3.170107in}{1.307018in}}%
\pgfpathlineto{\pgfqpoint{3.171964in}{1.308210in}}%
\pgfpathlineto{\pgfqpoint{3.172893in}{1.308210in}}%
\pgfpathlineto{\pgfqpoint{3.174750in}{1.309411in}}%
\pgfpathlineto{\pgfqpoint{3.175679in}{1.309411in}}%
\pgfpathlineto{\pgfqpoint{3.177536in}{1.310617in}}%
\pgfpathlineto{\pgfqpoint{3.178464in}{1.310617in}}%
\pgfpathlineto{\pgfqpoint{3.180322in}{1.311802in}}%
\pgfpathlineto{\pgfqpoint{3.182179in}{1.312408in}}%
\pgfpathlineto{\pgfqpoint{3.188679in}{1.315410in}}%
\pgfpathlineto{\pgfqpoint{3.189608in}{1.315410in}}%
\pgfpathlineto{\pgfqpoint{3.191465in}{1.316623in}}%
\pgfpathlineto{\pgfqpoint{3.193322in}{1.316623in}}%
\pgfpathlineto{\pgfqpoint{3.195179in}{1.317847in}}%
\pgfpathlineto{\pgfqpoint{3.196108in}{1.317847in}}%
\pgfpathlineto{\pgfqpoint{3.198894in}{1.319674in}}%
\pgfpathlineto{\pgfqpoint{3.205394in}{1.322720in}}%
\pgfpathlineto{\pgfqpoint{3.207251in}{1.322720in}}%
\pgfpathlineto{\pgfqpoint{3.209108in}{1.323961in}}%
\pgfpathlineto{\pgfqpoint{3.210037in}{1.323961in}}%
\pgfpathlineto{\pgfqpoint{3.212822in}{1.325783in}}%
\pgfpathlineto{\pgfqpoint{3.217465in}{1.327632in}}%
\pgfpathlineto{\pgfqpoint{3.218394in}{1.327632in}}%
\pgfpathlineto{\pgfqpoint{3.220251in}{1.328858in}}%
\pgfpathlineto{\pgfqpoint{3.222108in}{1.329479in}}%
\pgfpathlineto{\pgfqpoint{3.228609in}{1.332555in}}%
\pgfpathlineto{\pgfqpoint{3.229537in}{1.332555in}}%
\pgfpathlineto{\pgfqpoint{3.231394in}{1.333794in}}%
\pgfpathlineto{\pgfqpoint{3.233252in}{1.333794in}}%
\pgfpathlineto{\pgfqpoint{3.235109in}{1.335028in}}%
\pgfpathlineto{\pgfqpoint{3.242538in}{1.338708in}}%
\pgfpathlineto{\pgfqpoint{3.243466in}{1.338708in}}%
\pgfpathlineto{\pgfqpoint{3.245323in}{1.339965in}}%
\pgfpathlineto{\pgfqpoint{3.246252in}{1.339965in}}%
\pgfpathlineto{\pgfqpoint{3.248109in}{1.341215in}}%
\pgfpathlineto{\pgfqpoint{3.249966in}{1.341215in}}%
\pgfpathlineto{\pgfqpoint{3.251824in}{1.342459in}}%
\pgfpathlineto{\pgfqpoint{3.252752in}{1.342459in}}%
\pgfpathlineto{\pgfqpoint{3.254609in}{1.343709in}}%
\pgfpathlineto{\pgfqpoint{3.255538in}{1.343709in}}%
\pgfpathlineto{\pgfqpoint{3.257395in}{1.344964in}}%
\pgfpathlineto{\pgfqpoint{3.258324in}{1.344964in}}%
\pgfpathlineto{\pgfqpoint{3.260181in}{1.346216in}}%
\pgfpathlineto{\pgfqpoint{3.262038in}{1.346832in}}%
\pgfpathlineto{\pgfqpoint{3.268538in}{1.349966in}}%
\pgfpathlineto{\pgfqpoint{3.269467in}{1.349966in}}%
\pgfpathlineto{\pgfqpoint{3.271324in}{1.351207in}}%
\pgfpathlineto{\pgfqpoint{3.272253in}{1.351207in}}%
\pgfpathlineto{\pgfqpoint{3.274110in}{1.352474in}}%
\pgfpathlineto{\pgfqpoint{3.275967in}{1.353104in}}%
\pgfpathlineto{\pgfqpoint{3.282467in}{1.356269in}}%
\pgfpathlineto{\pgfqpoint{3.283396in}{1.356269in}}%
\pgfpathlineto{\pgfqpoint{3.285253in}{1.357549in}}%
\pgfpathlineto{\pgfqpoint{3.287110in}{1.357549in}}%
\pgfpathlineto{\pgfqpoint{3.288968in}{1.358822in}}%
\pgfpathlineto{\pgfqpoint{3.289896in}{1.358822in}}%
\pgfpathlineto{\pgfqpoint{3.292682in}{1.360728in}}%
\pgfpathlineto{\pgfqpoint{3.299182in}{1.363912in}}%
\pgfpathlineto{\pgfqpoint{3.301039in}{1.363912in}}%
\pgfpathlineto{\pgfqpoint{3.302897in}{1.365204in}}%
\pgfpathlineto{\pgfqpoint{3.303825in}{1.365204in}}%
\pgfpathlineto{\pgfqpoint{3.305682in}{1.366505in}}%
\pgfpathlineto{\pgfqpoint{3.306611in}{1.366505in}}%
\pgfpathlineto{\pgfqpoint{3.308468in}{1.367800in}}%
\pgfpathlineto{\pgfqpoint{3.309397in}{1.367800in}}%
\pgfpathlineto{\pgfqpoint{3.311254in}{1.369113in}}%
\pgfpathlineto{\pgfqpoint{3.316825in}{1.371707in}}%
\pgfpathlineto{\pgfqpoint{3.317754in}{1.371707in}}%
\pgfpathlineto{\pgfqpoint{3.319611in}{1.373015in}}%
\pgfpathlineto{\pgfqpoint{3.320540in}{1.373015in}}%
\pgfpathlineto{\pgfqpoint{3.322397in}{1.374309in}}%
\pgfpathlineto{\pgfqpoint{3.323326in}{1.374309in}}%
\pgfpathlineto{\pgfqpoint{3.325183in}{1.375637in}}%
\pgfpathlineto{\pgfqpoint{3.326111in}{1.375637in}}%
\pgfpathlineto{\pgfqpoint{3.327969in}{1.376950in}}%
\pgfpathlineto{\pgfqpoint{3.329826in}{1.377603in}}%
\pgfpathlineto{\pgfqpoint{3.336326in}{1.380912in}}%
\pgfpathlineto{\pgfqpoint{3.337255in}{1.380912in}}%
\pgfpathlineto{\pgfqpoint{3.339112in}{1.382244in}}%
\pgfpathlineto{\pgfqpoint{3.340040in}{1.382244in}}%
\pgfpathlineto{\pgfqpoint{3.341898in}{1.383585in}}%
\pgfpathlineto{\pgfqpoint{3.343755in}{1.383585in}}%
\pgfpathlineto{\pgfqpoint{3.345612in}{1.384917in}}%
\pgfpathlineto{\pgfqpoint{3.346541in}{1.384917in}}%
\pgfpathlineto{\pgfqpoint{3.348398in}{1.386270in}}%
\pgfpathlineto{\pgfqpoint{3.349326in}{1.386270in}}%
\pgfpathlineto{\pgfqpoint{3.351184in}{1.387599in}}%
\pgfpathlineto{\pgfqpoint{3.352112in}{1.387599in}}%
\pgfpathlineto{\pgfqpoint{3.353969in}{1.388950in}}%
\pgfpathlineto{\pgfqpoint{3.354898in}{1.388950in}}%
\pgfpathlineto{\pgfqpoint{3.356755in}{1.390303in}}%
\pgfpathlineto{\pgfqpoint{3.357684in}{1.390303in}}%
\pgfpathlineto{\pgfqpoint{3.359541in}{1.391651in}}%
\pgfpathlineto{\pgfqpoint{3.360470in}{1.391651in}}%
\pgfpathlineto{\pgfqpoint{3.362327in}{1.393024in}}%
\pgfpathlineto{\pgfqpoint{3.363255in}{1.393024in}}%
\pgfpathlineto{\pgfqpoint{3.365113in}{1.394377in}}%
\pgfpathlineto{\pgfqpoint{3.366041in}{1.394377in}}%
\pgfpathlineto{\pgfqpoint{3.367898in}{1.395737in}}%
\pgfpathlineto{\pgfqpoint{3.369756in}{1.396437in}}%
\pgfpathlineto{\pgfqpoint{3.373470in}{1.398482in}}%
\pgfpathlineto{\pgfqpoint{3.374399in}{1.398482in}}%
\pgfpathlineto{\pgfqpoint{3.376256in}{1.399875in}}%
\pgfpathlineto{\pgfqpoint{3.377184in}{1.399875in}}%
\pgfpathlineto{\pgfqpoint{3.379042in}{1.401258in}}%
\pgfpathlineto{\pgfqpoint{3.380899in}{1.401258in}}%
\pgfpathlineto{\pgfqpoint{3.382756in}{1.402636in}}%
\pgfpathlineto{\pgfqpoint{3.383685in}{1.402636in}}%
\pgfpathlineto{\pgfqpoint{3.385542in}{1.404015in}}%
\pgfpathlineto{\pgfqpoint{3.390185in}{1.406760in}}%
\pgfpathlineto{\pgfqpoint{3.392042in}{1.407475in}}%
\pgfpathlineto{\pgfqpoint{3.393899in}{1.408189in}}%
\pgfpathlineto{\pgfqpoint{3.394828in}{1.408189in}}%
\pgfpathlineto{\pgfqpoint{3.396685in}{1.409566in}}%
\pgfpathlineto{\pgfqpoint{3.397614in}{1.409566in}}%
\pgfpathlineto{\pgfqpoint{3.399471in}{1.410953in}}%
\pgfpathlineto{\pgfqpoint{3.400399in}{1.410953in}}%
\pgfpathlineto{\pgfqpoint{3.402257in}{1.412348in}}%
\pgfpathlineto{\pgfqpoint{3.403185in}{1.412348in}}%
\pgfpathlineto{\pgfqpoint{3.405042in}{1.413743in}}%
\pgfpathlineto{\pgfqpoint{3.406900in}{1.414453in}}%
\pgfpathlineto{\pgfqpoint{3.410614in}{1.416535in}}%
\pgfpathlineto{\pgfqpoint{3.411543in}{1.416535in}}%
\pgfpathlineto{\pgfqpoint{3.413400in}{1.417957in}}%
\pgfpathlineto{\pgfqpoint{3.414328in}{1.417957in}}%
\pgfpathlineto{\pgfqpoint{3.416186in}{1.419343in}}%
\pgfpathlineto{\pgfqpoint{3.417114in}{1.419343in}}%
\pgfpathlineto{\pgfqpoint{3.418971in}{1.420756in}}%
\pgfpathlineto{\pgfqpoint{3.420829in}{1.420756in}}%
\pgfpathlineto{\pgfqpoint{3.422686in}{1.422170in}}%
\pgfpathlineto{\pgfqpoint{3.423614in}{1.422170in}}%
\pgfpathlineto{\pgfqpoint{3.425471in}{1.423583in}}%
\pgfpathlineto{\pgfqpoint{3.432900in}{1.427781in}}%
\pgfpathlineto{\pgfqpoint{3.434757in}{1.427781in}}%
\pgfpathlineto{\pgfqpoint{3.436615in}{1.435576in}}%
\pgfpathlineto{\pgfqpoint{3.440329in}{1.436427in}}%
\pgfpathlineto{\pgfqpoint{3.444043in}{1.438709in}}%
\pgfpathlineto{\pgfqpoint{3.444972in}{1.438709in}}%
\pgfpathlineto{\pgfqpoint{3.447758in}{1.441221in}}%
\pgfpathlineto{\pgfqpoint{3.451472in}{1.442268in}}%
\pgfpathlineto{\pgfqpoint{3.453329in}{1.444012in}}%
\pgfpathlineto{\pgfqpoint{3.454258in}{1.444012in}}%
\pgfpathlineto{\pgfqpoint{3.456115in}{1.445332in}}%
\pgfpathlineto{\pgfqpoint{3.462615in}{1.446277in}}%
\pgfpathlineto{\pgfqpoint{3.468187in}{1.447751in}}%
\pgfpathlineto{\pgfqpoint{3.471901in}{1.448233in}}%
\pgfpathlineto{\pgfqpoint{3.474687in}{1.448917in}}%
\pgfpathlineto{\pgfqpoint{3.477473in}{1.448819in}}%
\pgfpathlineto{\pgfqpoint{3.479330in}{1.450216in}}%
\pgfpathlineto{\pgfqpoint{3.480259in}{1.450216in}}%
\pgfpathlineto{\pgfqpoint{3.482116in}{1.451588in}}%
\pgfpathlineto{\pgfqpoint{3.483045in}{1.451588in}}%
\pgfpathlineto{\pgfqpoint{3.484902in}{1.452971in}}%
\pgfpathlineto{\pgfqpoint{3.485830in}{1.452971in}}%
\pgfpathlineto{\pgfqpoint{3.487688in}{1.454319in}}%
\pgfpathlineto{\pgfqpoint{3.489545in}{1.454999in}}%
\pgfpathlineto{\pgfqpoint{3.496045in}{1.458387in}}%
\pgfpathlineto{\pgfqpoint{3.496974in}{1.458387in}}%
\pgfpathlineto{\pgfqpoint{3.498831in}{1.459737in}}%
\pgfpathlineto{\pgfqpoint{3.499759in}{1.459737in}}%
\pgfpathlineto{\pgfqpoint{3.501617in}{1.461078in}}%
\pgfpathlineto{\pgfqpoint{3.503474in}{1.461728in}}%
\pgfpathlineto{\pgfqpoint{3.509974in}{1.465043in}}%
\pgfpathlineto{\pgfqpoint{3.510903in}{1.465043in}}%
\pgfpathlineto{\pgfqpoint{3.512760in}{1.466338in}}%
\pgfpathlineto{\pgfqpoint{3.513688in}{1.466338in}}%
\pgfpathlineto{\pgfqpoint{3.515546in}{1.467651in}}%
\pgfpathlineto{\pgfqpoint{3.517403in}{1.467651in}}%
\pgfpathlineto{\pgfqpoint{3.519260in}{1.468934in}}%
\pgfpathlineto{\pgfqpoint{3.520189in}{1.468934in}}%
\pgfpathlineto{\pgfqpoint{3.522046in}{1.470203in}}%
\pgfpathlineto{\pgfqpoint{3.527617in}{1.472736in}}%
\pgfpathlineto{\pgfqpoint{3.528546in}{1.472736in}}%
\pgfpathlineto{\pgfqpoint{3.530403in}{1.473986in}}%
\pgfpathlineto{\pgfqpoint{3.531332in}{1.473986in}}%
\pgfpathlineto{\pgfqpoint{3.533189in}{1.475243in}}%
\pgfpathlineto{\pgfqpoint{3.534117in}{1.475243in}}%
\pgfpathlineto{\pgfqpoint{3.535975in}{1.476467in}}%
\pgfpathlineto{\pgfqpoint{3.536903in}{1.476467in}}%
\pgfpathlineto{\pgfqpoint{3.538760in}{1.477687in}}%
\pgfpathlineto{\pgfqpoint{3.539689in}{1.477687in}}%
\pgfpathlineto{\pgfqpoint{3.541546in}{1.478895in}}%
\pgfpathlineto{\pgfqpoint{3.542475in}{1.478895in}}%
\pgfpathlineto{\pgfqpoint{3.544332in}{1.480071in}}%
\pgfpathlineto{\pgfqpoint{3.546189in}{1.480668in}}%
\pgfpathlineto{\pgfqpoint{3.552689in}{1.483595in}}%
\pgfpathlineto{\pgfqpoint{3.553618in}{1.483595in}}%
\pgfpathlineto{\pgfqpoint{3.555475in}{1.484747in}}%
\pgfpathlineto{\pgfqpoint{3.557332in}{1.484747in}}%
\pgfpathlineto{\pgfqpoint{3.559190in}{1.485867in}}%
\pgfpathlineto{\pgfqpoint{3.560118in}{1.485867in}}%
\pgfpathlineto{\pgfqpoint{3.561975in}{1.486979in}}%
\pgfpathlineto{\pgfqpoint{3.565690in}{1.488087in}}%
\pgfpathlineto{\pgfqpoint{3.568476in}{1.490254in}}%
\pgfpathlineto{\pgfqpoint{3.574047in}{1.491325in}}%
\pgfpathlineto{\pgfqpoint{3.575904in}{1.492351in}}%
\pgfpathlineto{\pgfqpoint{3.579619in}{1.493396in}}%
\pgfpathlineto{\pgfqpoint{3.581476in}{1.494408in}}%
\pgfpathlineto{\pgfqpoint{3.585190in}{1.495388in}}%
\pgfpathlineto{\pgfqpoint{3.592619in}{1.498296in}}%
\pgfpathlineto{\pgfqpoint{3.596334in}{1.499248in}}%
\pgfpathlineto{\pgfqpoint{3.598191in}{1.500169in}}%
\pgfpathlineto{\pgfqpoint{3.602834in}{1.501062in}}%
\pgfpathlineto{\pgfqpoint{3.604691in}{1.501972in}}%
\pgfpathlineto{\pgfqpoint{3.607477in}{1.502832in}}%
\pgfpathlineto{\pgfqpoint{3.613048in}{1.504507in}}%
\pgfpathlineto{\pgfqpoint{3.616763in}{1.505337in}}%
\pgfpathlineto{\pgfqpoint{3.618620in}{1.506149in}}%
\pgfpathlineto{\pgfqpoint{3.621406in}{1.506944in}}%
\pgfpathlineto{\pgfqpoint{3.629763in}{1.509211in}}%
\pgfpathlineto{\pgfqpoint{3.633478in}{1.509925in}}%
\pgfpathlineto{\pgfqpoint{3.635335in}{1.510623in}}%
\pgfpathlineto{\pgfqpoint{3.639049in}{1.511308in}}%
\pgfpathlineto{\pgfqpoint{3.641835in}{1.511957in}}%
\pgfpathlineto{\pgfqpoint{3.652049in}{1.514368in}}%
\pgfpathlineto{\pgfqpoint{3.658550in}{1.515406in}}%
\pgfpathlineto{\pgfqpoint{3.668764in}{1.517200in}}%
\pgfpathlineto{\pgfqpoint{3.678979in}{1.518196in}}%
\pgfpathlineto{\pgfqpoint{3.684550in}{1.518711in}}%
\pgfpathlineto{\pgfqpoint{3.691979in}{1.519262in}}%
\pgfpathlineto{\pgfqpoint{3.835912in}{1.520241in}}%
\pgfpathlineto{\pgfqpoint{3.843341in}{1.521051in}}%
\pgfpathlineto{\pgfqpoint{3.852627in}{1.522070in}}%
\pgfpathlineto{\pgfqpoint{3.857270in}{1.522924in}}%
\pgfpathlineto{\pgfqpoint{3.863770in}{1.523868in}}%
\pgfpathlineto{\pgfqpoint{3.880485in}{1.527371in}}%
\pgfpathlineto{\pgfqpoint{3.884199in}{1.528387in}}%
\pgfpathlineto{\pgfqpoint{3.891628in}{1.530201in}}%
\pgfpathlineto{\pgfqpoint{3.895342in}{1.530968in}}%
\pgfpathlineto{\pgfqpoint{3.897199in}{1.531761in}}%
\pgfpathlineto{\pgfqpoint{3.900914in}{1.532570in}}%
\pgfpathlineto{\pgfqpoint{3.902771in}{1.533384in}}%
\pgfpathlineto{\pgfqpoint{3.906485in}{1.534243in}}%
\pgfpathlineto{\pgfqpoint{3.913914in}{1.536897in}}%
\pgfpathlineto{\pgfqpoint{3.917629in}{1.537834in}}%
\pgfpathlineto{\pgfqpoint{3.925057in}{1.540701in}}%
\pgfpathlineto{\pgfqpoint{3.929700in}{1.541666in}}%
\pgfpathlineto{\pgfqpoint{3.931558in}{1.542665in}}%
\pgfpathlineto{\pgfqpoint{3.935272in}{1.543670in}}%
\pgfpathlineto{\pgfqpoint{3.938058in}{1.545188in}}%
\pgfpathlineto{\pgfqpoint{3.942701in}{1.546744in}}%
\pgfpathlineto{\pgfqpoint{3.946415in}{1.547817in}}%
\pgfpathlineto{\pgfqpoint{3.953844in}{1.551008in}}%
\pgfpathlineto{\pgfqpoint{3.954773in}{1.551008in}}%
\pgfpathlineto{\pgfqpoint{3.956630in}{1.552127in}}%
\pgfpathlineto{\pgfqpoint{3.960344in}{1.553212in}}%
\pgfpathlineto{\pgfqpoint{3.967773in}{1.556559in}}%
\pgfpathlineto{\pgfqpoint{3.968701in}{1.556559in}}%
\pgfpathlineto{\pgfqpoint{3.970559in}{1.557692in}}%
\pgfpathlineto{\pgfqpoint{3.972416in}{1.558260in}}%
\pgfpathlineto{\pgfqpoint{3.977059in}{1.559959in}}%
\pgfpathlineto{\pgfqpoint{3.977987in}{1.559959in}}%
\pgfpathlineto{\pgfqpoint{3.979845in}{1.561118in}}%
\pgfpathlineto{\pgfqpoint{3.980773in}{1.561118in}}%
\pgfpathlineto{\pgfqpoint{3.982630in}{1.562275in}}%
\pgfpathlineto{\pgfqpoint{3.983559in}{1.562275in}}%
\pgfpathlineto{\pgfqpoint{3.985416in}{1.563448in}}%
\pgfpathlineto{\pgfqpoint{3.986345in}{1.563448in}}%
\pgfpathlineto{\pgfqpoint{3.988202in}{1.564612in}}%
\pgfpathlineto{\pgfqpoint{3.989131in}{1.564612in}}%
\pgfpathlineto{\pgfqpoint{3.990988in}{1.565786in}}%
\pgfpathlineto{\pgfqpoint{3.991916in}{1.565786in}}%
\pgfpathlineto{\pgfqpoint{3.993774in}{1.566959in}}%
\pgfpathlineto{\pgfqpoint{3.994702in}{1.566959in}}%
\pgfpathlineto{\pgfqpoint{3.996559in}{1.568132in}}%
\pgfpathlineto{\pgfqpoint{3.997488in}{1.568132in}}%
\pgfpathlineto{\pgfqpoint{3.999345in}{1.569328in}}%
\pgfpathlineto{\pgfqpoint{4.001202in}{1.569907in}}%
\pgfpathlineto{\pgfqpoint{4.007703in}{1.572883in}}%
\pgfpathlineto{\pgfqpoint{4.008631in}{1.572883in}}%
\pgfpathlineto{\pgfqpoint{4.010488in}{1.574075in}}%
\pgfpathlineto{\pgfqpoint{4.012346in}{1.574075in}}%
\pgfpathlineto{\pgfqpoint{4.014203in}{1.575269in}}%
\pgfpathlineto{\pgfqpoint{4.015131in}{1.575269in}}%
\pgfpathlineto{\pgfqpoint{4.017917in}{1.577667in}}%
\pgfpathlineto{\pgfqpoint{4.019774in}{1.577667in}}%
\pgfpathlineto{\pgfqpoint{4.021632in}{1.578877in}}%
\pgfpathlineto{\pgfqpoint{4.023489in}{1.579482in}}%
\pgfpathlineto{\pgfqpoint{4.028132in}{1.581293in}}%
\pgfpathlineto{\pgfqpoint{4.029060in}{1.581293in}}%
\pgfpathlineto{\pgfqpoint{4.030918in}{1.582516in}}%
\pgfpathlineto{\pgfqpoint{4.031846in}{1.582516in}}%
\pgfpathlineto{\pgfqpoint{4.033703in}{1.583728in}}%
\pgfpathlineto{\pgfqpoint{4.034632in}{1.583728in}}%
\pgfpathlineto{\pgfqpoint{4.036489in}{1.584937in}}%
\pgfpathlineto{\pgfqpoint{4.037418in}{1.584937in}}%
\pgfpathlineto{\pgfqpoint{4.039275in}{1.586156in}}%
\pgfpathlineto{\pgfqpoint{4.040204in}{1.586156in}}%
\pgfpathlineto{\pgfqpoint{4.042061in}{1.587376in}}%
\pgfpathlineto{\pgfqpoint{4.042989in}{1.587376in}}%
\pgfpathlineto{\pgfqpoint{4.044847in}{1.588570in}}%
\pgfpathlineto{\pgfqpoint{4.045775in}{1.588570in}}%
\pgfpathlineto{\pgfqpoint{4.047632in}{1.589795in}}%
\pgfpathlineto{\pgfqpoint{4.048561in}{1.589795in}}%
\pgfpathlineto{\pgfqpoint{4.050418in}{1.591005in}}%
\pgfpathlineto{\pgfqpoint{4.051347in}{1.591005in}}%
\pgfpathlineto{\pgfqpoint{4.053204in}{1.592200in}}%
\pgfpathlineto{\pgfqpoint{4.055061in}{1.592200in}}%
\pgfpathlineto{\pgfqpoint{4.057847in}{1.594018in}}%
\pgfpathlineto{\pgfqpoint{4.064347in}{1.597070in}}%
\pgfpathlineto{\pgfqpoint{4.066204in}{1.597070in}}%
\pgfpathlineto{\pgfqpoint{4.068061in}{1.598278in}}%
\pgfpathlineto{\pgfqpoint{4.068990in}{1.598278in}}%
\pgfpathlineto{\pgfqpoint{4.070847in}{1.599491in}}%
\pgfpathlineto{\pgfqpoint{4.071776in}{1.599491in}}%
\pgfpathlineto{\pgfqpoint{4.073633in}{1.600727in}}%
\pgfpathlineto{\pgfqpoint{4.074562in}{1.600727in}}%
\pgfpathlineto{\pgfqpoint{4.077347in}{1.602538in}}%
\pgfpathlineto{\pgfqpoint{4.081990in}{1.604391in}}%
\pgfpathlineto{\pgfqpoint{4.082919in}{1.604391in}}%
\pgfpathlineto{\pgfqpoint{4.084776in}{1.605606in}}%
\pgfpathlineto{\pgfqpoint{4.085705in}{1.605606in}}%
\pgfpathlineto{\pgfqpoint{4.087562in}{1.606840in}}%
\pgfpathlineto{\pgfqpoint{4.088491in}{1.606840in}}%
\pgfpathlineto{\pgfqpoint{4.090348in}{1.608048in}}%
\pgfpathlineto{\pgfqpoint{4.092205in}{1.608650in}}%
\pgfpathlineto{\pgfqpoint{4.098705in}{1.611750in}}%
\pgfpathlineto{\pgfqpoint{4.099634in}{1.611750in}}%
\pgfpathlineto{\pgfqpoint{4.101491in}{1.612937in}}%
\pgfpathlineto{\pgfqpoint{4.103348in}{1.613550in}}%
\pgfpathlineto{\pgfqpoint{4.109848in}{1.616594in}}%
\pgfpathlineto{\pgfqpoint{4.111706in}{1.617197in}}%
\pgfpathlineto{\pgfqpoint{4.116349in}{1.619038in}}%
\pgfpathlineto{\pgfqpoint{4.118206in}{1.622751in}}%
\pgfpathlineto{\pgfqpoint{4.120063in}{1.622751in}}%
\pgfpathlineto{\pgfqpoint{4.121920in}{1.624237in}}%
\pgfpathlineto{\pgfqpoint{4.125635in}{1.625051in}}%
\pgfpathlineto{\pgfqpoint{4.129349in}{1.625270in}}%
\pgfpathlineto{\pgfqpoint{4.132135in}{1.625914in}}%
\pgfpathlineto{\pgfqpoint{4.135849in}{1.627775in}}%
\pgfpathlineto{\pgfqpoint{4.136778in}{1.627775in}}%
\pgfpathlineto{\pgfqpoint{4.138635in}{1.636249in}}%
\pgfpathlineto{\pgfqpoint{4.139564in}{1.636249in}}%
\pgfpathlineto{\pgfqpoint{4.141421in}{1.637793in}}%
\pgfpathlineto{\pgfqpoint{4.142349in}{1.637793in}}%
\pgfpathlineto{\pgfqpoint{4.144207in}{1.639085in}}%
\pgfpathlineto{\pgfqpoint{4.146064in}{1.639085in}}%
\pgfpathlineto{\pgfqpoint{4.147921in}{1.641298in}}%
\pgfpathlineto{\pgfqpoint{4.148850in}{1.641298in}}%
\pgfpathlineto{\pgfqpoint{4.150707in}{1.642686in}}%
\pgfpathlineto{\pgfqpoint{4.156278in}{1.643780in}}%
\pgfpathlineto{\pgfqpoint{4.158136in}{1.645471in}}%
\pgfpathlineto{\pgfqpoint{4.165564in}{1.646154in}}%
\pgfpathlineto{\pgfqpoint{4.167422in}{1.647850in}}%
\pgfpathlineto{\pgfqpoint{4.168350in}{1.647850in}}%
\pgfpathlineto{\pgfqpoint{4.170207in}{1.649401in}}%
\pgfpathlineto{\pgfqpoint{4.173922in}{1.650402in}}%
\pgfpathlineto{\pgfqpoint{4.176707in}{1.650980in}}%
\pgfpathlineto{\pgfqpoint{4.178565in}{1.647722in}}%
\pgfpathlineto{\pgfqpoint{4.182279in}{1.647817in}}%
\pgfpathlineto{\pgfqpoint{4.184136in}{1.653851in}}%
\pgfpathlineto{\pgfqpoint{4.185065in}{1.653851in}}%
\pgfpathlineto{\pgfqpoint{4.187851in}{1.650364in}}%
\pgfpathlineto{\pgfqpoint{4.189708in}{1.656256in}}%
\pgfpathlineto{\pgfqpoint{4.193422in}{1.655281in}}%
\pgfpathlineto{\pgfqpoint{4.196208in}{1.653364in}}%
\pgfpathlineto{\pgfqpoint{4.198994in}{1.660953in}}%
\pgfpathlineto{\pgfqpoint{4.200851in}{1.661828in}}%
\pgfpathlineto{\pgfqpoint{4.204565in}{1.663657in}}%
\pgfpathlineto{\pgfqpoint{4.208280in}{1.666133in}}%
\pgfpathlineto{\pgfqpoint{4.211994in}{1.667209in}}%
\pgfpathlineto{\pgfqpoint{4.213851in}{1.667209in}}%
\pgfpathlineto{\pgfqpoint{4.215709in}{1.668937in}}%
\pgfpathlineto{\pgfqpoint{4.216637in}{1.668937in}}%
\pgfpathlineto{\pgfqpoint{4.218494in}{1.670278in}}%
\pgfpathlineto{\pgfqpoint{4.222209in}{1.670413in}}%
\pgfpathlineto{\pgfqpoint{4.224066in}{1.672188in}}%
\pgfpathlineto{\pgfqpoint{4.224995in}{1.672188in}}%
\pgfpathlineto{\pgfqpoint{4.226852in}{1.674043in}}%
\pgfpathlineto{\pgfqpoint{4.231495in}{1.674902in}}%
\pgfpathlineto{\pgfqpoint{4.233352in}{1.675393in}}%
\pgfpathlineto{\pgfqpoint{4.235209in}{1.670787in}}%
\pgfpathlineto{\pgfqpoint{4.236138in}{1.670787in}}%
\pgfpathlineto{\pgfqpoint{4.237995in}{1.672079in}}%
\pgfpathlineto{\pgfqpoint{4.238924in}{1.672079in}}%
\pgfpathlineto{\pgfqpoint{4.240781in}{1.677483in}}%
\pgfpathlineto{\pgfqpoint{4.241709in}{1.677483in}}%
\pgfpathlineto{\pgfqpoint{4.243567in}{1.679286in}}%
\pgfpathlineto{\pgfqpoint{4.248210in}{1.680378in}}%
\pgfpathlineto{\pgfqpoint{4.250067in}{1.680869in}}%
\pgfpathlineto{\pgfqpoint{4.252853in}{1.679608in}}%
\pgfpathlineto{\pgfqpoint{4.256567in}{1.680844in}}%
\pgfpathlineto{\pgfqpoint{4.258424in}{1.681903in}}%
\pgfpathlineto{\pgfqpoint{4.259353in}{1.681903in}}%
\pgfpathlineto{\pgfqpoint{4.261210in}{1.683048in}}%
\pgfpathlineto{\pgfqpoint{4.264924in}{1.683813in}}%
\pgfpathlineto{\pgfqpoint{4.266782in}{1.685401in}}%
\pgfpathlineto{\pgfqpoint{4.267710in}{1.685401in}}%
\pgfpathlineto{\pgfqpoint{4.269567in}{1.698306in}}%
\pgfpathlineto{\pgfqpoint{4.270496in}{1.698306in}}%
\pgfpathlineto{\pgfqpoint{4.272353in}{1.699727in}}%
\pgfpathlineto{\pgfqpoint{4.276067in}{1.700580in}}%
\pgfpathlineto{\pgfqpoint{4.277925in}{1.702239in}}%
\pgfpathlineto{\pgfqpoint{4.278853in}{1.702239in}}%
\pgfpathlineto{\pgfqpoint{4.280710in}{1.703792in}}%
\pgfpathlineto{\pgfqpoint{4.281639in}{1.703792in}}%
\pgfpathlineto{\pgfqpoint{4.283496in}{1.705205in}}%
\pgfpathlineto{\pgfqpoint{4.285353in}{1.705205in}}%
\pgfpathlineto{\pgfqpoint{4.287211in}{1.706472in}}%
\pgfpathlineto{\pgfqpoint{4.288139in}{1.706472in}}%
\pgfpathlineto{\pgfqpoint{4.289996in}{1.703426in}}%
\pgfpathlineto{\pgfqpoint{4.290925in}{1.703426in}}%
\pgfpathlineto{\pgfqpoint{4.292782in}{1.708177in}}%
\pgfpathlineto{\pgfqpoint{4.295568in}{1.705961in}}%
\pgfpathlineto{\pgfqpoint{4.297425in}{1.708107in}}%
\pgfpathlineto{\pgfqpoint{4.302068in}{1.707542in}}%
\pgfpathlineto{\pgfqpoint{4.303925in}{1.709928in}}%
\pgfpathlineto{\pgfqpoint{4.304854in}{1.709928in}}%
\pgfpathlineto{\pgfqpoint{4.306711in}{1.702341in}}%
\pgfpathlineto{\pgfqpoint{4.307640in}{1.702341in}}%
\pgfpathlineto{\pgfqpoint{4.309497in}{1.704769in}}%
\pgfpathlineto{\pgfqpoint{4.313211in}{1.704963in}}%
\pgfpathlineto{\pgfqpoint{4.315069in}{1.706262in}}%
\pgfpathlineto{\pgfqpoint{4.315997in}{1.706262in}}%
\pgfpathlineto{\pgfqpoint{4.317854in}{1.707533in}}%
\pgfpathlineto{\pgfqpoint{4.318783in}{1.707533in}}%
\pgfpathlineto{\pgfqpoint{4.320640in}{1.709966in}}%
\pgfpathlineto{\pgfqpoint{4.325283in}{1.710801in}}%
\pgfpathlineto{\pgfqpoint{4.327140in}{1.712170in}}%
\pgfpathlineto{\pgfqpoint{4.330855in}{1.713750in}}%
\pgfpathlineto{\pgfqpoint{4.336426in}{1.716262in}}%
\pgfpathlineto{\pgfqpoint{4.337355in}{1.717292in}}%
\pgfpathlineto{\pgfqpoint{4.341998in}{1.717826in}}%
\pgfpathlineto{\pgfqpoint{4.343855in}{1.719095in}}%
\pgfpathlineto{\pgfqpoint{4.344784in}{1.719095in}}%
\pgfpathlineto{\pgfqpoint{4.347570in}{1.720989in}}%
\pgfpathlineto{\pgfqpoint{4.352213in}{1.722952in}}%
\pgfpathlineto{\pgfqpoint{4.353141in}{1.722952in}}%
\pgfpathlineto{\pgfqpoint{4.354998in}{1.724231in}}%
\pgfpathlineto{\pgfqpoint{4.355927in}{1.724231in}}%
\pgfpathlineto{\pgfqpoint{4.357784in}{1.725504in}}%
\pgfpathlineto{\pgfqpoint{4.358713in}{1.725504in}}%
\pgfpathlineto{\pgfqpoint{4.360570in}{1.726808in}}%
\pgfpathlineto{\pgfqpoint{4.361499in}{1.726808in}}%
\pgfpathlineto{\pgfqpoint{4.363356in}{1.728093in}}%
\pgfpathlineto{\pgfqpoint{4.365213in}{1.728740in}}%
\pgfpathlineto{\pgfqpoint{4.371713in}{1.731986in}}%
\pgfpathlineto{\pgfqpoint{4.372642in}{1.731986in}}%
\pgfpathlineto{\pgfqpoint{4.374499in}{1.740877in}}%
\pgfpathlineto{\pgfqpoint{4.375428in}{1.740877in}}%
\pgfpathlineto{\pgfqpoint{4.377285in}{1.743302in}}%
\pgfpathlineto{\pgfqpoint{4.379142in}{1.743302in}}%
\pgfpathlineto{\pgfqpoint{4.380999in}{1.745140in}}%
\pgfpathlineto{\pgfqpoint{4.391214in}{1.749700in}}%
\pgfpathlineto{\pgfqpoint{4.393071in}{1.750539in}}%
\pgfpathlineto{\pgfqpoint{4.394928in}{1.751377in}}%
\pgfpathlineto{\pgfqpoint{4.395857in}{1.751377in}}%
\pgfpathlineto{\pgfqpoint{4.397714in}{1.752858in}}%
\pgfpathlineto{\pgfqpoint{4.401428in}{1.753602in}}%
\pgfpathlineto{\pgfqpoint{4.403285in}{1.754687in}}%
\pgfpathlineto{\pgfqpoint{4.404214in}{1.754687in}}%
\pgfpathlineto{\pgfqpoint{4.406071in}{1.756137in}}%
\pgfpathlineto{\pgfqpoint{4.409786in}{1.757068in}}%
\pgfpathlineto{\pgfqpoint{4.411643in}{1.758237in}}%
\pgfpathlineto{\pgfqpoint{4.412571in}{1.758237in}}%
\pgfpathlineto{\pgfqpoint{4.414429in}{1.759862in}}%
\pgfpathlineto{\pgfqpoint{4.415357in}{1.759862in}}%
\pgfpathlineto{\pgfqpoint{4.417214in}{1.761087in}}%
\pgfpathlineto{\pgfqpoint{4.418143in}{1.761087in}}%
\pgfpathlineto{\pgfqpoint{4.420000in}{1.763148in}}%
\pgfpathlineto{\pgfqpoint{4.421857in}{1.764153in}}%
\pgfpathlineto{\pgfqpoint{4.423715in}{1.765157in}}%
\pgfpathlineto{\pgfqpoint{4.424643in}{1.765157in}}%
\pgfpathlineto{\pgfqpoint{4.426500in}{1.766395in}}%
\pgfpathlineto{\pgfqpoint{4.429286in}{1.767505in}}%
\pgfpathlineto{\pgfqpoint{4.431143in}{1.768763in}}%
\pgfpathlineto{\pgfqpoint{4.433001in}{1.768763in}}%
\pgfpathlineto{\pgfqpoint{4.434858in}{1.770038in}}%
\pgfpathlineto{\pgfqpoint{4.435786in}{1.770038in}}%
\pgfpathlineto{\pgfqpoint{4.437644in}{1.771419in}}%
\pgfpathlineto{\pgfqpoint{4.438572in}{1.771419in}}%
\pgfpathlineto{\pgfqpoint{4.440429in}{1.772907in}}%
\pgfpathlineto{\pgfqpoint{4.441358in}{1.772907in}}%
\pgfpathlineto{\pgfqpoint{4.443215in}{1.774043in}}%
\pgfpathlineto{\pgfqpoint{4.444144in}{1.774043in}}%
\pgfpathlineto{\pgfqpoint{4.446001in}{1.775685in}}%
\pgfpathlineto{\pgfqpoint{4.446930in}{1.775685in}}%
\pgfpathlineto{\pgfqpoint{4.448787in}{1.777299in}}%
\pgfpathlineto{\pgfqpoint{4.452501in}{1.778325in}}%
\pgfpathlineto{\pgfqpoint{4.454358in}{1.780137in}}%
\pgfpathlineto{\pgfqpoint{4.455287in}{1.780137in}}%
\pgfpathlineto{\pgfqpoint{4.457144in}{1.781570in}}%
\pgfpathlineto{\pgfqpoint{4.458073in}{1.781570in}}%
\pgfpathlineto{\pgfqpoint{4.459930in}{1.782962in}}%
\pgfpathlineto{\pgfqpoint{4.460859in}{1.782962in}}%
\pgfpathlineto{\pgfqpoint{4.462716in}{1.784245in}}%
\pgfpathlineto{\pgfqpoint{4.463644in}{1.784245in}}%
\pgfpathlineto{\pgfqpoint{4.465502in}{1.785392in}}%
\pgfpathlineto{\pgfqpoint{4.469216in}{1.786218in}}%
\pgfpathlineto{\pgfqpoint{4.471073in}{1.787174in}}%
\pgfpathlineto{\pgfqpoint{4.475716in}{1.786535in}}%
\pgfpathlineto{\pgfqpoint{4.477573in}{1.788791in}}%
\pgfpathlineto{\pgfqpoint{4.481288in}{1.789687in}}%
\pgfpathlineto{\pgfqpoint{4.484074in}{1.790220in}}%
\pgfpathlineto{\pgfqpoint{4.485931in}{1.790279in}}%
\pgfpathlineto{\pgfqpoint{4.487788in}{1.787268in}}%
\pgfpathlineto{\pgfqpoint{4.491502in}{1.788189in}}%
\pgfpathlineto{\pgfqpoint{4.491502in}{1.788189in}}%
\pgfusepath{stroke}%
\end{pgfscope}%
\begin{pgfscope}%
\pgfpathrectangle{\pgfqpoint{0.592318in}{0.451986in}}{\pgfqpoint{4.085832in}{1.402021in}}%
\pgfusepath{clip}%
\pgfsetbuttcap%
\pgfsetroundjoin%
\pgfsetlinewidth{1.003750pt}%
\definecolor{currentstroke}{rgb}{0.909804,0.254902,0.094118}%
\pgfsetstrokecolor{currentstroke}%
\pgfsetdash{{3.700000pt}{1.600000pt}}{0.000000pt}%
\pgfpathmoveto{\pgfqpoint{0.778038in}{0.515714in}}%
\pgfpathlineto{\pgfqpoint{1.770709in}{0.851536in}}%
\pgfpathlineto{\pgfqpoint{1.770709in}{0.851536in}}%
\pgfusepath{stroke}%
\end{pgfscope}%
\begin{pgfscope}%
\pgfpathrectangle{\pgfqpoint{0.592318in}{0.451986in}}{\pgfqpoint{4.085832in}{1.402021in}}%
\pgfusepath{clip}%
\pgfsetbuttcap%
\pgfsetroundjoin%
\pgfsetlinewidth{1.003750pt}%
\definecolor{currentstroke}{rgb}{0.909804,0.254902,0.094118}%
\pgfsetstrokecolor{currentstroke}%
\pgfsetdash{{1.000000pt}{1.650000pt}}{0.000000pt}%
\pgfpathmoveto{\pgfqpoint{1.770709in}{0.851536in}}%
\pgfpathlineto{\pgfqpoint{4.492431in}{1.772297in}}%
\pgfpathlineto{\pgfqpoint{4.492431in}{1.772297in}}%
\pgfusepath{stroke}%
\end{pgfscope}%
\begin{pgfscope}%
\pgfsetrectcap%
\pgfsetmiterjoin%
\pgfsetlinewidth{0.803000pt}%
\definecolor{currentstroke}{rgb}{0.000000,0.000000,0.000000}%
\pgfsetstrokecolor{currentstroke}%
\pgfsetdash{}{0pt}%
\pgfpathmoveto{\pgfqpoint{0.592318in}{0.451986in}}%
\pgfpathlineto{\pgfqpoint{0.592318in}{1.854007in}}%
\pgfusepath{stroke}%
\end{pgfscope}%
\begin{pgfscope}%
\pgfsetrectcap%
\pgfsetmiterjoin%
\pgfsetlinewidth{0.000000pt}%
\definecolor{currentstroke}{rgb}{0.000000,0.000000,0.000000}%
\pgfsetstrokecolor{currentstroke}%
\pgfsetstrokeopacity{0.000000}%
\pgfsetdash{}{0pt}%
\pgfpathmoveto{\pgfqpoint{4.678151in}{0.451986in}}%
\pgfpathlineto{\pgfqpoint{4.678151in}{1.854007in}}%
\pgfusepath{}%
\end{pgfscope}%
\begin{pgfscope}%
\pgfsetrectcap%
\pgfsetmiterjoin%
\pgfsetlinewidth{0.803000pt}%
\definecolor{currentstroke}{rgb}{0.000000,0.000000,0.000000}%
\pgfsetstrokecolor{currentstroke}%
\pgfsetdash{}{0pt}%
\pgfpathmoveto{\pgfqpoint{0.592318in}{0.451986in}}%
\pgfpathlineto{\pgfqpoint{4.678151in}{0.451986in}}%
\pgfusepath{stroke}%
\end{pgfscope}%
\begin{pgfscope}%
\pgfsetrectcap%
\pgfsetmiterjoin%
\pgfsetlinewidth{0.000000pt}%
\definecolor{currentstroke}{rgb}{0.000000,0.000000,0.000000}%
\pgfsetstrokecolor{currentstroke}%
\pgfsetstrokeopacity{0.000000}%
\pgfsetdash{}{0pt}%
\pgfpathmoveto{\pgfqpoint{0.592318in}{1.854007in}}%
\pgfpathlineto{\pgfqpoint{4.678151in}{1.854007in}}%
\pgfusepath{}%
\end{pgfscope}%
\begin{pgfscope}%
\pgfsetbuttcap%
\pgfsetmiterjoin%
\definecolor{currentfill}{rgb}{1.000000,1.000000,1.000000}%
\pgfsetfillcolor{currentfill}%
\pgfsetfillopacity{0.800000}%
\pgfsetlinewidth{1.003750pt}%
\definecolor{currentstroke}{rgb}{0.800000,0.800000,0.800000}%
\pgfsetstrokecolor{currentstroke}%
\pgfsetstrokeopacity{0.800000}%
\pgfsetdash{}{0pt}%
\pgfpathmoveto{\pgfqpoint{0.670096in}{1.160485in}}%
\pgfpathlineto{\pgfqpoint{2.287065in}{1.160485in}}%
\pgfpathquadraticcurveto{\pgfqpoint{2.309287in}{1.160485in}}{\pgfqpoint{2.309287in}{1.182707in}}%
\pgfpathlineto{\pgfqpoint{2.309287in}{1.776229in}}%
\pgfpathquadraticcurveto{\pgfqpoint{2.309287in}{1.798451in}}{\pgfqpoint{2.287065in}{1.798451in}}%
\pgfpathlineto{\pgfqpoint{0.670096in}{1.798451in}}%
\pgfpathquadraticcurveto{\pgfqpoint{0.647874in}{1.798451in}}{\pgfqpoint{0.647874in}{1.776229in}}%
\pgfpathlineto{\pgfqpoint{0.647874in}{1.182707in}}%
\pgfpathquadraticcurveto{\pgfqpoint{0.647874in}{1.160485in}}{\pgfqpoint{0.670096in}{1.160485in}}%
\pgfpathclose%
\pgfusepath{stroke,fill}%
\end{pgfscope}%
\begin{pgfscope}%
\pgfsetrectcap%
\pgfsetroundjoin%
\pgfsetlinewidth{1.003750pt}%
\definecolor{currentstroke}{rgb}{0.152941,0.235294,0.458824}%
\pgfsetstrokecolor{currentstroke}%
\pgfsetdash{}{0pt}%
\pgfpathmoveto{\pgfqpoint{0.692318in}{1.715118in}}%
\pgfpathlineto{\pgfqpoint{0.914540in}{1.715118in}}%
\pgfusepath{stroke}%
\end{pgfscope}%
\begin{pgfscope}%
\definecolor{textcolor}{rgb}{0.000000,0.000000,0.000000}%
\pgfsetstrokecolor{textcolor}%
\pgfsetfillcolor{textcolor}%
\pgftext[x=1.003429in,y=1.676229in,left,base]{\color{textcolor}\rmfamily\fontsize{8.000000}{9.600000}\selectfont Messwerte \(\displaystyle R_{xy}\)}%
\end{pgfscope}%
\begin{pgfscope}%
\pgfsetbuttcap%
\pgfsetroundjoin%
\pgfsetlinewidth{1.003750pt}%
\definecolor{currentstroke}{rgb}{0.909804,0.254902,0.094118}%
\pgfsetstrokecolor{currentstroke}%
\pgfsetdash{{3.700000pt}{1.600000pt}}{0.000000pt}%
\pgfpathmoveto{\pgfqpoint{0.692318in}{1.475848in}}%
\pgfpathlineto{\pgfqpoint{0.914540in}{1.475848in}}%
\pgfusepath{stroke}%
\end{pgfscope}%
\begin{pgfscope}%
\definecolor{textcolor}{rgb}{0.000000,0.000000,0.000000}%
\pgfsetstrokecolor{textcolor}%
\pgfsetfillcolor{textcolor}%
\pgftext[x=1.003429in,y=1.496253in,left,base]{\color{textcolor}\rmfamily\fontsize{8.000000}{9.600000}\selectfont Linearer Fit \(\displaystyle f(x) = a \cdot x\)}%
\end{pgfscope}%
\begin{pgfscope}%
\definecolor{textcolor}{rgb}{0.000000,0.000000,0.000000}%
\pgfsetstrokecolor{textcolor}%
\pgfsetfillcolor{textcolor}%
\pgftext[x=1.003429in,y=1.376646in,left,base]{\color{textcolor}\rmfamily\fontsize{8.000000}{9.600000}\selectfont \(\displaystyle a = 1347.0 \)}%
\end{pgfscope}%
\begin{pgfscope}%
\pgfsetbuttcap%
\pgfsetroundjoin%
\pgfsetlinewidth{1.003750pt}%
\definecolor{currentstroke}{rgb}{0.909804,0.254902,0.094118}%
\pgfsetstrokecolor{currentstroke}%
\pgfsetdash{{1.000000pt}{1.650000pt}}{0.000000pt}%
\pgfpathmoveto{\pgfqpoint{0.692318in}{1.265418in}}%
\pgfpathlineto{\pgfqpoint{0.914540in}{1.265418in}}%
\pgfusepath{stroke}%
\end{pgfscope}%
\begin{pgfscope}%
\definecolor{textcolor}{rgb}{0.000000,0.000000,0.000000}%
\pgfsetstrokecolor{textcolor}%
\pgfsetfillcolor{textcolor}%
\pgftext[x=1.003429in,y=1.226529in,left,base]{\color{textcolor}\rmfamily\fontsize{8.000000}{9.600000}\selectfont Fit Extrapolation}%
\end{pgfscope}%
\end{pgfpicture}%
\makeatother%
\endgroup%

	\caption{}
	\label{abb:P07_Meth_3}
\end{figure}

\begin{table}[h]
	\centering
	\caption{}
	\begin{tabular}{lrrrr}
\toprule
         Messreihe &  $R_{xx} (B=0) [\Omega]$ & $\partial_B R_{xy}$ &  E-Dichte$(\varnothing) [1/\si{m}^2]$ &  E-Mobilität \\
\midrule
 P07\_0611\_0708\_0109 &  106.743 &      1347 &              4.634e+15 &             15.77 \\
 P07\_0611\_0709\_0109 &  182.622 &      1354 &              4.608e+15 &             18.54 \\
 P07\_1115\_1213\_2012 &   77.843 &      1336 &              4.673e+15 &             21.45 \\
 P07\_1115\_2018\_1913 &  160.499 &      1352 &              4.616e+15 &             21.06 \\
\bottomrule
\end{tabular}

	\label{tab:Meth_3}
\end{table}

\subsubsection{Schwankungen in der Elektronenmobiltität}
Da wir für alle Methoden in guter Näherung die gleiche Elektronenkonzentration bestimmen konnten sind die bestimmten Mobilitäten weitesgehend unabhängig von der Berechnungsmethodik. Es liegt damit nahe, dass es sich nicht um lokale Messfehler o.Ä. handelt sondern um tatsächliche Unterschiede im Messaufbau / Ablauf. 

Die Elektronenmobilität hängt nur vom spezifischen Widerstand (und damit dem Längswiderstand bei $ B = 0 $) und der E-Dichte ab. Weil letztere aber für alle Werte näherungsweise konstant ist, müssen die Schwankungen aus Unterschieden im Längswiderstand stammen. Dafür sind mehrere Ursachen möglich. 

Auffällig ist, dass die erste Messreihe einen besonders signifikanten Ausreißer darstellt. Die dritte und vierte Messreihe liegen sehr nah beieinander, die zweite dazwischen. Wir halten daher den Einfluss von Magnetisierungseffekten für möglich: da die Hallgeometrie vor der ersten Messung keinem Magnetfeld ausgesetzt war (bzw. sich längere Zeit bei Zimmertemperatur befand) ist dies eine mögliche Erklärung für die Unterschiede zur dritten und vierten Messung. 

Eine weitere mögliche Erklärung wären Defekte in der Geometrie selbst. Wie bereits in Abschnitt ?? ausgeführt wirkte diese abgenutzt. Da Elektronen beim Ladungstransport an Störstellen gestreut werden, wären Defekte in der Lage, die Driftgeschwindigkeit und damit die Ladungsträgermobilität zu reduzieren. 
